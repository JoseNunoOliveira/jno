\documentclass[9pt]{article}
\usepackage{textcomp}
\usepackage[a5paper]{geometry}
%------ using color ---------------------------------------------------------
\usepackage{color}
%input{rgb}
\def\black{\color{black}}
\def\blue{\color{blue}}
\def\red{\color{red}}
%definecolor{myGray}{gray}{0.85}
%----------------------------------------------------------------------------
%-------------------------------------------------------------------------------
\pdfpkresolution=1200
%-------------------------------------------------------------------------------
%\usepackage[pages=all]{background}
%\backgroundsetup{
%scale=1,
%color=black,
%opacity=0.6,
%angle=0,
%contents={%
%\unitlength=1mm
%\framebox(148,210){}
%  }%
%}
%-------------------------------------------------------------------------------
% A5 = 148 mm x 210 mm
%-------------------------------------------------------------------------------
\textheight  180mm
\textwidth   110mm
\headheight  -1.0cm
\oddsidemargin  -7.0mm
\evensidemargin  -7.0mm
%usepackage[b4,center,cross,pdflatex]{crop}
%-------------------------------------------------------------------------------
\usepackage{llp}
%-------------------------------------------------------------------------------
\makeindex
%-------------------------------------------------------------------------------
\begin{document}
%-------------------------------------------------------------------------------
\def\verso{\newpage\thispagestyle{empty}~\newpage}
%-------------------------------------------------------------------------------
	\capa
	\verso
	\setcounter{page}{1}
%-------------------------------------------------------------------------------

\thispagestyle{empty}

~\vfill

\begin{center}\large\sc
	As
\vskip 0.9ex
	Atribulações de Eugénio
\vskip 3.0ex
	(Subsídio para a autobiografia)
\vskip 6.0ex
Ludgero Lopes Parreira
\vskip 6.0ex

  	\includegraphics[width=0.65\textwidth]{llp_files/llp.jpeg}
\vskip 6ex
Pôrto,
\vskip 1ex
Agôsto de 1938
\end{center}

\verso

\thispagestyle{empty} ~ \vskip 7em

{\small
\begin{center}\sc\normalsize Nota introdutória\end{center}

Com a publica\-ção deste romance de cunho autobiográfico, escrito pelo meu Pai entre 1924 e 1943, pretendo homenagear a sua memória, 125 anos passados do seu nascimento.

Com a sua personalidade tímida e discreta, ele descreve, neste texto que nos deixou, a paixão entre as personagens Eugénio e Maria Princesa, ficcionada na cidade do Porto na primeira metade do séc.\ XX. Essa fic\-ção mostra o carácter repleto de valores e princípios que o orientaram na sua luta contra as adversidades que enfrentou desde criança.
Assim se vê resgatada, também, a memória da sua vida, que muitos não imaginariam ser tão rica e interessante.

Órfão de tenra idade, constituiu sempre uma obsessão de meu Pai a constru\-ção de uma fa\-mí\-lia e de um lar harmonioso onde nada faltasse, sempre sob o preceito de uma economia ponderada –- a única que em sua opinião se traduz em poupan\-ça e equilíbrio.

Da leitura deste texto, escrito na sua ve\-lha \emph{Remington Portable model 5}, bem como da inúmera correspondên\-cia existente, compreendi me\-lhor, como fi\-lho, os anseios de meu Pai no rigor da minha educa\-ção, pelo que expresso a minha reconhecida gratidão.

Os opúsculos que publicou enquanto médico escolar mostram gran\-de preocupa\-ção com a higiene, alimenta\-ção e saúde física e moral dos jovens estudantes, inquieta\-ção que perpassa o próprio romance. Há trechos em que revela uma severa crítica social à época, enquadrada no período histórico que se vivia --- a eclosão da 2ª Guerra Mundial --- conflito que aliás descreve como espectador atento.

O texto deve ser lido tendo em conta o contexto cultural do Portugal de então e o percurso individual do meu Pai; apelo, por isso, à benevolência do leitor quanto a aspectos mais datados, nomeadamente uma concepção talvez demasiado tradicional da família e do papel da mulher na sociedade.

Na transcri\-ção do texto dactilografado achei por bem respeitar a grafia original, por achar que torna a percep\-ção do seu conteúdo mais autêntica.

Finalmente, que a apresenta\-ção deste romance despretensioso, um subsídio para uma autobiografia que nunca chegou a escrever, sirva para promover mais uma reunião de fa\-mí\-lia, culto que verdadeiramente herdamos dos nossos antepassados.

\rule{0pt}{1.5em}

~\hfill José Henrique Oliveira Parreira

\verso

\newpage \newpage\rule{0pt}{7em}\begin{center}\sc\normalsize Breve biografia do autor\end{center} \thispagestyle{empty}

Ludgero Lopes Parreira nasceu em Penha-Longa, Marco de Canavezes, En\-tre-Douro e Minho, a 23 de Junho de 1900, e faleceu no Porto em 1974.

Formado em Medicina pela Escola Médica do Porto (1930), iniciou a sua carreira clínica no Alentejo, em Portalegre, onde permaneceu três anos (1930-1933). Regressou ao Norte, exercendo, também, o cargo de Médico Escolar (1934-1970) em várias escolas e liceus da cidade do Porto, bem como o de assistente de vacina\-ção no BCG (1961-1970). Em 1948 torna-se médico efectivo dos Serviços Médico-Sociais e é posteriormente convidado para um lugar de Inspector-Médico nesses serviços.

Publicou mais de três dezenas de traba\-lhos académicos, sendo-\-lhe atribuída pela Funda\-ção Calouste Gulbenkian (1956) uma bolsa de estudo no estrangeiro que, por motivos pessoais, não pôde ser aproveitada. Recebeu um louvor publicado no Diário do Governo (2ª
série, nº 97, de 27-4-1934).

Do seu casamento com Ângela Martins Albuquerque Oliveira, em 1945, nasceu José Henrique Oliveira Parreira (1946), seu único fi\-lho, igualmente licenciado em Medicina pela Universidade do Porto (1972).

Este é o seu único texto literário conhecido.
} % small

\verso

\begin{center}\bf\large As Antas\end{center}

A rua das Antas desde o Monte Aventino ou alto da Avenida Fernão
de Maga\-lhãis até Silva Tapada era tortuosa e triste, ladeada por muros
limitando, à esquerda de quem sobe, a Quinta das Antas e à direita o depósito de carvão que os "funiculares" das minas de S.\ Pedro da Cova, num
ininterrupto vai e vem, ali despejam constantemente para o abastecimento
da cidade. Mais além férteis campos orlam dum e doutro lado o caminho.
O pavimento da rua, irregular e térreo, estava sempre coberto por abundante camada de pó negro do carvão que o vento levantava em nuvens para sujar desastradamente os transeuntes, ou que a chuva transformava em enfadonha lama.

De resto o movimento da rua era pequeno. Para desta zôna suburbana ganhar o centro da cidade era mais prático descer a Avenida Fernão
de Maga\-lhãis ou a Calçada das Antas para tomar o "eléctrico" na Praça
das Flôres, ou então atravessar a S.\ Crispim e ao Marquez, do que subir
tal caminho com o nome de Rua das Antas. Este caminho era mais conhecido pelas
gentes modestas que aos domingos de calor se dirigiam, em romaria, para
o local próximo dos Lameirais com os seus farnéis, contornando pela rua
da Vigorosa para se isolarem um pouco do bulício da cidade e respirarem o ar mais puro dos pinhais que ali existem. À semana só muito
cêdo servia de passagem a magotes de operários vindos dos lados da
Areosa e Ermezinde, que deixavam a rua de Costa Cabral em Silva Tapada para meterem ao caminho das Antas e se dirigirem às fá\-bri\-cas das
imedia\-ções do Bonfim, ou então à tardinha ao despegar do traba\-lho.

De-resto poucas pessoas passavam por sítio tão êrmo, com aspecto
mais de aldeia que de cidade. De noite, pior. A ilumina\-ção pública era
só a bastante para dar ao local aspecto ainda mais lúgubre.

À Rua das Antas vinha ainda desembocar uma vereda estreitamente
apertada entre muros, e sinuosa, que não é mais do que a Travessa de
Costa Cabral, que saindo perpendicularmente da artéria dêsse nome até ao termo da Rua da Alegria, se inflectia a seguir em arco para a esquerda indo juntar-se a distância à Rua das Antas depois de contornar a quinta
%marginpar{pdf = 2}
dêste mesmo nome.

Quási no ângulo de confluência dêstes dois caminhos, e a-dentro dos
muros da quinta, via-se a casa das Antas com jardim de gran\-des palmeiras, onde então estava instalado o Colégio de Santa Emília. Por de traz avistava-se um mirante encimando uma gruta. Um pouco mais ao lado, a casa
do pessoal da lavoura.

Quási não havia mais casas nas redondezas até às
ruas de Costa Cabral, de Silva Tapada e da Vigorosa. Os campos de luxuriante pascigo ocupavam tôda a área.

Falava-se vagamente num plano de urbaniza\-ção das Antas, já há muito aprovado pela Câmara Municipal, mas ali jazendo no esquecimento.
Esse plano aformosearia muito o local, transformando-o num luxuoso bairro
com a sua avenida larga até à qual viriam prolongar-se as ruas da Alegria, de Santos Pousada e Avenida Fernão de Maga\-lhãis, e donde irradiariam novas ruas em direc\-ção à Vigorosa, cortadas por várias outras perpendiculares. Situado na região mais alta da cidade, o local era na verdade aprazível e salubre, sendo de prever a rápida edifi\-ca\-ção do bairro, com proveito certo para os proprietários dos terrenos. Uma emprêsa
de urbaniza\-ção chegou a fundar-se com o fito de grangear bom rendimento para os capitais empregados no seu financiamento. Os campos seriam
removidos, e no seu lugar levantar-se-iam prédios modernos, bem expostos ao sol por todos os lados, obrigatoriamente ajardinados, dando para
arruamentos que em nada desmereceriam dum Pôrto actualizado. Uma nova
vida se augurava para esta parte da cidade. As negocia\-ções para êsse
fim multiplicavam-se. Estamos em fins de 1926. Havia mesmo quem adiantasse que a Avenida, ponto de partida do novo bairro, deveria ser inaugurada em 9 de Abril do ano seguinte.

De facto um dia veio em que os traba\-lhos preliminares começaram.
As bandeirolas alinhadas e as miras, as pranchetas e o teodolito, marcaram o início da obra. E os operários daí em diante, revezados por
turnos, traba\-lharam incessantemente de dia e de noite e em gran\-de número para que a Avenida pudesse ser aberta ao trânsito em Abril desde Costa Cabral até às Antas.

\begin{center}
:-:-:-:-:-:
\end{center}

%marginpar{pdf = 7}

\begin{center}\bf\large Eugé\-nio\end{center}

Eugé\-nio fi\-cara muito cedo sem os carinhos dos pais que a morte roubou. Não conheceu sequer a mai, vitimada pelo parto que lhe deu origem.
Do pai, são poucas e vagas as reminescências que lhe restam. Tinha Eugé\-nio
6 anos quan\-do êle lhe faltou. Nêle, o mais novo do bando, via o pai a última recorda\-ção da esposa querida, e por isso o idolatrava. Os seus restantes 4 fi\-lhos eram já rapazitos a quem não havia faltado o mimo materno; ao mais novo era agora necessário mascarar quanto possível a falta
da mãe. Mas a sua morte veiu tornar mais órfão o pequeno, a cujos desígnios estava naturalmente indicado que presidiria de ora àvante o irmão
mais ve\-lho.

Soletrando as primeiras letras e feito o 1º exame, o petiz dava esperan\-ças. Necessário era procurar-\-lhe carreira. Foi para um colégio do
Pôrto estudar o 2º grau.

Os internatos eram, nêsse tempo, verdadeiras prisões onde as crian\-ças
estavam completamente isoladas do mundo. Sem qualquer espécie de convivência, os educandos ali passavam monotonamente os dias, as semanas, os
mêses. Levantavam-se ás 6 horas, ainda de noite na época invernosa, para
se dirigirem sob formatura para o salão de estudo. Ao soarem as badaladas
para o pequeno almôço, lá iam sob a mesma formatura para o refeitório de
compridas mesas de mármore onde lhes era servida uma tigela de caldo
com uma fatia de boroa. Seguia-se a reza em comum. Depois assistiam ás
aulas que se estendiam pelo dia adiante com um intervalo para o almôço
do meio-dia seguido dum pequeno recreio. Após as aulas nova sessão de
estudo, jantar e dormida. O silêncio era a todas as horas rigoroso; no
estudo, como no refeitório, nas desloca\-ções em formatura como nas aulas,
a proibi\-ção da conversa era severa. O corpo da prefeitura estava bem à
altura da missão cometida. Os rigores da puni\-ção não se faziam demorar
para todo aquele que violasse o silêncio monótono daquele internato, onde se respirava uma atmosfera de terror paralizadora de todas as fun\-ções
orgânicas.

A quinta pertencente ao colégio mal os educandos a conheciam. E do
exterior daquele casarão de gran\-des galerias, com uma capela ao centro
e dois corpos laterais e simétricos com frente para um largo pátio murado, a que dava ingresso um enorme portão, nenhum vislumbre de liberdade
bafejava os enclausurados pretizes.

Os quartos domingos de cada mês eram os únicos dias de visita que
o rígido regulamento permitia ás fa\-mí\-lias. Então, nas tardes dêsses almejados domingos, alunos e fa\-mí\-lias espa\-lhavam-se pelas galerias, e os farneis trazidos para os pupilos visitados eram no fim da tarde distribuídos irmamente pela comunidade, para que os não visitados comparti\-lhassem do monte, todos enganando um pouco com estas iguarias o seu estomago
das misérias do ementário usual.

Este era, na verdade, precário. Afora a refei\-ção do meio-dia, por conduto tendo
um prato de farinha de pau, ou, ás terças-feiras, de arroz com
uma pequena fatia de toucinho, as demais refei\-ções, quer a matinal, quer
a vesperal, eram reduzidas a uma tigela de caldo de couves mal adubado,
acompanhado duma ra\-ção de boroa, no extenso refeitório de compridas mesas
de mármore. Ao fundo do refeitório fi\-cava a mesa dos professores e prefeitos, onde a comida era mais abundante e variada, e cujo cheiro e aspecto
constituíam os únicos aperitivos para os internados saborearem a sua. Á mesa,
como nas demais partes, podia ouvir-se o zumbido duma môsca sem custo.
Não se falava, e muito menos se recalcitrava contra qualquer deficiência
do serviço. Uma vez, era á noite, o caldo não tinha uma pedra de sal; estava intragável. Apesar do apetite, todos o punham de lado sem uma única
palavra de censura que a timidez não deixava articular; de-repente, um
rapaz mais atrevido levanta-se e participa ao prefeito de serviço o caso.
A bofetada que lhe custou a audácia soou pela sala e parecia zumbir aos
ouvidos de todos os comensais que a\-pro\-xi\-maram de si a tigela e a esvasiaram ainda que a custo. A mão do musculoso Padre Pedro fez o prodígio
de ver esgotadas todas as tigelas sem mais hesita\-ções.

\fotoE

Levavam os dias tempo a passar no meio de tão rigorosa disciplina, e
traziam-se contados os que faltavam para as férias mais próximas sempre
desejadas, com sofreguidão.

Eugé\-nio frequentava a 4ª classe. O professor era o director do colé\-gio, homem forte, de colarinhos gomados muito altos, e côxo. A maior satisfa\-ção que êle podia dar ao curso era aparecer ao almôço sem colarinho;
era então sinal certo de feriado, pois sempre que saia reservava-se para
mudar essa peça mais tarde. Outra satisfa\-ção era a iminência de trovoada,
para a qual todos conheciam a sua pusilanimidade; era já em tal emergência esperada a mu\-lher do director para o animar com a sua presença em
plena aula, e essa presença se era salutar para o marido não o era menos
para os alunos; as fúrias aplacavam-se como por encanto e os castigos
eram suspensos, e se ás vezes perante um fracasso maior naquilo que havia
obriga\-ção de estar mais sabido, se esboçava qualquer impulsivo acto ou
impreca\-ção contra o discípulo, logo a voz meiga da santa senhora intervinha a aplacar o seu impulsivo ânimo que depressa transformava em dócil.

Ao fim de 1 ano estava Eugé\-nio dado como pronto, e o director propunha-o para exame. O tutor opôs-se porém; queria que o seu irmãozito fi\-casse com mais sólidas bases e para isso devia repetir o estudo, o que
ficou assente não sem custo do director, pois o colégio era pobre e o internado fi\-caria a estorvar direitos que a outrem cabiam de facto. Tinha
porém uma atenuante. Estudioso e humilde, Eugé\-nio não lhe dava gran\-de traba\-lho. Os maus tratos aos outros infligidos não se estendiam a Eugé\-nio.
Os alcunhas com que o director apelidava todos os educandos, buscados na
nomenclatura zoológica, eram para Eugé\-nio substituídos por apelidos humanos que a sua imagina\-ção lhe prodigalizava.

Feito o exame da 4ª classe, após dois anos de internamento, Eugé\-nio
seguiu para casa. Que destino se lhe daria? Os rendimentos não autorizavam a sua estadia no Pôrto onde seguisse estudos liceais, para o que
todos aliás o julgavam propenso, e o irmão mais ve\-lho de cogita\-ção em cogita\-ção conseguiu que o director do colégio o deixasse ali estar mais
um ano, a pretexto de aprender uma arte fácil e leve, admi\-tin\-\mbox{do-o} na oficina de alfaiateria privativa, como ajudante. Não obstante estar assim, 
em princípio, estabelecida a orienta\-ção profissional de Eugé\-nio, um acontecimento imprevisto veio modifi\-car os desígnios que só á Providên\-cia
cabe chancelar. Vagou um lugar na secretaria, e o director aproveitou o
rapaz para êsse lugar de pequena responsabilidade, ao mesmo tempo que o
obrigava a frequentar a 4ª classe a título de rememora\-ção dos conhecimentos adquiridos.

Um ano assim se passou. E nas férias gran\-des seguintes um novo
facto providencial ia modifi\-car radicalmente o futuro de Eugé\-nio.
Anunciaram as gazetas por essa ocasião o estabelecimento de 6 pensões
gratuitas do Estado na Escola Nacional de Agricultura de Coimbra, para
estudantes necessitados. De entre a multidão de requerentes, Eugé\-nio foi
um dos admitidos, e, exultando de contentamento, começou a frequentar o
curso que, sem outros encargos além do enxoval e das viagens, lhe era facultado. As viagens, essas mesmas, eram reduzidas ás das férias gran\-des.

Estranhou muito Eugé\-nio a nova vida. O enclausuramento sofrido até
então transformou-se na mais franca das liberdades. Embora internados,
os alunos aqui entregavam-se a si próprios, libertados em absoluto da
praga dos prefeitos e da rigidez disciplinar atrofiadora da personalidade. Aos estudantes mais ve\-lhos era exclusivamente atribuída a vigilância dos mais novos; êstes serviam aqueles nos pequenos serviços domésticos dos seus quartos privativos. Á mesa, bem servida em qualidade
e quantidade, juntavam-se indistintamente com os alunos o Regente do colégio e os professores de francês e inglês, contratados das respectivas
na\-cio\-nalidades para o ensino destas línguas, os quais viviam no colégio
com o Regente. As aulas distribuídas pela manhã, após o primeiro almôço
%marginpar{pdf = 9}
de leite com café e pão com manteiga servidos á discri\-ção, e pelo
princípio da tarde após o almôço constituído por sopa, 2 pratos, pão, vinho
e fruta. Ás aulas da tarde seguiam-se os traba\-lhos práticos na espaçosa
quinta. O jantar, ás 17 1/2, era á guisa da refei\-ção anterior.
O espaço decorrido entre as 19 1/2 e as 21 horas era destinado a prepara\-ção das li\-ções para o dia seguinte, livremente para os mais ve\-lhos nos
seus quartos privativos, obrigatoriamente para os mais novos na sala de
estudo. Ás 21 1/2 horas servia-se o chá, e ás 22 a sineta tocava a silêncio.
Cada um tinha então a faculdade de continuar o estudo no seu quarto ou
deitar-se.

O Regente do colégio era ao mesmo tempo professor de latim e sociologia, abrangendo esta a his\-tó\-ria, a geografia e a economia política, e entrara para o seu lugar no mesmo ano em que se matriculou Eugé\-nio. Por
isso tratava êste, e os do mesmo curso, por condiscípulos. Era homem dedicado ás letras, passando noites seguidas de insónia com a luz acesa no
seu quarto a ler livros de literatura. Durante o dia passava horas a citar passagens interessantes de Camilo aos seus pupilos, entre os quais
se contava Eugé\-nio, abundantemente festejadas por estridentes garga\-lhadas
do próprio ledor, que os ouvintes, ou por encontrarem graça aos trechos,
ou às desproporcionadas garga\-lhadas, ou ainda por comprazer para com o
Regente, secundavam. A disciplina que ele mantinha era a mais benévola
possível; nunca necessitou de recorrer a castigos, e as suas raras admoesta\-ções eram sempre piadas indirectas bebidas nos bons autores literários,
em que era profundo, e que ada\-ptava admiravelmente ao caso ocorrido. De
resto, o respeito que todos lhe tinham era enorme. Nas aulas, e nos exames,
a ciência afluia torrencialmente em discursos, e o aluno interrogado não
tinha nunca ensejo de mostrar aquilo que sabia.

O curso durou 6 anos, visto Eugé\-nio nunca perder nenhum. Simultaneamente o nosso estudante frequentou o curso liceal professado na mesma
escola para aqueles que facultativamente nele quisessem ma\-tri\-cu\-lar-se,
faculdade esta de que só se aproveitou do seu curso um condiscípulo mais
-- o seu mais inseparável amigo desde o início até final da escola, que
seguiu depois o curso de Agronomia de que havia de vir a ser Lente.

Com 18 anos possuia assim Eugé\-nio uma carta de curso, além das habilita\-ções do liceu, em excelentes condi\-ções para concorrer a qualquer lugar. A menoridade, porém, excluiu-o implacavelmente do primeiro concurso
aberto, e Eugé\-nio teve de marcar passo durante dois anos e meio em serviços parcamente remunerados das Escolas Móveis Agrícolas, nas Caldas das
Taipas e em Santo Tirso, até que um dia recebeu um convite telegráfico
para ir preencher uma vaga na própria escola que o diplomara. Exultou
então de alegria Eugé\-nio, pois via pela primeira vez ir ser condignamente
remunerada a sua actividade, antevendo a rápida liberta\-ção da dívida contraída para a conclusão dos seus estudos. Veio a tempo o convite, porquanto
já meio desanimado, pensava em procurar traba\-lho na África onde já tinha lugar prometido.

A vaga que foi preencher na Escola de Coimbra foi a de Sec\-ção de
Pecuária, sob a chefia do veterinário. O facto conduziu Eugé\-nio a estudar este ramo da ciência, desviando do seu entusiasmo primitivo que era
o de seguir o curso do Instituto Superior de Agronomia, pois que o sonho
de seguir estudos superiores nunca o abandonara, e só a míngua de recursos suspendia. Uma tarde, estava Eugé\-nio ainda em Santo Tirso, recebeu
carta do seu amigo inseparável que então frequentava o 4º ano de Agronomia a anunciar-\-lhe uma visita por ocasião duma excursão do seu curso que
ia passar naquela localidade. Pois Eugé\-nio, após uma tarde de con\-ví\-vio
com o antigo condiscípulo, passou uma das suas raras noites de insónia,
voltando-se na cama a todo o momento, pensando nervosamente na cristaliza\-ção do seu cérebro em contraste com o progredimento do amigo. Poderia
qualquer pessoa que não conhecesse Eugé\-nio por tratar-se de uma crise
de inveja. Mas só quem o não conhecesse.

%marginpar{pdf = 10 }
Anto\-lhavam-se-\-lhe agora em Coimbra, bem remunerado, me\-lhores possibilidades de singrar. E durante dois anos e meio acumulou reservas que
iriam rasgar novos horizontes á sua vida.

Assim sucedeu. Serôdiamente, aos 23 anos, matriculou-se na Escola Superior de Medicina Veterinária, em Lisboa, e, ao mesmo tempo, na Escola Politécnica, a cursar os preparatórios de Medicina. As classifi\-ca\-ções obtidas eram óptimo prenúncio duma carreira bri\-lhante. Porém o seu irmão
mais ve\-lho que aconteceu ir a Lisboa em vésperas dos actos de Eugé\-nio
foi encontrá-lo tão desfigurado que procurou por todos os meios distraí-lo, durante a sua estadia ali, da intensa aplica\-ção ao estudo; e,
regressado à aldeia, fez ali constar que quasi não reconhecera o seu irmão em Lisboa. Efectivamente
Eugé\-nio chegou exausto ao fim do 1ª ano dos cursos de Veterinária e Preparatórios de Medicina, e tinha de optar por um deles na certeza de não
poder abarcar ambos com tamanha aplica\-ção. Acrescia o facto de nêsse tempo o curso veterinário ter pouca saída, antes da cria\-ção dos partidos
municipais hoje existentes. Um outro facto ainda coagia a saída de Lisboa.
As suas economias ameaçavam esgotar-se com as despesas de pensão, propinas
e livros.

Optou então pelo curso médico e obteve a transferência para a Universidade do Pôrto, reco\-lhendo a casa dum seu irmão ali residente que á
indústria se dedicava. Este era o imediato ao mais ve\-lho, e tinha bem cedo
manifestado a sua nega\-ção pelos livros e a sua aplica\-ção á boémia que
\-lhe valeu o casamento imposto pelo tribunal com uma menor, criada de servir, analfabeta mas inteligente e nervosa, de quem teve 3 fi\-lhos que ao
tempo estudavam os primeiros anos do liceu.

Eugé\-nio ocupou-se então de explicar as li\-ções aos seus sobrinhos,
como indemniza\-ção pela comida e dormida que recebia graciosamente. A sua
tarefa saia-\-lhe, porém, árdua, pois que todo o tempo fora das aulas, de tarde e noite, era pouco para mastigar o alimento cerebral que áqueles custava a digerir -- um, o mais novo, Adélio de nome, por hereditária nega\-ção
para os estudos, outra, a mais ve\-lha, por incapacidade de assimila\-ção, e
ainda outra, a do meio, por simples cabulice. A mãi repontava com as notas
obtidas pelos fi\-lhos no liceu, e Eugé\-nio, a quem não fi\-cava um só momento
para os seus estudos, sentia ter assumido um encargo superior ás suas
fôrças, incompreendido por parte dos pais dos explicandos que na deficiência de escolaridade destes viu antes o desleixo do explicador. E com
isto, só durante as férias podia Eugé\-nio rever os seus programas; e no
final do ano tinha sempre que deixar exames para Outubro afim de me\-lhor
se preparar durante as férias gran\-des.

\begin{center}\bf\large Um encontro\end{center}

Da  casa dos parentes com quem morava, no Monte Aventino,
Eugé\-nio passava habitualmente nas Antas para tomar o carro eléctrico em
Silva Tapada.
Entre os caminhos que se lhe anto\-lhavam para alcan\-çar o carro,
nas suas idas para a Faculdade de Medicina,
passou agora a preferir êste, como mais curto. De capa e batina, livros
debaixo do braço, lá ia cotidianamente, sempre às mesmas horas no cumprimento dum dever que o seu brio, mais do que a imposi\-ção do curso livre, lhe impunha. A chuva torrencial de tempestade nunca o convidou a
usar do direito de faltar às aulas, como qualquer dêsses catarros febris
que convidam à cama todo o mortal menos imprudente. Quando êle faltasse a uma aula haveria de estar gravemente doente, e êsse facto parece
nunca nin\-guém o verificou. Não; era sua norma não faltar a aula alguma, senão por motivo de fôrça maior. E êsse motivo nunca apareceu. Cêdo, após o pequeno almoço, êle lá ia a entrar no "eléctrico" em Silva Tapada pela rua poeirenta ou transformada em lamaçal, atestada pelos seus
sapatos sempre sujos ao entrar na Faculdade. Ao meio dia ou vinha a
casa almoçar ou ingeria uma merenda levada consigo, conforme o tempo
disponível. À tarde regressava para rever as suas li\-ções. Assim
passavam os dias, as semanas, os mêses.

A sua freima era exclusivamente a do estudo. Era o seu único passatempo. Nunca nin\-guém o vira nêsses centros de cavaco tão frequentados pelos estudantes, ou nos cafés. Poucas pessoas se gabavam do seu
con\-ví\-vio, tão limitado à fa\-mí\-lia, aos livros e a um condiscípulo de fora da cidade. Para quê o cavaco, se a má-língua, a intriga, eram invariavelmente o seu epílogo? E Eugé\-nio abominava a má língua. A vida a\-lheia
nunca lhe interessou. Não era por egoista falta de sociabilidade, ou
por autista concentra\-ção do seu espírito; bem pelo contrário, Eugé\-nio
vibrava com o mundo exterior e abria-se com os seus amigos. Mas êstes
eram muito seleccionados e, para tal poderem ser considerados, havia a
certeza de os ca\-ra\-cte\-ri\-zar uma vida sã. Sem outras ambi\-ções além da
obten\-ção dum curso limpo e bem classifi\-cado, a modéstia era o seu apanágio.
%marginpar{pdf = 12}
Dir-se-ia mesmo que a modéstia constituia a sua única vaidade. Se valor
tinha só nos exames êle se revelaria a quem assistisse. Solicitado
a cooperar como dirigente em qualquer modalidade de associacionismo académico, em
que a sua personalidade podia porventura destacar-se, a resposta era sistematicamente a de que não dispunha de tempo para o desempenho do lugar como se tornava imprescindível para
tudo aquilo a que estivesse ligado o seu nome; e não insistissem, porque
não transigia, e logo manifestava a sua contrariedade se lhe replicassem que os outros traba\-lhariam por si. Não. Nunca poderia subscrever
o seu nome senão ao traba\-lho honesto, e nunca como elemento decorativo.

Discussões vulgares evitava-as. Cada qual lá tinha a sua razão,
tri\-lhando o mais das vezes caminhos diferentes para atingir o
mesmo objectivo. Por isso não gostava de intervir. Nem deixava adivinhar o seu pensamento, que não lhe interessava impõr aos outros que possuiam como
ele cérebro raciocinante.

Monotonamente avan\-çava, pois, a vida de Eugé\-nio na mais calma e
pacata placidez.

Um dia regressava da última aula da manhã, para almoçar. Era 5ª feira, e frequentava nessa altura o 1º ano.
Vinha atrazado uns minu\-tos da sua habitual pontualidade em virtude dum
incidente ocorrido no carro ao subir a rua de Santa Catarina por alturas da Fontinha; incendi\-ou-se o motor e tiveram os passageiros de mudar de carro. Estavamos na Primavera, e o dia não o desmentia. O sol
distribuia poa\-lhas de ouro ténue sôbre a terra, e a atmosfera estava
calma.

Ao cruzar a Rua das Antas com a futura Avenida, já terraplanada,
depara com uns o\-lhos que o impressionaram. Uma rapariga linda como
os amores, cabelos pretos, testa ampla, rosto oval, tez branca de cêra, altura menos
que mediana, formas arquitecturais e arredondadas. A idade tanto quanto é possível adivinhá-la numa mu\-lher,
andaria pelos 18 anos. Junto, outra rapariga mais esbelta e alta, a mesma tez de cêra, uma cara de boneca, talvez um pouco menos de idade. Deviam ser irmãs, e despediam-se duma senhora que acompanhavam à porta
da casa.

Eugé\-nio moderou o passo. Aquêles o\-lhos sonhadores perturbaram- -no % JNO ALERT!
profundamente. E o almoço  já não decorreu com a calma do costume...

O fenómeno não era tão banal como
poderia parecer a primeira vista.

Vivendo para o traba\-lho, Eugé\-nio
nunca fôra daqueles que se deixam perturbar por qualquer rapariga com que
cruzam, ou que se entreteem a galantear
torto e a direito quem passa. De rapaz sério o acoimavam por essa razão.

Uma só rapariga, era êle mais crian\-ça, o havia sensibilizado deveras
por forma a enlear aquele cora\-ção,
que para muitos dos seus companheiros
se afigurava ser de gêlo, incapaz de
amar, de oferecer guarida a uma afei\-ção
das que perduram. Essa afei\-ção teve o
destino de todas que não passam da vulgaridade: extinguiu-se.

Hoje o seu cora\-ção palpitara de
novo, em face de alguém que tivera
a magia de o subjugar. Por invulgar, dêste facto decorre quanto a simpatia co\-lhida à primeira vista foi gran\-de.

Bendito o atrazo do carro que proporcionou tão belo encontro!

A Avenida já não era de ora avante somente o caminho obrigatório para os seus estudos. Passou a ser o passeio post-prandial em que
haviam de decorrer felizes as digestões.

Os traba\-lhos da nova avenida decorriam com enorme actividade. O
dia proposto para a sua inaugura\-ção aproximava-se. O local desde aquela quinta-feira tornara-se para Eugé\-nio mais encantador. Já não era
só distrac\-ção proporcionada por a\-quê\-le intenso vai e vem da multidão
de operários ali empregados. Era acima de tudo a expectativa constante de tornar a ver a\-quê\-les o\-lhos cuja imagem não mais lhe saiu do
pensamento. Já por esta ocasião também a rua das Antas estava a ser
retifi\-cada para dar lugar à rua de Bartolomeu Dias em que já se viam
assentes as ca\-lhas onde o "eléctrico"\ havia de passar muito em breve.

Os dias passavam porém sem que Eugé\-nio tornasse a ver o objecto
das suas pesquisas. Não seria dali a\-quê\-le belo rosto oval, de o\-lhos lindos como os amores, e teria sido por incidental casualidade o seu encontro naquela habita\-ção? Tomava-se necessário investigar.

Nas traseiras da quinta estava aberto um horto ao público. Aí
se vendiam plantas hortenses e flores. As investiga\-ções não deviam
pois ser difíceis. Restava arranjar pessoa insuspeita capaz de desempenhar
o papel de investigador. Eugé\-nio adextrá-la-ia, que para isso tinha
de sobejo o jeito que lhe faltava para entrar em ac\-ção. A timidez paraliza\-\mbox{va-lhe} todos os movimentos sempre que ditados por um facto emocional. E para êle as investiga\-ções desejadas respeitavam já um
facto que mantinha em elevado tonus a sua emotividade.

%marginpar{pdf = 15}

Um domingo era dia apropriado para a compra de flôres. Um simples
ramo seria o pretexto de um pouco de conversa com o jardineiro. Assim
foi. Um petiz não se tornaria suspeito. E o pequeno Adélio, um rapaz do
1º ano liceal, astucioso, era encarregado da grave missão.

Demorou uma comprida hora a compra do ramo, enquanto Eugé\-nio passeava impacientemente nas proximidades, augurando bons frutos à missão
a ajuizar pela demora. Regressou, enfim, com um ramo arranjado a primor
de rosas; à beleza das flôres rescendendo perfume e frescura,
juntava-se a poesia da sua proveniência. O ramo era um encanto. Avolumava-se, porém, a impaciência pelos resultados para que o ramo de flôres fora simples pretexto. E Adélio, com flagrante cara de satisfa\-ção
por ter consciência de bem se haver desempenhado da sua missão, começou:

-- Atravessado o horto, deparei com um homem mais baixo que alto,
entroncado, corado, aparentando 50 anos, fato escuro sem casaco e colete desabotoado a mostrar o peito da camisa branca. Meio escondido
por entre os arbustos, andava a limpar as plantas com esmero de algum
ramo sêco, ao mesmo tempo que alinhava mais estèticamente os pequenos
vasos que a granel se amontoavam no jardim. Só queria que visse, acrescenta Adélio, a órdem, o asseio, de todo horto. Dá vontade de lá entrar
só para o admirar. Mas vamos ao caso que interessa. O nosso homem, embevecido na faina do seu traba\-lho, não dava pela minha aproxima\-ção. Saudei-o, e só então o\-lhando-me, correspondeu cortejamente com os bons-dias,
menino. Declinei o meu desejo dum ramo de flores, que êle se prontificou a co\-lher depois de interromper a sua faina. Não tenho pressa, repliquei-\-lhe. E joguei a primeira cartada: Êste horto não pertence aos proprietários desta casa junta? Eu creio que conheço essa fa\-mí\-lia...

-- Pertence, sim senhor. Deve conhecer a fa\-mí\-lia, deve. É muito conhecida nestas redondezas. O meu patrão é o Sr. Frazão, a que muitos
chamam doutor por falar muito bem e ser muito viajado. O\-lhe que tem
corrido todos os paízes por êsse mundo fora.

-- Sei, sei; e não é pai dumas meninas que se chamam... E aqui Adélio deixou arrastar a última palavra como que a invocar os nomes de
que se não lembrava.

-- Maria Princesa e Julieta, acode o jardineiro.

-- Exactamente, retorquiu Adélio.

A armadi\-lha dera resultado. A astúcia de Adélio conseguira já
alguma coisa que, pelo menos, lhe dava ânimo para prosseguir:

 -- Eu conheço muito bem essas meninas, embora há muito que as não vejo. Eram boas meninas. Se os meus cálculos não
fa\-lham, deverão andar aí pelos 18 anos de idade...

-- Creio que mais, menino. A menina mais ve\-lha, a Mariazinha, se não
estou em êrro, anda pelos 20. Deve fazê-los daqui a pouco tempo, pois costumam ser por esta ocasião os seus anos.

-- Nesta casa não estava instalado ainda há pouco um colégio, o
de Santa Emília?

-- Estava, sim senhor. Mas o patrão que morava noutra casa sua, lá
para os lados da Boavista, resolveu vir habitar esta. Dantes êste
altio era êrmo e o sr. Frazão achava-se me\-lhor lá para a\-quê\-les lados.
Mas agora que se está a romper a avenida conseguiu despedir o inquilino e resolveu alojar-se cá para os altos. Essa avenida a êle se deve,
e é em gran\-de parte construída em terras suas. Consta que ofereceu a
terra necessária à Câmara, futurando bom negócio na venda de terrenos marginais para edifi\-ca\-ções. Aquilo é que êle é financeiro! A mim dá-me pena ver tão boas terras de lameiro e mi\-lho serem arrazadas. Mas
êle é quem põe e dispõe... De resto isto aqui vai fi\-car muito bonito.
Quem viu a antiga rua das Antas desde o Monte Aventino para cá, e quem
a vê agora!... Já é outra rua, com outro nome, e o "eléctrico"\ já aqui
passa há pouco tempo. De vinte em vinte minutos temos "eléctrico" da
Praça para aqui, por Santa Catarina e Costa Cabral para dar volta por
Silva Tapada até aqui abaixo ao depósito do carvão. Chama-se o carro
das Antas. O menino se mora perto bem devia conhecer isto como era. Por
o tal caminho que agora se transformou numa boa rua, pela travessa de
Costa Cabral até ao Seixal, era perigoso andar de dia, quanto mais de
noite! Agora, a dar crédito ao que tenho ouvido, vai fi\-car aqui um centro que se pode comparar à Praça e à Bata\-lha. O\-lhe quem conheceu também, como eu conheci, o cora\-ção da cidade lá em baixo, e como aquilo está hoje tudo modernizado. O menino não faz ideia do que era aquilo,
porque não é dêsse tempo. Eu nunca fui a Lisboa, mas diz que há lá umas
avenidas novas longe do centro da cidade, muito bonitas, e que estas
que aqui se vão fazer não fi\-cam muito abaixo.

-- De casa chamaram pelo sr. António, e o jardineiro obedeceu prontamente. Despedi-me e retirei-me discretamente, satisfeito por julgar
bem cumprida a minha missão.

Eugé\-nio mostrava bem visivelmente ainda maior satisfa\-ção. Não se enganou ao confiar tão melindrosa missão ao pequeno Adélio. Estava agora
de posse de preciosos dados. Sabia já o nome e a presumível idade de
quem tanto o impressionara. Até o nome lhe parecia encantador como a
pessoa que o possuía.

Os dias decorriam sempre, com baldadas o\-lhadelas na passagem por a\-quê\-le sítio. Só uma ou duas 5$^{\mbox{\tiny as}}$ feiras à mesma hora pôde presenciar o
quadro tal qual o avistara semanas antes.

A Princesinha de rosto oval com uns o\-lhos fascinadores e a irmã
Julieta, despedindo-se daquela hóspede. Esta devia, pelos modos, ser uma
professora que à mesma hora de tôdas as 5$^{\mbox{\tiny as}}$-feiras lhe vinha ministrar
a li\-ção à sua residên\-cia.

Não foi já sem saudades que Eugé\-nio teve de partir para a sua aldeia
a gozar as férias da Páscoa. Projectava reduzir a ausência e estar
de volta no 9 de Abril, tanta vontade tinha em assistir à inaugura\-ção
da Avenida, ou porque tão prêso espiritualmente se achasse à\-quê\-le novo
bairro, ou porque tal acto lhe multiplicasse as probabilidades de ver
quem desejava. Seguiu para o Douro. A visão da\-quê\-les o\-lhos não o abandonava um instante, lá longe, naquela casinha sertaneja, isolada do mundo, rodeada de
montes, que o viu nascer e que lhe dava guarida distante umas boas duas
horas a cavalo da esta\-ção do caminho de ferro mais próxima.

O tempo mudou. O vento sul fez cair enormes catadupas de água,
incessantemente, na\-quê\-les dias, que pareciam arrastar tudo adiante dos
aguaceiros que desciam torrencialmente dos montes e enchiam os caminhos. A aldeia era medonha. Avizinhava-se cada vez mais o 9 de Abril,
e o tempo não dava esperan\-ças de me\-lhorar. E Eugé\-nio viu-se enfim impossibilitado de atravessar a\-quê\-les montes a tempo de assistir à almejada inaugura\-ção. O pesar foi gran\-de, mas a travessia do caminho era
de todo impossível; não havia senão que conformar-se com a sua pouca
sorte.

A cidade já tinha para si atractivos que até então nunca lhe encontrara. Abominava-a antes, com o baru\-lho insuportável dos automóveis
e "eléctricos", o seu ar irrespirável por tantos gases deletérios das
fá\-bri\-cas, pela ostenta\-ção das indumentárias, tão em contraste com o sossêgo, o ar puro, a modéstia da sua aldeia onde a vida era mais pura. Nem
aqui se viam caras coradas que o não fôssem pela pródiga natureza, como na cidade se vêem. A cidade, que horror! replicava Eugé\-nio com um
misto de dó e antipatia, sempre que lhe puxavam conversa, a cidade onde
o sol não tem nascente nem poente, e o luar não entra, a cidade onde
não há natureza e nem sempre a arte conta; a cidade onde tudo é hipocrisia, a cidade dos cafés e cinemas, a noite trocada pelo dia e o dia
pela noite, não me seduz. A mu\-lher da cidade fugiu da vida doméstica,
nobre e sã, para viver no lodaçal mundano. Ela levanta-se para jogar
o ténis e até o jôgo da bola, monta a cavalo, fuma cigarros, frequenta
os bailes num à-vontade que tresanda a miséria e é visitada pelos maridos das suas amigas. Na aldeia a vida é sã, a mu\-lher tem a única côr
com que foi dotada, e vive exclusivamente para o marido e para os fi\-lhinhos que constituem o mais doce enlêvo, a mais alta ex\-pres\-são e a
suprema finalidade da sua exis\-tên\-cia; resumem a sua felicidade, as suas
ambi\-ções. E o lar é um ninho de amôr da\-quê\-les entes que só vivem uns
para os outros, estranho ao mundo e a tudo, na mais santa comunhão de
ideias. A vida é calma e pacata, sem as emo\-ções da cidade. O pudor tem
qualquer coisa de sagrado que se torna necessário conservar, e a honra
tem ainda religioso signifi\-cado.

Eugé\-nio era transigente e, mais, tinha um profundo respeito pela
coerência das suas palavras como dos seus actos. Palavra ou ideia que
uma vez proferisse só muito dificilmente seria contraditada em qualquer emergência. Era na verdade escravo da sua palavra, e também das
suas ideias, embora só até ao momento em que se convencesse do êrro.
Não era teimoso. Não era dos que mais facilmente se deixavam vencer
pela sensibilidade, deixando que esta se sobrepusesse à razão. Certamente que era homem, mas com uma certa fleuma que o levava a apreciar
as coisas por cima, com independên\-cia bastante e pondera\-ção.

Desta vez porém Eugé\-nio deixou-se divagar nas asas da sensibilidade. Possuído duma gran\-de afei\-ção viu sossobrar a razão. Aquêles o\-lhos
enfeitiçaram-no. E Eugé\-nio teve pela primeira vez consciência da sua
própria inferioriza\-ção, ainda agravada pela impossibilidade de reagir.
Com pouco contacto com o mundo, foi sempre com sensível consterna\-ção
que assistiu a essas intrigas da vida diária, a êsses mil conflitos
com que se topam a tôdo o momento, cujo macabro mecanismo nunca compreendeu. Pois se a razão está à vista de todos, porque êste degladiar constante da animadversão? Porque não há paz onde a terra chega de sobejo
para todos? Eugé\-nio teve agora, por experiência própria, de acreditar
numa fôrça superior à da razão, a da sensibilidade, fôrça brutal que
esmaga e inferioriza o homem reduzindo-o à animalidade pura e simples.
Discernindo, viu como a avidez, geradora da mania da revindica\-ção, base das gran\-des dissen\-ções, se filia na sensibilidade. Eis a explica\-ção da constante instabilidade da sociedades, das fa\-mí\-lias, dos indivíduos. É sempre a mesma sensibilidade que dirige tôdas as nossas actividades e subjuga a nossa razão. É a vitória da animalidade sôbre a
inteligência, a impotência do freio inibidor do cérebro sôbre a afectividade.

Eugé\-nio retido pelos rigores da invernia na aldeia, pôde enfim
regressar à cidade ainda antes de terminadas as férias. Estava-\-lhe
reservada uma a\-gra\-dá\-vel surpreza. Encontrou-se nessa mesma tarde no
"eléctrico" com uma destas raras beldades, em cuja fisionomia reconheceu traços de alguém que já tinha visto alguma vez, não sabia onde.
Aos seus o\-lhos cabiam maravi\-lhosamente aquelas palavras do P.$^{\mbox{\tiny e}}$ Vieira: ``a maior graça da natureza, e o maior perigo da graça,
são os o\-lhos. São duas luzes do corpo, são dois laços da alma''.
E pensou para consigo quanto mais interessante era esta rapariga do
que a primeira. Indubitavelmente mais interessante, tez corada de romã. Mas, coisa exquisita, os traços fisionómicos pareciam a final serem
os de Princesa. Seria alguma irmã que Eugé\-nio nunca tivesse visto?
Não. Eram os traços dela mesma, como dela mesma era a pessoa com
quem Eugé\-nio viajava e achava nêsse dia mais encantadora que nunca.

Depois viu-a várias vezes. Tornaram-se mais frequentes os encontros. Os o\-lhares de Eugé\-nio cruzavam-se com os dela frequentemente,
e ele perturbava-se. O seu cora\-ção interrompia-se num momento. Deixaram de ser encontros fortuitos para serem premeditados.

Eugé\-nio tinha já uma vasta rede de polícia em intensa actividade.
Arranjara a saber com pormenores a vida de Princesa. A sua mãi, uma
senhora de esmerada educa\-ção, tinha orgu\-lho especial naquela fi\-lha.
Educava-a a seu modo, na vida da casa. Entendia ser esse o me\-lhor dote que
um dia lhe deixaria. De ilustra\-ção literária só a suficiente para ilustrar o lar que viesse a constituir -- por\-tu\-guês, música, lavores, e pouco
mais. A professora de por\-tu\-guês era uma senhora que ia a casa às 5$^{\mbox{\tiny as}}$ feiras, e que Eugé\-nio já por mais de uma vez vira.

Educava-as como se fazia uns séculos atrás. Embora fôsse gran\-de a
fortuna que tinha a deixar-\-lhes, me\-lhor fortuna lhe ambicionava que era
de fazer das suas fi\-lhas umas boas dirigentes de casa. Que me\-lhor
dote lhes poderia deixar que êsse? Daí viria a felicidade do futuro lar, e
com a felicidade a maior riqueza e a única que resistia a tôdas as
vicissitudes. Não as queria para meninas duma sociedade exibicionista, ôca e abalada, de duvidosa reputa\-ção, mas sim para participantes
duma sociedade modesta e sã.

Não era difícil à mãi educar uma fi\-lha assim. Por temperamento
Princesa amava o traba\-lho, por curiosidade queria conhecer tôdas as
modalidades da vida doméstica e, activa como era, recreava-se pra\-ti\-can\-do-as. Traba\-lhava muito, muito, e a vida de casa já não tinha para ela
segrêdo algum. Além disso, era nas mais sólidas bases de economia que decorria a
educa\-ção: de todos os produtos da quinta se fazia dinheiro naquela
casa, não permitindo que se estragasse nada. A modéstia era o timbre
daquela fa\-mí\-lia. Aparato, nenhum. Aquêles vestidos, um verde e outro
escarlate nas duas irmãs, os chapéus, tôda a indumentária, eram intermináveis. Gastavam muito tempo em casa; não o perdiam cá por fora nas
exibi\-ções mundanas.

A biografia daquela gente embevecia Eugé\-nio. Não se havia enganado nos seus vaticínios, e a simpatia por Princesa cimentava-se cada
vez mais solidamente depois da\-quê\-les dados biográficos. Era a razão
que vinha lan\-çar um facho de luz na sensibilidade; razão e sensibilidade harmonizavam-se perfeitamente para erigirem em amôr aquela simpatia.

Aproximava-se uma festa infantil. Princesa a quem não faltava
virtude alguma era generosa para com os infelizes e tinha um fraco
pelas criancinhas. A sua ternura manifestava-se nêsses mil afagos
com que atraia os petizes, e êstes adoravam-na. Era então protectôra
dum estabelecimento de educa\-ção infantil, que todos os anos promovia
uma festa a favor dos seus pupilos. Nessa festa que estava anunciada
para breve, no salão dum diário da cidade, Princesa tomava parte activa em muitos números do programa.

Era ao anoitecer, quan\-do já Eugé\-nio rondava o portal do edifício
em que se ia realizar o sarau. Viu-a entrar, e depois de permutarem os
o\-lhares do costume, Eugé\-nio entrou também. A festa foi uma demonstra\-ção
de arte. Coube a Princesa a alocu\-ção de abertura. Nos bailados, nas
récitas, nuns solos de piano, Princesa tomou sempre parte, e as ova\-ções
da assistência sucediam-se. Em Eugé\-nio cada vez se arreigava mais aquela ingente simpatia por ela.

Restava manifestar-\-lha. Mas como? Eugé\-nio era um emotivo, a quem
todas as manifesta\-ções da vida de rela\-ção se paralizavam à vista do
ídolo que amava. O chão que pisava faltava-\-lhe sob os pés, e a voz embargava-se-\-lhe. Não sabia que posi\-ção dar aos braços. Sentia mudar de
côr, e baixava os o\-lhos ao ser por ela o\-lhado. Eu sei lá que mais fenómenos se passavam na\-quê\-le ser, para quem o domínio de si desaparecia
completamente ao ver Princesa! Dirigir-se-\-lhe era, pois, impossível.
Nem pensar em tal. Só por escrito. E na noite de S. João uma carta lhe
era entregue por mão própria, que recebeu. Eugé\-nio começava por pedir
\-lhe permitisse a exterioriza\-ção da sua simpatia em pálidas mas bem
sentidas palavras. Ousadia talvez, essa exterioriza\-ção era ditada por
um cora\-ção moço e o cora\-ção não mente. O silêncio porventura prolongado é que seria uma mentira. E para que serviria o silêncio? Tornar-se-ia necessário aniquilar o alicerce fundamental da biologia -- o Amôr -- para que se coartasse o direito de manifestar o sublime sentimento que em si encerra tão bela impressão. E êsse direito não só não
pode ser coartado, como ainda sôbre nós impende o direito de o usarmos
quan\-do bem sentido, como quan\-do êle é ditado pelo mais íntimo e nobre
do nosso ser -- o cora\-ção! E justifi\-cado assim o atrevimento, Eugé\-nio
terminava dizendo julgar-se credor duma resposta benévola.

Não respondeu Princesa, talvez por impossibilidade de o fazer, pois
que em breve seguia para Vizela. E Eugé\-nio retirava-se após os seus
exames para a aldeia. Terminava o 3º ano médico, e ia restabelecer-se
do penoso traba\-lho escolar, e buscar no campo um pouco de lenitivo para a calmaria desse Junho ardente e também... dêsse cora\-ção em ebuli\-ção.

Como se passariam as férias de Eugé\-nio é lícito supô-lo. Em con\-tí\-nuas lucubra\-ções, a sua ideia não mais se afastou do móvel afectivo
que tão decisivamente o prendeu. E no regresso à cidade novamente lhe
escreveu, solicitando a almejada resposta à sua carta; mas foi em
vão. Cairia em poder de seus pais esta última carta enviada pelo correio? O regime severo que reinava na educa\-ção filial, a ajuizar pelas
informa\-ções co\-lhidas, deixavam-no supor. Estavamos já no Outono, que
nêsse ano foi de muita chuva, e os meses passaram-se, intermináveis,
sem Eugé\-nio tornar a encontrar Princesa.

Nem o ano seguinte, de 1929, modificou o aspecto das coisas por forma
evidente. O estabelecimento de educa\-ção infantil protegido de Princesa realizava a sua festa costumada do final do ano lectivo, numa das
suas salas, por uma tarde de Junho. O salão já estava literalmente cheio
quan\-do Eugé\-nio chegou, envergando o seu trajo académico e sobraçando
uma pasta com laço amarelo. Não havia quási lugar vago. Ficou de pé,
encostado à parede, e o facto não o contrariou. Me\-lhor podia, desta forna, estender a sua vista exploradora, e mais notado se tornava também.
E assim rapidamente avistou á distância a sua Princesa. Mais uma vez
se cruzaram os o\-lhares de ambos. Princesa veio sentar-se ao pé de onde
Eugé\-nio estava, e os o\-lhares prosseguiram incessantemente como setas
de Cupido.

Muito raros foram os encontros em todo êste longo ano. Apenas
talvez mais um na Foz, e nada mais. Andavam as duas irmãs num grupo
de amigas, de entre as quais Ofélia, sobrinha da professora de por\-tu\-guês,
era das mais íntimas de Princesa. E esta, com Ofélia sairam frequentes
vezes do grupo, para Princesa mais discretamente poder saborear e retribuir os galanteadores o\-lhares de Eugé\-nio.

%marginpar{pdf = 24}

O ano que se lhe sucedeu não foi mais pródigo em acontecimentos. Algumas trocas de o\-lhares a-quan\-do do regresso das li\-ções de música, ao
tomar o carro em Sá da Bandeira, e pouco mais. Terminava Eugé\-nio os seus
estudos. Ia possuir um diploma universitário após um curso, como epílogo das suas atribula\-ções escolares. Era já para si uma satisfa\-ção. Mas
não lhe bastava. Amanhã, atirado para onde os caprichos do futuro o determinarem, necessitaria de comparti\-lhar a sua felicidade com alguém mais.
Esse sonho dominava o seu pensamento.

Debalde o António Ferreira o dissuadia de prosseguir na sua obcessão que ameaçava eternizar-se sem finalidade alguma.

-- O\-lha, Eugé\-nio, dizia-\-lhe com bom humor, o seu amigo: Há muitas mu\-lheres.
Não te é difícil arranjar coisa me\-lhor. Deixa-te de paixões, que elas
ainda se riem por cima. Não te prendas tanto e goza me\-lhor o mundo, que
para tomar a vida a sério é muito cêdo.

Eugé\-nio repontava sempre. -- Nunca encontrou nem jamais encontraria
uma rapariga que tão profunda impressão causasse nêle. E que, de resto, neste século
XX as raparigas com os predicados exigidos não abundam, antes são cada
vez mais raras. Tu bem o sabes, continuava Eugé\-nio. E eu, pelo que conheço visualmente e pelas informa\-ções de boa fonte àcêrca de Princesa, ela é
única pessoa com os requisitos que exijo.

-- António Ferreira ria-se. Como essa há muitas. A questão é procurá\-las. Com a obcessão que te cega é que não vês mais nin\-guém. Precisas
de reagir.

Eugé\-nio não lhe dava ouvidos. Essas palavras soavam-\-lhe mal e
reco\-lhia-se ao silêncio. Só no silêncio pensando nela é que achava
con\-fôr\-to para o seu mal. E assim passava horas e horas.

Um dia conversando com um amigo, o Dr.\ Rocha, ultimanista de Letras, descobriu que Ofélia era amiga das suas irmãs e que as visitava amiudadas vezes ao domingo. Eugé\-nio pensava que relacionando-se com as amigas de Princesa
mais fácil se lhe tornaria criar perante esta o ambiente favorável
que era mister. Não podia pois perder esta amiga íntima, e logo se planeou ali a forma de se relacionar com ela. Apareceria aos domingos por
casa do amigo, num pitoresco sítio à beira-rio, até que se proporcionasse o ensejo. As visitas eram tanto mais naturais quanto é certo que a fa\-mí\-lia Rocha era descendente
da sua aldeia, onde passava épocas a veranear, e Eugé\-nio mantinha com todos os seus componentes amistosas rela\-ções. Assim fez.

Nem sempre a sorte abandona os infelizes. Eugé\-nio no domingo imediato fazia-se anunciar em casa do estudante de Letras, e quan\-do o amigo
correu a recebê-lo logo lhe comunicou a meia voz que Ofélia estava ali.
A apresenta\-ção não se fez demorar, e a amiga de Princesa, risonha, afirma
não lhe ser desconhecida a fisionomia de Eugé\-nio.

-- Eu conheço o sr. Doutor; donde, é que me não ocorre.

-- A mim também a sua cara me não é desconhecida, diz humoristicamente Eugé\-nio. Não me recordo donde, mas talvez pensando um pouco me
venha à mente. Não se lembra com certeza?

-- Não. Não me lembro.

Durante a tarde, em ameno con\-ví\-vio conversou-se muito. Ofélia, as
irmãs e irmãos do estudante de Letras, e Eugé\-nio. Êste intencionalmente derivava a conversa para interpelar Ofélia se já se lembraria por
que artes é que lhe não era estranho a ela.

-- Não há forma de me lembrar, respondia invariavelmente Ofélia.

-- Não costuma ir à Foz? Não será porventura daí? Preguntou finalmente Eugé\-nio.

E Ofélia num último esforço de evoca\-ção solta um ah!

-- Ah! não diga mais. Já sei!

O ano lectivo estava a findar, e era necessário agir. Em Maio e Junho duas novas cartas seguiram pelo correio, que tiveram a mesma sorte
das anteriores.

Em vão o António Ferreira pretendia dissuadir Eugé\-nio de continuar
nesta pesada odisseia que muito bem sabia quanto custava à saúde dêste. Via-o apreensivo, e só estava bem derivando tôdas as conversas para Princesa. Era a ideia única de Eugé\-nio, a ideia obcecante que o dominava em tôdas as modalidades da sua actividade. Mostrava-\-lhe a sua
atitude incorrecta de deixar obstinadamente as cartas sem resposta.

Eugé\-nio justifi\-cava-a. De certo os pais sequestravam-\-lhe as cartas, e Princesa nem sequer tinha conhecimento delas. Incorrec\-ção, por
forma alguma. Nada há que permita tal hipótese.

O único processo que se anto\-lhava a Eugé\-nio era o de se lhe dirigir pessoalmente. Mas conhecia bem as suas possibilidades, e não dava volume à ideia. Pois se em só a vendo ao longe todo ele tremia...

\centerline{\rule[-1em]{0pt}{2.5em}:-:-:-:-:-:-:-:}

\begin{center}\bf\large O novo médico\end{center}

Não obstante os precalços sofridos, sobretudo relativos á escassez de
tempo com que lutava, Eugé\-nio singrava no seu curso sem desfalecimento, as classifi\-ca\-ções obtidas fi\-cavam pelo preço do seu intensivo traba\-lho, sem coac\-ção de qualquer espécie junto de quem tinha a missão de
julgar no final de cada ano. E só no final do curso médico, assediado
pelo sentimento de responsabilidade, e também depauperado de saúde, optou
por uma casa de pensão em que, com mais tempo livre, havia de ultimar com
distin\-ção o seu espinhoso curso, embora à custa da realiza\-ção de algumas dívidas mais que o não amedrontavam. A pensão era económica mas satisfazia plenamente as suas exi\-gên\-cias. Medíocre para outros temperamentos que conheciam uma pensão cada mês, resmungadores por ofício de tudo
quanto lhes serviam, satisfazia muito bem a Eugé\-nio habituado desde tenra
idade a comer de tudo sem restri\-ções de caprichos. Disso mesmo se vangloriava, prestando homenagem ao seu Pai, sempre que tinha ensejo, por assim o ter educado.

Terminou enfim, Eugé\-nio, a sua formatura. Guardou a sua carta de alforria
num misto de alegria e indiferença. Ela marcava o final das suas lides escolares mas ao mesmo tempo o início da luta pela vida. Desenha-se perante os seus o\-lhos o gran\-de ponto de interroga\-ção que é o futuro. Não era isso porém que o dominava. Vencera sempre na vida, e de
hoje em diante também havia de sair vitorioso. A carta de formatura
tomava-o feliz, por um lado. Mas alguma coisa lhe faltava para que
a sua felicidade fôsse completa, e essa alguma e única coisa era o
ver-se compreendido por Princesa. Então, sim, é que veria realizado o
seu sonho.
Já um ano antes o facto, que constara vagamente entre os condiscíplos, tivera influência na biografia de Eugé\-nio inserta no tradicional livro dos Quintanistas.\footnote{Nota: do original consta o apontamento a lápis \emph{“Transcrever aqui versos que falam de amor..."}, referindo-se com certeza aos que constam do livro dos Quintanistas do autor, ver Figura \ref{fig:figA}.}

\figA

\marginpar{\red VERIF\\ ZH}

\def\OMITIDOS{
\begin{quote}\small
“Vinde a vê-lo, cachopas, vinde a vê-lo,
\\
Este jovem, simpático Doutor!
\\
Q’ às difíceis manobras d’escalpelo,
\\
Junta ainda... umas práticas d'amor!

Correi a vê-lo! e reparai atentas,
\\
Do seu andar na angélica maneira,
\\
Nas galochas sem bri\-lho, poeirentas,
\\
Tão próprias ao Dr.\ Lopes Parreira!

Tem medo à água...  e gestos delicados...
\\
Uns languidos requebros, completados
\\
Pelas voltas qu’imprime ao guarda-chuva

Que vem a ser, co’as galochas despolidas
\\
As duas coisas para êle mais queridas
\\
E fi\-cam, ao perfil, como uma luva.”
\end{quote}
}

O seu curso estava completo, é certo. Lá estava a atestá-lo o diploma, e os termos dêste deviam ser motivo de orgu\-lho para Eugé\-nio. Não
faltava lá o "laudabiliter et honorifice adeptus est"\ dos alunos
que se haviam destinguido nos estudos. Em nada desmerecia êste diploma do anterior que já possuia doutro curso, e ainda das altas classifi\-ca\-ções obtidas em outros estudos que encetára. O aspecto material
da sua vida deveria estar garantido. Mas nem só de pão vive o homem.
Eugé\-nio sentia que muito mais lhe faltava que era a afei\-ção de alguém
que enchesse de luz o seu espírito. E êsse alguém só podia ser preenchido por uma pessoa única: -- Princesa.

%marginpar{pdf = 28}

Se de todo fôsse impossível conquistar a afei\-ção daquela por quem
suspirava, então esforçar-se-ia por a esque\-cer. Mas esquecê-la afigurava-se-\-lhe esforço muito superior às suas fôrças. Aquêles o\-lhares trocados haviam-se vinculado indelevelmente em Eugé\-nio.

Eugé\-nio teve de retirar-se para o Alentejo. Tinha ali fa\-mí\-lia, e
esta desafiou-o a tentar lá clínica. Lá foi. Eram 77 léguas de distância que talvez servissem de lenitivo para o seu tormentoso mal.
Foi recebido de braços abertos. Ao jantar, após a chegada, uma pequena afi\-lhada, botão de rosa a abrir-se para a vida, recitou-\-lhe de surpresa as seguintes quadras:

\begin{quote}\small
Três anos que se passaram
\\
Sem eu o tornar a ver;
\\
Três anos longos e tristes
\\
que não poderei esque\-cer.

Se eu conseguisse umas asas
\\
Com que pudesse voar,
\\
Quando acabou o seu curso
\\
Alegre o ia abraçar.

Se meus Avós e seus Pais
\\
Vivessem nês\-te bocado,
\\
Que santa alegria tinham
\\
De ver um fi\-lho formado!

Aceite, querido Padrinho,
\\
Esta pequena florinha,
\\
Pedindo que se não esqueça
\\
Desta sua afi\-lhadinha.
\end{quote}

O novel médico entrou com o pé direito. Uns casos difíceis de
entrada deram-\-lhe fama. O seu prestígio aumentou mais quan\-do constou que até Lentes de Medicina, de Lisboa, ao serem
consultados por doentes vindos das mãos de Eugé\-nio ata\-lhavam logo que
estavam muito bem entregues e mais não lhes podiam fazer. O novel médico conseguira grangear a estima e simpatia dos doentes, e a admira\-ção dos mestres da capital. E a sua fama crescia de vento em pôpa.
Achavam-no muito meticuloso no exame clínico, e exaltavam o cuidado
com que seguia as doenças. Pois se o sentimentalismo de Eugé\-nio até
vibrava em uníssono com o do doente... acrescentava o público, querendo realçar a extrema afectividade do médico que se ufanavam de ter
no seu meio.

A felicidade seguia-o. Materialmente ia singrando. Moralmente,
ainda me\-lhor. Para o sentimentalismo que o ca\-ra\-cte\-ri\-zava, o simples
facto de uma cura era para si já a mais alta recompensa do traba\-lho
dispendido. Verdadeiro sacerdote, sentia verdadeiro prazer em arrancar das garras da morte qualquer ente. Os proventos materiais a que
tinha real direito tinham para si valor secundário. Sempre que regressava a casa de ver um doente a fa\-mí\-lia lia bem visivelmente na sua fisionomia a mudança de estado do doente.

Com os seus colegas locais manteve sempre as me\-lhores rela\-ções.
A deontologia profissional observava-a severamente, com frequentes prejuizos dos seus interesses.

As horas vagas eram preenchidas com laboriosas pesquisas de elementos para a confec\-ção dum livro destinado a médicos, que publicou
e a que a imprensa e os colegas fizeram lisongeiras referências. E
mais um traba\-lho publicou em fo\-lheto.

Assim decorreram os três anos em que ali permaneceu. O mal long\-ínquo, porém, avassalava-o constantemente, e daí um sentimento de nostalgia a chamá-lo para o Norte. Ninguém ali conhecia os segrêdos da
sua vida, e era em silêncio que curtia as suas penas. O que será feito
de Princesa? Prendada como é, não lhe faltarão galanteadores, pensava
Eugé\-nio umas vezes. Outras vezes iludia-se a si próprio lembrando-se
que os dotes que tão bem a distinguiam não eram, felizmente para êle,
os mais procurados pela mocidade petulante de hoje, vasia de todo o
discernimento e ôca de senso; uma rapariga sem os artifícios da pintura e as vaidades do luxo, e a exibi\-ção das ruas e dos salões, não possue atractivos para a\-quê\-les outros que no matrimónio vêem o mais banal dos actos, que o divórcio desfaz como o sôpro a luz da vela. Estas
cogita\-ções animavam um pouco Eugé\-nio. Todos os dias se lembrava dela.
\marginpar{ZH}Quando [sentia?] algum gran\-de prazer -- e os maiores prazeres eram as
curas difíceis -- pensava sempre quanto mais feliz era em poder parti\-lhar com alguém êsse prazer espiritual. Êsse alguém não era senão Princesa. E nos momentos mais difíceis evocava religiosamente a imagem
sublime dessa Dulcineia resplandescente de beleza, e a imagem representada ante os seus o\-lhos como uma visão mágica velava-\-lhe as  dificuldades
e dava mais radiosa tonalidade ao seu espírito alquebrado.

Durante os 3 anos de ausência somente tinha de longe em longe
vagas notícias da sua exis\-tên\-cia. O António Ferreira exercia clínica
próximo do Pôrto, e conhecendo o prazer que com isso dava ao seu amigo,
incluia sempre nas suas cartas qualquer alusão chistosa com o habitual bom
humor, desprovida de sentido pejorativo:

-- Há dias, passando pelo Pôrto vi os lobos que desceram ao povoado. E acrescentava um ou outro comentário que o encontro lhe havia sugerido. Umas vezes informava encontrá-la mais nutrida, outras vezes menos; com aspecto de mais saúde umas vezes, e de menos saúde outras. E
a isto se limitavam as informa\-ções que até Eugé\-nio chegavam. Em meados
porém dêsse período de ausência, em 1932, teve de vir ao Norte com demora de algumas semanas. Rondou a casa de Princesa, viu-a algumas vezes e resolveu escrever-\-lhe. Mas como? Pelo correio arriscava-se a não
\-lhe chegar às mãos a carta. Só havia um processo e Eugé\-nio lançou mão
dêle. Conseguiu convencer, embora muito a custo, que uma criada da casa lha levasse. E uma semana depois outra criada trazia oralmente a
resposta de que "a menina mais ve\-lha do sr. Frazão mandou dizer que não
vale a pena andar a perder tempo"; mas Eugé\-nio pôde saber que ela não
estava em casa, e ficou pensando se a iniciativa desta resposta teria
porventura outra proveniência.

\centerline{\rule[-1em]{0pt}{2.5em}:-:-:-:-:-:-:}

\begin{center}\bf\large Regresso ao Porto\end{center}

Após um concurso público a que Eugé\-nio se sujeitára para obter
plena aprova\-ção tinha de sair do Alentejo. Convidado a esco\-lher de entre
as vagas existentes, optou pelo Porto e para ali voltou.

Estar-\-lhe-ia destinada a mesma tormenta? Eugé\-nio previa-a. A tormenta consistia sobre tudo na dúvida de ser correspondido por aquela a
quem amava. Era certo que os seus o\-lhares não evitavam os de Eugé\-nio,
mas a verdade é que não respondia às suas cartas. Esta dúvida cruel dominava inteiramente Eugé\-nio. Procurou não a ver, tentou deixar de pensar
nela. As palavras com que lhe viera António Ferreira, há anos, bailavam-\-lhe na mente; há muitas mu\-lheres, tinha-\-lhe ouvido dizer e concordava;
mas o seu pensamento só albergava uma. E debalde procurava varrer de si
a idea dominante.

Instalou-se nos altos da cidade, num quarto particular. Quarto modesto, quanto bastava para as suas exi\-gên\-cias, e muito aquém das suas possibilidades que ao tempo eram em larga escala destinadas a educar pessoas de
fa\-mí\-lia menos afortunadas a quem nunca regateara todo o bem-estar possível. Nem precisava de ser luxuosa a instala\-ção para os hábitos modestos
de Eugé\-nio. Era decente, banhado pelo sol logo de manhã, confortável e
sossegado, e era o bastante. A mobília aumen\-\mbox{tou-a} suficientemente por
forma a poder dispôr com método as suas coisas e a dar-\-lhe a comodidade
indispensável. Não se sentia mal com a vida simples que sempre vivera.
Quem o quizesse visitar ou abandonava a cerimónia antes de entrar ou
deixaria de repetir a visita. Era decente, e nêle cabiam todas as pessoas
que fôssem amigos sinceros, fossem êles de que categoria fôssem. Eugé\-nio
só distinguia duas categorias — a dos amigos e a dos indiferentes —
e da primeira não fazia cerimónia. De-resto lembrava-se muitas vezes
com mágua quanto de palaciano tinha o seu quarto ao pé dêsses mi\-lhares
tugúrios onde um quinto da popula\-ção citadina vegeta lugubremente.

Eugé\-nio só era exigente na higiene e num pouco de con\-fôr\-to. O primeiro dêsses dois factores era imprescindível para a conserva\-ção da saúde
que podia gabar-se de possuir e que tinha como um dos mais imperiosos
e sagrados deveres procurar manter íntegra. O segundo tornava-se necessário para a boa disposi\-ção que permanentemente o ca\-ra\-cte\-ri\-zava e que, como
uma parte da saúde, era sagrada.

Depois o quarto não se destinava a gran\-de permanência. Com o hábito
de levantar cêdo, Eugé\-nio teria de cêdo o abandonar para passar muitas
horas no traba\-lho fora.

Diga-se de passagem que os hábitos de simplicidade de Eugé\-nio, a
saúde e a boa disposi\-ção, levavam-no e considerar-se feliz, não completamente, para o que só faltava a reparti\-ção dos seus afectos — que até
 nisso era rico— mas remediadamente. A acrescentar a isto, não lhe faltavam recursos a que as faculdades extraordinárias de traba\-lho lhe davam jús.

O seu consultório fi\-cava pertinho, a dois passos, na artéria mais movimentada
da cidade alta. Tinha um defeito. Por ali haveria de passar
 frequentes vezes Princesa, que Eugé\-nio desejava fazer por não ver. Uma
 vez mesmo, por ocasião de um peditório a favor dos tuberculosos, Eugé\-nio
 retirou-se ao aperceber-se da sua aproxima\-ção para a não ver. Mais cedo
ou mais tarde, porém, tinha que a ver por aquela rua constituir passagem
obrigatória para o centro da urbe. Pior defeito ainda, descobriu-o Eugé\-nio em breve, era o de quasi defronte funcionar uma escola de corte que
Princesa frequentava. Mas seriam defeitos êsses? Eugé\-nio assim lhe
chamava, mas nem êle sabia se o sentia. Eugé\-nio não queria ver Princesa,
e ao mesmo tempo e incoerentemente procurava vê-la e adorá-la na sua
beleza admirável...

Pouco depois da chegada de Eugé\-nio ao Porto começava a ani\-\mbox{mar-se} a
praia da Foz. Era a época calmosa. Estavamos em férias e a calidez da cidade
tornava-se insuportável. Eugé\-nio dispunha de algumas horas por dia. A praia,
ali a dois passos, convidava. Uma tarde foi até lá.

\fotoB

As barracas estavam cheias, e o areal do Mo\-lhe coberto de crian\-ças
semi-nuas, bronzeadas pelo sol, em alegre chilreadeira, brincando despreocupadamente com a areia ou correndo diante das ondas, enquanto outras gritavam
no banho. Ás 5 1/2 horas o Mo\-lhe enchia-se de fa\-mí\-lias que os carros eléctricos despejavam com o fito de temperarem um pouco com a briza marítima
o calor excessivo da cidade, ou de descansarem um pouco no verde-glauco do
horizonte marítimo o sistema nervoso sobrecarregado pelo excesso de traba\-lho.

O Mo\-lhe, até essa hora quasi deserto, transformava-se então num denso
formigueiro, a que a garrida policromia das louçãs femininas dava interessante relêvo.

Em breve, porém, no meio da mole de gente que passeava, Eugé\-nio reconheceu Princesa. Esta frequentava a praia, e tinha lá a sua barraca. Eugé\-nio
quis furtar-se a admirá-la, na esperan\-ça de poder esque\-cer quem tanto o atormentava. Mas não pôde. A sua indumentária leve, de praia, realçava-\-lhe os
encantos da formosura. E a tormenta recomeçou dali em diante.

A êsse primeiro passeio, casual, havia agora de suceder-se a romagem
obrigatória todos os dias, como que atraído maquinalmente pelo enlêvo sentido
com o encontro de Princesa. Dirigia-se para lá ás tardes com alguma revista no bolso, e deitava-se nos penedos a ler depois de desembaraçado do casaco e do chapéu.

Assim passava horas, deleitado com as blandícias da brisa e com a vista da imensidade do mar. Para mais dali podia contemplar, na sua barraca, a
figura angélica de Princesa, oferecendo aos o\-lhos de Eugé\-nio um empolgante
quadro de casta beleza, sentada, o semblante como que animado dum ar sobrenatural de melancólica ex\-pres\-são, ocupada nos lavôres, sempre rodeada por
crian\-ças a brincar em redor, entre si ou com ela mesmo — vivo atestado duma
bondade que mais fascinante tornava a sua candura, ou então absorta em alguma leitura.

Ás 5 1/2 horas ia dar uma volta pelo Mo\-lhe, já coberto de gente.

O António Ferreira também frequentava aquela praia. E ambos os amigos,
em grupo com outros, cruzavam muitas vezes com o grupo em que passeava Princesa, a irmã e outras raparigas de entre as quais uma loira. Os companheiros notaram frequentes vezes em Eugé\-nio qualquer coisa de estranho, uma
mudan\-ça de voz, ao cruzar com a\-quê\-le grupo numeroso. E interpelaram-no, pretendendo investigar qual das do grupo merecia tal simpatia a Eugé\-nio que
até não podia deixar de trair na sua voz. E logo por unanimidade consertaram que fôsse a loira. Eugé\-nio não se deu por achado e António Ferreira
não os desmentiu. De ora avante era a loira do grupo quem acobertava a curiosidade dos companheiros de Eugé\-nio nas trocas dos o\-lhares deste com
Princesa...

Um dia Eugé\-nio, incitado pelo António Ferreira, encheu-se dum pouco
mais de coragem. Cruzou, sozinho, as do grupo, e fitou Princesa. Esta fitou-o
também energicamente, e Eugé\-nio manteve com energia o o\-lhar. Estava então
todo o grupo debruçado sobre as guardas de ferro do Mo\-lhe, voltado para o
mar, e Princesa tivera, para fitar Eugé\-nio, de se voltar obliquamente para
traz onde êle se encontrava encostado nas guardas do outro lado do Mo\-lhe.
A timidez, se assim se pode chamar ao vulgar abaixamento do o\-lhar quan\-do
este dá com o da pessoa verdadeiramente amada, dissipara-se pela primeira
vez em Eugé\-nio.

Tornava-se necessário dar uma solu\-ção ao tormentoso problema: ou a
vitória resplandescente de alegria para Eugé\-nio, ou o total aniquilamento
do seu espírito tumultuoso de apaixonado. Era para Eugé\-nio uma questão sumamente vital. O o\-lhar feérico de Princesa era uma esperan\-ça. Mas de
futuro, nas passagens dum pelo outro se ela o\-lhava algumas vezes, seguia com indiferença outras vezes. O mesmo enigma de sempre!

Escrever-\-lhe-ia mais uma vez, na firme resolu\-ção de esque\-cer definitivamente o passado no caso da carta não ter me\-lhor sorte que as
anteriores. Eugé\-nio concentrou-se um pouco. Que dizer-\-lhe? Abominou
sempre as palavras hipócritas, e temia que as suas fossem tomadas como
tal. As declara\-ções de amôr redundam sempre numa série de lugares
comuns, mais obra de verborreia romântica do que ex\-pres\-são do sentimento pessoal. A dificuldade estava em levar a Princesa a convic\-ção da
sinceridade que repassava as suas palavras. Não constituiria, de-resto, um testemunho bastante da afei\-ção de Eugé\-nio por Princesa, a constância e persistência com que a admirava e seguia desde há anos?

Escreveu-\-lhe pelo correio uma longa carta na qual procurou traduzir o me\-lhor que pôde e soube a sinceridade que a inspirava. Ei-la:
 (...)\footnote{Não se encontra no original dactilografado o texto desta carta. }

%marginpar{pdf = 35}

Alguns dias passados estava Eugé\-nio abancado a uma mesa de chá,
com a família de António Ferreira. O grupo de Princesa veiu ocupar a
mesa contígua. Nem uns nem outros o costumavam fazer, mas naquela tarde coincidiram as ideias. Princesa ia sentar-se de costas para a mesa de Eugênio quan\-do notou a presença dêste; mudou então para de-fronte, e não mais retirou os o\-lhos dos de Eugé\-nio. Este sentiu-se feliz,
e um novo sôpro vital o animou. Embora tôda ela de enigmas, parecia ser
fora de dúvida não ser indiferente a Princesa. Mas acidentes análogos se haviam dado noutras emergências, e nova dúvida turvava o espírito de Eugé\-nio. Desta vez parecia porém mais posta à prova a sua
simpatia, tornava a pensar, avolumando mais a dúvida que ameaçava tornar-se crónica.

Esperaria o dia seguinte na esperan\-ça de ver corroborada a forma como Princesa se houve para consigo. Só a sequência dos factos
podia confirmar ou anular a radiosa esperan\-ça de Eugé\-nio. O pior é
que no dia imediato Princesa não apareceu. Nem nos que se lhe seguiram. O verão estava no final, e Princesa partia para Trás-os-Montes
onde a fa\-mí\-lia ia assistir às co\-lheitas. Mais uma desilusão.

A praia deixou de ter aquela vida que Eugé\-nio estava acostumado
a encontrar-\-lhe. Parecia-\-lhe soturna. O mar, mesmo êsse, perdeu para
ele todos os encantos. Eugé\-nio já não encontrava interêsse algum na
Foz, e reco\-lheu-se à vida pacata habitual. O António Ferreira, seu único confidente, ouviu-\-lhe como desabafo estas palavras: -- Parece que
um véu negro me rodeou. Vejo tudo soturno e até a luz do sol me parece mais ténue. A vida não me traz alegria.

António Ferreira obtemperava sempre: -- O\-lha, meu caro Eugé\-nio, deixa-te de tristezas. Vira-te para outro lado, que mu\-lheres abundam e das
que te servem. Tu é que, obcecado como andas, não as vês. Supões que
só existe essa a quem tanto te prendeste? Segue o meu conse\-lho: procura outra e não penses mais nessa.

Eugé\-nio, cabisbaixo, respondia: Era de facto o único remédio o esquecimento. Consegui-lo era conseguir o me\-lhor alívio para o mal que
me acabrunha. Mas isso é impossível. A afei\-ção que desde há anos voto a Princesa está muito arreigada em mim. E tu bem sabes; bem sabes
quanto gosto dela, e quanto impossível é substituí-la no meu cora\-ção.

-- Demais, continuou o António Ferreira, tu não exibes a\-quê\-les requisitos de indumentária que as raparigas exigem. Tu andas decente e
nada mais, o que para elas não basta. Querem mais. Um fato irrepreensivelmente cintado na moda e constantemente vincado a esmero, um bocadinho de perfume num cabêlo delambido, uns modos efeminados e outras
futilidades que elas são exímias peritas em apreciar.

-- De facto a mu\-lher é tôda de futilidades, mas Princesa, com o senso prático que possue e modesta como é, e de modos naturais sem qualquer espécie de afecta\-ção, tem de ser fatalmente uma excep\-ção. A modéstia no vestuário até deverá agradar ao seu espírito. Quanto a perfumes...  suponho bem que, lavando-me todos os dias, não terei cheiros
desagradáveis a mascarar.

-- Tu não sabes o que são mu\-lheres. Princesa é como as outras.
Pren\-dem-se só com essas pequeninas coisas, e o valor do homem, a sua
forma\-ção moral e intelectual, faculdades de traba\-lho, etc,
são coisas para elas muito secundárias. Exigem de preferência que os rapazes ostentem exterioriza\-ções
espectaculosas não só no chiquismo da sua apresenta\-ção impecável como nas exibi\-ções de salão, ainda
que ôcos interiormente; que se mostrem nos bailes ou na rua a temperar o seu ócio com os madrigais que
\-lhes dirigem, ainda que imbecis.

\centerline{\rule[-1em]{0pt}{2.5em}:-:-:-:-:-:-:}

\begin{center}\bf\large Cerco a Princesa\end{center}

Em Eugé\-nio subsistia como pesado fardo a eterna dúvida. Ela
gostará ou não dêle? A última carta, como as anteriores, chegaria ao
seu destino? Se soubesse por forma inequívoca que era indiferente a
Princesa, conformar-se-ia. Mas qual o signifi\-cado da\-quê\-les o\-lhares com
que ela tantas vezes correspondeu aos seus?

Como poderia tirar-se de dúvidas? O seu cora\-ção incandescente,
ardendo em labaredas, incendiou o cérebro. Congeminou intimamente e mil
coisas lhe passaram pela mente. O cérebro passou a ser um vulcão. Se
conseguisse captar simultâneamente a simpatia de Princesa e dos Pais...
Mas como?

%centerline{\rule[-1em]{0pt}{2.5em}:-:-:-:-:-:} JNO: RISCADO

O estabelecimento de educa\-ção infantil protegido por Princesa,
em que Eugé\-nio entrára há anos, era duplamente simpático a êste: era-o pelo fim educativo que realizava, e de mais a mais em criancinhas
da 2ª infância tão amadas por êle, e por funcionar sob a égide benemerente daquela protectora. Conseguiu que lhe fôssem aceitos gratuitamente os seus serviços profissionais e começou a ir por lá, concorrendo com entusiasmo e por todos os meios ao seu alcance para o bem-estar
da\-quê\-les miúdos, na sua maior parte muito pobrezitos. A Directora do
estabelecimento via com imensa simpatia a ac\-ção de Eugé\-nio. Numa parede do salão maior ostentava-se garbosamente uma bela fotografia ampliada de Princesa e sua irmã, como gratidão aos benefícios tão gentilmente prestados. Mas não fi\-cava por aqui a actividade de Eugé\-nio.

%centerline{\rule[-1em]{0pt}{2.5em}:-:-:-:-:-:-:} JNO: riscado


Na cidade havia uma explora\-ção de determinado produto alimentar,
com destino exclusivo a criancinhas. Era uma explora\-ção que custara
muito dinheiro ao seu proprietário, muito dispendiosa, e de lucro minguado, mas para a qual êste vivia com tôda a sua alma. Falava dela com
enternecimento, e discutia com calor os seus altruistas fins e com verdadeiro conhecimento de causa as bases científi\-cas em que tal obra
se fundava. Honrava o país, como única no género, e estava instalada em
obediência aos mais exigentes princípios de higiene. Dos países mais
avan\-çados da Europa, que o proprietário visitou, veio a inova\-ção destinada a produzir retumbante eco nos meios científicos. A explora\-ção estava à disposi\-ção dos médicos, que o proprietário, Sr.\ Frazão, tinha muito gôsto em receber.

Eugé\-nio arquitectou maneira de se relacionar com o sr. Frazão.
Estudaria a fundo o  problema, que ademais lhe interessava como médico:
Rebuscaria os livros nacionais e estrangeiros, e visitaria mesmo outras indústrias análogas.

Um bom par de mêses levou Eugé\-nio no seu estudo intenso e extenso com que deveria adquirir suficientes bases para se insinuar no espírito do Sr.\ Frazão. Mas êste estudo tinha agora para Eugé\-nio um
outro interêsse, o interêsse das coisas da ciência. Entusiasmou-se
duplamente. Nunca tinha pensado a fundo em tal problema, que tanto se
ajustava à sua profissão. Se alguma vez tivera de o abordar no decurso dos seus estudos, foi de passagem como gato sôbre brasas. Em Portugal eram raros a\-quê\-les que se dedicavam ao caso, que a Eugé\-nio se afigurava agora, com conhecimento de causa, como fundamental no abaixamento da taxa de mortalidade infantil que tanto preocupa os higienistas
de todo o mundo.

Estava ainda apto a convencer os incrédulos que chasqueavam dos
processos modernos de que vagamente se falava, como é de uso e costume para com todos os gran\-des empreendimentos.

Estudou muito tempo ainda. Houve quem o informasse que o sr. Frazão era águia no assunto, e que aficionado pela sua explora\-ção como era,
pregava prolongadas li\-ções de muitas horas seguidas, de tardes inteiras,
a todo a\-quê\-le que lá caísse. Eugé\-nio não queria fazer como os mais,
que se limitavam a ouvir. Desejava antes discutir o caso para me\-lhor
ainda o aprofundar. Nisso estava empenhado o brio de quem, como
Eugé\-nio, nunca gostava de se encontrar em falso em qualquer assunto
científico, antes procurava sempre atacá-lo a fundo.

Estavamos na primavera de 1935.

Por uma tarde de domingo, após o almôço, resolveu-se a visitar aquela explora\-ção. Já tinha visitado as outras, e esta fi\-cara propositadamente para o fim.
A caminhada era curta, como o espaço da sua casa á do Sr.\ Frazão; ainda mais curta lhe pareceu, tão embebido
nos seus pensamentos que lhe perpassavam no cérebro, tantos e tão diferentes, que o a\-lheavam completamente do
que via. Que faria se a visse, a ela?
E mais: se lhe proporcionasse falar-\-lhe? Vacilava então, desejaria renunciar à visita, possuído todo ele duma
timidez estranha, e tornava-se necessário recobrar ânimo para prosseguir. O
cora\-ção batia por momentos apressadamente. As pernas teimavam em não sustentar o corpo, como que vergando sob
ac\-ção duma corrente eléctrica, e o
passo tornava-se incoordenado.

Chegou enfim, ao portão. Para pôr
termo á hesita\-ção crescente, apressou-se a puxar o cordão da pesada sineta,
que repetidas vezes teve de fazer soar.
Entretanto, os minutos de espera eram
como semanas de ansiedade. Que se iria
passar, ali dentro daqueles muros, para
onde se dirigem todos os seus pensamentos, e onde está a sede de todos os
seus sonhos? Como seria recebida a
sua pessoa? Como é que lhe iria parecer o interior daquela quinta encantada, de que só entrevia te\-lhados de celeiros e estábulos, e um motor de vento, e, mais longe, a casa de Princesa?

O portão abriu-se, enfim. O Sr.
Frazão passeava perto, em companhia
duma fi\-lha, e o grupo pôde ser avistado imediatamente por Eugé\-nio, mas êste perturbou-se-\-lhe a vista a tal ponto
que nem sequer reconheceu quem acompanhava o Sr.\ Frazão, apesar da pequena distância. Seria Princesa? Seria
a irmã Julieta?

O que é certo é que quem quer que
fosse desapareceu misteriosamente, e o
Sr.\ Frazão, já sozinho, acorreu ao encontro do visitante que acabara de fazer-se anunciar, e que aliás já esperava, porque Eugé\-nio o havia prevenido de véspera
pelo telefónio.

Todas as instala\-ções foram então
pormenorizadamente mostradas a Eugé\-nio.
Este não
pôde deixar de louvar o meticuloso asseio, a técnica empregada e a iniciativa e orienta\-ção reveladas em empreendimento de tal magnitude.
Seguiu-se a tarde tôda na conversa, pois o Sr.\ Frazão além do saber
técnico que possuía era um explêndido cavaqueador. Só ao anoitecer,
e já considerados dois amigos, se separaram para jantar.

Não surtiu outro efeito a visita do que as amistosas rela\-ções
travadas e o conhecimento da interessante indústria. Voltou lá mais
duas vezes a breve trecho, e eis tudo. E corresponderam-se ainda algumas vezes pelo telefónio ou por cartas sôbre assuntos cuja discussão começou no encontro de ambos.

Passados meses Eugé\-nio recebia uma carta do Sr.\ Frazão, na qual
\-lhe fazia um pedido a favor do portador. Depois de informado
Eugé\-nio reconhecia-se incapaz de levar a bom final a tarefa, tão superior ela era às suas fôrças. Foi uma luta titânica em que por vezes
Eugé\-nio esmorecia a ponto de lan\-çar mão do telefónio para transmitir
nervosamente ao Sr.\ Frazão quanto difícil era a tarefa que lhe impusera, quanto duros os obstáculos a vencer apesar dos seus esforços reais,
e sugeria-\-lhe o refôrço do pedido por intermédio de oûtrem que viesse
em auxílio de Eugé\-nio. O Sr.\ Frazão limitou-se a reiterar a confian\-ça
a Eugé\-nio.

Duma das vezes em que o dasalento se apoderou de Eugé\-nio, foi Laurinda,
uma prima de Princesa já sua conhecida, que o atendeu.

-- Sabia bem das dificuldades, declarou. O sr. Frazão não está,
mas pode estar certo de que êle se não dirigirá a mais alguém. E,
acrescentou ironicamente: Bem sabemos todos aquilo que tem feito
pelo caso do nosso parente comum, e o que não conseguir não o conseguirá mais nin\-guém.

O que é certo é que Eugé\-nio conseguiu, com gran\-de gáudio seu e
após momentos de extrema ansiedade, levar a bom termo o desejo do Sr.\
Frazão.

O cérebro de Eugé\-nio em vulcão continuava em plena actividade.
Tudo aquilo que pudesse interessar directa ou indirectamente a Princesa era também para si motivo do maior interêsse e simpatia.

Um dia foi visitar no exercício da sua profissão o feitor da quinta do Sr.\ Frazão. Morava êle longe, no centro da cidade. Pior, porém, que com o
pai estava uma fi\-lha do feitor. Era uma interessante rapariga loira,
para quem as ilusões dos seus 20 anos se tinham já apagado perante
a doença que a minava havia muitos anos e agora reduzia aquela vida
aos últimos redutos. Não aparentava gravidade o estado da doença a
quem a visse. A frescura da sua beleza estava destinada a acompanha-la fielmente até à última morada, assim como a vivacidade dos seus
o\-lhos e a ajudeza da inteligência. Conhecia muito bem o seu estado periclitante, e premeditava com elevado estoicismo a sua sorte para breve.

Eugé\-nio, encarregado de lhe assistir em substitui\-ção temporária
do seu médico, visitava-a todos os dias. Ju\-lho ia excepcionalmente
quente, o que à doentinha trazia maior dificuldade em satisfazer de
ar a\-quê\-les pulmões carcomidos. A ansiedade gran\-de com que via desfo\-\mbox{lhar-se} precocemente, uma por uma, as pétalas daquela flôr levavam-na a chorar, a chorar muito. É que no fundo e a despeito do seu aparente estoicismo devia ser uma revoltada por se ver arrebatada dum mundo onde
sonhara quimeras que nunca pudera realizar.

Quando Eugé\-nio entrava, podia ver sempre em Linda -- êste era o
seu nome abreviado -- os o\-lhos marejados de lágrimas. Sentava-se a
seu lado, e confortava-a. Ameaçava substituir por feia o seu nome se
\-lhe tornasse a ver lágrimas. Conversavam muito em seguida, todos os
dias, durante uma boa parte da tarde. A temperatura desse cálido Ju\-lho levava Linda a convidar Eugé\-nio, esbaforido de calor e salpicado de suor por
sobre a face congestionada, a tirar o casaco, a pôr-se \mbox{á-vontade}. Eugé\-nio não
se fazia rogado. Para quê? Se está na índole de toda a gente procurar o
á-vontade em todos os actos, e em especial na indumentária, para que fingir
cerimónia hipócrita, como acontece nas inditosas termas, nas praias, e até no
campo, com o azombar da naturalidade que ca\-ra\-cte\-ri\-za a natureza? Eugénio, com a mesma naturalidade que ca\-ra\-cte\-ri\-za a natureza e o seu temperatento, tinha por norma nunca esperar por segundo convite para se pôr á-vontade, e isso tanto no passeio, como numa visita, como num jantar.

Linda esquecendo-se dos seus sofrimentos contava muitas peripé\-cias
da sua juventude cheia de toda a espécie vida, documentadas com fotografias várias. Eram
diabruras de toda a espécie tornadas em motivo de alegria constante para
as amigas e companheiras.

Pode dizer-se que fôra criada com Princesa e com a irmã, as suas duas
maiores amigas da infância. A mãi destas, a Sr.ª D.\ Maria Frazão, queria-\-lhe
como se fosse sua fi\-lha. Devia tanto, tanto, a esta senhora... Ainda agora,
antes de se ausentar para Trás-os-Montes, afirmara levá-la bem atravessada
no cora\-ção. Aquela fotografia em que Linda se vestira de rapaz, de cigarro
na mão, encostada sob a janela em que assomava a Sr.ª D.\ Maria, era a diabrura que mais tinha enlevado a mãi das suas amigas queridas.

Destas amigas recebia Linda cartas, que lia com sofreguidão. Ansiava
dia a dia por saber da mais nova das irmãs, de Julieta, recentemente
operada e que por motivo de convalescença fez retirar a fa\-mí\-lia para a sertaneja e longínqua aldeia onde tinha as suas propriedades.

Eugé\-nio ignorava se Linda era conhecedora da sua simpatia para
com Princesa. Ou porque sabia prender dêste modo a aten\-ção de Eugé\-nio,
ou porque a sua amizade lho determinava, o certo é que as conversas de
Linda derivavam todos os dias sem excep\-ção para a pessoa de Princesa,
sôbre quem havia sempre que dizer de novo. E Eugé\-nio ouvia, ouvia, esquecia-se completamente dos outros afazeres. Até mesmo encaminhava
já a sua vida no sentido de lhe fi\-car a tarde livre. E quan\-do por qualquer motivo Linda demorava a desviar a conversa no sentido desejado,
Eugé\-nio levava-a com discre\-ção a fazê-lo.

\newpage % JNO

Nenhum pormenor sôbre a vida de Princesa ficou decerto desconhecido de Eugé\-nio.

-- Nunca vi rapariga assim, dizia Linda. Vive para o traba\-lho e
não lhe cresce tempo para se distrair. Foi sempre assim. Já em crian\-ça era a que tinha aspecto mais senhoril e o pensar mais ponderado.
Hoje, criada num ambiente de virtudes que no mais alto grau possue
a mãi, Princesa tem tôdas as qualidades duma dona de casa, como ex\-pres\-são mais alta das virtudes que considero apanágio duma mu\-lher. A sua
modéstia realça-\-lhe sôbremaneira o valôr real intrínseco. Nunca nin\-guém lhe viu o mais leve artifício sôbre a pele, nem sequer o simples
pó de arroz.

-- É de facto preciosa tal rapariga, no meio desta sociedade tão
contaminada pelos modernismos que dissolvem radicalmente os sagrados
laços da fa\-mí\-lia. As mu\-lheres que por aí se vêem, precipitando as características dum século vindouro, vivem tudo para o exterior num sentido totalmente deturpado da sua missão. A casa deixada ao abandono das criadas
não é mais a\-quê\-le lar que o marido deveria encontrar e prender no repouso do seu traba\-lho fatigante, mas o desmantelamento anárquico que o
afugenta. Quando a mu\-lher sai à rua prende mais a aten\-ção do transeunte que admira as cores apetitosas em que a droga transformou a sua
pele, os lábios, as unhas das mãos, e até, para cúmulo, dos pés, ou que se
substituiu às suas sobrance\-lhas. Para onde caminhará isto, com tal
gama de desvarios? Eu conheço-as desde aquelas que se poderiam classifi\-car de século XX, simplesmente empoadas, até às que se exibem num
futurismo estúpido do rosto e cabelos coloridos a condizer com a coloraçao do vestido e calçado do dia, autênticos bonecos que para nada
mais servem do que enfeitar a casa. São as do século XXX. As do termo intermédio, século XXV, pintam desmedidamente a cara, as sobrance\-lhas os lábios e as unhas, duma só côr, passam três
quartas partes do dia no ténis, na rua, nos chás, baile, desafiam a irreverência de tudo e todos com a lasciva nudez dos
peitos, ombros, braços e pernas, não criam fi\-lhos ou porque lhes
desmancham a estética ou porque lhes pesam nos braços como brasas
escaldantes, e não sabem cozinhar porque nunca tiveram tempo para aprender e as mãos perdiam o mimo da frescura, e que coleccionam as contas
da modista à espera duma heran\-ça com que o marido as possa solver.
Caso curioso: essas mu\-lheres atentam inclusivamente contra as leis da natureza. Numa rápida
análise da escala animal imediatamente se nota
ser sempre o sexo masculino que se reveste das
côres mais berrantes para seduzir o outro sexo.
Veja por exemplo a plumagem das aves, do pavão,
do galo, do faisão. O macho é sempre o chamariz
da fêmea, e nunca inversamente, como acontece com
todos os seres vivos... excepto com o ser humano
ultra-moderno em que o pudor desapareceu.

Mas além das do século XXV e entre aqueles extremos há classifi\-ca\-ção para a maioria do que para aí se vê.

Ao século XIX pertencem aquelas que cozinham e dirigem uma casa,
vivem para a fa\-mí\-lia constituída que manteem indestrutível com a ternura e carinhos que só ela conhece, e que se preocupa em transmitir aos
seus fi\-lhos o nome herdado dos pais, que não exige do marido um vestido quan\-do êste lho não possa dar.

-- Já tinha notado que o Sr.\ Doutor detesta a vida fictícia moderna, como eu também detesto. Os princípios que defende são apanágio
da vida sã que eu já lhe adivinhara. Ainda bem que há um rapaz, e para mais da sua cultura, que abomina a podridão. Pois o\-lhe que aquela
fa\-mí\-lia de que tanto lhe tenho falado é a encarna\-ção do seu sentimento.

-- Por isso, e porque constitue nos
tempos de hoje um caso invulgar, tenho
ouvido com o maior interêsse as narrativas que se lhe referem. Porque,
continuou Eugé\-nio, pondo a sua habitual
ironia de bom humor nas palavras, porque a continuar este estado de coisas,
teremos nós, os rapazes, de aprender a
cozinhar, a embalar o bebé, e todos os
mais serviços domésticos, para que a
refei\-ção esteja preparada e a casa
arrumada e o menino adormecido ao chegar da esposa que regressa dos seus
deveres mundanos, inglòriamente a\-lheada das mais nobres fun\-ções que são
as que lhe competem.

-- Oxalá não dêem aquelas raparigas com um estouvado que não lhes
admire outra coisa além da sua fortuna. Não o merecem por princípio
algum. Prevejo que aquela quinta tão bela cá na cidade, em que o Sr.
Frazão tanto gôsto tem e que conserva uma maravi\-lha, depressa seja
desfeita. Porque, paralelamente ao que acontece com as mu\-lheres são
poucos hoje os rapazes dignos de constituir fa\-mí\-lia. A dissolu\-ção é
exactamente a mesma.

-- Exactamente, confirmou Eugé\-nio.

-- A maioria dos rapazes tem um sentido de vida muito diferente
do verdadeiro. O rapaz ponderado, desejoso de ser útil à sociedade
pelo traba\-lho honesto, metódico em todos os actos, senhor duma forma\-ção moral que a\-ma\-nhã lhe dê foros dum chefe de fa\-mí\-lia exemplar, quási que desapareceu, para ser substituído pelo boémio estúrdio, vadiando
ociosamente pelas ruas da cidade, estroinanto quanto possue ou quanto
pode vir a possuir, estragando a sua saúde na vertigem dos gozos materiais de tôda a espécie, péssimo chefe de fa\-mí\-lia pelo exemplo, péssimo
pai pela saúde arruinada que legará aos fi\-lhos e péssimo homem duma
sociedade em que figura como inútil ou até pernicioso.

E prossegue: um considera como ex\-pres\-são máxima da vida a valorisa\-ção da própria pessoa e da fa\-mí\-lia que chefiará, e ainda da sociedade a que pertence. O outro não encontra outra finalidade na vida
que o gôzo animal. Aquele é o que transmite incólume o bom nome dos
pais aos seus fi\-lhos; êste não se lembra já do que foram os pais, nem lhe
importam os fi\-lhos. O primeiro é o que traba\-lha por aumentar o bem
estar do seu lar que êle ama acima de tudo; o segundo é que não conhece sequer o signifi\-cado do lar. O primeiro é ainda a personifi\-ca\-ção
da honra; o segundo não tem personalidade alguma. Aquêle é o que produz, êste o que destroi. Aquêle é o que cansado de traba\-lho encontra
o repouso reconfortante no seio da fa\-mí\-lia; êste é o que cansado do
seio da fa\-mí\-lia repousa na tavolagem e no prazer. A maioria dos rapazes de hoje é, em suma, a janotice ôca e inútil, desprovida de todo o
sentimento e tendo como únicos predicados a elegância da cintura e
as luvas a realçar o seu maneirismo grácil.

-- A constitui\-ção do lar tende a ruir perante a desorientada forma que preside à sua organiza\-ção, acrescentou Eugé\-nio. Muitas são as
vezes em que os cônjuges não vivem naquela atmosfera harmoniosa de
ideias e sentimentos que deve ser a única característica do lar. Poucas, muito poucas, são aquelas em que a comunhão de ideias e sentimentos tornam a vida do lar no mais dôce paraíso terrestre. Porquê? Necessariamente porque os temperamentos, o modo de ser dum dos nubentes
não se coaduna com o do outro. Quantíssimas vezes é o interêsse material que preside a tais enlaces, a avidez da simples ma\-té\-ria tôrpe
que atrai os indivíduos de sexo diferente, a mentira, enfim, prestes a
desmascarar-se para precipitar o lar num inferno? O amôr não se impõe
pelo bri\-lho do outro, mas pela justaposi\-ção dos feitios. Sempre que
uma sólida forma\-ção moral, objectivada no respeito mútuo, pronto a aplanar desde o início os pequenos incidentes, e na no\-ção do dever a suplantar a no\-ção do direito; sempre que a disposi\-ção temperamental não é
a mesma, sôbretudo exteriorizada numa atrac\-ção real e natural que é
a simpatia, e em costumes e hábitos idênticos, de entre os quais ressalta o método no arranjo das coisas; sempre que as faculdades intelectuais se não aproximam, segue-se o inevitável cortejo de dissidên\-cias
que do ninho de amôr fazem o perpétuo suplício daquelas almas penadas
que em má hora se juntaram.

Infelizmente são estas almas que em maior quantidade se vêem
por aí, e de que nós nos não apercebemos por não surpreendermos a sua
vida íntima. Pelo facto são responsáveis tôdos a\-quê\-les factores que
se podem resumir numa só ex\-pres\-são: não se estudarem convenientemente os entes antes da resolu\-ção conjugal, deixados prender pela sensibilidade efémera da beleza que desaparece dum momento para o outro
como espiralas de fumo, ou pela embriaguez do dinheiro cuja abundância
mais dissolve do que constroi felicidades.

Linda apoiava com acenos de cabeça as afirma\-ções de Eugé\-nio, e
acrescentou:

-- Poucas vezes tenho ouvido falar com tanto acêrto. Se todos pensassem assim... A sociedade está atolada no miserável lodaçal que
vemos e de que é impossível desembaraçá-la.

-- O que se passa a-dentro do lar não é mais do que o reflexo do
que se vê cá fora. A sensibilidade e a ambi\-ção desmedidas põem em
crise constante o juízo. A sensibilidade que nos leva a sermos fascinados pelo sentido do prazer ou da simpatia mais que pela luz da razão:
e a ambi\-ção que nos leva a julgarmo-nos sempre prejudicados nos nossos
direitos de revindica\-ção a julgarmo-nos sobrecarregados nos deveres.
Todos nós pretendemos ter mais direitos que a\-quê\-les que temos, e menos
deveres. Entendemos ter o direito de invadir a esfera dos outros,
de participar dos benefícios dos outros, de traba\-lharmos menos e
dispender mais horas no ócio, de sermos tratados com urbanidade, de
termos me\-lhor mêsa, de não termos que dar satisfa\-ções dos nossos actos,
de viajarmos comodamente; e não reconhecemos êstes mesmos direitos
nos outros que parece terem mais de que nós o dever de se deixarem
invadir na sua esfera de ac\-ção, de restringirem os benefícios co\-lhidos,
de traba\-lharem mais e terem menos descanso, de receberem as nossas grosserias, de serem mais sóbrios na alimenta\-ção, de darem satisfa\-ções, dos seus
actos, de se deixarem acotovelar numa viagem. É a falta de respeito,
a convic\-ção de que somos mais do que os outros, ambi\-ção de sermos mais
do que na realidade somos. Um funcionário não tolera que os superiores
o tratem com menos amabilidade, e é incorrecto para com os seus inferiores. Estes não estão nas mesmas condi\-ções de exigirem da\-quê\-le o
que êle exige dos seus superiores? Um industrial julga-se no direito de arrecadar somas avultadas no fim do ano para estroinar num mês
no estrangeiro, e não reconhece ao seu pessoal o direito duma retribui\-ção que me\-lhor o compense do seu traba\-lho e lhe dê a garantia duma alimenta\-ção me\-lhorada; direitos para si, deveres para o operário. O burocrata, com umas limitadas horas de serviço por dia, direito à aposenta\-ção,
direito a um mês de licença todos os anos e a outras licenças por
doença, etc, etc., não tolera que o seu criado, que traba\-lha de dia e noite e não tem regalias de espécie alguma além do ordenado, adoeça uns
dias. Um proprietário residente na cidade sem outros incómodos que
o de receber o rendeiro que lhe vem pagar a anuidade tem o direito
de exigir cada vez maior renda, enquanto que o rendeiro tem o dever
de morrer à fome.

Todo o indivíduo tem o direito de impôr o seu pensar a quem o
ouve, e a quem o ouve nega-se o direito de discordar das suas ideias.
Em ma\-té\-ria de política sucede o mesmo. Todos se arrogam o direito
de mandarem e nin\-guém se arroga o dever de obedecer. Todos o de criticar e nin\-guém o de ser criticado. Os partidos são retalia\-ções de
obcecados que não se compreendem porque a paixão lhes cega o juízo.
E que signifi\-cado teem os partidos? Dois grupos políticos inimigos ôntem, conluiam-se amanhã para combater um terceiro; e no dia seguinte não
admira que êste último esteja mancomunado com um daqueles. Os filiados
deixam de ser cérebros pensantes para se tornarem acorrentados á idea
dum só homem, cujas ambi\-ções e interêsses servem, na esperan\-ça em geral
de verem os seus interêsses e ambi\-ções servidos. Não é a luta de ideas,
mas sim a dos interesses quasi sempre, que impulsiona o indivíduo; mesmo
assim acontece se, nem sempre a separa\-ção é suficientemente nítida
para justifi\-car a intolerância, e ainda que o fôsse não é fácil dizer
desapaixonadamente, de que lado está a verdade que os adversários disputam
com igual direito. Bastará dizer que aquilo que hoje temos como verdade
a\-ma\-nhã deixa de o ser. Por isso a discussão resulta quasi sempre estéril
acarretando incompatibilidades pessoais, até da\-quê\-les que eram amigos e
se sobrepõe á amizade o falso jugo do partido a que o facciosismo dá
intenso colorido que só a razão desvanece. Não é tão pouco a filia\-ção
partidária que ca\-ra\-cte\-ri\-za o indivíduo, mas o seu temperamento. Não vê
muitos indivíduos com o rótulo de liberais serem os mais intransigentes
déspotas na sociedade ou a-dentro da fa\-mí\-lia, e, inversamente, como outros
se dizem retrógrados e são por actos e factos estruturalmente liberais?
O rótulo exterior nada vale; o que vale é o interior que só a conduta
traz á superfície.

Podia, enfim, estar um dia inteiro a enumerar factos demonstrativos de
quanto os direitos e a razão estão longe de ser compreendidos, e de quanto a
a falta dessa compreensão actua na
sociedade. A força do direito não existe, mas sim o direito da força.
Quanta harmonia, quanta paz, quanto bem estar, adviriam para
a sociedade do domínio do juízo sôbre a sensibilidade... Menos isen\-ções
pessoais, menos retalia\-ções políticas, menos ideias subversivas. Mais respeito mútuo.

E prosseguiu: Talvez se possa filiar nessa falta de compreensão,
no desequilíbrio que existe constantemente entre o juízo, por um lado,
e a ambi\-ção e a sensibilidade, por outro, um vício de forma\-ção que destroi as amizades e separa os indivíduos. É a inveja. Este sentimento
leva-nos a só ver o bem estar dos outros que desejaríamos para nós.
Se um amigo modifi\-ca a sua situa\-ção para me\-lhor no sentido duma mais
favorável situa\-ção na vida, eis-nos a guerreá-lo com a mais pérfida
das armas -- a inveja. Mas, caso curioso, que tanto inferioriza a nossa
forma\-ção moral: não lhe invejamos o esfôrço que fez para conseguir
elevar-se e não pensamos na sorte dos desgraçados inferior à nossa.

-- Ver as coisas desapaixonadamente, apreciá-las a frio, criticá-las imparcialmente, bilateralmente, pelo lado que nos respeita e pelo que nos poderia respeitar, devia ser a norma de quem se preza, de
quem tem um cérebro a caracterizá-lo de racional. A meu favor ou
contra mim, a favor ou contra um parente ou amigo, a nossa razão deve-nos dar um testemunho fiel, não o testemunho da sensibilidade que é
as mais das vezes a nega\-ção da verdade a depôr na balan\-ça da justiça.
E a verdade deve colocar-se muito acima de todo e qualquer móvel
afectivo. O acto criminoso dum desconhecido tem o mesmo cunho de importância que o mesmo acto cometido por um amigo ou parente. Uma ac\-ção
deve ser apreciada pelo seu valor e bem independentemente da pessoa
que a praticou ou do grupo a que pertence.

Pense bem, e encontrará que em redor dessa crise de juízo é que
giram todos os desvarios da sociedade, talqualmente todos os astros
giram em redor do sol. Só há uma diferença: é que o sol sobeja de potência para arrastar disciplinadamente os inúmeros astros satélites, quanto o juízo
fa\-lha para se impôr à\-quê\-les dois resumidos factos -- sensibilidade
e ambi\-ção.

-- Nas ciências como na política, em todas as actividades da socieade, continuou Eugé\-nio, a crise de que falo tem sempre o mesmo fundo.
Até na ciência, repito. Veja como as maiores aquisi\-ções da ciência relativas à conserva\-ção da nossa vida, à profilaxia do câncro, por exemplo,
passam desapercebidos pela massa popular, ao contrário do que acontece
com notícia dum novo invento que momentâneamente destruirá uma cidade com os seus mi\-lhares de vidas. A ciência não é aquela arma com
que procuramos defender-nos das vicissitudes da contingência, prolongando a vida e procurando-\-lhe maior rendimento em todos os sentidos,
mas sim a arma posta ao serviço da destrui\-ção das vidas para satisfazer ambi\-ção e saciar o ódio e as paixões. Sempre a ambi\-ção e a sensibilidade, como vê, que é o mesmo que dizer sempre a falta de juízo!

-- Tem razão, diz Linda.

-- Mas o lar, dizia eu, ressente-se como a sociedade em geral, do
domínio do juízo sôbre a sensibilidade. Vou mostrar-\-lhe um outro aspecto sob que naufraga a harmonia do lar, e que podemos generalizar
à harmonia da sociedade. É um aspecto fundamental na felicidade das
fa\-mí\-lias e dos povos, porque é a base da economia, que frequentemente é
a mola que acciona todas as dissidên\-cias. Poucas são de casas onde
há um or\-ça\-men\-to. Gasta-se mais do que se pode. Quantas fa\-mí\-lias se
vêem por aí exibindo supérfluas riquezas no vestuário e na habita\-ção,
para quem correm todos os dias os credores, o joa\-lheiro, a modista, o
alfaiate, os simples padeiro ou leiteira, a mendigar, em vão, o pagamento
da sua conta já atrazada! São as bacanais, os automóveis, tudo quanto
pode ruir a economia duma casa quan\-do não há uma sólida fonte de receita que lhes faça frente. E onde há hoje uma sólida fonte de receita?
Poucas vezes. Note a falta de sensibilidade dessa gente, que não cora
ao passar por tantos credores, e a deficiência de juízo que os impede
de pegar num lápis a verem o que podem gastar em face dos proventos
seguros, e ainda a ambi\-ção de se mostrarem mais do que são na realidade.
Tôda a pessoa tem necessidade de fazer mensalmente o seu or\-ça\-men\-to
para que as despezas nunca ultrapassem as receitas, nem sequer as
igualem; antes fiquem sempre aquém prevenindo as contingências do dia
de a\-ma\-nhã. Que felicidade sente todo a\-quê\-le que, com todos os compromissos solvidos no último dia de cada mês, pode proclamar bem alto
não dever nem um centavo a quem quer que seja! É uma elementar questão de senso administrativo.

-- Há ainda uma casta de pessoas que tiram às necessidades do seu
estômago ou do seu con\-fôr\-to para satisfazerem as necessidades do luxo
supérfluo, acrescentou Linda.

-- Essas criaturas são simplesmente abjectas. Sacrifi\-cam a quantidade e a qualidade da comida, que é o mesmo que dizer a saúde do
seu organismo, para me\-lhor poderem esticar o or\-ça\-men\-to em favor dum
vestuário mais espaventoso da moda, duma jóia mais rica, duma assinatura nos teatros, de
tôdas as manifesta\-ções mundanas, enfim. Regateiam o prêço dos alimentos que não importa serem de inferior qualidade porque se não vêem
depois de ingeridos; mas não regateiam o prêço da sua pompa vistosa.
Todos podem andar decentes a-dentro das suas posses. Todos podem ter
as comodidades que o seu or\-ça\-men\-to comporte. Podem e devem. O dinheiro só tem valor quan\-do convertido em con\-fôr\-to para nós. Amontoado,
sem utilidade, é a avareza. Repugna tanto à minha consciência des\-per\-di\-çar
inutilmente um tostão, como faltar com o necessário à saúde física ou ao
bem estar moral.

-- O gastador estroina e o avaro são por igual aberra\-ções da natureza humana, prosseguiu Eugé\-nio. Um e outro são miseráveis. A ambi\-ção é própria do homem,
e traduz uma qualidade salutar quan\-do dentro dos devidos limites. Um pouco
de ambi\-ção material e social, ou seja o desejo de me\-lhorar a situa\-ção financeira e de se elevar perante a sociedade, um pouco de ambi\-ção moral ou
o predomínio pelo carácter, são próprias do homem activo e honesto. Os
avaros abundam em ambi\-ção que aos estroinas falta. Uns e outros são mais
que inúteis, são prejudiciais. O económico que se não priva do indispensável é o verdadeiro equilíbrio. Em ma\-té\-ria de ambi\-ção, ou avidez, ainda
todos aqueles que nunca estão satisfeitos com aquilo que possuem e pretendem
mais, sempre mais; lan\-çam mão de todos os meios, desde o conflito da
rua até ao do tribunal, em demanda de novos direitos como se o mundo lhes
não chegasse. Confundem-se com os avarentos. Outros ávidos, que reputo de equilibrados, e a que poderemos chamar conservadores, limitam pelo contrário a
sua ambi\-ção a conservarem aquilo que de direito lhes pertence e a pouco
mais, condicionado pelas necessidades da vida, respeitando sempre os direitos dos outros. Para que serve o dinheiro além dum certo limite? Eu se
possuísse o supérfluo destiná-lo-ia a obras de caridade. Construiria um
hospital modelar, com o seu corpo clínico e de enfermagem rigorosamente
selecionados, exclusivamente destinado á indigência, onde não seria regateado o con\-fôr\-to material e moral possíveis. Não disse, há pouco, que o
dinheiro não tinha valor se não quan\-do trocado pelo con\-fôr\-to do possessor?
Não havia me\-lhor con\-fôr\-to moral para mim, me\-lhor consola\-ção para a minha
consciência, que fazer bem a êsses desgraçados que nem na doença ou nos
últimos dias de vida, teem o necessário para mitigar a sua miséria? Serviços absolutamente gratuitos, repito-o, com expulsão imediata de qualquer empregado que recebesse a mínima espórtula, para que o serviço prestado resultasse mais desinteressado e equitativo.

-- Que bom cora\-ção o seu, replicou Linda. Mas porque se não lembra de
preferência dos tuberculosos?

-- A ideia exposta não exclue a sua. Para êsses, um pavi\-lhão especial.
Sabe porque não falo de muitas outras obras de caridade, e só desta? Porque um dos factos que mais tem ferido a minha sensibilidade é o duma doente internada numa enfermaria do Hospital, a que tive de prestar serviço
como estudante de medicina. Entrara para ali na véspera, ardente em febre pela
casa dos 40, gemendo com dores no peito abaulado pela acumula\-ção de
pus na pleura e atroz falta de ar. Quando transpunha o limiar da enfermaria para a observar, a enfermeira avan\-ça para mim com longos protestos. A
doente não a deixara dormir toda a noite. Tão impertinente criatura nunca
\-lhe fôra dado ter de aturar. Acalmei o fervente mau humor alegando o sacerdócio que a sua missão constituia sobretudo em face de doentes que, como
este, estava em deplorável estado de saúde. Momentos depois abordava a doente que  se apressou a desabafar: -- Que a misericórdia tinha sido ímpia para
si, porquanto toda a interminável noite, ardendo em febre, pedira um gole de
água que lhe mitigasse a sêde, obtendo por resposta única e sistemática a
ordem de se calar e deixar dormir o pessoal; que a manifesta desigualdade
com que era tratada provinha de não dispor duma pequena lembran\-ça que pudesse dar ao pessoal, visto ser extremamente pobre. Não acha horroroso êsse quadro, que aliás tantas vezes se reproduz? Creia que me impressionou
deveras, e não me sai mais da mente. Quem sabe se esta criatura seria uma
dessas muitas que não teem um beiral a abrigá-las dos horrores do inverno
e um bocado de pão a suavizar a fome! Que ao menos na doença tenha um pouco de carinho e con\-fôr\-to que lhe ajude a suportar a dôr. Não acha que o
deleite de bem-fazer aos desprotegidos da sorte é um bom juro para o capital empatado numa obra de assistência? Pois era êsse juro que eu ambicionaria para os meus capitais supérfluos se fosse rico.

%marginpar{pdf 56}

Noutros dias divagam sôbre o método no arranjo das coisas. Eugé\-nio no exercício da sua profissão entra nas casas mais dispares. Umas,
pobrezinhas, mas onde uns modestos solitários de flores, bem arranjados,
dão logo alegria à vivenda, assim como a limpeza, a órdem na disposi\-ção
das coisas, o método no arranjo dos móveis dão a me\-lhor disposi\-ção a
quem entra. Noutras casas, muitas vezes ricas, o desalinho de tudo aquilo é flagrante à mistura com o pó e as teias de aranha; não há um
quadro por modesto que seja, ou umas florinhas simples, a emblezar um
pouco aquela habita\-ção duma aridez que choca; não há senso estético
na arruma\-ção, não há uma mu\-lher que saiba prender ao lar o marido
com essas futilidades que encantam e atraiem e são o repouso do espírito de quem chega a casa cansado de um dia de traba\-lho. Péssimas
escolas para os fi\-lhos, estas casas causam o tédio do marido que come
apressadamente as suas refei\-ções para se refugiar nos cafés.

Assim conversavam horas e horas, Eugé\-nio e Linda, no alegre con\-ví\-vio de duas almas sãs, de dois amigos que tão bem se compreendiam.
A doentinha esquecia-se do seu terrível mal. Sentia-se mesmo bem. E
só quan\-do Eugé\-nio se retirava ela podia concentrar-se no estado precário da sua saúde. Chorava então muito, e os pais queixavam-se da
irritabilidade do seu carácter progressivamente agravada com a
doença.

-- Tenham paciência, respondia Eugé\-nio. Façam por não a contrariar.
Lembrem-se de que é uma doente. Mesmo pelo que respeita à alimenta\-ção
ela prometeu-me comer mais, e tenho a certeza de que cumprirá. Pela
persuasão consegue-se dela tudo sem que se irrite. Não vêem o que acontece com certas prescri\-ções que lhe tenho feito? Bem sei que a contrariam, mas consigo que ela me obedeça sem constrangimento. Eu sei
convencê-la.

Linda obedecia cègamente ao seu médico, de facto. Nunca no decurso das suas conversa\-ções houve motivo de desacôrdo, quanto mais no que
\-lhe era prescrito como paliativo para a sua doença. Entretanto Eugé\-nio sentia-se feliz por as suas divaga\-ções serem compreendidas por
Linda, que só interrompia para elucidar amiúde: -- Princesa é exactamente assim, como o Sr.\ Dr.\ concebe a mu\-lher. Parece que está a retratá-la.

Eugé\-nio estava mesmo já convencido do conhecimento de causa com
que Linda lhe falava insistentemente de Princesa. E não se enganava.

Um receptor de rádio berrava durante todo o dia, altissonante, discos já
estafados. Que indesejável praga para um indivíduo normal, quanto mais
para um doente! Logo se combinou ali a maneira de desem\-baraçar Linda de
importuna vizinhan\-ça.

Há gente que possue rádio não para se deliciar com a música, mas para
incomodar os vizinhos. A música impregna o ambiente de beleza que se repercute benefi\-camente em todo o nosso sêr; à mesa facilita a digestão; no
traba\-lho lima a aspreza; em todas as emergências proporciona bem-estar e
eleva-nos. Mas é música digna dêsse nome. O rádio veio infelizmente desvirtuar o senso musical. O piano raro se ouve já nos seus acordes maravi\-lhosamente delicados, como delicado é o sentimento feminino que o faz vibrar.

\begin{center}\bf\large Prosseguem as investiga\-ções\end{center}

A vasta polícia de investiga\-ções continuava, entretanto, em actividade. Dir-se-ia seu chefe o Sr.\ Mendes, pessoa ponderada, gozando de
generalizada considera\-ção na cidade onde era muito relacionado. Estava ligado pelo ofício à alta finan\-ça, e era primo de Eugé\-nio. Homem
já pesado na idade, tinha pelo seu primo elevada considera\-ção a julgar pelas pessoas do seu con\-ví\-vio a quem falava de Eugé\-nio em termos
verdadeiramente lisonjeiros. Não eram poucas as vezes que ao fazer
a apresenta\-ção do primo a algum amigo, êste ripostava logo: -- já o conheço de tradi\-ção, de tanto que o seu primo me tem falado de si; e
apressava-se a documentar a sua afirma\-ção com algumas honrosas passagens da biografia de Eugé\-nio, de entre as quais avultava o facto
de dever a si, ao seu esfôrço, tudo quanto é na sociedade.

O Sr.\ Mendes estava em contacto directo ou indirecto com numerosas fa\-mí\-lias da cidade sôbre o assunto do interêsse de Eugé\-nio. Era
negócio sôbretudo de senhoras. As informa\-ções co\-lhidas é que eram
frequentemente muito diferentes, opostas mesmo, inverosímeis muitas
vezes. O que se passava a-dentro daquela habita\-ção da fa\-mí\-lia Frazão
continuava sempre o mesmo enigma. Umas vezes o Sr.\ Mendes estava habilitado a informar Eugé\-nio que Princesa tinha já o seu amoroso compromisso. Outras vezes dizia que as cartas de Eugé\-nio não deviam ter
chegado ao destino, porque o ambiente disciplinar, quási conventual,
daquela casa, levava os pais a fazer cerrada selec\-ção da correspondên\-cia recebida, tendo como certo ser sequestrada às fi\-lhas tôda a carta
com o carácter das de Eugé\-nio.

Deve dizer-se em abono da verdade que o Sr.\ Mendes não procedia
desde há muito nas suas investiga\-ções por sua espontânea vontade, mas
sim para dar satisfa\-ção a Eugé\-nio. Começara, sim, por ver bem a aproxima\-ção dêste com Princesa, mas dada a sequência que os factos levavam,
dizia a todos os que o ouviam: -- É uma pena o que se passa com Eugé\-nio. Aquêle rapaz, com o valor que tem e os predicados que o ca\-ra\-cte\-ri\-zam, podia fazer um bom casamento, como merece, e para isso não lhe
faltava com quem. Mas embeiçou-se por aquela rapariga e não vê mais
nenhuma. Nunca vi coisa assim. O tempo vai-se passando, os anos vão
rodando, e Eugé\-nio não há meio de mudar de ideias. E é minha convic\-ção que ela lhe conhece o fraco e faz pouco dêle. Pois a dar-se comigo muito menos daquilo que se tem dado com êle, eu odiaria profundamente aquela rapariga. Acresce que ela não é digna dêle, como a princípio se me afigurava. As investiga\-ções a que tenho procedido atestam que êle está
muito enganado. Mas não há nin\-guém que o convença da realidade das coisas. Ninguém!

Às vezes o Sr.\ Mendes agarrava Eugé\-nio num domingo e conversavam
muito ambos tôda a tarde. A conversa era necessariamente derivada para Princesa.

-- Ela não é nada do que julgas, começava por atacar o Sr.\ Mendes.
Ela é uma trocista que se ri do teu fraco. Sabe que te faz andar a cabêça à roda e explora-te. Como todas as raparigas, é vaidosa e gosta
de te ver a persegui-la.

-- Perdão! -- obtemperou Eugé\-nio: Nunca lhe notei o menor vislumbre
de vaidade, troça ou incorrec\-ção; antes pelo contrário. Isso não é verdade!

-- São as próprias amigas, companheiras de Princesa, que o afirmam.
E afirmam mais: -- Que, ao contrário do que pensas, ela pinta-se. Que
sabe muito bem fazê-lo, por forma a enganar o mais esperto, dizem-no
as que consigo privam.

-- É uma dessas calúnias com que as 'amiguinhas' se entreteem.
Não me admiram êsses processos desleais de quem, por não chegar aos
seus calcanhares, pretende mordê-la. É uma autêntica falsidade isso que lhe atribuem.

E o Sr.\ Mendes perante a energia posta nas palavras de Eugé\-nio
não insistia nêsse ponto, procurando investir por outro lado:

-- Mas ela troça de ti, afirma-se.

-- Não acredito. Poderia fazê-lo uma, duas, o máximo três vezes, sem
que eu desse por isso. São porém decorridos anos desde que a conhêço,
e nunca vi o menor indício de troça. Nunca! Tenho mesmo dela um con\-ceito
suficientemente elevado para não poder acreditar na mínima incorrec\-ção da sua parte.

-- Pois o\-lha, dizia o Sr.\ Mendes: bastava para mim que uma rapariga me fizesse andar a cabêça à roda uma centésima parte do que te tem
feito Princesa, para eu a odiar, desde que não me correspondesse.

-- Quem me dera poder manejar êsse sentimento, o ódio, a meu
belo prazer! Se eu conseguisse sentir ódio por aquela rapariga, estava resolvido o maior problema da minha vida. Tudo tenho conseguido
quanto tenho querido. Tudo! E fui sempre feliz em todos os empreendimentos. Só em fazer-me compreendido por Princesa é que não. Foi êsse o meu primeiro fracasso. Conseguir esquecê-la seria o meu maior
desejo, mas é impossível. Medicamento heroico seria o ódio, êsse sentimento tão próprio do homem, que em maior dose o leva às mais vis e torpes ac\-ções, e em dose moderada ao simples desprêzo da pessoa de quem
há qualquer ressentimento. Mas eu nunca senti ódio por nin\-guém;
nunca soube o que isso é senão pelo que vejo nos outros. Se pudesse
criar em mim tal sentimento, criá-lo-ia sim para esta emergência como único remédio que se me anto\-lha para o mal que me acabrunha. Mas
é coisa que não sinto, que não há forma de poder sentir, por mais que
o procure. Tenho tentado representá-lo no meu cérebro, tenho procurado convencer-me que a odeio, simplesmente para a esque\-cer. Em vão!
É tudo fictício! Basta o\-lhá-la e tudo se desvanece como por encanto.

-- Isso só constitue mais uma prova do teu carácter, dos bons
sentimentos que o emolduram, e da coerência, da notável constância
que costumas pôr em todos os teus actos, e da extraordinária persistência com que procuras realizá-los. Mas Princesa não te compreende.
Não é digna de ti. E é pena -- tenho-o dito a tantas pessoas -- que tu
não reconheças a tua superioridade real para te convenceres de que me\-lhor mereces.

-- Me\-lhor não mereço, porque não o concebo. Princesa sintetiza o
meu pensamento.

\centerline{\rule[-1em]{0pt}{2.5em}:-:-:-:-:-:}

\vskip 1em %JNO

Um dia o Sr.\ Mendes reproduziu a Eugé\-nio uma conversa que Princesa tivera com uma senhora amiga da casa, antiga condiscípula da mãi
num colégio que frequentaram em pequenas:

-- Eu pretendo, disse Princesa, um rapaz honesto, activo, traba\-lhador, modesto, económico e inteligente. Não me preocupa a fortuna. Naquelas qualidades encontro o marido que idealizo e que julgo digno
de mim.

-- Acho esquisito que pensando tu assim queiras deixar fugir
a\-quê\-le que possue todos êsses requisitos e mais uma enorme simpatia por ti. São êsses precisamente os predicados de Eugé\-nio. Basta
dizer-te que se fez exclusivamente à sua custa, sem auxílios estranhos,
pondo em ac\-ção uma notável vontade para se elevar até onde quis, alcan\-çando um diploma dos mais belos na sociedade, como o é o de médico, à
custa de muito traba\-lho honesto, de muita economia e inteligência. Quem
o vê não dirá o extraordinário valor da\-quê\-le rapaz que a modéstia encobre para os que não o conhecem, mas realça para os que o conhecem.
Pois o\-lha que não arranjas quem mais digno seja de ti. Digo-to como
tua amiga que sou, e como amiga de infância de tua mãi.

Todos êstes informes eram ouvidos religiosamente por Eugé\-nio.
Não era o juízo lisonjeiro que de si faziam, e de que tantos testemunhos
recebia amiúde, que tanto lhe interessavam. Calava profundamente no
seu espírito, é certo, mas estava muito habituado a criar em todas as
pessoas do seu con\-ví\-vio outros tantos amigos pelas terras que pisara.
Mas tôdas as conversas sôbre Princesa tinham para êle a sonância especial de conversas queridas. Na impossibilidade de lhe falar, só estava
bem a falar ou a ouvir falar sôbre ela.

O ambiente de que há muito rodeavam Eugé\-nio em presença de Princesa não lhe era indiferente a êle. Era a revivescência da esperan\-ça.
Pois tão lisonjeiramente conhecido dela, como o devia ser já através
de várias amigas, aquela sua indiferença deveria manter-se indefinidamente, teimosamente?

Acontece um dia passar por um antigo condiscípulo que desfecha a
Eugé\-nio a seguinte observa\-ção: -- Então essas cartas?

-- Que cartas? -- Preguntou Eugé\-nio intrigado.

-- Eu sei de tudo; não te faças de novas. Para aquela rapariga
que bem sabes...

Já pelo espírito de Eugé\-nio passara desde a primeira pregunta
do condiscípulo a intui\-ção daquela a quem êste se queria referir. Mas
como podia êle saber o passado?

Não sossegou enquanto não procurou o condiscípulo. Este sabia
de tudo através das suas irmãs, amigas de Princesa. E narrou-\-lhe que
esta não correspondia à afei\-ção de Eugé\-nio por simpatizar com um rapaz que conhecêra num baile realizado em sua casa no último carnaval.

Eugé\-nio ao ouvir estas palavras sentiu o chão faltar-\-lhe sob os pés. Uma punhalada que lhe atravessasse o cora\-ção não
o teria ferido mais profundamente na sua sensibilidade. Ficou exausto. Muito a custo pôde concentrar as forças necessárias para replicar ao condiscípulo nu\-ma voz anelante:

-- Mas não tenho visto nada de suspeito nas rondagens feitas
às suas imedia\-ções.

-- Não sei. Só sei que é verdade o que te digo, confirmou o condiscípulo.

A dúvida aumentava em Eugé\-nio. Sempre a eterna e cruel dúvida
àcêrca de Princesa. Se era certo o que o condiscípulo lhe dizia, impu\-\mbox{nha-se} o esquecimento. Mas estava já tão habituado a notícias falsas...
O certo é que nin\-guém se lhe tornára suspeito nas vizinhan\-ças
da residên\-cia de Princesa. Não gostou entretanto Eugé\-nio que as suas
cartas andassem pelas bôcas do mundo. Princesa recebera-as, não as devolvêra, e propalava-as.

\centerline{\rule[-1em]{0pt}{2.5em}:-:-:-:-:-:-:}

\begin{center}\bf\large Em procura do esquecimento\end{center}

Um só remédio se afigurava heroico para esque\-cer Princesa como
desejava, em face da sua tão indefinida atitude. Era encontrar alguém
que conseguisse atrair a sua simpatia. A afei\-ção viria depois, a ver
se poderia substituir-se no espírito de Eugé\-nio ao lugar ocupado por
Princesa. Era um mal necessário de que Eugé\-nio teria de lan\-çar mão
quanto as circunstâncias lho impunham. Seria a única forma de esque\-cer Princesa, uma forte afei\-ção que o prendesse a uma outra rapariga.
Tarefa difícil, muito difícil mesmo, mas que era necessário tentar.
Impu\-\mbox{nham-no} o brio, o bem senso e a saúde de Eugé\-nio.

Uma vez encontrou num "eléctrico"\ uma rapariga cujo perfil
\-lhe lembrou o de Princesa. Era realmente interessante; não tinha os
o\-lhos de Princesa, mas tinha o mesmo cabêlo e o rôsto era parecido. De
perfil, lembrava Princesa. Encontrou-a mais algumas vezes. Viera passar uma temporada à cidade, ainda para maior coincidên\-cia, a uma casa
de fa\-mí\-lia amiga e vizinha de Princesa.

Quem era ela? Era o que faltava saber. Para isso Eugé\-nio encarregou um colega da\-quê\-les lados, o Baptista, dando-\-lhe todos os elementos
necessários à identifi\-ca\-ção da rapariga, e começou a aparecer às noites
em casa dêle para tomar conhecimento do que o amigo havia apurado.

-- Pela descri\-ção que me fez, reconheci hoje a rapariga em que me
falou, disse Baptista uma noite. O retrato dela que você me pintou é
exacto. Felicito-o pelos seus bons gôstos. Ia eu com minha mu\-lher,
e ambos a achámos muito interessante. Ainda não pude saber, porém, nada a seu
respeito.

Certa noite resolveram os dois, Baptista e Eugé\-nio, darem uma volta por perto. A atmosfera estava calma e apetecia um pouco de ar livre.
Passavam próximo da casa de Princesa, quan\-do Baptista observa a Eugé\-nio: -- Você parece que, pelos modos, tem especial predilec\-ção por êstes sítios. Constou-me há tempos que andou por aqui muito prêso das suas afei\-ções...

-- Quem lho disse? -- Preguntou Eugé\-nio surpreendido.

-- Várias pessoas, afirmou Baptista. É mesmo voz corrente. Em conversas
que tenho tido aqui nas vizinhan\-ças teem-me preguntado, por vezes, se
eu conheço um colega meu que gosta muito duma das fi\-lhas do Sr.\ Frasão. E indicam-me o seu nome.

-- Eugé\-nio não pôde negar, manifestando ao amigo o seu desgôsto
por andar tão espa\-lhado o facto que tanto dominou e domina ainda o
seu espírito. E acabou por lhe confessar que o maior empenho que tem
na rapariga de quem lhe falou é precisamente a necessidade de esque\-cer
a ideia fixa da tal fi\-lha do Sr.\ Frazão.

As noites sucediam-se e o Baptista nada apurava de interêsse.
Eugé\-nio para lá caminhava sempre. Frequentava também a casa de Baptista uma sua parente e vizinha para quem Eugé\-nio começou a notar não
ser indiferente a sua presença.

-- É uma rapariga de muito bons sentimentos e uma rica herdeira,
disse Baptista uma vez a Eugé\-nio, referindo-se à sua parente. Comparo
muito os sentimentos dela com os seus. Habita um palacete aqui próximo.
Fala-me muito de si com simpatia.

Eugé\-nio começava a ver deturpada a finalidade das suas visitas.
O Baptista afinal nada adiantava sôbre a rapariga de quem fi\-cara de
co\-lher informes, e parecia antes interessado em estreitar as rela\-ções
de Eugé\-nio com a parente. Eugé\-nio vendo o caminhar das coisas e a trai\-ção da armadi\-lha, começou
por rarear discretamente as visitas e acabou por lá não pôr os pés.

-- Você está zangado comigo? -- Preguntou um dia o colega a Eugé\-nio
ao encontrarem-se na rua. Apareça. A minha parente vai a minha casa
tôdas as noites, e pregunta muito por si. Não sei que mal lhe fizemos
para deixar de aparecer.

-- Afazeres, pretextou Eugé\-nio.

Parece que quanto mais Eugé\-nio procurava evitá-los mais os encontros ocorriam; passado pouco tempo os dois colegas quási que se
esbarravam de novo.

-- Você arranjou-ma bonita! -- Exclamou Baptista. -- A minha parente
não me larga a casa a preguntar por si e eu sou quem tenho de a aturar. Quere você ir no próximo domingo a um gran\-de pic-nic para que
estamos convidados nos arredores duma cidade minhota? Deve ser um
sucesso! Nós vamos todos, e você podia acompanhar-nos.

-- Impossível. Já tenho um compromisso para êsse dia, disse Eugé\-nio.

Entretanto a rapariga de perfil parecido com o de Princesa não fôra
mais vista por Eugé\-nio. A simpatia por Princesa ainda desta vez não
sofrêra interrup\-ção. O que continuava a dar que pensar a Eugé\-nio era
a sua atitude indecifrável e ainda o facto de propalar o caso. Já através de duas origens tinha Eugé\-nio conhecimento de que Princesa espa\-lhava o passado entre ambos. E tal procedimento não podia deixar de
ser considerado por êle como um verdadeiro sacrilégio atendendo à
pureza da afei\-ção que lhe dispensa.

Resolveu escrever a Princesa novamente, mas desta vez por forma
mais incisiva, e endereçou-\-lhe a seguinte carta:

"Persiste V. Exª. em continuar indiferente às minhas cartas
sucessivas. Nestas, repassadas de tôda a sinceridade nas suas palavras, procurei traduzir a elevada simpatia e admira\-ção pela sua pessoa.

"Perante tal silêncio sou levado a presumir que não sou correspondido. Resta-me nês\-te caso procurar esque\-cer o passado -- tarefa difícil pelo muito que indelévelmente se gravou de V.Exª.\ no meu espírito,
mas que hei-de conseguir com o esfôrço que se tornar necessário dispender.

"Sendo-\-lhe eu indiferente, as minhas cartas em seu poder nada podem interessar a V.Exª., e com a sua devolu\-ção encontrará meio, desde
que assim o pretenda, de não tornar a ser importunada por quem se deixou prender bem preso pelos sedutores predicados seus. Embora eu nada tenha a arrepender-me do que lhe declarei, porque é a ex\-pres\-são
do que sinto, eu pretendo, ainda que muito a custo, esque\-cer tudo isto,
no caso, repito, de V.Exª, não querer corresponder à afei\-ção que me prende à sua pessoa.

"E agora um aparte antes de prosseguir: Tôdas as cartas escritas
estão tornadas públicas ao máximo, decerto por V.Ex.\ e interpostas
pessoas. Pouco importa. As suas palavras dizem o que sinto; e, embora
mal redigidas e sem pretensiosismos, foram bem inteligíveis. Os numerosos juizes que as julgaram, se são inteligentes, tê-las-ão compreendido; e se teem sensibilidade não terão aconse\-lhado o procedimento
que V. Exª. tem tido para comigo que eu suponho não ter feito por merecer.
Mas o que é certo é que nem as cartas, nem a minha insistência em manifestar-\-lhe por outras formas a minha simpatia, são de molde a justifi\-car qualquer espécie de rèclamo, ou a estimular quaisquer vislumbres de vaidade, atendendo à sua muito modesta procedên\-cia que só é
elevada na admira\-ção que por V. Exª. nutre, e atendendo ao facto de
que a vaidade é sentimento que nunca lhe conheci e mesmo julgo incapaz de se arraigar num espírito bem formado como o de V. Exª.

"Prosseguindo no exposto, que é a razão de ser desta, julgo, enfim, de
ser tempo de V.Exª.\ tomar uma resolu\-ção definitiva e inequívoca pela
qual anseio, aceitando os testemunhos de afei\-ção manifestada ou desfazendo-se das minhas cartas.

"Nesta última conjuntura não a tornarei a importunar em parte alguma e de nenhuma forma, e de qualquer pequena incorrec\-ção que porventura tenha involuntariamente cometido (de que a consciência aliás me não acusa)
me penitencio.

``Termino desejando-\-lhe tantas felicidades como para mim próprio e
subscrevendo-me com a máxima considera\-ção."

\centerline{\rule[-1em]{0pt}{2.5em}:-:-:-:-:-:}

Princesa leu a carta e devolveu-a. E desculpou-se de não juntar as
cartas anteriormente recebidas por motivo de as ter inutilizado. A carta
de Princesa era sêca de laconismo e dactilografada. Subscrevia-a uma
assinatura intencionalmente ilegível.

Eugé\-nio ao recebê-la só disse para consigo: -- Pronto! O caso está
definitivamente arrumado. A redac\-ção da carta enviada a Princesa foi
em moldes tais que dava categoricamente por terminada tôda a espécie de
rela\-ções de Eugé\-nio desde que aquela persistisse em não responder afirmativamente.

Procuraria no traba\-lho o esquecimento do passado, a mitiga\-ção da dôr
profunda que a resposta da carta lhe trouxera. No traba\-lho encontraria
o isolamento do mundo necessário ao seu estado de espírito, talqualmente
o que professa o vai encontrar no convento. Mas, numa cruel contradi\-ção
com seus pensamentos, interrogava-se a si próprio: Traba\-lhar, para quê,
se não tinha com quem parti\-lhar o objectivo ou a alegria do traba\-lho? De
que lhe valem as canseiras sem existir como estímulo o amor que conduza
a um fim útil para a vida -- a constitui\-ção e a solidifi\-ca\-ção dum
lar?

Eugé\-nio sofrera como que uma intensa descarga nervosa ao ler a resposta, que o prostrou esgotado. A esperan\-ça que lhe sorrira antes de a abrir,
transformara-se agora numa desilusão superior às suas forças que o deixou inerte.

Ao mesmo tempo revoltava-se consigo próprio. Que fraqueza esta, a
minha! -- exclamava. Porque não hei-de reagir contra êste marasmo que me
punge? Que culpas estarei expiando, e que não conheço, para que a natureza se mostre tão cruel para comigo? Não poderei encontrar outra mu\-lher que mereça a minha afei\-ção?

O sol tornara-se dum verme\-lho baço, e espa\-lhava sobre a terra uma
luminosidade prenunciadora de tempestade. Nuvens translúcidas apareciam
ao longe no horizonte, dispersas, caminhando em direc\-ção a nós para se juntarem a outras mais escuras que, cada vez mais encobriam o azul do céu.
A atmosfera tornou-se irrespirável. De-repente o vento começou a fazer
rodopiar a poeira do chão, e a escuridão deu ao ambiente uma tonalidade
medonha de trovada que se avizinha.

Oh meu Deus! continuava meditando Eugé\-nio; até a natureza se associa
á tristeza da minha alma, como que a confirmar a pureza da minha afei\-ção
incompreendida! E que singular coincidên\-cia de em momentos análogos de
decep\-ção que Princesa lhe tem feito sofrer, a natureza cobrir com o manto
da sua tristeza a tristeza íntima de Eugé\-nio!

A natureza assim associada aos tristes pensamentos de Eugé\-nio, mais
convidava à lúgubre sequência das suas medita\-ções. Tão desprotegido
se via pela felicidade, que aos outros sorri! Não seria merecedor duma
alma como a de Princesa, a única que tanto o tem feito sofrer e a única
que constitue para si o precioso relicário de todos os seus afectos?
Seria antes providencial a sua sistemática evasiva, pressupondo a tortura
que lhe daria um dia a morte ao ter de a abandonar, seme\-lhante à indiferença com que partiria para o Além sem assistência afectiva da mu\-lher
que ama?

Dando assim pasto a sentimentos que reflectiam a depressão que o
avassalava, Eugé\-nio pensava em fugir para longe, para muito longe, para
um além-mar muito distante, isolado de tudo e de todos, onde nunca mais visse nem ouvisse falar do objecto da sua paixão.

Por outro lado procurava, como em idênticas conjunturas anteriores,
imaginar defeitos físicos ou morais em Princesa que o conduzissem a poder bradar para consigo, como lenitivo para a sua dôr: Que loucura a minha! Enganei-me. Não era êste o ideal que sonhei. Mas a imagem de Princesa aparecia-\-lhe imediatamente, ainda mais bela e impoluta aos seus o\-lhos
como que a chamá-lo á realidade que só uma forçada imagina\-ção podia profanar.

Eugé\-nio era uma destas naturezas que, no meio da sua concentra\-ção,
necessitam dum amigo íntimo para desabafar as suas tristezas como também
não resistem a descarregar as suas alegrias. Nêsse confidente só, Eugé\-nio
encontrava como que a válvula de seguran\-ça prestes a descarregar a demasia
dos sentimentos que lhe assaltavam a alma, como uma caldeira a alta
pressão necessita aliviar um pouco de vapor.

O confidente era António Ferreira. Com ele se abria totalmente, contando-\-lhe sempre que tal se lhe oferecia, um o\-lhar mais demorado e esperançoso que conquistara a Princesa, e todas as mais pueris demonstra\-ções
de não indiferença que o acaso dum encontro proporcionara. E desta alegria
parti\-lhava certamente o amigo.

Desta vez era a tristeza, o tédio, que adquiria alta pressão no ânimo de Eugé\-nio.

O António Ferreira, ao saber do inglório epílogo de tantos anos de
intensa bata\-lha, logo veio com as palavras costumadas:

-- Vira-te para outro lado enquanto é tempo. Repito-te que mu\-lheres
como essa há muitas, e essa não é quem julgas. Tu é que só a vês de baixo
do prisma da tua obcessão, daquela obcessão de que estás possuído.

-- Nada disso. Os predicados de Princesa não são vulgares e nunca
poderei encontrar rapariga por quem tanto me sinta cativado.

-- O\-lha que isso em ti não é mais do que aquela qualidade que possues
e de que aliás te podes orgu\-lhar, da tenacidade com que pensas obstinadamente em levar a cabo todos os teus actos. E como estás acostumado a vencer sempre, é que estranhas. Nenhum rapaz teria a paciência que tu tens.

-- Não me interessam os outros, para quem a manifesta\-ção de simpatia
por uma rapariga é coisa geralmente banal. Ou porque a achem bonita, ou
por qualquer outra causa trivial, apressam-se a declarar a sua afei\-ção.
Se pega, pega; se não, pouco se importam. São os galanteadores profissionais.
Comigo o caso não é o mesmo, como tu sabes. Foi um conjunto considerável
de qualidades que me determinou a manifes\-\mbox{tar-me}, não por um impulso de
momento, mas pela sua radica\-ção no meu espírito tantos anos.

Eugé\-nio passava longas horas no seu quarto invadido pelo tédio. De
que lhe valia assim a vida? Uma vida só de traba\-lho, para quê? Tudo lhe
parecia mais soturno. Sentava-se á mêsa sem apetite e a comida custava
a transpor-\-lhe a garganta.

Depois sonhava muito. Sonhava umas vezes encontrar-se numa casa contígua á
de Princesa, ou no mesmo quintal dela a vê-la e a admirá-la no seu vai-vem.
Uma vez sonhou estar a conversar com ela e com a irmã, lado a lado, como
tres amigos antigos, no parapeito duma janela onde chegavam aromas inebriantes de flôres dum jardim próximo.

Doutra vez, após uma tarde em que os o\-lhos de Princesa se encon\-tra\-ram mais demoradamente com os seus, Eugé\-nio sonhou -- doce sonho de ilusões! — conversar com a sua amada. Estavam os dois na praia, sentados
sôbre a areia. O mar estava manso e calmo, e a superfície espe\-lhava os
últimos raios do sol prestes a esconder-se no ocaso. Miríades de crian\-ças brincavam ao seu redor, dando mais vida ao areal com as suas diabruras por entre a melopeia aguda de avesitas chilreantes. Princesa estava
mais bela do que nunca! Os cabelos escuros engrinaldavam-lhe a cabeça
tão bem moldada; a tez das suas faces era dum belíssimo veludo branco-rosado; os lábios de carmesim deixavam ver os dentes de uma alvura sem
rival e escapar um hálito de mil etéreos aromas; as formas superiormente esculturadas, a elegância do seu porte, mostravam a Eugé\-nio a superioridade de Princesa entre as mais belas do seu sexo, a sua incomparável pulcritude tornando-a ideal entre as ideais. Eugé\-nio sentia-se fascinado pela meiguice do seu o\-lhar e deleitado pelo seu encantador sorriso. Princesa escutava Eugé\-nio que lhe murmurava:

-- Amo-a com todo ardor. O longo tempo em que esta paixão se apoderou de mim é testemunho bastante da sua solidez. A dúvida em que tenho
estado durante anos de ser correspondido não esmoreceu em mim essa simpatia porque o meu espírito insiste em não encontrar guarida noutro cora\-ção que não o seu. A sua imagem não me abandona e a ela recorro nos
transes em que necessito dum apoio moral. \mbox{Apresenta-se-me} então como uma %JNO
visão sobrenatural estilizada de angélica beleza, que outra não é do que a
ex\-pres\-são sublime do natural com que a estou vendo, a amparar-me nos passos
difíceis.

Princesa dizia então, e as suas palavras produziam em Eugé\-nio o efeito de sons musicais de transcendental harmonia:

-- Compreendi o seu amor; é belo e sincero, e correspondo-\-lhe com a beleza e sinceridade do meu.

Eugé\-nio acordou. Tudo se havia esvaído, como por encanto, como uma
espirala de fumo se esvai nos ares; tudo se havia esvaído menos a adorável imagem de Princesa que lhe ilumina o cora\-ção, lhe vivifi\-ca o espírito,
o consola nas horas de tristeza e o fortalece nas horas de fraqueza.

Eugé\-nio teve desejos de morrer no lugar em que a vira em sonho. A
negra morte também deve ser boa quan\-do se é infeliz. E também nela se
deve conter felicidade porquanto no seu letal manto se teem emaranhado
muitas criaturas que na terra eram felizes, e muitas paixões, e muitos
belezas moças, e talvez muitas ilusões como a de Eugé\-nio. Esta efémera
vida é uma série de dramas em que o nosso cora\-ção é o mais das vezes o
protagonista. E se a vida é um estúpido contrassenso quan\-do se não ama,
porque o amor é uma predestina\-ção humana, {torna-se} horrivelmente estuosa %JNO
quan\-do se ama e se não é compreendido, \mbox{torna-se} aborrecida, insuportável,
neurasténica, quan\-do se teme desabafar naquela, porque o cora\-ção palpita,
essa explosão de amor apaixonado que há tantos anos em Eugé\-nio tumultua.

Uma tarde Eugé\-nio encontrou-se eventualmente na Foz com um amigo da
sua aldeia. Era um amigo de infância, que há bastante tempo não via. Abraçaram-se, e ainda mal refeitos estavam dos cumprimentos trocados, já o
amigo declinava a descoberta feita momentos antes da simpatia de Eugé\-nio
por Princesa, a qual lhe fôra mostrada.

-- Como sabes tu? Indagou com espanto Eugé\-nio.

-- Muito simplesmente. Tenho andado aqui a passear com as raparigas
Rochas que a conhecem e me contaram várias peripécias passadas entre ti
e ela. Elas mesmo já o fizeram constar na nossa aldeia, e agora acabam de
mostrar-me a tua apaixonada.

-- E como o sabem elas?

-- Da maneira mais natural. Contaram-me que o sabem por uma rapariga
amiga comum de ambas as fa\-mí\-lias, com quem uma vez te encontraste em sua
casa...

Era Ofélia a quem o amigo de Eugé\-nio se referia que, com a sua inconfidên\-cia, fizera já chegar a notícia até à aldeia daquele.

\centerline{\rule[-1em]{0pt}{2.5em}:-:-:-:-:-:-:-:}

A tensão de espírito de Eugé\-nio não se modifi\-cava de noite e
dia. De dia pensava sempre em Princesa. Ao mesmo tempo queria esquecê-la e não a tornar a ver, mas como que movido por estranha mola todos os seus passos se dirigiam maquinalmente para onde a pudesse encontrar e ver. Retirou-se da cidade uma temporada, mas em vão! Só no traba\-lho encontrava repouso para o seu espírito e nas horas vagas distraia-se escrevendo traba\-lhos para publicar.

Assim se passou o resto do verão e todo o inverno imediato.

\begin{center}\bf\large Zequinha e Toninho\end{center}

% (1936)
Aproximava-se com o verão seguinte, de 1936, a época balnear. Já Eugé\-nio
voltava a ver adiante de si as tormentas da vista de Princesa. Estava na sua
mão, é certo, não aparecer na praia em que era certo encontrá-la. Mas
sabia já que não resistia a ir até lá. Quem lhe dera revestir-se da
fôrça de vontade precisa para deixar de ir à praia! Não indo, com mais
probabilidades de êxito procuraria esque\-cer Princesa. Já sabia por
experiência própria -- e bem dura experiência! -- quanto lhe aumentava
o sofrimento a vista dela; mas também sabia quanto eram baldados
os seus esforços, a ajuizar pelos anos anteriores, em procurar resistir à frequência da praia.

Antes, deu-se uma cena que bastante impressionou Eugé\-nio. Entrou
num eléctrico que ia repleto de passageiros. Ficou na plataforma também repleta, e avistou, sentadas num banco, a mãi e irmã de Princesa. Logo que vagou um lugar de sentado, apressou-se a aproveitá-lo. Até aqui,
nada de extraordinário. Mas passados momentos, vagou um novo lugar e
adianta-se Princesa para o tomar. Donde surgiu ela? Sem dúvida, da
plataforma. Tinha passado desapercebida a Eugé\-nio por entre a gente
aglomerada naquele recinto. Pior ainda era a ideia de desconsidera\-ção
intencional que podia atravessar a mente de Princesa, a ver Eugé\-nio
adiantar-se para se sentar sem qualquer aten\-ção para com uma senhora.
Pois se êle nunca o faria a qualquer outra senhora, quanto mais a quem
tão especial considera\-ção lhe merecia? Revoltou-se contra o seu próprio procedimento, involuntário é verdade, mas denotando uma distra\-ção
que não estava no seu modo de ser. Pensou em dar-\-lhe uma explica\-ção.
Mas como, se nunca se atrevera a dirigir-se-\-lhe?... Podia dar a
explica\-ção uns meses depois ao Sr.\ Frazão, num jantar em que o acaso
levou a encontrar-se com Eugé\-nio. Ficaram ambos lado-a-lado, e conversaram bastante amigavelmente. O Sr.\ Frazão, que nos primeiros encontros
com Eugé\-nio desconhecia as ideias dêste a respeito da sua fi\-lha, já agora estava ciente do passado. Eugé\-nio não se atreveu a dar-\-lhe aquela
necessária explica\-ção, e o caso ficou assim mesmo bem contràriamente à sua vontade.

Desde há muito que dominava o pensamento de Eugé\-nio dirigir-se pessoalmente a Princesa. Pensara-o por vezes na praia, mas a timidez própria do
apaixonado embaraçava-\-lhe a ac\-ção. Avolumava-se porém a ideia, visto que
baldados tantos esforços para a tornar ciente da sua afei\-ção só restava a perspectiva de lhe falar.

Princesa frequentava uma li\-ção de lavores, nas proximidades de Eugé\-nio.
Êste via-a entrar e sair dali todas as semanas ás mesmas horas do mesmo
dia. Projectou abordá-la no seu caminho, entre a sua residên\-cia e a aula.
Mas como seria recebido? A sua timidez não lhe dava foros para encarar
friamente o caso de ela não o atender, alegando por exemplo não o conhecer...
No entanto, como que fortalecido de coragem, resolveu-se um dia a agir nêsse
sentido, e esperou Princesa. Infelizmente fa\-lhou o plano porque Princesa
não apareceu nesse dia á costumada li\-ção.

A felicidade não protegeu Eugé\-nio ainda desta vez, e as férias do
verão vinham pôr termo a todas as actividades lectivas.

Veio enfim a época balnear. Como nos anos anteriores, procurou
não ir à praia frequentada por Princesa. Procurou mas não o conseguiu.
Foi lá uma vez só a título de experiência, protestando não voltar. Viu-a e... continuou a ir todos os dias. Que prisão a de Eugé\-nio por
a\-quê\-les o\-lhos feiticeiros! Os o\-lhos de Princesa magnetizavam-no e
Eugé\-nio contentava-se só com vê-la, o\-lhá-la, admirá-la. Todos os seus esforços no sentido de opôr a vontade ao desejo resultavam inúteis.

Um incidente estava reservado agora para o acordar da letárgica
abulia. Por meados da época corrente começou a ver alguém em conversa
amena com Princesa. O caso repetiu-se alguns dias. Nunca a vira com
rapaz algum. Mesmo êsses banais e frequentes galanteios observados
nas praias, ou fora, na cidade, da massa anónima para as raparigas que passam, nunca Eugé\-nio observou que algum se dirigisse a Princesa. Sempre o seu aspecto
ponderado, senhoril, de sempre, e nunca Eugé\-nio observou, em ocasião alguma, que qualquer rapaz se lhe
dirigisse a não ser para trocar um rápido e cerimonioso cumprimento
e nada mais. Até nisso Princesa se distinguia de todas as mais raparigas, realçando aos o\-lhos de Eugé\-nio a sua sensatez como única. Basta
dizer-se que é rapariga que nunca nin\-guém viu às janelas da sua casa.
O facto agora observado por Eugé\-nio tornava-se por isso mais suspeito, e a suspei\-ção tinha foros de verdade ao ligar êste facto com a informa\-ção que o condiscípulo lhe trouxera no ano anterior.

Só tinha um caminho a seguir. Era imperioso abandonar a frequência daquela praia que lhe dilacerava cada vez mais impiamente o cora\-ção. E abandonou-a discretamente.

Mas a eterna dúvida perseguia Eugé\-nio. Todo o inverno seguinte
rondou as imedia\-ções de Princesa e nada de suspeito observou. Que se
passaria? Haver-se-ia desfeito a névoa que assombrava o espírito de
Eugé\-nio? Era uma hipótese.

As informa\-ções obtidas pelo Sr.\ Mendes eram escassas, quan\-do não
opostas, e os seus conse\-lhos afinavam pelos do António Ferreira.

-- Que feitiço que essa rapariga te lançou, que não há argumentos,
nem factos, nem nada, que te convençam a abandonar a ideia que em tão
triste momento nela puseste! -- observavam os dois amigos confidentes de
Eugé\-nio. E prosseguiam: que feitiço êsse que te domina a fôrça
de vontade que sempre mostraste em todos os demais actos da tua vida!
Que supremacia ela tomou, sôbre ti, que tens muito mais valor que cem
raparigas como essa. Ainda se ela fôsse bonita...

-- Para mim é-o, replicava Eugé\-nio. Que me importa que nin\-guém lhe
encontre a beleza que eu encontro, se é para mim que escô\-lho?... Atenta
naquela can\-ção de Manuel Monteiro que diz:

\begin{quote}\small
Há quem procure saber		\\
Qual é a mu\-lher mais bela	\\
De formas esculturais.		\\
A mais formosa mu\-lher,	\\
A mais linda, é sempre aquela	\\
De quem nós gostamos mais.
\end{quote}

E o mesmo cenário de sempre se anto\-lhava a Eugé\-nio. Se a encontrava por acaso na cidade, parecia paralizar-se-\-lhe toda a vida
um momento. O cora\-ção batia mais, as pernas tremiam-\-lhe, a voz embarga\-\mbox{va-se-lhe} se estava conversando com alguém. Como sempre, umas vezes ela
fitava-o, outras seguia indiferente.

\centerline{\rule[-1em]{0pt}{2.5em}:-:-:-:-:-:-:-:}

Por esta ocasião frequentava assiduamente a casa de Eugé\-nio um seu colega, rapaz acabado recentemente de se formar e que à noite lhe aparecia sempre
para o acompanhar num passeio que a ambos auxiliasse a digestão. Ainda pessoas
suas vizinhas havia que insistiam em o tratar por Zèquinha, nome familiar que
recordava a infância em que tinham visto desenvolver-se o novo licenciado.
Rapaz de hábitos simples, até certo ponto parecidos com os de Eugé\-nio, acamaradava bem com êste. Onde havia mais contraste era na linguagem. Enquanto
Eugé\-nio não se atrevia a usar de termos que não pudessem ser ouvidos deante
de tôda a pessoa, ainda que da maior respeitabilidade, Zequinha era livre na
sua fraseologia que nem sempre uma senhora ouvia sem corar... Não lhe levemos
a mal, que o caso é frequente nas nossas escolas donde se sai sobejamente habilitado em falar em por\-tu\-guês rico de calões, uns menos indelicados que outros mais obscenos. Eugé\-nio é que sempre reagiu contra êsse costume, timbrando em ser impecável na correc\-ção com que se exprime fôsse deante de quem fôsse; os calões soavam mal aos seus ouvidos, assim como os estrangeirismos numa língua já de si rica, como o é a nossa.

São pequenos pormenores que não podem opor-se à franca camaradagem que
os dois colegas estreitaram, de mais a mais como espírito tolerante de Eugé\-nio, para quem a responsabilidade das ac\-ções pode caber quási sempre exclusivamente a quem as pratica, além de terem uma causa determinante, o temperamento ou outra, que as origina e lhes dá vulto.

A parte das noites ocupada no passeio era dedicada às discussões mais
variadas, sempre num ambiente de amena harmonia. Ora versavam sôbre casos de
doenças observados por um e outro, sôbre novas técnicas com que a ciência enrique\-cera a nobre profissão de curar ou mitigar sofrimentos, raras vezes sôbre assuntos políticos em que não demoravam em estar de acôrdo, e, muitas vezes, sôbre assuntos de doenças cardíacas em que falavam menos como profissionais da medicina que como rapazes livres e alodiais.

Foi numa conversa desta última modalidade que o Zequinha uma noite manifestou a Eugé\-nio a estranheza que lhe causava o facto de se conservar ainda sem mudan\-ça de estado, com os ponderáveis prejuízos de que êste se queixava amiúde da falta de comodidades, inerentes à hospedagem sem o carinho da fa\-mí\-lia constituída. Tanto mais que, a\-cres\-centou, a conduta pacata e sóbria que
conhecia a Eugé\-nio, era tôda de molde a justifi\-car tal passo, e tanto mais ainda que possue uma riqueza afectiva nada de desprezar, e uma independên\-cia material satisfatória. To\-lhê-lo-ia a avultada no\-ção da responsabilidade, que
tantas vezes lhe havia notado?

-- Sim, desabafou Eugé\-nio. Que quere? Acha pequena a responsabilidade
do casamento nos tempos de hoje, em que a educa\-ção da mu\-lher a deslocou da
sua dupla missão -- a direc\-ção doméstica e a maternidade, para a frivolidade
do exibicionismo que na me\-lhor das hipóteses a torna absolutamente inútil
no falso lar?

-- Nem tôdas as mu\-lheres tem essa educa\-ção, ata\-lhou Zèquinha.

-- Bem sei que não, mas poucas se aproveitam as que hoje teem, alia\-da
à beleza natural da sua escultura, aquela outra beleza de que temos frequenemente falado dependente duma sólida forma\-ção moral e doméstica que cada
vez mais a rareia.

Seguiram-se uns minutos de silêncio, que Eugé\-nio interrompeu para prosseguir:

-- Você, declarando há pouco a exis\-tên\-cia em mim duma riqueza afectiva,
pôs implicitamente de lado a ideia da antipatia pela constitui\-ção do lar.
Efectivamente eu sou dotado duma afectividade incompatível com a vida solitária. A necessidade de repartir com alguém a afei\-ção é para mim imperiosa;
assim como a sociabilidade, que me torna comunicativo e familiar, me constrange a socorrer-me de pessoa com quem me expanda e abra francamente sem restri\-ções nessas mil pequenas vicissitudes alegres ou tristes que necessitam de
um cora\-ção em que encontrem eco e tempêro. Como vê, longe, muito longe de ser
refratário ao casamento, tenho-o como absolutamente indispensável.

-- Estou de acôrdo em tudo, sustentou Zèquinha, excepto na barreira insuperável em que tropeça para satisfazer a necessidade que o seu bem estar e,
mais que isso, o temperamento, reclamam... É certo que você não se exibe, isola-se demasiado e até a sua posi\-ção social passa desapercebida.

-- O gran\-de estôrvo à consecu\-ção do legítimo desejo, constituiria ma\-té\-ria
para muitas horas de narra\-ção. Verei se em poucas palavras lho posso resumir.
Há bastantes anos, são já decorridos 7, que deparei com uma rapariga por quem
me senti deveras arrebatado. Notei em breve que também não lhe era eu indiferente. Bem
informado, pude apurar que sob todos os aspectos tal rapariga preenchia completamente o meu ideal pela educa\-ção esmerada que os pais timbravam em \mbox{dar-lhe} em manifesto contraste com a educa\-ção moderna. Muitas foram as peripécias sucedidas que consolidaram em mim aquela afei\-ção, e a aumentaram para
a transformar numa ideia fixa. Nunca mais se me deparou uma formosura que
me fizesse vibrar tão intensamente. O pior, porém -- e aqui começa a odisseia --  é que lhe escrevi uma, duas, várias vezes, sem obter resposta alguma. E quan\-do
nos encontravamos depois de lhe haver escrito, continuava a notar nela não lhe
ser eu indiferente. Para comigo aventei justifi\-ca\-ções ao seu silêncio,
nos primeiros tempos. Cheguei a crer que recaísse sôbre ela o odioso dos pais,
a\-quê\-le am\-bien\-te que me diziam conventual, por culpas que a mim cabiam; e essa persuasão torturava-me. Mais tarde, deixei de ver essa causa a justifi\-car
silêncio, e não tornei a poder aperceber-me da estranha contradi\-ção entre
seus o\-lhos que me fitavam e o silêncio com que se obstinava a responder-me.
Rodaram anos e as cenas que narro repetiam-se. O meu pensamento não se distraia
daquela que constituia o meu sonho, e todos os esforços em ler o que sôbre
mim se encerraria naquela alma enigmática eram baldados. E os meses, os anos,
continuaram a rodar. Baldados esforços foram ainda aqueles que fiz para desvanecer a sua imagem sempre viva no meu cérebro. Convenci-me que nunca poderia mais enamorar-me de outra criatura.

-- É estranho o seu relato. É de facto deveras enigmático!

Eugé\-nio passou a descrever algumas peripécias passadas entre ambos, que
recordava como saudosos momentos vividos, algumas das quais tiveram testemunhas oculares. Reproduziu como em duas mesas contíguas de chá, na Foz, ela não
tirara dêle os seus o\-lhos, a ponto de alguém da sua comitiva preguntar: que
me\-lhor quere? Reproduziu como idêntica cena observou no mo\-lhe da praia, e
como lhe foi notado ver-se bem que Eugé\-nio não lhe era indiferente. Relatou
como uma vez, no mesmo mo\-lhe, estando distraído a o\-lhar para o mar, notou-se
fitado por ela ao voltar-se de-repente e como ela corou então. Outra vez, na
esplanada, varrendo com o o\-lhar o areal, numa manhã de Agôsto, reparou ao pas\-sar os o\-lhos por determinado ponto como ela o estava fitando...

-- Pois essa rapariga, prosseguiu Eugé\-nio, ocupou-me uma boa por\-ção da vida, a me\-lhor por\-ção.

-- Agora percebo eu a razão porque tão assiduamente frequenta a praia
da Foz. Ela ainda a frequenta?

-- Sim, respondeu. Sempre rodeada por crian\-ças na sua barraca, que tão bem
sabe atrair, proporciona aos meus o\-lhos um quadro que me não canso de
contemplar.

-- Até na atrac\-ção para as crian\-ças se parece consigo, ata\-lhou Zequinha.
Mas necessita de mudar de ideias. Não cristalize assim. Que paciência a
sua! Procurando, não lhe será difícil encontrar outra seme\-lhante. Águas
passadas não moem moinhos. Faço-\-lhe a justiça de supor que a obcessão
não o cega a ponto de não lhe agradar mais nenhuma rapariga...

Eugé\-nio pareceu não ouvir esta observa\-ção, limitando-se a retorquir:

-- Só lhe conheço no meio de tantos e tão sedutores encantos um defeito -- ser rica. E você que conhece um pouco da minha vida, na sua simplicidade com que me orgu\-lho, e na avidez reduzida, e ainda na franqueza que ponho em todas as palavras, bem compreenderá o desdém que me merece o dinheiro a\-lheio conquanto, a exigi-lo, numa mu\-lher, seria o estritamente suficiente
para equilibrar o encargo acrescido com o casamento.

-- Ainda na riqueza não vejo desigualdade alguma, observou o colega. A
posi\-ção social que alcançou, as faculdades raras de traba\-lho que possue, a
independên\-cia material que adquiriu, a saúde que lhe observo, são valores
que desafiam confronto com a me\-lhor fortuna. Procurar outra que satisfaça
ás exi\-gên\-cias do seu cora\-ção, não se me afigura difícil, e você precisa de não
des\-per\-di\-çar tempo. Ainda não encontrou?

-- Não! respondeu secamente Eugé\-nio. Ou antes: encontrei, sim, mas a
mesma. Não percebe? Eu explico-me. Pouco tempo decorrido após a ter visto
pela primeira vez, encontrei-me num caro eléctrico com uma rapariga que me
chamou a aten\-ção. De fisionomia idêntica, apresentava uma beleza ainda superior á primeira. Segui-a, e ao fim de certo percurso, pude convencer-me
que era ela mesma a beldade que então se me afigurara outra pessoa. Não
vê nês\-te episódio a contra-prova que confirma quanto eu gosto dela, por quem
a minha simpatia foi uma única vez, desde que a conheço, ficticiamente ofuscada por momentos, para se instalar noutra presumível rapariga que no final
de contas era a mesma?

Era já tardia a hora a que se separaram, e Eugé\-nio sentia-se fatigado
com a conversa que chamava ao primeiro plano da consciência sentimentos
tão gratos ao seu espírito, é certo, mas ao mesmo tempo surtindo o efeito
do avivamento duma chaga cruenta, com a depressão nervosa que se lhe segue.

Necessitava Eugé\-nio de ver absorvido o tempo que tinha disponível para lhe atenuar um pouco a obcessão que o perseguia sistemáticamen\-te. O tempo inutilmente desperdiçado causava-lhe horror. Desde há
muito que lhe impressionava a ignorância que a todo momento tinha
de confessar perante os termos mais elementares de automobilismo que
ouvia. Necessitava estudar qualquer coisa sôbre tão oportuno assunto.
A sua ignorância parecia-\-lhe injustificável no século da via\-ção acelerada. Além dos conhecimentos teóricos, podiam ser-\-lhe úteis uns elementos de me\-câ\-ni\-ca e condu\-ção de automóvel.

Um seu vizinho, o Toninho, foi ao encontro do desejo de Eugé\-nio.
Punha à sua disposi\-ção o seu carro, ciência e habilidade técnica. O
pai permití-lo-ia, estava disso certo. Eugé\-nio aceitou.

Toninho era um rapaze\-lho com todos os predicados dum homem de
senso. Fi\-lho de pais outrora abastados, teve de suspender os estudos
por determina\-ção médica, e revoltava-se por ser um inútil na sociedade. Causava-\-lhe tédio a inac\-ção. Conheceu o desejo de Eugé\-nio. A oportunidade que agora se lhe oferecia de ser útil a êste agradava-\-lhe. Timidamente, ofereceu-se para o ensinar. Faltava-\-lhe somente carta de
condu\-ção para que facilmente se arranjaria qualquer pessoa em satisfa\-ção às exi\-gên\-cias da lei.

Eugé\-nio em breve já manejava o volante na Estrada da Circunvala\-ção. Sentia entusiasmo pela condu\-ção e pelos conhecimentos da me\-câ\-ni\-ca. Teve porém em breve de suspender a aprendizagem em virtude dos afazeres
que lhe ocupavam agora mais tempo.

O inverno e a primavera passaram depressa.

Toninho passou a ser um dos companheiros favoritos de Eugé\-nio,
de quem êste apreciava as qualidades de conduta. Era rapaz humilde,
de bons sentimentos e muito económico. Como todos os mortais tam\-bém
tinha defeitos, é certo, mas Eugé\-nio podia corrigir-\-lhos. A inac\-ção levava-o a trocar a noite pelo dia e o dia pela noite, em gran\-de parte,
o que ao fim de pouco tempo era levado a modifi\-car tanto mais que tal
desregramento era pernicioso à sua saúde precária. A inconstância das
suas predilec\-ções é que poderia temperar um pouco a exagerada constância na ideia fixa de Eugé\-nio. Não o conseguiu, porém, como nin\-guém. Era useiro e vezeiro em invocar a palavra de honra por tudo e por nada.

-- Eu acredito no que me diz, interrompia então Eugé\-nio, ou pelo
menos, na boa-fé que põe no que afirma. Não necessita de vir em seu
auxílio o testemunho da sua honra. Isso só se justifi\-ca em quem não
é habitualmente verdadeiro no que diz, e para êsses nem a invoca\-ção
da honra é testemunho bastante, porque a mentira de que habitualmente
se servem já é a nega\-ção da honra. Quem muito jura, muito mente, diz o
ditado. Tôda a palavra proferida deve ser verdadeira, tôda a palavra
é, pois, de honra. Escusado é invocá-la quan\-do somos verdadeiros.

Toninho era um fervoroso apóstolo do jôgo da bola. Estava inscrito num clube de que era intransigente partidário. Abespinhava-se
todo quan\-do alguém fazia comentários à infrac\-ção dos seus adeptos, nesses desvarios tão inerentes ao despôrto da competi\-ção. Os homens do
seu grupo eram intangíveis. Quanto os adversários eram maus, grosseiros e desleais, os seus tinham todos os predicados de estimar.

Eugé\-nio tinha de intervir constantemente:

-- No seu grupo deve haver do bom e do mau. Sem partidarismos preconcebidos, nós devemos criticar com o juízo e não com a sensibilidade
para sermos justos e imparciais. As más ac\-ções do seu grupo são tanto
de notar como as boas ac\-ções dos adversários.

-- O sr. Doutor fala assim por ser inimigo do despôrto, observou
uma vez Toninho.

-- Puro engano. Todos os jogos, quan\-do educativos, merecem a minha
aprova\-ção. Não é o caso do jôgo de bola, onde a deslealdade, a bestialidade e a grosseria são apanágio dos adversários que se disputam
a primasia da vitória material, do número de pontos, sem o\-lhar aos meios
de que se servem para atingir tão antipático fim. Se o jôgo da bola fôsse feito comedidamente, usando mais da agilidade, da aten\-ção e da combina\-ção num conjunto inteligente, sem descurar a lealdade e cor\-recção
devidas ao adversário, como em alguns países se observa, e se, por outro
lado, não excedesse as possibilidades fisiológicos de cada jogador, êle
seria útil, êle constituiria um salutar exercício educativo. E enquanto,
porém, como acontece correntemente entre nós, êle não possue aquelas
características para descambar numa série de violências desmedidas, tal
despôrto é um exercício indesejável, perigoso, depauperador do organismo, desorganizador do espírito; o indivíduo asseme\-lha-se então ao animal que ataca e se defende com a única arma ao seu alcance -- a fôrça
muscular -- desprendido dos atributos superiores que a inteligência e
a forma\-ção moral lhe conferem; é então a animalidade que predomina, fazendo descer o homem aos mais baixos degraus da escala zoológica.

E continua:

-- Não acha caricato que um campo desportivo seja transformado
numa arena onde a fúria e a bestialidade duns são atiçadas pela sensibilidade doentia doutros que o policiamento, sempre indispensável,
dificilmente acalma? Que espectáculo tão degradante!

-- Mas modifi\-cados num sentido moralista, os desportos deixam de
ter interêsse...

-- O interêsse do jôgo está no desenvolvimento físico e não no
espectáculo emocionante que êle possa indevidamente trazer-nos. O nos\-so temperamento quente, impulsivo, de meridionais, necessita antes de
morigera\-ção e não de espectáculos que o exaltem mais.

-- De facto onde está um por\-tu\-guês, está um impulsivo, observa Toninho. Tantos conflitos que se vêem diáriamente na rua tão facilmente
evitáveis...

-- Desde que houvesse calma, acrescentou Eugé\-nio. É só calma que nos
falta, essa calma que não é mais do que o domínio do cérebro, do juízo, sôbre a sensibilidade que em demasia abunda. Nisso está a verdadeira educa\-ção. O conflito está geralmente num exagêro de amôr próprio e
na concomitante deficiência de amôr pelo próximo. Todos nos julgamos
mais daquilo que somos e, sôbretudo, mais que os outros. A nossa sensibilidade a isso nos conduz, sem que o senso corrija o defeito. Julgamo-nos com o direito de ofender todo o transeunte na rua, mas defenda-se
quem quer que seja de pretender atentar levemente contra o nosso amôr-próprio!%
... Fervi\-lha o bofetão e o ponta-pé. Eis a génese do conflito a que
os espectadores dão mais vulto acudindo a favor de um ou de outro contendor, não levados pela justiça do seu pensamento mas sim pela simpatia a que a sensibilidade conduz.

-- De facto, diz Toninho, ainda noutro dia observei um caso signifi\-cativo: um rapaz ao meu lado dirige uma chufa a uma rapariga. Fis-\-lhe
sentir a incorrec\-ção do seu procedimento e preguntei-\-lhe se tinha irmãs, ao que respondeu afirmativamente. Se algum desconhecido se dirigisse a sua irmã, tinha que lho tolerar, disse eu. -- Isso tolerava eu,
repontou o outro rapaz; partia-\-lhe imediatamente a cara!

Toninho possuia uma forma\-ção moral boa e não lhe foi difícil grangear
a estima de Eugé\-nio. Passou a frequentar a casa dêste tôdas as noites. Ao
jantar a batedela à porta era já conhecida. E o Toninho dava imediato ingresso na sala de jantar de Eugé\-nio ou, se tardava um pouco mais, no seu quarto
onde êste se preparava para sair a fazer o quilo. Saiam ambos, numa volta
pelas imedia\-ções até às 10 horas e meia em que reco\-lhiam. Assim se passaram semanas e mêses.

Por uma noite pardacenta de inverno, Toninho apareceu mais triste. Não
dizia nada e parecia a\-lheado a tudo quanto ouvia.

-- O que tem você hoje que eu estranho? -- preguntou-\-lhe Eugé\-nio.

-- Sinto-me irritado, disse, pondo nestas palavras umas reticências de
quem quere  dizer mais, desabafar alguma coisa que lhe vai no ín\-timo:

-- Porquê? Desabafe! Aconteceu-\-lhe alguma coisa de anormal?

-- Já que o Sr.\ Dr.\ me permite uma liberdade que tanto me desvanece...
Já que me põe tanto à-vontade com o à-vontade dos seus modos e que me dá
tão franca hospitalidade...

-- Vá, desabafe e deixe-se de tanta retórica, ordenou dôcemente Eugé\-nio.

-- Lembra-se de me falar há pouco da miséria social que por aí vai
tão generalizada e que tão importunamente se reflete no seio das fa\-mí\-lias?
Da desharmonia que reina entre os casais, umas vezes por hegemonia de temperamento outras de educa\-ção?
 Não sei a que propósito veio a conversa,
mas o que é certo é que foi com indizível aten\-ção que o ouvi. Confesso-\-lhe
hoje que as lágrimas embaciaram-me a vista por vezes durante a sua conversa.
É que...

-- Diga, desabafe! Não percebo onde quere chegar.

-- É que em minha casa reina essa mesma desharmonia.

-- Hein?! Estava muito longe de o supôr, bem pelo contrário, afirmou estupefacto Eugé\-nio!

-- Pois já que me põe tanto à vontade, confesso-\-lhe que sim. A cena hoje
passada foi uma das muitas a mimosear o jantar. O meu Pai, é um desnorteado
como tenho dito por vezes, que divide os afectos e encargos por duas
casas. Ainda se o fizesse equitativamente... Mas não.

-- Isso não causa admira\-ção. Sempre que tal acontece, não sei por que artes, os afectos vão sempre em maior quantidade para a concubinagem. É
caso que tenho verifi\-cado na generalidade. O amôr ilícito suplanta sistemáticamente o lícito, para mais destruir a felicidade do lar, nessas mil e
uma coisas em que se pode exteriorizar. São os afectos e tudo o mais. Tudo é
derivado como por encanto para a nova alian\-ça. O homem passou a ter tôda a confian\-ça na amante a quem dedica agora mais afecto, aumentando-a de
prestígio os seus o\-lhos, com desprêzo absoluto dos afectos que à legítima
espôsa deve e da confian\-ça que esta de preferência lhe deveria merecer. É
a psicologia de todo bígamo. É o desmoronamento da felicidade do lar, o
desvirtuamento do matrimónio, o desmantelamento da fa\-mí\-lia.
%
Mas continue, e desculpe-me o àparte.

-- O meu Pai foi rico. Supôs que o dinheiro nunca mais se lhe acabava, e
os prazeres eram a razão da sua vida ociosa. Infelizmente  a fortuna era em títulos
e dívida de vários estados, que suspenderam os pagamentos sem que haja esperan\-ças de recuperar mais o perdido. Foram-se-\-lhe muitas centenas de contos, talvez alguns mi\-lhares, embora. Á abastan\-ça sucedeu a parcimonia e, hoje
mesmo, está a sujei\-\mbox{tar-se} ao crédito que ainda possue. Ora, casa onde não há
pão, todos ra\-lham e nin\-guém tem razão, diz o ditado e é bem certo. Não é isso porém o que mais me intristece, porquanto todos nós tinhamos de nos resignar com a sorte, e eu era o primeiro que me resignava na qualidade de
fi\-lho. Já mesmo não falo na obriga\-ção que tôda a pessoa tem de traba\-lhar, rica
ou pobre que seja, não só para se precaver contra as contingências do
a\-ma\-nhã como para ser útil à sociedade e a si próprio. Estas palavras tenho-as ouvido por vezes ao Sr.\ Dr.\ e acho-as muito acertadas. Mas o que mais
me revolta, o que me revolta acima de tudo, é que enquanto passamos priva\-ções
em nossa casa, na outra sei que nada é regateado.

-- É o caso de que lhe falei. É o comum. Se assim não acontecesse, seria
uma excep\-ção. Não me admira, pois. O mal foi ter dado tal passo, que tôda a
pessoa deve evitar tanto mais quanto pensar nas consequências inevitáveis
do acto, que explanei e que você acaba tristemente de me objectivar. E como o
homem é um ser pensante, ou deve sê-lo; como tem uma consciência ou deve tê-la, a que obedecer; e como é um responsável, ou deve sê-lo, dos seus actos
perante a sociedade, e muito especialmente perante a mu\-lher que desposou e
que passou a ser uma metade do seu ser, é que tem por restrita obriga\-ção
pensar, considerar e medir as responsabilidades de tudo quanto possa ferir
a susceptibilidade intangível da harmonia e respeito inerentes aos sagrados
laços da união de que hão-de provir as vergônteas chamadas a perpetuar a
fa\-mí\-lia.

-- Ai, o Sr.\ Dr.\ não calcula quanto sôfro, desde os primeiros momentos
em que vejo como princípio da borrasca os ares a turvar-se. Primeiro a
ansiedade, depois a tormenta, e no final a revolta. Apetece-me fugir de casa,
fugir para onde não seja mais visto, e de onde não torne a aparecer. As labaredas
dum inferno não me torturavam mais. Não calcula.

-- Calculo, sim. Creia que sinto a sua dôr nas suas palavras. Di\-\mbox{go-lhe}
mais: numa fa\-mí\-lia em casa de quem estive hospedado, da\-\mbox{va-se} comigo precisamente o mesmo: a anciedade, a tormenta, a revolta, desde os prenúncios até ao
desenrolar da borrasca. Compreendo bem a sua dôr, porque já a senti, e ainda
porque vibro intensamente com todos os desvarios sociais e sôbretudo quan\-do êstes se refletem na fa\-mí\-lia que para mim é o que há de mais sagrado.
Compreendo muito bem, creia-o, a sua dôr. Só a não compreendem a\-quê\-les, infelizmente em tão gran\-de número, que deixados arrastar pelo prazer animal,
são os mais indígnos chefes de família duma sociedade cada vez mais corrupta.

Toninho sentiu bem a consola\-ção de quem se vê compreendido nas suas
palavras, para poder desabafar, dizer tudo, contar a miséria do seu lar que
a desharmonia põe todos os dias à prova nessas cenas degradantes da animalidade
humana.

Nessa noite o passeio durou mais tempo, prolongando-se a conversa\-ção
até mais tarde que o costume, durante a qual conversa\-ção Eugé\-nio procurou
acalmar a dôr de fi\-lho que experimentava Toninho. Passava de meia noite
quan\-do se separaram. Ao despedirem-se ainda Eugé\-nio lhe disse:

-- Sirva de lenitivo à sua sensibilidade o facto de não estar sozinho
na dôr que o caso lhe traz. Infelizmente são muitos os que sofrem êsses
precalços. São muitos, muitos! O\-lhe: procure levar a sua cruz ao calvário
o me\-lhor que possa, serenamente. Procure dormir sossegado.

Não devia ter dormido muito sossegado, porém, a\-quê\-le espírito e Eugé\-nio bem
o compreendia com a sua sensibilidade pare\-lha.

A prova que não dormiu atesta-o a conversa da noite seguinte que Toninho
depressa encaminhou para o mesmo assunto:

-- Que vergonha para a vizinhan\-ça! Ainda que côrra a fechar as janelas, é impossível
a\-lhear os vizinhos do que se passa na minha malfadada casa.
Impossível! Todos o sabem, todos ouvem a vozearia que reflecte a desavença
dos meus Pais.

Eugé\-nio tentou ainda acalmá-lo:

-- Não lhe dêem cuidado os vizinhos. É uma vergonha, é, mas para êles é
o pão nosso de cada dia. Infelizmente o hábito em que estão de ouvir as
mesmas cenas passadas nas outras vivendas embotou-\-lhes a sensibilidade. Já
nin\-guém repara nessas coisas.

Eugé\-nio forçou um pouco a nota que esta resposta exprime, porquanto
sabe quanto os vizinhos e sôbretudo as vizinhas sentem prazer em escutar
o que com os outros se passa, por um dêstes fenómenos que a psicologia dificilmente explica. Recordando-\-lhes o passado em suas casas, deviam aborrecê-los as cenas dos outros, mas assim não acontece.

E Toninho continuava:

-- Quanto aprêço eu tenho pelo Pertunhas da esquina da minha rua! Falo
do Pai da Milita. Naquela fa\-mí\-lia a mais pequena desinteligência se dilue
logo que assoma. Homem, mu\-lher, e a\-quê\-le casal de fi\-lhos vivem na mais dôce harmonia. Que riqueza a vida da\-quê\-le lar, onde o ambiente rescende a felicidade! Penso muitas vezes naquela gente.

O Pertunhas da esquina era homem já idoso que começara por ser modesto empregado duma padaria, subira a sócio e hoje era senhor \mbox{absoluto}. Educado no traba\-lho, só para o traba\-lho vivia, hoje ainda em que já tinha a sua
independên\-cia sólida. O mundo para si limitava-se a poucas léguas em redor,
no restrito espaço em que se incluia a sua enorme freguesia. E não necessitava de mais a sua vida simples e modesta, que a abastan\-ça de agora em nada
alterava. Todo o seu prazer era a felicidade da espôsa e dos fi\-lhos, um rapaz que andava nos estudos e acabava de ser apurado para a  militan\-ça, e uma
fi\-lha um pouco mais nova que o irmão. O viver dêstes em nada destoava da
educa\-ção dos pais senão na cultura.

A fi\-lha estudara também uns anos num colégio, que abandonou para se
dedicar à vida doméstica -- culinária, corte e um pouco de piano.

Esbelta, elegante, Milita era um amôr. Não admiraria muito que se lan\-çasse na torrente desvairada da época, mas tal não sucedeu pelo menos por enquanto. Modesta no vestuário e nos modos, era o enlêvo da família e o alvo
para onde muitas setas de Cupido se dirigiam. Com êxito ou sem êle, não só
ela o sabia como também a vizinhan\-ça que via sucessivamente aparecerem novas rondas a substituir outras rà\-pi\-da\-men\-te destronadas.

Toninho frequentava desde a infância a casa de Milita, e foi num ambiente de amizade que os dois vizinhos cresceram. Amizade pura? Ou mais
que isso? Procuremos sondá-lo, tanto quanto é possível sondar cora\-ções.

Toninho não se fartava de falar dela a Eugé\-nio. É uma rapariga como
poucas, dizia amiúde. É daquelas que o Sr.\ Dr.\ me ensinou a apreciar pela
modéstia da sua apresenta\-ção, sem que artifícios de qualidade alguma escondam a\-quê\-le belo rôsto de pérola, isolada do mundo, vivendo só para a fa\-mí\-lia
e para casa.

-- É oiro. É o que lhe serve.

-- Não é isso o que quero dizer. Não me liga a ela mais que o sentimento de pura amizade nascido na infância.

-- Procure transformar, sublimando-o, êsse sentimento. Ela ofere\-\mbox{ce-lhe}
todos os requisitos para o fazer feliz. Acresce que o muito que lhe tenho
ouvido falar dela me autoriza a convencer-me que ambos gostam muito um do
outro.

-- Ora isso é que eu não sei, retorquiu Toninho. Há dois anos ofere\-ceu-se-me ensejo de conduzir certa conversa nêsse sentido, muito discretamente,
muito a mêdo. Que era muito nova para pensar nisso, foi a resposta que obtive.

-- E então? Volte ao caso. A resposta à primeira invectiva deve aumentar
mais o conceito que faz dela. O contrário é que era de fazer esmorecer. Queria ver nela o reflexo dessas tantas raparigas que fazem profissão de aceitar os galanteios do primeiro que lhes aparece?!

-- Mas é que... quanto a mim... também não sei...

-- Mau! -- replicou Eugé\-nio. Você conhece-a me\-lhor que nin\-guém. E que gosta
dela sei-o eu. Que mais quere?

-- Receio que a\-ma\-nhã goste de outra, e o facto de ser vizinho desta acarretaria dissabores.

-- Ai o Toninho inconstante! -- observou Eugé\-nio, dando uma tonalidade de
repreensão às suas palavras. Deixe-se disso. Ou gosta, ou não gosta. Se
sim, ande para diante.

-- A dizer a verdade, Milita é muito interessante e muito boa rapariga.
Mas... vou confessar-\-lhe. Uma rapariga há de quem gostei mais. Conheci-a há
dois anos e fez nascer em mim uma paixão atroz. É uma linda morena de
Trás-os-Montes. Viera passar aqui próximo umas semanas. Todos os dias conversavamos, e as horas corriam desapercebidas perante a doçura idílica do
nosso con\-ví\-vio. Muitas vezes premeditei dizer-\-lhe aquilo que eu sentia em
mim, e para lá me dirigia com essa resolu\-ção; mas a voz embargava-se-me, e
nunca pude adiantar conversa no sentido desejado. E ela bem o sabia, bem o
reconhecia, tanto que chegou a dizer-me um dia:

-- Quere que o carro ande adiante dos bois? Mas eu sentia-me
na sua frente de tal forma acanhado, que nem assim ousei. A certa altura
retirou para a aldeia e não a tornei a ver. Já lá vão dois anos.

E continuou:

-- Essa rapariga é quem dirige a casa lá na sua aldeia, e tem tôdos os
atributos duma rapariga sã. Tem um defeito: pinta-se.

-- Defeito que é muito gran\-de, interrompeu Eugé\-nio. Para mim basta essa mentira com que pretende realçar a beleza, para que eu desconfie serem
mentirosos todos os atributos que lhe vê o seu cora\-ção apaixonado. Deixe
que fale assim. E digo apaixonado porque a timidez com que se houve na sua
presença é própria dos apaixonados. E é essa mesma paixão que o leva talvez
a ver nela os atributos que idealiza em vez de ver a\-quê\-les que de verdade
ela possue. Mas eu não a conheço e formulo uma simples hipótese. Milita, ao
contrário, nada tem que seja mentira. O caso é, porém, consigo, e releve o
ter-me adiantado de mais. Já sabe que digo sempre o que sinto e quási sempre com uma franqueza rude.

Toninho no à-vontade que sentia em face de Eugé\-nio, confessou-\-lhe hesitar entre as duas raparigas, mas mostrando que a aldeã o seduzira de preferência.

\hskip 1em

\begin{center}\bf\large Eugé\-nio persiste na ofensiva\end{center}

E novo verão surgiu, nova época balnear, novos protestos de
não mais pôr os pés na praia da Foz...

Não exprimira bem, de resto, na sua última carta enviada há dois anos a Princesa, a resolu\-ção de a não tornar a seguir?!

Como é óbvio, voltou a frequentar a praia. Novas trocas de o\-lhares, intermeados com indiferença daquêles o\-lhos predestinados a amarfanhar o espírito de Eugé\-nio, já aliás tão alquebrado.

Até Eugé\-nio numa mal dis\-far\-ça\-da altivez, pretendeu muitas vezes simular indiferença à sua passagem, que mais o torturava ainda como todos
os actos que não correspondessem firmemente ao sentimento que os originara.

O mais interessante é que os encontros de Princesa com o desconhecido, que
tanto se haviam tornado suspeitos de Eugé\-nio no ano anterior, não mais
foram por êste verifi\-cados. De mais, nunca vira nada durante o inverno à roda
da casa de Princesa como era natural no caso de serem fundados os seus
tristes presságios. E a época balnear decorreu sem outros incidentes que
a\-quê\-les em que era protagonista Eugé\-nio.

Uma vez mais pensára em escrever-\-lhe. Mas a resolu\-ção tomada há
dois anos bem expressa na carta que lhe enviou? Nem a sua palavra anterior o demoveu.

Estavamos em Setembro de 1937. Os amigos de Eugé\-nio aconse\-lhavam-no a não exteriorizar mais a sua simpatia por Princesa. As mu\-lheres
quanto mais se envaidecem, pior, esclareciam êles. Mas que importava a
Eugé\-nio, alma sã que nunca soube ocultar aquilo que sente, para quem
as palavras são a verdadeira ex\-pres\-são do que pensa, sem o mí\-ni\-mo vislumbre de hipocrisia, que esta carta fôsse mais um elemento destinado
a satisfazer vaidades? Ela era sincera como as anteriores, e isso
bastava para a sua consciência. Os o\-lhos de Princesa continuavam sempre a ser o sol que ilumina os passos de Eugé\-nio, e a sua mágoa era
não lhe poder falar de viva voz para lhe dizer quanto a admira, quanto
se extasia ao vê-la, quanto feliz se sente a pensar nela.

E escreveu uma nova carta, muito simples, reiterando-\-lhe a ex\-pres\-são da sua simpatia que o tempo longe de obscurecer aviva ainda mais.
Disse-\-lhe que os momentos mais felizes da sua vida são a\-quê\-les em
que a vê, em que adora a sua beleza; e quanto gosta de a ver ao pé do
mar, a cuja imensidade se compara a imensidade da sua venera\-ção e cujas ondas revôltas são bem o espê\-lho do seu espírito revôlto por não
ser compreendido pela única entidade que o seduz e prende! E terminava pedindo uma resposta que fi\-cava anciosamente aguardando.

E ficou aguardando a resposta pedida com a anciedade de quem espera a decisão da sorte da sua vida. Uma borboleta que se aproximava
de Eugé\-nio, rodopiando ao seu redor, de asas muito brancas, parecia prenunciar-\-lhe boas novas. Eugé\-nio não era supersticioso, mas qual é o
espírito forte que não tem os seus afrouxamentos em momentos de gran\-de
gravidade? De gran\-de gravidade, sim, porquanto Eugé\-nio não tinha rebuço
em dizer que a exis\-tên\-cia de Princesa era a única razão da sua.

Poucos dias passados recebia um sôbrescrito timbrado da casa de
Princesa. Os dedos tremiam-\-lhe ao abrí-lo, a sua respira\-ção suspendeu-se e o cora\-ção bateu mais apressado. Tirou o seu conteúdo, e verificou que êste era, pura e simplesmente constituido pela carta que escrevera e que
Princesa lhe devolvia sem uma única palavra. Caiu exausto. O resto
da tarde passou-o sentado à secretária, inerte, como que a\-lheado de tudo o que o circundava.

Bem forte era a sua complei\-ção para aguentar tantas e tão cruéis
punhaladas, pensava Eugé\-nio com os seus botões! Que ímpia se mostrava para
consigo a Providên\-cia nês\-te particular! Que rapariga essa, que nin\-guém
vê falar com quem quer que seja estranho ao seu sexo e que obstinadamente se nega a receber as suas declara\-ções bem sinceras de simpatia!
Haverá na sua intui\-ção de mu\-lher alguma coisa que lhe mostre não ser
Eugé\-nio digno dela? Mas Eugé\-nio numa rigorosa introspec\-ção não encontra nada em desabono próprio.

E sonhava sempre, acordado ou dormindo, em Princesa, única mu\-lher
que o inspirava e arrebatava, única mu\-lher que dominava como por encanto todo o seu pensamento e o amarfanhava loucamente com a sua indiferença.
E revivia, concentrado, os seus o\-lhares magestosos. Várias vezes mesmo
Princesa foi surprendida por Eugé\-nio a o\-lhá-la espontâneamente, e córava então de pudor que lhe realçava ainda mais a sua formosura. Eugé\-nio recordava
ainda o êxito da experiência que fizera há tempos de acompanhar algumas tardes na praia umas raparigas conhecidas para estudar o efeito produzido em
Princesa, verifi\-cando quanto a esta não era indiferente o facto.

Enquanto revivia, recordando-os, êstes e outros pormenores que tinham concorrido para cimentar a sua paixão, Eugé\-nio procurava auscultar a própria razão para lhe preguntar: os móveis que levam Princesa a conduzir-se assim para consigo não seriam os da curiosidade de preferência aos da afei\-ção? Mas
a\-quê\-le rubôr que notou por vezes subir-\-lhe a face, de-repente, ao encarar consigo? Nem a imagina\-ção hipertrofiada no sentido da sua ideia o poderia nunca levar a fantasiar coisas fictícias, e estas eram daquelas bem objectivamente notadas.

Mesmo supondo não passarem de mera fantasia da imagina\-ção a\-quê\-les pormenores, isto é, pretendendo supor-se que Eugé\-nio era indiferente a Princesa, ainda
restaria preguntar qual o motivo porque esta nunca mostrou a mínima contrariedade ao ser assediada com a persegui\-ção tenaz da\-quê\-le, como era natural acontecer e ao contrário do que sempre se observou. Nunca Eugé\-nio viu no semblante de Princesa o menor indício de aborrecimento, e muito menos a tentativa de
o evitar.

A atitude de Princesa aparecia como sempre inexplicável, e cada vez mais.
O que se passará a-dentro da\-quê\-le relicário que é a sua alma, tão hermética,
como cruelmente fechado, que não permite que nada transpire aos o\-lhos de Eué\-nio? Oh! Se lhe fosse possível escalpelar a\-quê\-le cora\-ção! Eugé\-nio daria
tudo por o sondar, por penetrar na\-quê\-le mistério que tão adversamente amachucava o sensivel cora\-ção de Eugé\-nio!

Dar-se-á o caso de Princesa ser um dêsses temperamentos frios como gêlo, incapaz de afeiçoar-se a alguém? Esta última hipótese já Eugé\-nio a considerou muitas vezes, afigurando-se-\-lhe plausível em certas ocasiões, pondo-a
de parte noutras ao recordar certas manifesta\-ções inequívocas do passado em
que se mostrara não ser indiferente para êle. A atrac\-ção para as criancinhas
que sempre a rodeiam, que a amam porque ela as ama, em nada depõe a favôr da
frieza do seu cora\-ção.

Os meses continuaram a rodar sem que nada se tornasse suspeito a Eugé\-nio nas imedia\-ções da vivenda de Princesa, a levantar o veu do insondável enigma que envolve aquela vida.

O Zèquinha naquela noite do desabafo de Eugé\-nio foi deitar-se pensando
no estranho caso.

Eugé\-nio mostrara-se sempre até aí sisudo e dum estoicismo que o punha aos o\-lhos de Zèquinha como incapaz de se deixar apaixonar por qualquer rapariga, tais e tantas eram as exi\-gên\-cias que reputava indispensáveis
na mu\-lher para como tal a considerar. O amigo de Eugé\-nio augurava lá para
consigo que tais exi\-gên\-cias eram irrealizáveis, e uma vez lhe arriscou
essa objec\-ção. O que Eugé\-nio acabara de confessar-\-lhe era uma revela\-ção.
Quem conseguiu requestar tão profundamente o seu afecto era por certo alguém que no mundo feminino havia de marcar pela beleza das suas formas
e do seu carácter, e ainda pela educa\-ção doméstica, pois eram êstes os atributos que lhe ouvira mencionar frequentes vezes como indispensáveis á
mu\-lher.

Quem seria ela? Vários factos lhe perpassaram pela mente, em que o
Zèquinha pretendeu encontrar o fio da meada.

Eugé\-nio tinha uma predilec\-ção notável por certa rota nos passeios
que ambos davam todas as noites. Por mais do que uma vez o Zèquinha pretendera mudar de itinerário, observando, sem atinar com as razões, que Eugé\-nio
quan\-do aquiescia era com certa contrariedade mal dis\-far\-ça\-da.

Duma outra vez que o acompanhara ao cinema, notou que no intervalo
da sessão, quan\-do se iluminou o recinto e ambos percorreram com o o\-lhar
a assistência, Eugé\-nio não pôde esconder a perturba\-ção produzida pelo conteúdo de determinado camarote a ponto de lhe pedir que não o\-lhasse naquela direc\-ção. Não se explicou o companheiro do Zequinha, mas não
foi difícil a êste notar que a mudan\-ça de côr de Eugé\-nio, a mudan\-ça do timbre da voz e o estado de indiferença para a sua conversa a contrastar com
a expansibilidade anterior, tinham o aspecto dum choque nitidamente afectivo.

Retomara Eugé\-nio, por êste tempo, a aprendizagem da condu\-ção de automóvel para aproveitar ùtilmente o resto das férias e para se distrair
um pouco. O automobilismo constituía de facto bom passatempo com que
se recreava. Foi mais além o seu entusiasmo. Deitou balanço às suas
economias, e encorajou-se a fazer a aquisi\-ção dum veículo económico. Preparou-se bem, não fôsse aí encontrar o primeiro insucesso como examinando, e alcançou a sua carta de condu\-ção.

\begin{center}\bf\large Na missa\end{center}

Em princípios de 1938 criava-se uma nova freguesia eclesiástica nas
cercanias de Princesa. A sua igreja, em constru\-ção, já possuía uma instala\-ção provisória nos rés-do-chão com destino ao culto. E de facto, antes
mesmo da aprova\-ção da nova freguesia já se realizavam ali os actos religiosos do costume. Já havia sido inaugurada na presença da autoridade episcopal e isso bastava para que o público da futura freguesia ali acorresse
à missa, confissões ou quaisquer outros actos. Nisso poupavam tempo e caminho os paroquiais fiéis em busca da satisfa\-ção espiritual que até aí tinham
de procurar em igrejas mais distantes.

O caso parecia não interessar a Eugé\-nio. Já muitos anos antes, era êle
estudante ainda, vagueava pelas igrejas mais próximas, aos domingos, procurando encontrar Princesa nas diversas missas rezadas desde as das horas
mais cêdo até às do meio-dia. Nestas romagens pelas igrejas de mais palpite, Eugé\-nio perdeu muitos e muitos domingos, os bastantes para se convencer
que Princesa não frequentava os actos religiosos. Por isso nada interessava a Eugé\-nio a edifi\-ca\-ção da\-quê\-le templo.

Um domingo, porém, resolveu-se a ir à nova igreja. O Zequinha acompanhou-o. O acto da missa tinha começado há pouco, e os retardatários ainda entravam
em gran\-de número, se bem que a igreja já estivesse quási repleta de devotos.

O\-lhou em redor quan\-do, com surprêsa sua, notou a presença de Princesa
na companhia da irmã e dos avós. De joe\-lhos, cabêça curvada, vestida de preto,
embebida na ora\-ção, tinha um não sei quê de místico que mais punha em destaque a sua formosura.

Terminado o religioso acto, Eugé\-nio postou-se de fronte da porta de
saída. Princesa ao transpôr o limiar depressa o avistou, e fitou-o com
insistência e por uma forma inequívoca antes de prosseguir. Não havia andado uma dúzia de passos no adro exterior quan\-do parou para se demorar a
conversar com o seu grupo que formara roda por forma a que Princesa pôde
fixar demoradamente Eugé\-nio.

Não passaram despercebidos ao Zèquinha, que permanecia junto do amigo,
estas trocas insistentes de o\-lhares, repetidas já dentro de escassos minutos, e a contra-prova tirou-a momentos depois quan\-do o grupo se retirava
do adro e os mesmos o\-lhos se voltavam para se encontrar com os de Eugé\-nio.
O Zèquinha preguntou quem seria aquela que simpatizara com Eugé\-nio.
Êste desejou manter a incógnita, simulando indiferença num enco\-lher de ombros ao responder-\-lhe que a não conhecia.

O Zequinha é que não se convenceu da indiferença, que a fisionomia de
Eugé\-nio traíra por modo bem flagrante ainda que involuntariamente, e resmungou:

-- Hum!, aqui há coisa. Ou me engano muito ou uma das do grupo, a mais
esbelta, estava no cinema, na\-quê\-le camarote para que você me pediu que não
o\-lhasse...  O seu enigma está descoberto...

Eugé\-nio não respondeu.

A frequência daquela igreja passou agora a tornar-se obrigatória para Eugé\-nio. Para lá caminhava de ora-avante todos os domingos, ora a pé,
ora no automóvel, a assistir a missa das 11, na dôce esperan\-ça de ver a
sua adorada Princesa, sob o mesmo tecto, respirando o mesmo ar que ela, mergu\-lhado no mesmo ambiente de beatitude, ouvindo as mesmas práticas de moral no Evange\-lho. Eugé\-nio achava-a mais admirável e sentia-se feliz em a poder adorar extasiado.
À saída, os o\-lhares trocaram-se com satisfa\-ção para Eugé\-nio por
forma sistemática durante alguns domingos. Na retirada, Eugé\-nio seguia
o grupo de Princesa até à sua casa ou à dos avós, adiante da sua algumas
centenas de metros. De uma das vezes, nês\-te percurso, uma amiga que se despediu no meio do caminho para enveredar por uma rua transversa deu ensejo
a Princesa para trocar um o\-lhar furtivo para trás onde vinha Eugé\-nio; doutra vez, fi\-cando-se para último lugar à entrada de casa o\-lhou-o vivamente
enquanto fechava o portão. E outras cenas análogas se sucederam. Depois,
como de costume, umas vezes era na\-quê\-les o\-lhares inequívocos, quer à saída
quer durante o percurso no regresso a casa, outras vezes a indiferença.

Eugé\-nio debalde procurava mil explica\-ções para esta atitude dúbia,
para esta notável duplica\-ção da personalidade de Princesa. Um caso de
neurastenia? -- pensava finalmente muitas vezes.

Num domingo de vésperas, de Carnaval, à saída da missa, Eugé\-nio seguiu Princesa que se fazia acompanhar pela irmã. Ambas davam o centro a uma amiga, e o grupo dirigia-se em
direc\-ção à do Avô, ao fundo da Avenida.

Em certa altura notou Eugé\-nio que Julieta fazia quaisquer gestos com os antebraços cruzados por detraz das costas, de interpreta\-ção duvidosa. Não quis dar-se, porém, por achado. Mas os gestos repetiam-se, e o
Zequinha, que acompanhava Eugé\-nio, pretendeu invectiva-la, furioso. E mais furioso ficou quan\-do Eugé\-nio o procurou dissuadir da interpreta\-ção dada.

-- Pois quê? Você tem os o\-lhos tão tapados que não vê?

Podia muito bem serem movimentos inconscientes, observou Eugé\-nio. Decerto que tratando-se de outra pessoa acreditaria no signifi\-cado obsceno
do gesto; da irmã de Princesa é que nunca! Porque, a acreditá-lo, seria negar os princípios em que esta fa\-mí\-lia era educada. Não! Não tinham o direito, nem ele, nem o Zéquinha, de o fazerem.

-- Que cegueira a sua! Os argumentos não destroem os factos, e êstes
estão bem á vista. Pois não vê como se repete o gesto, e como, para o confirmar, ela e a amiga do meio, o\-lham para traz em ares de escârneo?

Era bem verdade tudo quanto o Zequinha observava, e Eugé\-nio, meio vencido, notou que Princesa não podia responsabilizar-se pelos actos duma irmã
mais leviana...

E enquanto o Zèquinha insistia, contraditado por Eugé\-nio, que a atitude
de Julieta não podia deixar de ser de acôrdo com a irmã, o grupo entrou finalmente na casa para onde se dirigia, não sem que Princesa, que desapareceu
em último lugar para fechar o portão, volvesse um o\-lhar para traz a cruzar
com o de Eugé\-nio.

E este acabou por deixar de frequentar aquela igreja, em que já se sentia dominado pela neurastenia possivelmente contaminada por Princesa... O
bom humor que, apesar das vicissitudes passadas, ca\-ra\-cte\-ri\-zava permanentemente o estado de Eugé\-nio, ia-se apagando e êle era o primeiro a dar por isso.

\begin{center}\bf\large Uma transmontana\end{center}

Estávamos em pleno verão. Haviam decorrido poucas semanas sôbre a conversa atraz
narrada do apaixonado Toninho quan\-do êste, numa tarde, apareceu mais cêdo
que de costume em casa de Eugé\-nio, numa  excita\-ção febril que ao seu amigo
não passara desapercebida.

Afigurara-se-\-lhe ter visto a transmontana que tanto o seduzira dois
anos antes, e queria tirar-se de dúvidas. Tinha quási a certeza de a ter
visto à janela onde se havia hospedado, mas tal fôra a como\-ção sentida que
não se atreveu a certifi\-car-se da inesperada apari\-ção. A companhia de Eugé\-nio novas fôrças lhe trariam para me\-lhor poder observar o facto.

Os dois lá foram. Era realmente ela, mas a Toninho ainda desta vez se
paralizaram os órgãos da voz para lhe dirigir os elementarís\-si\-mos cumprimentos que lhe devia. E assim passaram.

A impressão de indiferença que Toninho mostrou só podia subsistir
em quem não saiba o que são estas coisas. Não é indiferença. Sob essa aparência esconde-se um torvelinho de ideias opostas entre si, em que a razão
e a sensibilidade se guerreiam mas sem que predomina uma ou outra que antes
são como que recalcadas, mas em que a timidez toma papel importante.

Nem sempre o acaso é estéril em consequências. Passados poucos nomentos, Toninho e Eugé\-nio cruzavam com a transmontana que se dirigia a um marco postal para enviar notícias da sua chegada à fa\-mí\-lia.

O encontro era por ambos desejado. Por Toninho sabe-se bem porquê; por
Eugé\-nio, rápidamente apresentado, para conhecer e estudar quem tanto seduzia
o amigo.

Não lhe havia passado despercebida a ela a passagem de Toninho a quem teve
ocasião de repreender pela indiferença mostrada. E o Toninho esforçando-se por mascarar o seu embaraço que a voz trémula mais fazia avolumar, lá se justificou o me\-lhor que pôde.

O tempo decorreu. Grande seria a indiscri\-ção querer penetrar nos idilios que entre ambos sucederam. Isso é lá com êles.

Um pormenor das conversas de Toninho não havia porém escapado a Eugé\-nio: a
transmontana vivia numa vila próxima da aldeia da fa\-mí\-lia Frazão, e tanto
bastava para que Eugé\-nio tivesse de procurar em conversa desvendar um pouco o enigmático véu que aos seus o\-lhos enchia de interroga\-ções a vida de
Princesa que sempre continuava a preencher os seus pensamentos.

Eugé\-nio, numa noite de longa conversa, preguntou à transmontana se conhecia a fa\-mí\-lia Frazão. Que sim, respondeu. E veladamente a conduziu a dar-\-lhe informes, sem
que ela nem de longe supusesse a inten\-ção da sua curiosidade.

-- O sr. dr. Frazão, informou ela, tem duas fi\-lhas, a mais nova das quais
tem o mesmo nome que eu. Não me ocorre o da mais ve\-lha. Conheci-as num
baile no meu conce\-lho; tornou-se notada a modéstia da sua apresenta\-ção
e uma delas desculpou-se que não foram me\-lhor prevenidas para a aldeia.
O que teem de mais curioso aquelas raparigas é que nin\-guém as vê a conversar
com rapazes, como é natural e próprio das nossas idades.

-- Pois sei de alguém que faz o seu pé-de-alferes à mais ve\-lha, arriscou Eugé\-nio.

-- Pelo que eu conhêço dela, quási posso afirmar que será mal sucedido
quem quer que ele seja.

-- Então ela, interrogou Eugé\-nio, não há-de gostar de nin\-guém?

A interlocutora hesitou um pouco e acrescentou:

-- Talvez, por isso mesmo que gosta de alguém, é que não dá confian\-ça
mais nin\-guém...

A curiosidade de Eugé\-nio aguçava-se cada vez mais.

-- Então já confessa que, dessas raparigas, pelo menos a mais ve\-lha, fala com alguém?

-- Consta lá na terra que ela tem aqui, na cidade, alguém de quem gosta
muito, parece que um médico, mas que não é correspondida.

A informa\-ção, mesmo deturpada como era, não se tornava desinteressante para
Eugé\-nio. A última parte devia referir-se-\-lhe a êle, como era de supor.
Conquanto fôsse precisamente o contrário da verdade, o que é certo era já
constarem os seus amores na aldeia de Princesa, por aquelas terras longínquas das faldas da Serra
de Bornes, talqualmente desde há muito constavam na aldeia de Eugé\-nio...

E o assunto da conversa\-ção mudou imediatamente, para que a transmontana se não apercebesse do interesse de Eugé\-nio. Falaram nos passatempos predilectos da morena transmontana: montava a cavalo, guiava Vau 8 e jogava hockey em patins.

Separaram-se a horas tardias, e Toninho pretendeu saber as im\-pres\-sões
sôbre a sua morena.

-- Pinta-se, monta a cavalo, guia Vau-8, joga hockey em patins... Não sei.
Isso é consigo. Só a si compete decidir-se.

Toninho suspendera radicalmente a frequência da casa de Milita. O caso
não é de admirar porquanto o tempo que os dias duram não era demasiado para
manifestar os arroubamentos de afei\-ção que o prendiam à morena transmontana.
Mais de admirar se tornava a forma como falava agora de Milita, de quem passou a referir-se com o mais absoluto desdém.

A transmontana anunciou a sua retirada para a aldeia natal. Se houve
lágrimas de parte a parte, só aos dois fervorosos namorados compete dizê-lo.
Mas é crível que sim. Não é impunemente que duas almas se separam após tão
dôces e numerosos momentos de idílio como a\-quê\-les que a Natureza lhes proporcionou. Também é crível que jurasse amenizar um pouco a cruel separa\-ção com
as frequentes cartinhas do estilo. Deixemo-los lá em paz, a pensar um no outro, visto que para o amôr não há distâncias. Nunca as houve quanto mais hoje
em que êsse factor desapareceu.

Os dias continuaram a desenrolar-se monòtonamente, pelo menos para os
estranhos àquêle episódio. Toninho decerto para suavisar saudades retoma os
hábitos anteriores entre os quais avulta o de frequentar a casa de Milita.

Num dia próximo e sem mais preâmbulos, com estupefac\-ção de Eugênio, Toninho confessa-\-lhe que já não sabe de novo de quem gosta mais -- se de Milita, se
 da transmontana. Embora de tipos diferentes, ambas merecem a sua afei\-ção.
Ambas realizam o seu ideal de beleza física.

Eugé\-nio não pôde deixar de repreender tão volúvel, tão inconstante cora\-ção.

-- Ouça Toninho: quan\-do na escô\-lha que nós fazemos entre exclusivamente,
como móvel, a beleza física, o resultado é inevitavelmente êsse. Porque belezas físicas há muitas, a irresolu\-ção é fatal. Até na mesma mu\-lher a beleza
faria com o realce que lhe empresta a indumentária. Para a\-quê\-les que vêem
na beleza física o único atractivo das suas predile\-ções, então a afei\-ção seria
variável com o vestido do dia... Não acha?! Acresce que a nossa disposi\-ção
de espírito que nos leva a julgar do belo é já de si um pouco variável. Mais
grave ainda é o facto de a beleza física ser de tôdas as belezas a mais efémera. Uma rapariga hoje bonita perde a\-ma\-nhã tôda a sua estética, sôbretudo
após as núpcias; o inverso também se observa, e êstes factos são explicados
pela ciência. Ora pois se a nossa afei\-ção estivesse sòmente adstrita ao
físico mutável a instabilidade do nosso espírito seria a nega\-ção perentória do matrimónio e da fa\-mí\-lia. Mas quan\-do aliamos ao nosso juízo crítico a aprecia\-ção
da beleza moral, então um novo horizonte se nos depara; esta beleza tem outra
estabilidade muito diferente porque assenta em alicerces seguros como são o
temperamento e os hábitos da mu\-lher que esco\-lhemos. E com alicerces seguros a
afei\-ção é segura e o edifício em que o lar se constroi desafia a borrasca
 mais tempestuosa. O edifício é então, e só então, seguro como o marfim, consolidado como é então, e só então,  pelos indestrutíveis laços do amôr.

E continuou:

-- A sua volubilidade vem precisamente de que a sensibilidade anima demasiado o seu ser, não lhe permitindo que escô\-lha entre duas belezas físicas
que se lhe anto\-lham, belezas que a\-ma\-nhã ainda mais agravariam a inconstância
do seu espírito com a inconstância da sua dura\-ção. Lance sôbre essa fugaz
característica do belo um facho de luz da sua razão que o leve a ver outros atributos do belo moral, e decida-se então. Verá como essa decisão será então perdurável. A beleza física é indispensável mas está muito longe de bastar. Leia
Júlio Deniz e medite a paixão de Guida e a volubilidade de Daniel, e a paixão
final dêste com aquela cheia da mais bela eleva\-ção moral como elevada é a
beleza moral que ca\-ra\-cte\-ri\-za Guida.

Coincidiram as últimas palavras de Eugé\-nio com a passagem pela frente
da casa de Princesa, e qualquer pessoa avisada, teria notado que a inflexão
que êle deu à sua voz ao proferí-las coincidiu ainda com uma o\-lhadela para
aquelas janelas onde a ilumina\-ção era como de costume o único indício de vida.

O itinerário de Eugé\-nio era sempre a\-quê\-le, não porque as passagens por
ali lhe dessem ensejo de ver os o\-lhos lânguidos e pensativos de Princesa, mas porque a simples contempla\-ção daquelas paredes onde se abriga o objecto dos seus pensamentos lhe dava um prazer a que se não podia furtar, por ser
único que lhe era dado saborear.

Nem as prova\-ções que tanto assediavam Eugé\-nio lhe faziam mudar o rumo
dos seus passeios nem alterar a sua boa disposi\-ção humoral, por\-quanto ao passar ali de fronte tinha sempre uma pregunta de inocente ironia a fazer a Toninho no género desta:

-- Quem morará nesta casa, sabe?

-- Aqui mora o Sr.\ Frazão, de quem deve ter ouvido falar, respondia ingènuamente Toninho.

-- Não. Não conheço. Nem nunca ali vi nin\-guém. Nesta casa não haverá
raparigas?

 -- Isso não sei; de certo não. Nunca vi aqui nenhuma...

-- O que não quere dizer que as não haja.

-- Não me parece. E então nês\-te local, onde as raparigas fazem tôdas as
noites avenida na mais espectaculosa liberdade com os rapazes vizinhos...
Esta e outras respostas seme\-lhantes, um tanto provocadas intencionalmente, tinham para Eugé\-nio o sabor dum galardão com que um estranho distinguia,
sem o saber, Princesa aos o\-lhos de Eugé\-nio, e constituiam para si momentos de
indizível felicidade.

\begin{center}\bf\large Na barbearia do Matias\end{center}

A barbearia do Sr.\ Matias era a mais asseada das proximidades de Eugé\-nio. Sem luxo, muito bem iluminada por duas portas envidraçadas, umas paredes
muito caiadas de que se destacam dois gran\-des espe\-lhos sôbre uma estreita
banca onde se estende o arsenal de nava\-lhas de barbear, máquinas de cabelo
de tamanhos vários, pinceis, pentes, frascos de essências, etc.; duas cadeiras
para o serviço profissional e mais uma meia dúzia para os freguêses que esperam, e uma pequena mesa com diários e revistas, completam o recheio.

A ajuizar pelo número de pessoas que ali estacionam é de augurar bom
rendimento ao Matias. Porém o serviço só abunda nas horas de saída do pessoal dos escritórios ou das fá\-bri\-cas, isto é, por ocasião do descanso ao meio
dia e nas horas que precedem o jantar. Fora disso, comentam-se as notícias
trazidas por uns e outros.

O Sr.\ Jeremias, dono do ta\-lho da esquina, homem atarracado e pan\-çudo, com
o pescoço metido entre os ombros e um boné sôbre a cabeça, uma face congestiva e uns óculos armados em tartaruga, mãos atraz das costas corcovadas, é dos
mais assíduos frequentadores da barbearia em ma\-té\-ria de cavaco, pois que dali
pode vigiar muito bem a entrada de freguêses na sua loja. O seu pai, homem
já trôpego pela idade, encostado a uma bengala e de cabeleira postiça, passa
por ter gran\-de fortuna em dinheiro, e assim deve ser pelo número de ta\-lhos
de que é proprietário e de prédios que possue.

O relojoeiro pegado também por ali aparece, rapaz novo de maneiras delicadas, e assim outros vizinhos.

Uns, palradores por ofício, outros ouvintes por hábito, o que é certo é
que as falcatruas, os escândalos, as falências, os crimes, a guerra no Oriente
e a perda independên\-cia dos Sudetes, o jôgo da bola, o noivado dum vizinho
e a doença dum outro, um cartaz aparecido no centro da cidade, tudo são factos
que constituem boa ma\-té\-ria prima para passa-tempo. Desta forma o Sr.\ Matias,
e o seu oficial, estão sempre aptos a entreterem os freguêses que chegam com a notícia cujo gènero me\-lhor agrade ao já conhecido temperamento de
cada um.

-- Oh Sr.\ Matias, há lugar para uma barba? -- pregunta ainda de fora da
porta um recém-chegado, que acrescenta ao sentar-se: Rápida, que tenho pressa,
sim?

-- Vem hoje tão enervado, Sr.\ Oliveira? -- atreveu-se o barbeiro a inquirir o freguês, que puxava pelo relógio.

O Sr.\ Oliveira era o gerente duma casa de venda de automóveis. Tinha
a receber um pagamento dum cliente a quem mandara já pela sexta vez cobrar
umas centenas de escudos á conta da liquida\-ção dum rico carro de luxo. Estavam novas presta\-ções a vencer-se, e o cliente ainda não solvera a de há três
mêses. O cobrador trazia como resposta que não está, umas vezes, ou que não
pode atender, outras vezes, que volte daí a dias, ainda outras.

-- Estou informado que já mudou de fornecedor de gasolina duas vezes por
falta de pagamento. E mais: os credores que correm diariamente a sua casa, a
pedir a esmola dalgum dinheiro para amortiza\-ção do débito, são despedidos com
a grosseria das suas maneiras; de entre êstes contam-se o merceeiro, o padeiro e até a  leiteira. Na modista a conta anda por bastantes mi\-lhares de escudos que o nosso homem se nega a pagar, resmungando que nada tem com o que
a mu\-lher manda fazer. Há dias o carvoeiro, depois de o fazer parar no carro,
em plena rua, invectivou-o de punhos cerrados, e foi assim que conseguiu receer um débito atrazado. Por tudo isto, resolvi lá ir eu em pessoa.

%centerline{\rule[-1em]{0pt}{2.5em}. . . . . . .}

Uma escassa meia hora passada, a campainha do rico palacete do barão do
Chasco retinia com firmeza. E sem mais preguntas, o Sr.\ Oliveira faz-se anunciar pelo criado e pede para falar ao Sr.\ Barão. Mas o criado resmungou que
não estava.

-- Há equivoco da sua parte, certamente, pois vi entrar há pouco o seu amo.

Ah, então... Mas são ordens que tenho. Ele não recebe nin\-guém. Reco\-lheu esta noite muito tarde, já de madrugada, e está a descansar um pouco.

-- Sei disso. Jogou no casino da Póvoa, e depois de perder quanto levava
regressou um pouco incomodado... Não admira. Bem, eu esperarei que S. Exª
acabe o seu descanso.

O Sr.\ Oliveira acende um charuto, e mete outro nas mãos do criado.

-- Muito obrigado, mas eu não estou habituado a coisa tão fina.

-- Fume, ordenou o visitante. Ora quem me\-lhor pode fumar dos caros, do
que quem está ao serviço de tão boa casa?!

-- Está muito enganado, senhor. Ah, como as aparências iludem... Com
mu\-lher e cinco fi\-lhos, recebo cento e cinquenta escudos mensais com que os
hei-de sustentar e vestir. Ainda se os recebesse pontualmente...  Pois o\-lhe,
senhor, nem com o dôbro me pagava só o traba\-lho que tenho todos os dias em
aturar aquilo a que o meu amo chama os impertinentes.

-- O quê? -- preguntou o Sr.\ Oliveira.

-- Impertinentes, elucidou, são os cobradores. Logo que vejo a\-pro\-xi\-mar-se
alguém com uma pasta, ou papéis na mão, digo logo que o amo não está, que volte
passados dias. Que quere? São ordens...  Se tivesse dinheiro no bolso, seria
eu o primeiro a contentá-los só para os não aturar. Até tenho vergonha. E
ainda por cima é sôbre mim que se descarrega a ira do amo ao informá-lo
das pessoas que vinham para receber.

Decorria assim a conversa na escadaria da porta principal. O porteiro,
meio estonteado do charuto que saboreava, dizia pormenores que o interlocutor ouvia com interesse.

A aproxima\-ção dum rapaz veio fazer suspender a conversa. Muito direito,
lia-se nos dizeres do boné da sua farda o nome da confeitaria a cujo serviço
estava.

-- Vinha trazer...

-- O patrão não está, interrompeu o porteiro, desdobrando a factura que
tornou a devolver.

E voltando-se, segredou ao ouvido do Oliveira:

-- 10:500\$00, referentes ao ano passado... Contam-se por dúzias as vezes que
para cá teem caminhado da Confeitaria.

O rapaz limitou-se a murmurar entre dentes: ainda hoje não recebi um
centavo. Já corri 15 fregueses, e a resposta é sempre a mesma. Como notou
que o escutavam, continuou: -- Tudo donos de bons palacetes, e de um eu sei que
tem 4 automóveis todos luxuosos. Muitos destes senhores, que eu conheço, e aos
seus carros, vejo-os passar todas as noites à minha porta, em Valadares, em
direc\-ção a Espinho, acompanhados com senhoras que não são suas esposas, para
regressarem a deshoras. E o pior é de mim, a quem o patrão já ameaçou de despedir se continuassem a passar-se dias sem receber conta alguma.

Voltou costas o cobrador de confeitaria e o porteiro, enco\-lhendo os
ombros, explicou ao Sr.\ Oliveira:

-- É tudo assim hoje. Só queria que ouvisse há dias aqui o empregado da
Casa das Modas. A conta ali atingia uns noventa e cinco mil escudos. Ainda
há dias foi acrescentada com 10 contos e meio, importância de 3 casacos de peles que a Srª Baroneza encomendou para si e para as meninas, todos
iguais, que são uma maravi\-lha, porque eu conheço o artigo, mas por que
eu não dava dez tostões e meio. O que interessa, porém, é que a empregada trazia uma letra para os patrões assinarem, da dívida, que a casa pudesse
descontar. A letra assinaram-na êles, mas a casa veio queixar-se de que nenhum banco a descontava por falta de cobertura.

-- Mas o Sr.\ Barão devia ter muito crédito. Pois se só as suas propriedades de Valflor dão um rendimento anual que orça por 150 contos... E as
quintas que tem no Douro e Traz-os-Montes, que valem bem á vontade, ambas,
cinco mil contos?

-- Ah, então não sabe? Isso está tudo hipotecado, e em vésperas de ser
vendido. Só para o casamento da mais ve\-lha teve o Sr.\ Barão de vender
a quinta de Urjais que lhe deu 250 contos. Agora está a viver quasi só dos
rendimentos do Alentejo, cujas  herdades apenas dão uns 200 moios de trigo,
60 hectolitros de azeite e uns 10 quintais de cortiça. O que é isto para
a gran\-deza em que foi acostumado? Só em Paris, nês\-te último verão, constou
aqui que êle gastou, nos 2 mêses em que lá esteve, trezentos contos! E foi
sozinho... O\-lhe que de vestidos para a Srª Baroneza e para as fi\-lhas
trouxe oito malas para a esta\-ção do Porto, e outras tantas fui eu levantar
à esta\-ção de Gaia, que depositei, sem mais saber do seu destino, mas aqui posso afirmar que elas não entram. E as recep\-ções que aqui os senhores dão?!
Bem vê que há uma menina a casar... Cada baile fi\-ca por uma fortuna, e tudo
serve de pretexto para mais um, como sejam festas de anos na fa\-mí\-lia, aniversários do casamento dos Srs. Barões ou da menina, ou ainda dias de gala da
antiga fa\-mí\-lia real.

Mas agora reparo eu, continuou o porteiro, que o charuto me pôz a falar de mais. Ai que se os meus amos o soubessem, não estava mais um momento
ao seu serviço. Assegura-me que esta conversa não passa daqui!

E perante a afirma\-ção do Sr.\ Oliveira o porteiro asseverou que, de-resto, era ali que ganhava a sua vida e que não tinha se não a dizer bem dos
seus amos. -- Uns revezes na fortuna, quem os não tem? É de certeza que a
maioria dos que por aí vivem em bons palácios, ostentam automóveis e bri\-lhantes nos dedos, vive á custa de empréstimos. E senhoras da alta! Nem quero
falar mais. Mas aqui, com os senhores, nada disso acontece. Eles podem pagar
o luxo, e pagam-no. A demora está em que ás vezes as despesas são gran\-des, e
os rendimentos mal chegam... Nem nisto é nada á vista do pai do sr. Barão,
que, em vida, acendia charutos com notas de cem mil reis para se divertir, naqueles tempos em que cem mil reis valiam o que hoje valem dois contos e
meio.

Ouve-se baru\-lho no interior do palacete. -- É o amo, informa o porteiro,
que já saiu do seu quarto. Eu vou anunciá-lo, acrescentou, retirando-se.

Dirigindo-se por um corredor, o Melo encontra-se com a Madalena, uma
criada de sala que entrara para o serviço na véspera. Esta vinha de mau humor, e ao dar com o Melo reponta:

-- Ao fim do mês também quero ver a pontualidade comigo. Então não quere saber? Atendi agora o telefónio. Era da carvoaria que falavam, a preguntar quan\-do podiam mandar receber a conta, para evitar que o cobrador perdesse mais tempo em caminhadas baldadas. Transmiti o recado ao Sr.\ Barão, que
me disse que não podia tratar agora disso. E virando-se para mim chamou-me
importuna e invectivou-me para que não voltasse a incomodá-lo.

-- Pois eu, observou o Melo, vou anunciar-\-lhe um cobrador da agência de
automóveis.

-- Cai Troia, meu Deus, disse a Madalena.

E o Melo, seguindo o seu caminho, bateu timidamente á porta do escritório.

-- Entra, mandou o Barão. Que queres?

-- Está ali o Sr.\ Oliveira que pretende falar a V.\ Exª, respondeu humildemente o porteiro enquanto se dobrava em ângulo recto em prolongada vénia.

-- Qual Sr.\ Oliveira? -- preguntou o Barão. Não conheço nenhum Oliveira.
Dize que não estou.

-- Já lho disse, Sr.\ Barão. É o sr. Oliveira dos automóveis. Mas êle
pelos modos viu entrar V. Exa e há já mais de uma hora que está á porta.
Afirma que não arredará pé sem lhe falar.

-- Apre! -- resmungou o Barão, colérico. Então que entre.

O Sr.\ Oliveira foi imediatamente introduzido no escritório. Curvandose respeitosamente, expoz as dificuldades do negócio e as necessidades de
realizar dinheiros em dívida. Eis a razão porque se encontrava ali, certo
que S. Exª não deixaria de o atender.

-- Amanhã falaremos, disse o barão.

Pedia desculpa de insistir, mas tinha de ser hoje, observou com firmeza
o Oliveira. Tinha de solver nêsse mesmo dia um compromisso, e não podia
adiá-lo. De-mais já vinha de longe o débito, e estava cansado de mandar cobrar ao Sr.\ Barão sem resultado. E resolutamente:

-- Sr.\ Barão: Não saio daqui sem o pagamento dêste recibo. São 3 presta\-ções já que estão em dívida, no valor de 12 contos.

-- Não insista. Amanhã liquido.

-- Sr.\ Barão: Queria evitar-\-lhe o desgôsto de ainda hoje o automóvel
voltar para a casa... V.\ Exª sabe que, pelas condi\-ções estipuladas no
contracto, nada poderá revindicar das presta\-ções pagas que são pertença da
casa a título de desvaloriza\-ção da mercadoria.

-- Insolente! -- vociferou o barão. Com que então o senhor vem aqui para me insultar como a qualquer plebeu insolvente! A nobreza da minha linhagem obriga-me a convidá-lo a sair imediatamente da minha casa. Leve consigo o carro.

E premindo o botão duma campaínha, o barão deu ordens para que puzessem
o carro ás ordens do Sr.\ Oliveira, enquanto êste lhe estendia uma fo\-lha desdobrada com a quita\-ção das quantias devidas.

-- Não pretendo mais nada da sua casa, terminou o barão.

-- Peço perdão, mas eu já não tencionava entabolar mais contractos
com V.\ Exª.

O sr. Oliveira deu as boas-tardes, a que o barão não respondeu, e retirou-se.

Cá fora, o motorista com a sua farda azul em que luziam gran\-des botões
amarelos, lastimava a sua sorte ao entregar o carro que puzera a traba\-lhar.
De-certo iria ser despedido. Ainda ao menos se o patrão lhe pagasse os 5
mêses que lhe devia...

E o carro transpôz, enfim, o limiar do portão, sob o o\-lhar humedecido do
motorista, para rodar velozmente para a cocheira donde saira.

Ao mesmo tempo, o barão, com os cotovelos assentes na secretária, descansando a cabeça sôbre os punhos cerrados, invectivava a Providên\-cia tão hostil para a nobreza, cujos resplendores se apagavam a o\-lhos vistos ante a
plebe vitoriosa. Ai tempos, tempos! suspirava êle, mergu\-lhado na sua ira.


\centerline{\rule[-1em]{0pt}{2.5em}. . . . . . . . . . . . . . . . . . . .}


Enquanto se desenrolava êste episódio a distância, sucediam-se como habitualmente as conversas na barbearia do sr. Matias sôbre os mais diversos
factos. Era a hora a que costumava juntar-se o grosso da freguesia.

-- Então temos guerra ou não? -- pregunta um, como complemento ás boas tardes.

-- A Alemanha prefere ir tomando conta de tudo em paz. Por meio dos
plebiscitos ela vai apanhando o que lhe convém, responde outro do lado.

A conversa generaliza-se agora, por unanimidade, na necessidade de esmagar os países com tendên\-cias imperialistas, como focos de desordem con\-tí\-nua.

-- Não é o Dr.\ Eugé\-nio quem vai ali a passar? E o sr. Jeremias que
acabara de entrar para pescar com os seus óculos quem passa, aponta atravez
dos vidros. Já sabem que êle está noivo?

-- Com quem? -- inquirem uns do lado.

-- Vai casar com uma fi\-lha do Frazão. Namoram há muito. Ainda ôntem os
vi juntos na Foz. Corre que o casamento está para breve. E dirigindo-se a
um rapaz que se conservara sentado: -- Não é verdade, Toninho? Você é que
bem sabe disso, visto dar-se muito com êle.

-- Eu de nada sei, nem êle mo dizia, replicou o Toninho.

-- Ora deixe-se disso. Não queira deitar poeira aos o\-lhos dos outros!

-- Palavra que nada sei, confirmou.

Já ao tempo se comentava noutro sector de tão ilustre assembleia a
próxima elei\-ção do clube recreativo a que a maioria dos circunstantes pertencia. Opinavam uns que a direc\-ção do João Bernardes fôra a que me\-lhores passeios anuais proporcionara, como a volta ao Minho e a ida ao Algarve por
ocasião das amendoeiras em flor. Opinavam outros que o Fonseca lhes prometera uma excursão à i\-lha da Madeira, e que era homem para cumprir. -- Á Madeira! -- gritou toda aquela gente. Votemos todos pelo Fonseca, oh amigos!

E ali ficou resolvido que a direc\-ção presidida pelo Fonseca seria a
mais conveniente, embora tivessem de combater uma corrente afecta ao Bernardes.


\centerline{\rule[-1em]{0pt}{2.5em}. . . . . . . . . . . . . . . . . . . .}


Passados alguns dias, Eugé\-nio visitava uma sua antiga doente. Era uma
senhora idosa, da casa dos 80, espanhola, cujo sofrimento do cora\-ção levara
há já duas dezenas de anos um médico do Brazil, onde então se encontrava, a prognosticar-\-lhe curta vida. Mas lá ia resistindo, e agora, em Portugal,
a sua bronquite assanhara-se para mais dar que fazer ao cora\-ção já falido.
Ainda por cima aparecera-\-lhe a diabetes.
Há três anos esteve caída em coma e já nin\-guém dava nada por ela, a ponto de
os fi\-lhos chegarem a combinar com outros herdeiros, residentes na África, as
parti\-lhas a fazer dentro de breve. Insistia por que Eugé\-nio conhecia o seu
organismo como nin\-guém, visto tratá-la desde há anos, e não queria outro médico. Só em vê-lo entrar, todos os sofrimentos se lhe diluiam como por encanto e a face tornava-se logo risonha para dar os "buenos dias" num
por\-tu\-guês espanholado. Como é seu hábito, Eugé\-nio sentava-se junto da cama
da doente e conversava um bocado antes de começar o exame clínico.

Desta vez veio á baila o que se constava sôbre o próximo casamento de
Eugé\-nio.

-- Caso curioso, replicou o médico, eu de nada sei. Mas com quem?

-- Pues aqui se supe tudo. Nestas redondezas habla-se mucho disso. Quiere que lo diga? Es com uma hermosa fi\-lha del Sr.\ Frazão. Supuenho serem
dignos um del otro. E tiengo dito aqui mucha vez: Les desejo muchas e muchas felicidades, como amigo que consideramos el senhor Doutor.

E Eugé\-nio interrompeu a conversa, preparando o auscultador para examinar
a doente.

\begin{center}\bf\large Na aldeia\end{center}

Com o contínuo e monótono rodar dos anos, impossível de travar como
impossível é travar a evolu\-ção da humanidade que faz um folgazão adolescente onde há pouco não estava mais que uma crian\-ça engraçada, e um adulto onde estava um adolescente, que traz a vè\-lhice num abrir e fechar de o\-lhos
onde ainda recentemente floria a jovialidade, com êste monótono e implacável rolar dos anos a monotonia da vida de Eugé\-nio mantém-se a mesma.

Não era de admirar que o cabêlo encanecido já de há muito viesse exteriorizar o que se vai naquela alma cheia de afectos incompreendidos, nêsse cora\-ção tão ocupado pela paixão duma mu\-lher. Não era de admirar que o
uso ou abuso da razão o tivesse levado a precipitar-se num dêsses abismos
a que a condi\-ção humana está sujeita. Não era de admirar que ao amôr por
aquela mu\-lher substituísse o ódio.

Nada disso, porém. Nem os cabêlos ousam dizer o que se lhe vai no íntimo, nem a razão prodigiosa de coerência lhe permitiu desvi\-\mbox{ar-se} do caminho traçado, nem a pureza dos nobres sentimentos sofreu a mínima profana\-ção.

Em algures se lê que a loucura é inseparável do homem: umas vezes
toma-\-lhe a cabêça e deixa-\-lhe em paz o cora\-ção que nunca se empenha nos
desvarios a que ela é arrastada; outras vezes há na cabêça a frieza da razão, e ao cora\-ção desce a loucura para o perturbar com afectos. A loucura
de Eugé\-nio participa das duas modalidades, num equilíbrio ecléctico que
afirma em intensidade e em extensão, na mais alta e sublime ideia e sentimento
de amôr, em que participam simultâneamente o cérebro e o cora\-ção.

Por isso não é fácil ver modifi\-car-se a loucura de Eugé\-nio, se loucura
se pode chamar àquêle equilíbrio.

O tétrico verão está no seu apogeu com as tétricas consequências da
época balnear, que é como quem diz, Foz, ali próxima. Eugé\-nio tenta resistir mais uma vez, e uma vez mais para lá caminha como um autómato a quem a vontade se nega a opôr seu veto ao mí\-ni\-mo esfôrço.

Para que reproduzir os sucessos obtidos, se êstes em nada desmerecem
dos narrados tantas vezes para atestar uma vez mais quanto é monótono o
desfiar dos anos na costumada incoerência de Princesa?

Eugé\-nio não foi contudo tão assíduo nês\-te verão como nos anteriores,
e até em fins de Agosto resolveu retirar-se para umas termas e para a aldeia. Assim rolou um ano mais, em que a chaga de Eugé\-nio foi sangrar lá
para longe, ocultamente, no remanso das tílias e dos ulmeiros, com a imagem de
Princesa sempre presente no cérebro.

As termas são o refúgio duma sociedade em decomposi\-ção, encontrando
o expoente máximo nas exibi\-ções coreográfi\-cas chanceladas pelo anátema
do modernismo que tudo profana e corrompe, deturpa e estupefaz, em que o
hálito dos pares se entrechoca, fétido, como a ma\-té\-ria apodrecida pelo crime, em holocausto às termas de Roma.

Muito diferente é a aldeia. Aqui meditando no silêncio, podemos contemplar, mergu\-lhados numa letargia profunda, as belezas da noite com toda
a sua poesia, uma poesia de enigmas, com as sombras incertas e bruxuleantes das árvores, e a lua na sua fase ampla a circunscrever uma linha de
fogo nas etéreas alturas do céu, e as estrelas cintilantes, e os planetas,
e a superfície reflectora das águas fluviais...

Como tudo isto tão bem cala no espírito moço, e como tudo isto, nos transporta em arroubos de felicidade aos mares de rosas, em pensamentos ideais
e quiméricos!

Pensar, por exemplo, nêsse sêr infinitamente magnético, continente
duma multidão de perfumes inebriantes, bâlsamo das chagas passionais do
homem, altar onde assenta o modêlo de escultura; nessa mu\-lher que nos concedeu o amor solicitado, na esco\-lhida pelo nosso cora\-ção, em cuja companhia cortamos a monotonia das águas, com uma lancha cor de rosa, lan\-çando-\-lhe madrigais, esquecendo o mundo para conluirmos os cora\-ções em um só,
é o lenitivo que a Natureza mais prontamente oferece a quem ama. Porque,
como a noite, nada há mais cheio de poesia que êsse manancial de atractivos
que encerra a mu\-lher, como também, seme\-lhantemente á noite, nada há mais
enigmático que essa essa rosa das rosas que é a mesma mu\-lher -- ente para
quem se não sabe medir a oportunidade dum sorriso ou a sensa\-ção produzida
por uma amabilidade, ente, enfim, cujas portas do cora\-ção são tão delicadamente invioláveis como um sacrário que encerre a mais preciosa relíquia.

A aldeia de Eugé\-nio, recanto do Entre-Douro e Minho, tem toda a poesia
bucólica do terreno acidentado em que os horizontes variam por cada centena
de passos andados, e em que as modalidades do verde das searas, das relvas
e das frondosas árvores frutíferas, se multiplicam até ao infinito nos férteis vales, a anunciarem a abundância da Natureza pródiga.

Tem ainda outros encantos para Eugé\-nio a aldeia que o viu nascer. Todos os locais teem uma his\-tó\-ria ligada á sua infância. É um penedo que gostava de escalar outrora; é um sítio em que cometeu determinada diabrura,
que lhe valeu o prémio dum sorriso do seu Pai ou, pelo contrário, a admoesta\-ção; é um caminho íngreme por onde acompanhou até casa, algumas vezes
uma amiga de infância, com a tenra idade dêle, onde a simpatia mútua vegetava
num mixto de candura e ingenuidade. É uma pedra onde, já mais tarde, num sítio êrmo entre pinhais, encontrava comodidade para se instalar embebido nos
estudos médicos deixados para as férias, e á qual, ao meio da tarde, ia ter
uma gemada tonifi\-cante, pedra que hoje vai sendo conhecida pelo "penedo dos
dezassetes" por causa das classifi\-ca\-ções que Eugé\-nio tirava nas disciplinas ali repetidas.

É enfim uma multidão de coisas, as mais extravagantes, a que estão ligadas mil peripécias da sua infância que é como quem diz da sua vida. As
brincadeiras com os companheiros da escola primária, as longas caminhadas
que mais tarde tinha de fazer a cavalo para o caminho de ferro, nos estudos
do Porto, Coimbra e Lisboa, por madrugadas de frio enregelante atravez de montes, tudo isso se lhe representa na mente com profunda saudade dos tempos
idos.

Não era então a sua aldeia servida por estradas como hoje, em que de
todos os lados pode ser atingida; nem havia meios de comunica\-ção ao acesso
de todos, como hoje. A poesia do local era então maior, e a ingenuidade do
povo mais notável. Ter ido ao Porto, tornava-se nêsse tempo uma distin\-ção
de que poucos se vangloriavam.

Hoje a aldeia de Eugé\-nio é já atravessada por alguns automóveis diariamente, e já ali nin\-guém ignora os costumes da civiliza\-ção da cidade sem
excluir os artifícios de colorido da pele feminina. Os rádios também já por
ali vão aparecendo.

Não constitue assim a aldeia o local de repouso absoluto, de isolamento, entre pinhais, que busca toda a pessoa desejosa de descanso e de bons
ares a que tem jús após um ano de fadiga da cidade.

Em compensa\-ção, a vivenda de Eugé\-nio, situada num vale apertado entre
montes e retirada das estradas de comunica\-ção, é solitária e inacessível
aos ruídos da civiliza\-ção. Por sinal, ouvem-se dali me\-lhor, e vêem-se, os velocípedes da estrada da margem esquerda do rio Douro, ao longe, do que se
ouvem e vêem os que passam próximo na freguezia.

Eugé\-nio sente-se ali bem, tão perto da natureza como está. Sossêgo,
bons ares, alimenta\-ção sàdia e pura em que abundam as frutas co\-lhidas por
sua mão e ingeridas na mais natural sem-cerimónia, despreocupa\-ção na indumentária, boa amizade que em todos os habitantes conta, a vida simples e
salutar, enfim, a retemperá-lo das canseiras da cidade, eis o gôso que Eugério foi encontrar na sua aldeia natal nos restos do verão deste ano, além
dalguns passeios extensos que fez.

Sossegaria completamente o espírito? Não. Nem as ocupa\-ções das co\-lheitas a tomar-\-lhe o tempo lhe deslocaram da mente a imagem sempre presente
de Princesa. Uma vez viu-a em sonho. Formara-se um cortejo académico que
do jardim da vivenda de Eugé\-nio transpoz o portão tomando o caminho que
conduz ao rio. Uma modesta casa que existe abaixo, de caseiro, transformara-se na habita\-ção de Princesa. Esta e a irmã foram à beira do caminho ver
a passagem do cortejo, e retiraram em marcha acelerada, o que Eugé\-nio pôde
observar por entre os vinhedos e as frondosas frutíferas que separam ambas as habita\-ções.

Como havia de sossegar o espírito em local tão propenso à medita\-ção,
sobretudo quan\-do há uma ideia fixa a alimentá-lo?

Numa tarde pegou Eugé\-nio em papel e poz-se a escrever maquinalmente.
Muitas vezes o papel é o confessionário involuntário a que confiamos o
que se vai na alma. A resenha inconscientemente obtida constitue então
valioso relicário dos sentimentos, sem qualquer deforma\-ções que a
vista introspectiva é suscetível imprimir-\-lhe. Por isso não é de desprezar aquilo que Eugé\-nio escreveu, porque decerto andará em redor da ideia
fixa que o subjuga. Eis a transcri\-ção do autógrafo que maquinalmente redigiu:

-- ``Haverá idílio mais surpreendente que o de dois entes, beijados
pela frescura da aragem dos plátanos e dos carva\-lhos, imiscuídos nos
perfumes das tílias, e bafejados pela brisa do regato de água cristalina que saltita por entre amenos valados num murmúrio dôce, recordando a fase amorosa que antecedeu a sua união no paraíso terrestre,
que bem merecem? Ouvi-los lembrar-se, não obstante a felicidade que
ainda e sempre, talvez, os acompanha, ouví-los lembrar-se com saudade
da infância, quan\-do se afirmavam mutuamente amôr eterno entre os gorgeios que as avesitas alígeras, de múltiplos cambiantes, modulavam, parecendo lan\-çar-\-lhes uma bên\-ção superior em cada trinado, das ramarias
frondosas?

Oh! nada há mais dôce e belo que o amôr, cuja expansão os rigores
da borrasca ameniza e a noite gran\-de e fria de inverno torna poética
e a\-gra\-dá\-vel; que o cair da fô\-lha não estiola, nem a tenebrosa tempestade estorva!''

\begin{center}\bf\large NOVOS HORIZONTES\end{center}

Os mêses vão girando. As quadras de inverno não costumam ser ricas
de peripécias, antes pelo contrário monótonas, porquanto a sociedade cai
em letargia profunda. Tal, porém, não sucedeu êste ano de 1939.

O acaso, se é que existe, proporcionou novas modalidades na vida de
Eugé\-nio, como que tendentes a arrancá-lo á pacatez da sua habitual exis\-tên\-cia.

Um convite para acompanhar certa excursão á i\-lha da Madeira, que um
clube do Porto projectara, foi deferido. E, seguidamente, convidado para ingressar no mesmo clube, aceitou ainda.

Com a divisa de bem-servir, êste clube já de há muito conquistara
simpatia de Eugé\-nio pelas realiza\-ções práticas levadas a cabo por iniciativa sua.

Bem servir signifi\-ca contribuir na medida das suas fôrças para a felicidade e paz da humanidade, que, afinal, é a mais objectiva finalidade
da nossa exis\-tên\-cia. Por isso o fim do Clube apresentava-se ao espírito
de Eugé\-nio tão profundamente humano como belo.

Com efeito, em toda a sua vida, em todos os seus actos, ele tem procurado ser útil a-dentro das suas faculdades. Nem de outra forma concebe
vida. Sem actividade, ou com actividade mal conduzida, a vida não encontra
beleza, torna-se parasitária e causa estôrvo á sociedade. O prazer do
traba\-lho bem orientado e animado pelo ideal de bem-servir proporciona o
mais alto prazer da vida. Por isso não podia recusar o convite formulado
para ingressar nesta colectividade, cujo programa tão bem se coaduna com o
ideal da sua conduta, que é o de procurar valorizar quanto possível a
exis\-tên\-cia.

O facto que á primeira vista parece não ser digno de men\-ção especial,
por natural, vai ter gran\-de interêsse na vida de Eugé\-nio. Dêsse clube faz
parte o Sr.\ Frazão, e as fa\-mí\-lias dos associados, reúnem-se frequentes vezes com êstes para que o todo constitua uma só fa\-mí\-lia. Tal colectividade
haveria pois de proporcionar de futuro o estreitamento de rela\-ções entre
Eugé\-nio e Princesa.

O mundanismo falou logo, ao saber do caso. E não faltou quem se aprestasse a ir dizer á Sr.\ D.\ Maria Frazão que um médico havia que procurara
alistar-se na projectada excursão por causa duma sua fi\-lha.

A origem de tão infundado boato é que não é fácil descortiná-la, e
isso por três razões: porque Eugé\-nio nunca procurou alistar-se na excursão, nem como sócio, antes para isso foi convidado e instado; porque não
sabia á data se a fa\-mí\-lia Frazão se contaria no número dos excursionistas; e, finalmente, porque tentou desembaraçar-se do compromiso da viagem
ao saber mais tarde que quem o convidara havia desistido por motivo de
fôrça maior.

Não foi, pois, sem estranhar que Eugé\-nio soube do que se dizia a seu
respeito sem o menor fundamento. Mas não é isso tão próprio das pessoas
que derivam para a língua, ao serviço da imagina\-ção, a actividade que lhes
escasseia para fins mais proveitosos? Porque a verdade é esta: se bem
que o ingresso na sociedade da fa\-mí\-lia Frazão não fôsse de todo desa\-gra\-dá\-vel a Eugé\-nio, também não o alegrava demasiado por não entrever possibilidades de vitória na encarniçada luta desde há tantos anos travada.
Arma de dois gumes, a aproxima\-ção de Princesa tanto podia carrear o risonho porvir para o espírito de Eugé\-nio, como o desmoronamento definitivo dos sonhos arquitectados. Não foi, pois, sem resistência que acedeu ao convite, ao contrário do que se propalava, tanto mais que uma outra
contrariedade lhe trazia: a do aparato de que era preciso revestir-se,
tão oposto á sua maneira de ser.

Mas não ficou por aí o acaso que tão fértil havia de tornar o desenrolar dos acontecimentos. Por êste tempo eram os serviços de Eugé\-nio solicitados para a fa\-mí\-lia da Directora do estabelecimento infantil que
encontrámos anos atrás. Esta senhora protegera-o nas suas pretensas rela\-ções com a fa\-mí\-lia Frazão, e ela estava presa por amizade íntima aquela outra amiga da Srª.\ D.\ Maria que fizera já lisongeiras referências
a Eugé\-nio junto de Princesa. Eugé\-nio movido não só pelo brio profissional como também por um dever de gratidão, dever que
timbrava em respeitar religiosamente, obrigava-se a pôr notável carinho
e solicitude na pequena doentinha que lhe fôra confiada.

Esta casa, onde funcionara a escola a-dentro de cujas portas, numa
festa infantil de outrora, se proporcionou um encontro de Eugé\-nio com
Princesa, constituia ponto de reunião de amigas verdadeiras desta. Ofé\-lia
e sua irmã Marília, das mais íntimas amigas dela e primas da enfêrma, ali
se encontravam vezes amiúde com Eugé\-nio, e daí a pouco os três, sociáveis
como eram, estreitavam rela\-ções de mútua amizade. Tempo depois Eugé\-nio
ia mesmo a casa delas em serviço clínico requisitado para a sua mãi.

Esta fa\-mí\-lia, de que faziam parte, além de Ofélia e de Marília, seus
pais e o marido de Ofélia, era, pelo estreito parentesco com a primeira,
credora por afinidade da mesma gratidão de Eugé\-nio. No segundo andar que
os abrigava, nas imedia\-ções da igreja da Lapa, podia dizer-se com propriedade que não havia pais, fi\-lhos e genro, mas sim irmãos; a liberdade entre
uns e outros, a franca camaradagem, a harmonia transbordando de toda aquela gente eram incompatíveis com o habitual protocolo guardado entre país
e fi\-lhos. Todos ali traba\-lhavam concorrendo para o bem-estar comum. Riquezas materiais não se ostentavam, mas uma riqueza muito maior rescendia
de todo aquele ambiente -- a felicidade da harmonia. E que contagiosa era
essa felicidade! Era de desafiar o maior macambúzio a entrar ali sem
se deixar penetrar a breve trecho de tão a\-gra\-dá\-vel bem-estar como o
ali se respira, emoldurado pelo chistoso das conversas, pelas exibi\-ções
coreográfi\-cas, pelos sons do piano ou do canto. É o céu conquistado na
terra; é o contraste nítido e perfeito com o que acontece no seio da gran\-de maioria das fa\-mí\-lias em que as labaredas da desarmonia queimam os restos da vida desprotegida da afectividade e do bom-senso que nunca bafejaram tão soturnos lares.

Eugé\-nio era ali conduzido não só pelo cumprimento do dever como
ainda pelo bem estar que lhe proporcionava aquele ambiente tão a\-gra\-dá\-vel,
que o levava a demorar-se umas horas em ameno con\-ví\-vio com aquela fa\-mí\-lia.
Assim se passaram dias. A amizade mútua aumentava sempre. Veladas referências á simpatia de Eugé\-nio por Princesa, da parte de Ofélia e de Marília, começaram mesmo a traduzir a rutura dos restos cerimoniosos que
porventura existissem ainda.

Parecia haver de parte a parte o desejo de se abrirem francamente
todas aquelas almas sôbre aquilo que tanto preenchia o espírito de Eugé\-nio.

Derivavam-se tendenciosamente as conversas para a futura viagem á
Madeira, e faziam-se comentários àcêrca da travessia marítima com todos
os encantos de entre os quais sobrelevava um -- a companhia que Eugé\-nio
ia ter a bôrdo... ..

Inquiriu-se da arte coreográfi\-ca de Eugé\-nio; êste, tão afastado sempre do mundanismo, não dan\-çava. Verberou-se amargamente a falta, com a
qual sossobravam todas as possibilidades de estreitamento de rela\-ções de
Eugé\-nio com Princesa: e assentou-se em que Marília o poria pronto a dan\-çar até á data do embarque, para o que Eugé\-nio apareceria todas as noites em sua casa.

Decorreram as li\-ções sem novidade durante algumas noites. Eugé\-nio
aproveitava bem o tempo, aumentando a sua habilidade que o ia fazer nascer para o mundanismo, e Marília não escondia o seu orgu\-lho em assinalar
o progresso do discípulo, pondo em destaque a vontade dêste em atingir o
fim para que a dan\-ça não era mais que um meio.

Uma noite, após a li\-ção costumada, conversou-se demoradamente. Desapareceram os últimos preconceitos. Foi-se direito ao fim. Estáva\-mos a 10
de Março. O aniversário natalício da irmã de Princesa, que hoje completava 22 anos, foi o pretexto. Marília e Ofélia referiram pormenores das
façanhas de Eugé\-nio nas suas tentativas de aproxima\-ção com Princesa, que
mostravam bem quanto estavam na intimidade daquela sua amiga. Eugé\-nio de-resto sabia-o. Esses pormenores vieram por vezes esclarecer certas dúvidas
de Eugé\-nio, e outras vezes aumentá-las.

Uma estranheza causou logo de início as idades das duas irmãs, cujas
informa\-ções anteriores eram erróneas. Princesa ia ainda fazer 24 anos a
16 de Maio, quan\-do é certo que a ajuizar pelos cálculos do
jardineiro entrevistado devia ter alguns anos mais.

Mas graves confidên\-cias estavam agora reservadas a Eugé\-nio. A franqueza abriu-se de mais em mais. Ali ouviu Eugé\-nio quanto ele se havia tornado antipático ao grupo de Princesa, de que faziam parte Ofélia e Marília,
pela mesma razão com que se criam certas antipatias espontâneas que nada
explica. Não o conheciam, de -- facto, as companheiras de Princesa, e faziam
côro em criar-\-lhe uma situa\-ção de certo modo hostil.

-- Creia, rematou Ofélia, que o caso tem mudado radicalmente de aspecto.
Recorda-se daquelas raparigas de luto que há tempo aqui estavam de visita
e a quem o apresentamos? Também pertenciam ao grupo e eram das suas mais
renitentes inimigas; ao fim dumas horas de con\-ví\-vio confessaram-nos estarem
iludidas a seu respeito, e são desde então suas amigas. Nesta casa acontece
o mesmo, podendo vangloriar-se de nela só encontrar hoje amigos.

Eugé\-nio ouvia estupefacto êstes informes.

-- Pena é que o não conhecessemos há mais tempo, prosseguiu Ofélia
apoiada por Marília. Muito diferente teria sido a sequência dos factos que
se teem verifi\-cado. Há 3 anos todo grupo apoiou as simpatias dum belo
rapaz de quem Princesa se enamorou por, ocasião dum baile dado em 2ª feira
de Carnaval em casa dela. Interessante figura, esbelto, de cabelos encaracolados, vestindo irrepreensivelmente e dan\-çando a primor. Outros predicados,
porém, não tem, nem sequer o de saber articular uma conversa; e, pior que isso,
cedo se descobriu tratar-se dum vulgar aventureiro especializado em caça-fortunas. Está á frente duma casa de modas que, por sinal, ameaça ruína económica. Os Pais e Avós de Princesa contrariam tal aproxima\-ção, mas, talvez
por esse mesmo motivo, avolumou-se o entusiasmo dela pelo rapaz, porque Princesa é dotada de excessivo amor-próprio. Ela procurava-o, mais que ele a
ela, arranjando até nas compras pretexto para o ver no seu estabelecimento.

Um balde de água fria não teria produzido sôbre Eugé\-nio o efeito da
conversa ouvida. Afinal Princesa não era desprovida de temperamento afectivo, e a futilidade tão própria do sexo não era ainda desmentido aqui; as
suas simpatias iam sòmente para as banais exterioriza\-ções aparatosas do
mundanismo e não para a forma\-ção do homem. A superioridade de Princesa sofria pela primeira vez um abalo no conceito de Eugé\-nio, perante tão inesperada revela\-ção.

-- Mas, continuava Ofélia, o entusiasmo tem arrefecido sensivelmente. Tem
de facto. E a viagem á Madeira vem mesmo na ocasião mais oportuna, como que
de -- propósito. Admitimos mesmo a hipótese de os seus Pais procurarem na
aproxima\-ção de ambos o esquecimento do tal rapaz...

-- Não me é muito simpático pescar em águas turvas, repontou Eugé\-nio.
Servir de manequim nas mãos dos pais para conseguirem outros fins, não está
certo. Admitirei mesmo a hipótese de não procurar relacionar-me com Princesa depois das informa\-ções que ouvi. Conto de-resto com a antipatia da
Sra D.\ Maria; e o Sr.\ Frazão, êsse, parece-me bem ter nele um amigo a avaliar
pela forma como me trata sempre que nos encontramos. A irmã de Princesa, finalmente, sei quanto me hostiliza pelos sorrisos trocistas que nos seus
lábios encontro sempre que me vê.

-- Talvez esteja muito enganado. O Sr.\ Frazão é sobretudo um diplomata
que nunca mostra o que sente ou pensa; uma coisa se sabe apenas a respeito
de toda esta questão: é que afasta sistematicamente todo aquele que sonhar
ser pretendente á mão das suas fi\-lhas, não se sabendo bem o motivo desta
atitude. A Sra D.\ Maria é natural que tenha mudado a primitiva hostilidade
para consigo, pela razão exposta. A irmã de Princesa não é nada aquilo que
pensa: no sorriso dela não deve ver mais que a ingenuidade que a ca\-ra\-cte\-ri\-za. Receie mais da troça ofuscada que da simplicidade franca. Repito-o:
a viagem á Madeira vem na ocasião mais oportuna que se pode imaginar. Aproveite-a bem, e nêsse sentido todos nós fazemos ardentes votos. Quanto a pescar em águas turvas, ou servir de manequim, aconse\-lho-o a que se dê por desconhecedor do que aqui ouviu. Apesar de tudo, Princesa tem muitos merecimentos. É uma rapariga que tem predicados excepcionais, e constitue o ídolo dos
pais e avós. Fraquezas todas as teem, e é possível que para o facto contribuissem duas amigas que desde há 2 anos a acompanham muito e que conhecerá
a bôrdo como companheiras de viagem, para quem o valor do rapaz se resume na
trilogia -- ser boa figura, dan\-çar bem e ter dinheiro. Aproveite a ocasião de
se a\-pro\-xi\-mar de Princesa na excursão, e saiba vencer qualquer resistência
que muito naturalmente lhe aparecerá.

Dias depois, uma a\-gra\-dá\-vel notícia telegráfi\-ca dava ensejo a que se trocassem brindes no fim de jantar naquele ambiente sempre alegre, e agora com
razão ainda mais: Marília fora aprovada no curso recentemente ultimado. Eugé\-nio ao chegar para a sua li\-ção de dan\-ça veio encontrar a fa\-mí\-lia de Marília em festa, associando-se imediatamente com participa\-ção nos brindes. As
taças de champanhe esvasiaram-se não sem se haver bebido pela efectiva\-ção
das aspira\-ções de Eugé\-nio, por iniciativa de Ofélia e de Marília.

Após as revela\-ções ouvidas, uma pregunta teimava em aflorar ao espírito
de Eugé\-nio: se Princesa tinha as suas simpatias por outrem, que aliás
nunca Eugé\-nio encontrou nas imedia\-ções da sua casa, a que atribuir os o\-lhares
insistentes e inequívocos para sí? Só por troça se compreendiam então. E
pela sua mente perpassava rude a frase ouvida: -- receie mais da troça ofuscada que da simplicidade franca.

Uma noite formulou aquela pregunta a Marília, enquanto dan\-çavam um tango.

-- O quê? ripostou surpreendida a interlocutora. Não será engano seu,
resultante de falsa interpreta\-ção num sentido por si desejado? Outra pessoa
fôsse quem mo dissesse que não lhe daria crédito...

-- Pode crer. Não há, não pode haver falsa interpreta\-ção. Ofélia é disso
testemunha, por a acompanhar algumas das vezes em que o facto se deu. Até
duma vez em que na Foz o grupo se abancava a uma mêsa de chá vizinha, foi
Ofélia quem segredou a Princesa a proximidade de Eugé\-nio, e o facto motivou
a sua desloca\-ção para sítio mais conveniente donde pôde fitar constantemente Eugé\-nio de frente.

Marilia ouviu com estupefac\-ção esta e outras narrativas seme\-lhantes,
com alusão ao rubor que por vezes subia ás faces de Princesa ao ser surpreendida a fitar Eugé\-nio, e acabou por confessar não saber como interpretar o
estranho facto.

-- Prepare-se para a viagem á Madeira, e actue com serenidade -- rematou
Marília. Fixe bem o que lhe vou dizer. As raparigas, forçosa sou a confessá-lo,
pren\-dem-se mais com o exterior do que com o interior. É uma questão de psicologia feminina. Poucas são aquelas para quem a indumentária não é o principal chamariz. Sei que é avêsso a exibicionismos, mas necessita modifi\-car-se.
Aquelas, que, como eu, apreciam o rapaz pelo seu valor intrínseco, são a excep\-ção.
A mu\-lher é, por temperamento, sensível ás exibi\-ções. Sendo assim por
natureza, impossível se torna chamá-la á razão, que a sensibilidade para o
fútil suplanta. Reconheço representar isso uma inferioridade do sexo, mas
que quere? A sua psicologia tem de ir buscar elementos para a vitória. De
contrário é remar contra a maré. As excep\-ções não contam, tal a minoria que
constituem. Eu que já lhe disse não avaliar o rapaz pela exterioriza\-ção
mundana também tenho a minha futilidade: a preocupa\-ção do esmero com que
me apresento nas reuniões dan\-çantes. Vestido que já exibisse, não gosto de o
tornar a mostrar.

-- Na dan\-ça já vai menos mal, prosseguiu Marília. Não fará má figura. Só
tenho pena que não tenha praticado com mais raparigas, e receio que a emo\-ção
o embarace na viagem ao defrontar-se com Princesa. Esta dan\-ça muito bem,
mas a sua escola é a minha.

E os tangos, as valsas, os fox-trots, o lambeth walk e o palai glyd
sucediam-se, e eram todos eles já familiares para Eugé\-nio, a-menos na classifi\-ca\-ção das músicas que Eugé\-nio tinha dificuldade em etiquetar.

-- Quem me dera, dizia Marília poder transformar-me em insecto voador
para poder seguir viagem e segredar-\-lhe a espécie de música que se faz
ouvir... O seu êxito seria então completo.

A hora tardia, em que esta conversa se desenrolava, vincava-se já no
semblante da interlocutora a reclamar o justíssimo repouso de quem após
um dia de intensa actividade ainda se ocupa á noite com o ensino da dan\-ça.

\begin{center}\bf\large Preparativos de viagem\end{center}

Entretanto estávamos em vésperas de partida. Marília sempre solícita
em adextrar Eugé\-nio na arte de agradar, dava-\-lhe conse\-lhos frequentes. O cora\-ção da mu\-lher é um tesouro inexgotável de afectos, e o desejo de Marília
pelo bom sucesso de Eugé\-nio era bem patente.

Um pouco de amor-próprio é natural que pendesse no espírito de Marília
ao pretender a vitória de Eugé\-nio, que também era a sua. Porque a luta não
era agora simplesmente entre Princesa e Eugé\-nio, mas sim entre a primeira dum lado e o segundo e Marília do outro.

Eugé\-nio sempre avêsso aos exibicionismos da indumentária experimenta
pela primeira vez notável modifi\-ca\-ção sob êste aspecto. Já dan\-ça; faltava
agora a apresenta\-ção exterior, a única que, com a dan\-ça, interessa á frivolidade temperamental do belo sexo. Entrou o guarda-roupa em reboliço, e minuciosamente modernizados fatos, camisas, gravatas, calçado.

Incapaz até aqui de se preocupar com as futilidades do mundanismo ôco, Eugé\-nio deixa-se absorver com estas coisas. Nunca possuira um fato de cerimónia, de que agora necessitou, e mais coisas bem contrárias ao seu espírito
simples e modesto passaram a ter lugar na sua mala de viagem.

Eugé\-nio incapaz de gastar mal um centavo, abre agora a sua carteira
para dissipar dinheiro sem conta, tendo por único fito não fazer fraca figura no meio do pedantismo imbecil, esquecido já de quanto a ostenta\-ção esconde a escória social.

-- Noto que me inferiorizo, desabafou uma vez junto de Marília. Causa-me revolta a passividade com que me deixo dominar pelas exi\-gên\-cias duma sociedade combalida. Enoja-me mesmo.

-- Assim se torna necessário, porque só a inferioridade singra nês\-te
mundo, objectou Marília. Triste é dizê-lo, mas é a verdade. Todo modernizado,
desde indumentária até ao confortável carro que agora possue, a sua vitória
é quase certa. Orarei a favor da efectiva\-ção do seu objetivo, quan\-do
partir.

-- Onde está, porém, a modéstia que conheci em Princesa como a qualidade
mais simpática da sua pessoa?

-- São contos largos, elucidou Marília. Quando a conheceu, a sua situa\-ção
financeira estava bastante comprometida. A economia, aliada a modéstia, impunha-se então como necessidade. Depois a venda de terrenos onde a Avenida se
rompeu veio dar novo fôlego á fortuna exhaurida. Presentemente a situa\-ção
é desafogada, embora os rendimentos nem sempre bastem para cobrir as despezas.
Mas isso são já contos muito largos, como dizia. Só lhe acrescento que a
aproxima\-ção com aquelas duas novas amigas de que lhe falei modificou em
muito toda a sua maneira de viver.

E terminou, tomando ares de seriedade: Com o meu feitio comunicativo
escorrego em dizer-\-lhe tudo. Não digo mais nada, pronto! Tenho dito mais do
queria. -- E mudando de tom: O que vale é saber a quem o digo. O\-lhe que
segredos meus não os revelaria; são o mais precioso e intangível relicário
da mu\-lher.

Que precioso relicário de segrêdos terá aquele cora\-ção de Marília, não
é fácil sondar. Algum sonho róseo de quimeras, umas madeixas loiras a prendê-la no mais íntimo do seu sêr? Tentar escalpela-lo seria mesmo um sacrilégio.

Marília é dotada duma sensibilidade gran\-de, sobretudo para a mú\-si\-ca.
Começa sentindo a música, e sente a música meditando, interio\-rizando-se. O semblante reveste-se-\-lhe então da gravidade de quem tudo vive dentro de si. Mas
é só durante as audi\-ções musicadas que fazem vibrar aquele sêr por forma notável. Fora disso, Marília nenhuns vislumbres tem de autismo. A palavra dada
ao homem tem um duplo fim: nuns serve para exprimir o pensamento, noutros
para o esconder. Em espíritos francos como o de Marília, a palavra exterioriza fielmente o que se lhe vai na alma. Quando mesmo assim não acontecesse, os
o\-lhos sempre sorridentes, fulgurantes de sinceridade na sua matiz ver\-\mbox{de-glauca}, falariam por si. Na intimidade, Marília diz o que pensa, e pensa o que diz 
 sem artefactos. Até quem sabe se os segredos que existam naquele enigmático cora\-ção -- como enigmáticos são todos os cora\-ções de mu\-lher -- haverão
de ser revelados a Eugé\-nio, a-despeito-dos protestos feitos?

Marília alia ao temperamento franco a concentra\-ção ponderada, que lhe
dá jús a constituir-se em confidente de quem quer que seja. Com certo
orgu\-lho confessa que não sabe por que artes alguns rapazes se lhe teem
aberto já em confidên\-cia.

Um dia Eugé\-nio preguntou a Marília se Princesa é ou não expansiva.

-- Não, respondeu. Princesa tudo vive em si; é reservada, e nin\-guém
sabe o que pensa, só muito dificilmente exterioriza aquilo que sente.

Eugé\-nio, após tantos anos de luta persistente, tem ouvido com a franqueza peculiar de Marília revela\-ções extraordinárias. As modifi\-ca\-ções
da conduta sofridas por Princesa nos últimos anos tempos desloca\-ram-na
um pouco dos seus ideais. Onde pára aquela modéstia a realçar a beleza,
e a sobriedade afectiva a seleccionar sob a temperan\-ça da razão a esco\-lha dum par? Deslizada para o mundanismo, embebeu-se agora dos impulsos
a que conduz a mediocridade exigente de exibicionismos.

Eugé\-nio ainda aqui desmente a proverbial volubilidade atribuída ao
seu sexo. Passou gran\-de parte da sua mocidade em busca dum ideal que nestes tempos de materialismo cada vez se torna mais raro. Julgou encontrá-lo, e para êle viveu com tenacidade invulgar.

A figura outrora ideal de Princesa começa agora a decair, progressivamente, á medida que entra nos seus segredos através de pessoas que
consigo privam intimamente. Nuvem que passa, não deixando atrás de si indícios? Mais de-perto, Eugé\-nio desejaria estudar a sua psicologia durante a viagem, mas antevê-se que a brevidade desta, e ainda o caráter reservado de Princesa, obstem ao seu propósito.

Eugé\-nio tenta num último esfôrço adaptar-se um pouco mais à hipocrisia mundana. Será persistente nesta nova fase, êle que tão avêsso foi
sempre ao mundanismo? O temperamento não se modifi\-ca, e fácil é prever
que a máscara de que pretende revestir-se artificialmente será efémera.
E mais: começa a mostrar quão simpático se tornaria ocasionar-se-\-lhe
pretexto, de peso para desistir da próxima excursão...

Eugé\-nio vai desempenhar uma farça na viagem. Irrepreensivelmente
vestido, e dan\-çando qualquer coisa, vai assim apetrechado com as qualidades
primordiais que obscureceriam de-sobejo as que em ma\-té\-ria de carácter
porventura lhe faltassem. A apresenta\-ção vai agora sobrepor-se á conduta.
Aventureiro ou estroina que fôsse, desonesto ou ignorante, não importa. Vai
bem provido de vestuário, de maneira a poder exibir-se de modo diferente
a-dentro-de cada dia. Mostrar-se-á chique, enfim, e eis tudo. O resto é
imaterial e os tempos que correm são de materialismo. A moral e o intelecto não contam, e quaisquer fa\-lhas estariam supridas pelo espectáculo
da aparência que é o mais sedutor chamariz da mu\-lher.

Entretanto, pela cidade corre cada vez mais, em intensidade e em extensão, a próxima viagem de Eugé\-nio e de Princesa, abundantemente condimentada
pelos comentários ao sabor de cada um, como se fôra notícia importante
de que dependesse uma sucessão dinástica a pesar nos destinos da na\-ção.

\begin{center}\bf\large A travessia do Atlântico\end{center}

As férias da Páscoa chegam, finalmente. O tempo apresenta-se ora francamente chuvoso, ora densamente brumoso, por vezes tempestuoso, e sempre tristonho, a lembrar a passagem recente do equinócio. Fazem-se previsões quanto
ao tempo em que vai decorrer a próxima excursão, aprazada para a semana imediata ás festas pascais, mas os vaticínios são discordantes conforme o maior
ou menor optimismo de cada um. O domingo de Páscoa, véspera da partida, esteve
ainda péssimo de chuva miúda e vento.

Uma pregunta trazia Eugé\-nio na mente, desde há muito, para fazer a Marília, antes da partida. Fê-la agora:

-- Princesa não será capaz de qualquer incorrec\-ção no momento em que
me dirija a ela?

-- Não; não pense nisso. É certo que mudou muito desde o advento das
suas novas companhias... Mas não, não admitamos tal hipótese. Princesa é
muito correcta.

A contrastar com os dias anteriores, a manhã de segunda-feira aparece
radiosa de sol a querer transpôr as nuvens ainda espessas, por entre as quais
ele se mostra a espaços irregulares. Foi nesta manhã que Eugé\-nio partiu
para Lisboa, a juntar-se aos restantes excursionistas que em gran\-de parte
já tinham seguido para ali na véspera. De entre êstes últimos contava-se a
fa\-mí\-lia de Princesa.

O barco, embandeirado em arco e encimado pela flâmula do clube excursionista, ostentava ares de festa quan\-do Eugé\-nio nele entrou. O vai-vem dos
recém-chegados dava mais alacridade áquele ambiente de alegria, em que comumgavam cêrca de duas centenas de pessoas ávidas de sensa\-ções, para muitos novas, dum passeio marítimo.

A saída estava marcada para as 15 1/2 horas; um quarto de hora passado,
o vapor levantava ferro por entre acenos de despedidas. O vento soprava agora de rijo, e fazia bastante mar. A atmosfera carregada de bruma, deixando desprender agora uma chuva miudinha e basta, a custo permitia divisar o casario
urbano que se afastava gradualmente. Numa breve cerimónia consideram-se apresentados todos os excursionistas que de ora avante irão constituir uma só
fa\-mí\-lia, e serve-se o chá. Todos correm aos tombadi\-lhos no desejo de verem
os últimos panoramas de terra. Foi ali que pela primeira vez na viagem Eugé\-nio avistou Princesa. Eram 16 horas.

A pouco e pouco se passa o Mosteiro dos Jerónimos e a tôrre do Bugio, e
as águas do Tejo são substituidas pelas do Atlântico. A ondula\-ção é forte
com predomínio da vaga de estibordo a bombordo, e o barco rodopia em parafuso. Já no semblante de muitos se nota acentuada palidez, e em todos o embaraço de se deslocarem sem acrobacias de equilíbrio. As cadeiras de bordo,
dispostas no tombadi\-lho, são totalmente utilizadas por aqueles que só na posi\-ção horizontal encontram alívio para o mal que se avizinha.

O vapor afasta-se cada vez mais, e as trevas duma noite antecipada pela
chuva tempestuosa restringem em absoluto o horizonte aos limites do barco.
Apenas alguns reflexos da ilumina\-ção interior se espe\-lham sôbre a água circundante permitindo ver as vagas de travez na sua constante fúria.

O salão de festas é invadido pela gente moça que conserva o bem-estar.
Princesa ao piano toca um Lambeth, que Eugé\-nio e todos os mais dan\-çam conforme o permite o forte baloiço sentido, até que um criado, tangendo um timbale,
veio anunciar a hora de jantar.

Á mesa poucas pessoas se serviram, preferindo as restantes comer nos
lugares em que se haviam instalado horizontalmente.

A orquestra começou então, e dançou-se com alguma anima\-ção enquanto que
o número de baixas ia aumentando. Tarde já, toda aquela popula\-ção adormeceu,
uma parte espa\-lhada pelo jardim de inverno, outra pelo salão, e os mais animosos
instalados nos seus aposentos privativos.

Eugé\-nio acordou cedo. Pela vigia da sua câmara entravam raios de sol
dum dia límpido, mas o barco baloiçava ainda fortemente, como na véspera, embalando-o em vários sentidos. Em certos momentos afigurava-se-\-lhe faltar
apoio ao barco, para submergir na imensidade da água. Um ou outro safanão
mais violento sacudia de quan\-do em vez o barco.

Eram para si sensa\-ções novas que Eugé\-nio procurava analizar, ainda deitado. Impunha-se porém a curiosidade de apreciar o mar, e urgia que abandonasse a câmara. Senta-se no leito pronto a levantar-se, e então uma nova
sensa\-ção, desta vez trágica, o obriga a deitar-se rapidamente. Sentira vertigens, que a horizontalidade desvaneceu. A uma nova tentativa de se levantar
um estado nauseoso o acometeu, e, a outra, vómitos repetidos e impertinentes
dum conteúdo gástrico limitado aos sucos, que o prostraram.

Repetiram-se as tentativas frustradas por tão incómodo enjôo, até que,
num esfôrço máximo, saltou da cama para se vestir, sempre entre vómitos, meio
exausto, cambaleando a cada sacudidela do barco, e conseguiu enfim ir tomar
um pequeno almôço para ir engrossar as fileiras daqueles que nos tombadi\-lhos
procuravam o repouso nas cadeiras de bordo.

As fisionomias inexpressivas, e pálidas, eram já inúmeras. A quasi totalidade dos passageiros tinha-se ido abaixo. As laranjas com os seus efeitos
desenjoativos tiveram largo consumo; constituiram mesmo refei\-ção exclusiva
para muitos.

Após o almôço, a que poucos assistiram, Eugé\-nio recobrou ânimo. O mar,
de-resto, acalmou sensivelmente. Nos tombadi\-lhos superiores pôde então dar-se
largas aos folguedos de bordo; as corridas de sacos, as corridas a 3 pés, a
trac\-ção da corda, encurtaram a tarde dessa 3ª feira tragicamente começada.
Algumas pessoas permaneciam indiferentes ás competi\-ções desportivas, e dentre
elas contava-se Princesa, estendida na sua cadeira com as faces lívidas e a
modorra do enjôo.

Seguiram-se, após o serviço do chá, as corridas de cavalos
e de automóveis, para que algumas gentís passageiras passavam senhas diferentemente coloridas. Nas últimas corridas também a irmã de Princesa vendeu
senhas, mas, caso curioso, ao contrário do que acontecêra com as demais, passou
por Eugé\-nio sem lhas oferecer; e foi a sua côr, a verde, que venceu.

Á noite, após o jantar, dançou-se animadamente. Eugé\-nio não ficou muito
atrás dos restantes. Convidou todas as raparigas do salão para dan\-çar e, por
último, a irmã de Princesa. Além de nenhuma contra-indica\-ção o inibir de o
fazer, era um dever de cortezia. Julieta pareceu de-momento surpreendida com o
facto, mas acedeu. Princesa o\-lhando através duma janela do tombadi\-lho seguia
a dan\-ça. O pior porém é que ao repetir-se a música, Julieta escusou-se a continuar a dan\-çar com Eugé\-nio.

Servem-se refrescos alcoolizados com laranja e pedaços de fruta, e a um
e um todos se fôram separando para as suas câmaras de dormir ou, os mais pusilânimes, para as cadeiras dos tombadi\-lhos. Tornava-se necessário repousar,
para ao alvorecer ver ao longe o perfil de Porto Santo e, do nascer do sol,
gozar o panorama da i\-lha da Madeira, após um dia completo em que se não viu
se não água.

\begin{center}\bf\large Á entrada do Arquipélago\end{center}

Eugé\-nio acordou cedo, mal ainda se via. O mar estava então calmo, e a temperatura ameníssima. O receio de que se repetisse o mal-estar que o acometera
na véspera rapidamente se desvaneceu ao saltar da cama em direc\-ção á escoti\-lha
para me\-lhor apreciar o nascer do dia por sôbre as águas quietas do oceano. Nunca como então lhe pareceu tão aprazível a estadia sôbre o mar.

O silêncio era absoluto, apenas entrecortado pelo dôce sussurro das mansas
vagas.

Ao longe desenhava-se já, por entre a penumbra crepuscular, o con\-tôr\-no de
Porto Santo. Mais algumas dezenas de mi\-lhas percorridas estaremos, pois, no termo da excursão.

O que será dado ver aos nossos o\-lhos daqui a pouco? Quais as primeiras
im\-pres\-sões que sentiremos ao a\-pro\-xi\-marmo-nos daquelas belezas universalmente
aprègoados sôbre a Flor do Oceano?

Eugé\-nio havia-se preparado para a viagem, fo\-lheando todos os livros que
pôde consultar sôbre a I\-lha da Madeira; com os conhecimentos adquiridos chegara até a esboçar uma palestra, que não chegou a realizar-se por o mar nos não
haver proporcionado o necessário bem-estar compatível com a boa disposi\-ção
dos auditores.

E agora, que estamos nas águas do Arquipélago, já próximos da almejada i\-lha,
antecipemo-nos um pouco evocando um resumo da sua his\-tó\-ria. É precisamente
aquela palestra planeada, cuja leitura se segue, que nos vai mostrar êsse poucochinho de his\-tó\-ria, bordada com outras considera\-ções sôbre aquilo que se val
ver dentro de breves horas.

Quando em 1418 os cavaleiros da casa do Infante D.\ Henrique, Gon\-çalves Zarco e Tristão Vaz, se dirigiam para costa da Guiné, uma tempestade desviou-os da sua rota arrastando-os até uma i\-lha que, pelo feliz abrigo fornecido, foi denominada de Porto Santo.

Estava descoberta a primeira i\-lha do Arquipélago da Madeira, a 950
Km. dọ nosso cabo da Roca.

Surpreza maior, porém, estava reservada no ano seguinte àqueles navegadores. Prosseguindo para sudeste, uma nova i\-lha 15 vezes maior que o
Porto Santo se lhes deparou, densíssima floresta que o clima africano arrancara abundantemente daquele solo, faustosa cornucópia de flores e frutos a engalanar tão sorridente terra; tinha a configura\-ção dum barco
de qui\-lha voltada para o ar e de proa muito alongada, e media de comprido
65 Km, e de largo 22 sòmente. Estavam na i\-lha da Madeira, assim chamada
pela densidade florestal que a ca\-ra\-cte\-ri\-zava.

Demorou tempo a coloniza\-ção, sendo necessário que os incêndios abrissem lugar para as edifi\-ca\-ções, e que os escravos alcandorados a centenas
de metros de altura pelas escarpas dos píncaros com que a natureza vulcânica dotou a Madeira, construíssem as magnífi\-cas levadas para a condu\-ção de água dos planaltos até ás vertentes menos providas.

A cana de açúcar, trazida da Sicília, foi a primeira cultura explorada na i\-lha, chegando a produzir 400 mil arrobas de açúcar. Transplantada
depois a cana para o Brazil, decaiu a indústria sacarina ali. Começou então a
planta\-ção das vinhas, que em 1846 as doenças  destruiram, mas ainda hoje
constituem fonte de receita notável, pois em 1920,
atingindo somente metade de uma centena de anos atraz, produziram 55.500
hectolitros do afamado vinho da Madeira, ou seja 1/5 da produ\-ção vínicola do nosso Douro.

Uma cidade antiga, de ruas tortuosas e estreitas pavimentadas com pedras ovais de basalto negro, sem pó
nem lama, e servida por uma praia das mesmas pedras negras ergue-se hoje magestosamente em anfiteatro com os seus 25.000 habitantes, na costa sul da
i\-lha -- o Funchal -- em cuja baía os barcos de excursionistas costumam entrar de
manhã com um céu claro e o sol a nascer, para maior encanto lhe emprestar com a sua poa\-lha de ouro a beijar tão paradisíacas
paragens.

Paradisíacas, sim. A meio da distância entre os confins da Europa e da
África, nós iremos gozar durante 3 dias um clima de suavíssima amenidade,
umas paisagens onde o belo se casa com o horrível, uma luxuriosa vegeta\-ção
em que encontraremos nos primeiros 250 metros de altitude as bananeiras e
os catos, a cana do açúcar, o ananaz, a goiaba, a nona, e depois, subindo mais,
as vinhas, os cereais e pastagens e os castanheiros, os pinheiros, e, por último, os loureiros e as urzes. Eis o que nos será dado observar, a todos. Para
as senhoras um surpreendente mimo está reservado a deliciar os seus mimosos
o\-lhos: o mimo das rendas e bordados, cuja indústria atinge aqui os cumes
da delicadeza artística, ao lado das flores de cêra e de penas, e que de-certo
irão constituir indelével lembran\-ça de a\-gra\-dá\-vel excursão.

Paradisíacas paragens são aquelas para onde nos dirigimos, na mais salutar familiariadade rotária, em delicioso passeio, a demandar em êxtase de
felicidade a Madeira por sôbre as águas atlânticas outrora tão sulcadas
pelas galés portuguesíssimas dos navegadores sempre ávidos em acrescentar
novos mundos ao mundo! Hoje, como então, procuramos acrescentar novos mundos
ao mundo -- não novos mundos materiais, mas novos ambientes de harmonia entre
os povos, de que tanto carece o nosso orbe e em que tanto se empenha a gran\-de fa\-mí\-lia rotária.

Fôsse a harmonia em que viajamos o eco da harmonia da gran\-de massa de
cidadãos que povoa a terra, e teríamos conquistado o almejado céu em vida.
Para quê ambi\-ções, ódios, invejas e quejandos sentimentos que inferiorizam o
homem na escala animal, durante a tão efémera exis\-tên\-cia que temos sôbre o
globo? Mais paz entre os povos, menos dissidên\-cias, como mais paz e menos
dissidên\-cias ambicionaria ver a-dentro-do lar que é a maior e me\-lhor vida
que vivemos.

A profissão que abraço é daquelas que mais íntimo contacto proporcio
nam no seio das fa\-mí\-lias. Como é triste sentir, desde a classifi\-cada de alta
sociedade até á sociedade proletária, a lama em que vegeta grandíssimo número de fa\-mí\-lias, onde a harmonia é um mito e que a felicidade nunca bafejou!
Pouquís\-si\-mos são os lares em que se respira o salutar ambiente de amor a
impregnar as fa\-mí\-lias dos arroubos de felicidade a que todos temos direito.
Não necessitamos esperar a morte para merecermos os prémios celestiais ou
os castigos infernais, que os temos cá na terra durante a vida. Temos o céu
ou o inferno no nosso lar, consoante a directriz que dermos á nossa exis\-tên\-cia.

Desde o lar até ao todo universal, desde a unidade social até ao conjunto em globo, desde as rela\-ções entre os familiares até ás rela\-ções internacionais, uma crise há, tremenda, que aniquila as exis\-tên\-cias -- é a crise do
bom-senso. O senso é a alavanca que modifi\-ca radicalmente a nossa vida, transformando, como por magia, em prémio o que se nos anto\-lhava como castigo, em
céu o que não era outra coisa que inferno.

O bom-senso, apoiado numa inteligência sã, modifi\-ca a nossa exis\-tên\-cia.
Procuremos a paz entre os povos, a começar pela paz entre as fa\-mí\-lias de
que aquela depende.

Á inteligência humana devemos a seguran\-ça com que viajamos nês\-tes dôces momentos sôbre as águas do incomensurável mar, numa embarca\-ção segura
e confortável, seguindo uma rota sem desvios. Porque não transportamos nós
a inteligência para a rota da nossa vida, procurando o con\-fôr\-to do lar e a
harmonia da vida e a paz de todos ? Pois para que nos serve a inteligência
senão para nos adaptarmos da me\-lhor forma á vida?

Que a inteligência ilumine todos, e as retalia\-ções políticas, o facciosismo, deixarão de existir como elementos duma sensibilidade doentia que
nos atola no lodaçal em que ameaça submergir a sociedade. Apregoar e praticar a harmonia, desde o lar até á sociedade em geral, é fazer obra construtiva; é conquistar a paz, é conquistar a felicidade ; é viver ; é fazer
rotarismo, enfim. É mostrarmo-nos indivíduos que temos uma razão consciente a superiorizar-nos em face dos desvairamentos a que a mesquinhez da
sensibilidade arrasta os povos. É sobrepôr a inteligência, que nos deve
ca\-ra\-cte\-ri\-zar, ao ódio, á inveja, á ambi\-ção, ás malquerenças. Uma riqueza mais
persistente e muito superior á do ouro é a harmonia, única riqueza que nos
dá felicidade, pelo amor que traz ao lar e pela fraternidade que traz aos
povos.

Eis o que Eugé\-nio pensara dizer á laia de amena palestra, mas que pelo
motivo exposto não levou a efeito.

\begin{center}\bf\large A i\-lha da Madeira\end{center}

De o\-lhos postos na imensidade, e sorvendo a largos haustos aquele ar
puríssimo e tonifi\-cante que se respirava, Eugé\-nio mantinha-se ainda extático na escoti\-lha, quan\-do se ouviu, em alto som a advertência que a Madeira
estava á vista. Rufaram tambores por todos os recantos do barco, a acordar
os mais preguiçosos.

Eugé\-nio preparou-se rapidamente, e subiu ao convez superior. O sol
espe\-lhava já abundantes raios na superfície do mar.

Com efeito, a bombordo, começava a avistar-se ao longe o panorama incomparável
da Ponta de S. Lourenço, continuada por uma costa escarpada e árida. Os tombadi\-lhos são invadidos rà\-pi\-da\-men\-te por quasi todos os excursionistas, ávidos
da nova sensa\-ção, e as fisionomias pálidas e cadavéricas da véspera retomam,
como por encanto, a sua ex\-pres\-são normal. Assestam-se binóculos na ânsia de ver
mais.

Entretanto abordamos francamente a i\-lha. Uma cordi\-lheira de altas serranias ao longo de toda a i\-lha se ergue gigantescamente ao longe, dominando-a, e
em declive íngreme e abrupto apresenta-se á nossa vista toda a vertente sul.
A grandiosidade do aspecto dá a impressão dum sonho.

Á medida que o barco progride, pequenos aglomerados de habita\-ções se divisam ladeando de pressões do acidentado terreno, verdadeiros leitos de torrentes agora sêcas originadas nos cumes da serra que ao longe domina longitudinalmente a i\-lha. O Machico e Santa Cruz vão-nos fi\-cando agora para traz. Dispersas, alvejam a distância inúmeras casas seme\-lhando pedras caiadas de branco,
cada vez mais densas, até que o Funchal se começa a avistar com a sua baía
semi-escondida em arco de círculo.

Um gasolina encosta então ao barco, e uma comissão de recep\-ção entra a
apresentar as boas-vindas, entregando a cada um dos excursionistas o programa
minucioso da estadia na Madeira, juntamente com o mapa da i\-lha, um convite individual para cada um ir jantar na tarde dêsse dia a determinada casa particular, etc., tudo denotando uma organiza\-ção impecável.

Barquitos a remo circundam o casco do vapor, transportando bombeteirosrapazes bons nadadores que mergu\-lham para apanhar as moedas que os excursionistas lan\-çam á água.

O barco atraca finalmente ao novo cais da Pontinha. Uma largada de pombos festeja o acontecimento. E ao colocar-se o ponti\-lhão para terra, um rancho
de mu\-lheres madeirenses vem entregar a bordo um ramo de flores a cada passageiro.

Começa imediatamente o desembarque por entre os prègões dos ardinas que sobraçam diários locais onde se dedicam desenvolvidas e encomiásticas referências
á excursão. Um comboio de caminhetas, que aguardavam os recém-chegados, põe-se
agora em andamento, sob uma atmosfera pesada e quente, em direc\-ção à sede do clube do Funchal, onde são dadas as
boas-vindas e é fornecido um cálix de vinho da Madeira. Desculpem, disse entre outras
afirma\-ções o Presidente no seu discurso, o não termos esco\-lhido um local amplo e com maior
con\-fôr\-to para a vossa recep\-ção. Achamos mais interessante fazê-la no modesto
restaurante onde temos as nossas reuniões semanais, propositadamente modesto,
não por ser mais económico para nós, mas para fi\-carem disponibilidades
para um maior auxílio ás crian\-ças pobres.

Seguiram-se depois os cumprimentos no Palácio de S. Lourenço onde os
excursionistas fôram recebidos pelo Governador Civil e Presidente da Câmara.

Posta de novo em marcha, depois de visitada a Sé Catedral a caravana atravessa a cidade pavimentada com pedras negras e roliças de basalto, em direc\-ção
aos arredores. Estradas íngremes em que as caminhetas roncam em segunda ou
mesmo primeira velocidade, marginadas por casinhas caiadas de branco e te\-lhado
verme\-lho, muito limpas, deixando a espaços entrever densos canaviais sacarinos
ou culturas de bananeiras com os cachos ainda verdes e orladas por hortênsias e
outras flores multi-coloridas. A cidade fi\-ca lá em baixo encostada á baía,
e êste panorama, e outros duma magnificência inédita sucedem-se aos o\-lhos ávidos dos excursionistas. De quan\-do em vez as caminhetas param, e as objectivas
fotográfi\-cas disparam na ânsia de co\-lherem quadros de grandiosa beleza.

Eis o Pico de Barcelos, onde o horizonte é duma vastidão enorme de coloridos, onde as tonalidades do verde se multiplicam prodigiosamente, desde a
beira-mar até aos cumes da serra altaneira.

O apetite exacerbado por tão puros ares vai-se exteriorizando, porém,
nas exclama\-ções dos mais desesperados. O fuso horário local impõe um atrazo de 1 hora aos relógios continentais, o que complica um pouco o desejo
de satisfazer uma necessidade imperiosa.

Vamos entretanto descendo. Os carros galgam continuamente curvas e
contra-curvas, velozmente, a confirmar sempre a famosa perícia que os motoristas i\-lheus adquiriram aos o\-lhos de todo o mundo. Ainda antes do almôço
está marcada uma visita aos riquís\-si\-mos jardins do Reid's Palace Hotel,
que se estendem até á beira-mar, com uma luxúria de flores sem igual, verdadeiro éden onde loiras inglezas se estendem a beneficiar dos raios solares, semi-nuas, na mais impudica promiscuidade com o sexo contrário.

Segue-se o almôço no salão do Casino, primorosamente servido como todas as demais refei\-ções. Depois, a visita a Exposi\-ção de Bordados da Madeira na sede do Grémio dos Exportadores, onde delicadís\-si\-mos mimos passaram
como tela preciosa sob os o\-lhos dos visitantes.

-- Isto é simplesmente um encanto! exclamou Princesa junto dum lote
de pequenos lenços, cujos bordados eram autêntica filigrana.

E o resto da tarde passou-se no centro da cidade, livremente, percorrendo as ruas nos carros de monte, tão característicos do i\-lha, sem rodas,
deslisando sôbre apoios de ferro ensebados, puxados por juntas de bois, de
pequena estatura. Muitos excursionistas ostentavam já as carapuças típicas da Madeira, azuis com vivos verme\-lhos, e prolongadas no cimo por um rabicho azul. Outros, instalados nas mesas dos cafés, aproveitavam o tempo
para mandarem notícias aos continentais das suas rela\-ções.

O jantar decorreu com carácter de intimidade nas casas particulares
dos consócios i\-lhéus, distribuídos como tinham sido com antecedên\-cia os
continentais por cada um deles. A bordo apareceram á hora marcada, para
êsse fim, os hospedeiros a conduzirem os seus convivas. Uns e outros trajam de cerimónia para jantar, o que acontece pela vez primeira porque, na
viagem, embora assente a exi\-gên\-cia, o estado do mar não foi de-molde a permitir arranjos de indumentária.

Eugé\-nio foi jantar a casa dum colega seu, presidente do Clube i\-lheu,
onde foi exibir pela primeira vez o seu fatiote cerimonioso, ao qual aliás se
 adaptou sem custo. Feitas as apresenta\-ções, e servido como aperitivo um
"cocktail", jantou-se com aquele primor que as donas de casa sabem caprichar para com os seus hóspedes.

E o dia terminou por se juntarem todos no Casino. Eugé\-nio, com a fa\-mí\-lia hospedeira, foi dos primeiros que ali chegaram. Já se dan\-çava animadamente entre a colónia ingleza. Raparigas de torso nú, exibindo as suas
formas mais ou menos arquitecturais, pares dan\-çando tão encostados como se
formassem um todo indiviso, faces coladas, hálitos misturados, languidez sensual exalando-se de todos aqueles personagens -- eis o panorama que a Eugé\-nio foi dado observar logo de início.

E cogitou para consigo:

-- Que fiasco! É fora de dúvida que as fa\-mí\-lias excursionistas ao deparar com exibi\-ção tão atentatória contra o pudor das suas honestas esposas e fi\-lhas, se recusarão a entrar...

A ingenuidade de Eugé\-nio, acreditando em pudor quer das esposas ou fi\-lhas, quer dos maridos ou pais, mais uma vez foi notória. Que imperdoável
esquecimento da época que se atravessa! E que ignorância sôbre o que no
continente se exibe nas reuniões dan\-çantes!

As fa\-mí\-lias excursionistas começaram, com efeito, a chegar gradualmente. Com espanto seu, Eugé\-nio viu-as assomar á porta do salão, volverem um o\-lhar
sôbre a assistência e ingressarem imediatamente sem hesita\-ção. Viu dan\-çar
seguidamente toda aquela gente, sem observância de na\-cio\-nalidades na escô\-lha dos pares, o que é natural, indivíduos do sexo masculino sem distin\-ção
de idade ou de estado com mu\-lhers casadas, semi-nuas, como esquecidas dos
deveres que os laços matrimoniais lhes impuseram, e tudo isso com a maior
das indiferenças da parte das ilustres representantes do sexo feminino...

Pois não é toda a mesma sociedade que ali se reúne, a sociedade elegante, chique, de capitalistas e de nobres, a alta sociedade? Para quê então
o pudor entre pessoas de igual distin\-ção? Para quê a arcaica peia dos laços matrimoniais? -- Ve\-lharias impertinentes, anacrónicas, que nada justifi\-ca na actualidade, em que a mu\-lher revindica os mesmos direitos que a tradi\-ção conferia ao homem, e mais o de ser o\-lhada com superioridade, ciosa
de gôzo, sobrepondo os deveres mundanos aos domésticos, a exposi\-ção da carne
á pureza do espírito...

Só Eugé\-nio não o compreende, tão retrogradado anda da nossa épo\-ca. Sonha ainda a mu\-lher intangível no seu elevado pedestal de ve\-lha moral, enrubescendo ao ouvir uma palavra com assomos de obscena, isolada do mundo
e vivendo exclusivamente para a sua fa\-mí\-lia, fazendo da sua conduta impecável a arma poderosa que faz arredar tudo e curvar todos por onde passa;
sonha enfim, com ve\-lharias de museu, com coisas que de há muito deixaram de
existir.

Eugé\-nio não compreende que a mu\-lher de hoje aceite como graça, ou mesmo o exija com o delicadeza madrigalesca, uma frase mais apimentada de obscenidade, desde que venha dum rapaz elegante, aquela mesma graça que da boca
dum maltrapi\-lho seria tomada como a maior das afrontas para a sua dignidade;
num caso acoimam o rapaz de espirituoso, noutro de incorrecto. Se um rapaz
elegante se inibe de graças picantes, acoimam-no de sensaborão.

Eugé\-nio não compreende que a mu\-lher de hoje viva exclusivamente para
o prazer e se inferiorize ao dirigir a sua casa.

Eugé\-nio não abdica da ve\-lha moral. O mundanismo que sempre detestou
por questão de princípios inerentes ao seu temperamento, causa-\-lhe hoje náuseas
ao conhecê-lo um pouco mais de perto. E mais: a mu\-lher, que estava
habituado a colocar muito alto no seu conceito, cada vez mais se reduz a um simples
instrumento de gôzo, fa\-lho de toda a personalidade, verdadeiro monstro que de humano
só tem a sensibilidade e que a razão nunca penetrou.

Entretanto Eugé\-nio, encostado, seguia atentamente toda aquela bacanal
coreográfi\-ca. Do grupo de Princesa, a menos solicitada para dan\-çar era precisamente ela. Sucediam-se as músicas, e Princesa fi\-cava frequentes vezes
sentada, libertada das companheiras que bailavam quasi, ininterruptamente.

Foi num dêstes momentos, já de madrugada, que Eugé\-nio quis es\-tre\-ar-se.
Planeara muitas vezes  ir convidar Princesa para dan\-çar mas com a devida
discre\-ção para se não expôr á troça das amigas perante uma possível evasiva.
Tinha, é certo, na mente as palavras ouvidas a Marília de que aquela era incapaz duma incorrec\-ção, mas a dúvida persistia.

Em determinado momento resolveu-se. Já de-resto a bordo tinha dan\-çado
com todas as de mais excursionistas o que dava ao facto um aspecto normal,
se não de imposi\-ção social. Estava sem a maior parte do seu grupo, e conversava com um rapaz amigo de Eugé\-nio, quan\-do êste se lhe dirigiu.

-- V.\ Exª dan\-ça ?

E a resposta não demorou:

-- Agora não danço.

Eugé\-nio soube reprimir o embaraço que intimamente o avassalava, e convidou imediatamente com toda a fleuma outra rapariga do lado com quem foi
dan\-çar.

Haviam sido trocadas as primeiras palavras entre Eugé\-nio e Princesa.
Poucas é certo, mas as bastantes para ajuizar da hostilidade desta para com
aquele.

Marília tinha-se enganado. Verdade é que saíra da sua boca a afirma\-ção de que as irmãs Frazão se haviam modifi\-cado profundamente desde que
estreitaram rela\-ções com as suas novas amigas... Princesa nem sequer quis
justifi\-car a sua atitude, formulando qualquer desculpa, ou mantendo-a em futuras músicas para que outros a convidaram a dan\-çar.

Passava-se isto nas primeiras horas dum fatídico dia 13. Eugé\-nio não
dançou mais, e retirou-se, algum tempo passado, para bordo. Que se passaria
no seu espírito? Iria convencido desta feita que se tinha enganado durante tantos anos sôbre o carácter de Princesa? Daria agora razão aos amigos
que lhe afirmaram múltiplas vezes que Princesa era como as outras?Nem
Eugé\-nio sabia prescrutar o seu espírito naquela emergência de confusão mental, nem tãm pouco o tentou.

Na manhã seguinte o passeio estava destinado á vertente norte da ilha.
Todos os excursionistas já se espa\-lhavam pelo cais a procurar lugar nas
caminhetas, prevenidos com agasa\-lhos. E a caravana lá debandou, serra acima,
até á Ribeira das Cales numa altitude de 1420 metros, onde o frio por vezes
se fazia sentir bem áspero, para se começar a descer em direc\-ção ao Ribeiro
Frio onde o almôço estava preparado. A aridez dos cumes ta\-lhados em escarpas quasi a pique, por onde a estrada zig-zagueava com declives formidáveis,
contrastava agora com a fertilidade da costa sul. A distância,
o Atlântico. Mais longe os contornos de Porto Santo.

A visita a pé aos balcões do Ribeiro Frio, a 1,5 quilómetros daqui,
serviu de magnífico aperitivo para o almôço. É um precipício dalgumas centenas de metros, donde se disfruta um panorama horrível na sua magnitude,
no final do caminho todo êle ladeado à direita por uma série de outros
precipícios menos horríveis, onde vegeta sobretudo o loureiro.

O almôço que se seguiu foi servido no restaurante do Ribeiro Frio. Comida acentuadamente regional, em que os excursionistas saborearam uma sopa
de tomate e cebola, bifes de atum com mi\-lho frito, cozido á Madeirense e
cuscuz, pudim de pão, fruta e café, á mistura com a dan\-ça a que convidava o
ambiente familiar aqui respirado.

Seguidamente a caravana regressou á procedên\-cia. O Funchal, sempre
dominado do alto mas escondido a intervalos pelas colunas de nevoeiro ou
pelos pinhais, a meia encosta, lá está no fundo, sorridente na sua baía que
\-lhe realça a beleza. As caminhetas vão descendo, proporcionando sempre novos
horizontes de panoramas variados a cada curva que faz a estrada.
Passa-se o monte e o Terreiro da Luta, ainda entre matas de pinheiros, e mais
abaixo começam a aparecer as frutíferas do continente e em seguida os campos de bananeiras e cana sacarina, cada vez mais densos, até que chegamos
a cidade por uma estrada, numa extensão de alguns quilómetros, toda orlada
de flores, muitas flores, que aqui e ali grupos infantis atiram para dentro
das caminhetas.

Visitam-se agora os recheados armazéns da "Madeira Wine Association,
L.ed", onde se prova o magnífico vinho regional de que todos trazem uma
pequena garrafa.

O jantar é nês\-te segundo dia de permanência na Madeira servido no
maior e mais luxuoso hotel da i\-lha, e um dos mais caros da Europa -- o Reid'e
Palace Hotel. Serviço magnífico, exemplar, por um pessoal recrutado na Itália, a justifi\-car, com os aposentos onde nada falta, umas diárias de 1 a 8 libras.
Como nota interessante, também observada nalguns outros hoteis, há a registar o facto de o comensal ao puxar por um cigarro logo aparecer, pressuroso, um criado a acender-lho.

Acabado o jantar, um belo passa-tempo foi proporcionado aos visitantes,
no Teatro Municipal: um espectáculo de gala, estreia da interessante comédia "Fogo Preso", levada a efeito por amadores da sociedade madeirense.
De admirável desempenho, o entrecho era da mais patriótica oportunidade,
por flagrantemente anti-desna\-cio\-nalizante, e por isso altamente educativo.
A forma por que calou no espírito da assistência, que enchia literalmente
o magnífico teatro, foi bem francamente exteriorizada pelas intermináveis
ova\-ções com tanta justiça dispensadas aos figurantes.

A hora tardia a que terminou a récita prejudicou agora o baile no Casino, para onde uns se dirigiram durante pouco tempo,
preferindo outros seguir directamente para bordo em busca do apetecido
repouso.

Eugé\-nio foi dêstes últimos. No salão de festas do vapor dan\-çavam animadamente alguns convidados madeirenses que passavam um bocado da noite
com o comandante, elas, muito pintadas, fumando cigarros aparatosos de que
nem durante a dan\-ça se desembaraçavam, êles fumando discretamente nos intervalos de descanso.

O passeio do terceiro dia da Madeira não foi menos empolgante que os
anteriores. Seguindo a costa para poente, por Santa Cruz, até Machico, e inflectindo depois para a vertente norte até á Portela, umas série de pitorescos quadros se desenrolou aos o\-lhos dos excursionistas como gigantesca
tela. Era aqui a fragrante verdura em múltiplas tonalidades do Minho,
ali as leiras encostadas a declives brutais de Traz-os-Montes, acolá os
casebres pobres da Beira. Ao lado sempre o mar, deixando aflorar a pequena
distância as i\-lhas Desertas. Ao fim, a Portela com os seus horizontes deslumbrantes de beleza e amplidão.

O regresso fez-se pelo Santo da Serra, onde a Comissão de Turismo
quiz brindar a excursão fazendo-a saborear uma suculenta e variada refei\-ção de doces e vinho da Madeira, e, atravessando sempre para a costa sul
por entre pinhais, apareceu de novo o Funchal.

O Savoy Hotel forneceu agora o almôço, e, após uma visita á sua bela
piscina á beira-mar, ficou a tarde livre para compras na cidade. Descongestionaram-se as lojas dos seus preciosos bordados, que avalanches de excursionistas invadiam para trazerem valiosas recorda\-ções de tão mimosa
indústria.

Á noite foi-se jantar longe, ao restaurante Esplanade, situado na en-costa do monte, ao Terreiro da Luta. Jantar festivo, de despedida; muitos
discursos, dos quais, o último, foi uma entusiástica apologia á paz, cheia de
beleza e de primores de retórica que a todos encantou.

E terminou a noite por um baile de despedida no Casino, lindamente
ornamentado, e em cujo jardim se fizeram exibir grupos regionais e queimar
fogo de artifício. No salão dançou-se animadamente, misturando-se os trajes negros das casacas e dos smokings com os vestidos garridos, vaporosos
e decotados, enquanto outros se limitavam a apreciar a arte coreográfi\-ca
dos pares. Do número dêstes últimos fazia parte Eugé\-nio.

Mais um dia surge, o derradeiro de permanência na i\-lha. A visita hoje
é ao Cabo Girão, e a caravana de caminhetas lá segue, com paragem no Lido
cuja piscina á beira-mar foi objecto da mais encomiástica admira\-ção. Passa-se Câmara de Lobos, o pitoresco porto de pesca, e a estrada começa a subir em apertadas sinuosidades por entre socalcos de vinhas que do alto
se avista em declive abrupto ocupando enorme extensão e dando a impressão dum trecho do Douro.

O promontório do Girão é uma gigantesca fraga, que se eleva \mbox{a-pru}\-mo
a 401 metros de altura sôbre o mar, no cimo da qual está construído um
miradouro. Á chegada dos visitantes, um suculento almôço ali lhes foi servido ao ar livre, de carnes frias e frutas, vinhos e cervejas; o vinho verde, de Santo Tirso, constituiu a\-gra\-dá\-vel surpreza que os apreciadores já
não provavam desde o embarque.

Com a alegria que presidiu a êste almôço, contrastava agora a quasitristeza do regresso. Era o último adeus á i\-lha, de que tantas e tão gratas recorda\-ções todos levavam. Na volta, Câmara de Lobos aco\-lheu festivamente os visitantes que não quizeram passar por lá sem admirarem as belezas do seu porto. E novamente a caravana se pôz em marcha, e o Funchal apareceu para receber as despedidas.

Dentro de poucas horas o vapor levantaria ferro. Logo de manhã acenderam-se as suas caldeiras. O chá, servido a bordo em honra ás fa\-mí\-lias
madeirenses e á oficialidade, foi o remate do gran\-de cruzeiro. Às 18 1/2
horas locais o barco descolou do cais apinhado de gente que correspondia
com os lenços aos acenos de bordo.

Ninguém escondia a saudade com que partia, após tantos encantos vividos em pouco tempo. As aten\-ções com que os continentais fôram sempre cumulados durante esta estadia, desde a recep\-ção até final, a forma impecável
como foi organizado o programa e posto em prática, não são mais, de-resto,
que o reflexo do elevado civismo que ca\-ra\-cte\-ri\-za o público i\-lheu que bem
gravado ficou na memória de todos nós como li\-ção recebida. Que contraste, por exemplo, entre a compostura do elemento popular do Funchal, em cujo
seio se não ouve um palavrão menos cortez, e o povoleu continental sempre
desbocado á porfia!

Uma só tristeza, e gran\-de, ficou da visita: a desna\-cio\-naliza\-ção daquele lindo rincão de terra. As tabuletas são em gran\-de número impressas
em inglez, ou pelo menos lê-se êsse idioma ao lado do na\-cio\-nal. As gran\-des
emprêsas exploradoras do açúcar e do vínho são inglesas. Toda a gente fala
e entende o inglês, para agradar á numerosa colónia inglêsa que é detentora
dos me\-lhores palacetes dispersos pela i\-lha, e a pronúncia do idioma por\-tu\-guês ressente-se profundamente da influência inglêsa.

O sentimento que vibra no espírito dos madeirenses, é, porém, estruturalnente lusitano -- afirmam êles; e a desna\-cio\-naliza\-ção é somente aparente,
e condicionada pelo comércio excursionista que é essencialmente inglês. Os
por\-tu\-guêses do continente ainda não souberam tirar partido das belezas daquele bocado de paraízo que é a sua própria terra. Não a conhecem, não a
visitam, não a propagandeiam. E o resultado é aquele.

Não permitiu a escassa permanência na i\-lha prescrutar um pouco a alma
daquele povo, além dos hábitos de limpeza e de arranjo com que os mais pobres vivem no interior das suas casinhas modestas dispersas pela serra.
Nem o escasso tempo de estadia foi de-molde sequer a corrermos a por\-ção
oriental da i\-lha. Necessário tinha sido, para êste fim, ao menos o dôbro do
tempo.

O mar, que se conservara chão desde a chegada, começa a ameaçar no momento da saída. Já os mais pusilânimes se encaminham para as cadeiras dos
tombadi\-lhos, donde alguns não se haviam de despegar mais durante a viagem.
Funchal desaparece, Santa Cruz, Machico, a Ponta de S. Lourenço já acendeu
o seu farol, que por sua vez se vai gradualmente sumindo ao longe na escuridão da noite.

O mar começa então a acalmar, para em breve, ao passar ao lado de Porto Santo, proporcionar novas sacudidelas ao barco que depressa desapareceram.

O jantar decorreu normalmente, seguido de dan\-ça e do repouso tão necessário já a todos.

Alguns excursionistas instalaram-se em redor do receptor da T.S.F.

As notícias ouvidas a bordo pelo rádio aguçavam a curiosidade de todos.
A paz mundial estava preclitante uma vez mais, com a extrema\-ção de campos
favorecidamente acoimados de ideológicos.

O totalitarismo fascio-nazista ameaçava invadir a Europa.

Hitler, falando ao mundo, invectiva tudo e todos, declarando opôr a fôrça
à resistência que porventura encontre no seu pretenso direito de alcan\-çar
um lugar ao sol... como há 4 anos Mussolini, arrogando-se uma bravura desmedida, desafiava tudo e todos para que se opuzessem à sua entrada na Abissínia no altruista desejo de levar ali uma civiliza\-ção europeia.

A Itália, a Alemanha e o Japão formam o eixo duma ideologia na\-cio\-nal-sindicalista, imposta por ditaduras, em que a avidez de predomínio é a característica mais palpável, sob a forma de revindica\-ções de territórios a\-lheios,
invocando analogia racista ou direitos antigos.

Vai para 4 anos que a Itália fascista fez vibrar de cólera todo mundo civilizado com a conquista da Abissínia. Foi o terrível precedente que
ia pôr desde então em perigo os direitos dos povos fracos.

A Fran\-ça e a Inglaterra, últimos baluartes da democracia, adormecidos
sob os louros duma supremacia que dirigiu os destinos do mundo, acordam então da letargia para reforçar esquadras e aumentar a avia\-ção.

A Sociedade de Defeza das Na\-ções, empreza criada por uns e outros, reúne
para evitar os desmandos dum fi\-lho irrequieto que procura absorver o mais
fraco. E enquanto apela para a fôrça da razão e o ameaça com os rigores da
disciplina, ele, desobediente e atrevido, rompe com as hostilidades, apoiado
na razão da fôrça, e invade a Etiópia. As intimida\-ções do eixo franco-britânico não o deteem na sua carnificina monstruosa, e as ameaças da S.D.N.\ responde que, com ela, sem ela ou contra ela, há-de vencer seja porque preço fôr.
E o mundo, desafiado por um homem, começa a ver nessa atitude um balão de
ensaio, depois um acto de loucura paranóica, e acabou por assistir a consuma\-ção dum facto que só as atitudes decisivas dum, e a indecisão doutros, podiam
tornar realizável.

Sempre assim foi. Quando dois homens se encontram pela primeira vez, o
mais ávido e activo procura conquistar a supremacia sôbre o outro; se ambos
são tímidos, aquele que primeiro refrear a timidez será entre eles o indivíduo dominador. O que acontece com os indivíduos, acontece com os povos. Eis
a génese do despotismo. Mussolini é, em avidez e actividade, em audácia e temeridade, o condutor de povos, falando com arrogância aos seus apaniguados e
desafiando com as suas palavras destemidas o mundo.

O eixo franco-britânico vacilou pela primeira vez nos anais da his\-tó\-ria, sentindo-se oprimidos a fôrça gálica e o orgu\-lho inglez. A arrogância
inclinou o fiel da balan\-ça a seu favor, contra a fôrça. A Etiópia foi derrotada, perante o cruzamento de braços de todos, e o Negus não é hoje mais que
um súbdito britânico.

O sistema democrático sofreu a primeira gran\-de machadada. A Sociedade
das Na\-ções faliu. O balão de ensaio italiano surtira o me\-lhor resultado.

Hitler não fala alto ao mundo, não empolga multidões, não electriza
o seu povo para o conduzir, como faz Mussolini. Não é um condutor de povos.
Recorre à astúcia, a que se convencionou chamar, diplomacia. Procura antes
que tudo a oportunidade na fraqueza do adversário e, se a não encontra, provoca-a semeando a intriga. A sua actua\-ção é surda, os seus queixumes de necessidades colonizadoras contrastam com a arrogância de Mussolini.

A timidez e a pondera\-ção do eixo Paris-Londres cedem em presença da
audácia aliada à perspicácia do outro eixo. O sistema democrático reconhece
a sua impotência.

A Espanha lan\-ça-se na fogueira da guerra. Debatem-se precisamente ali
as duas ideologias opostas e irreconciliáveis entre gente da mesma na\-ção. Acorrem os dois eixos a enviar lenha que mantenha o fogo. Francezes, inglezes e
russos com espanhois governamentais: italianos, alemãis e portuguezes com
espanhois de Franco. As abundantes reservas de ouro da Espanha transformam-se em canhões sinistros que arrazam o país. Forjam-se à pressa tratados de
não-interven\-ção entre os dois eixos interessados, mas os tratados são papeis
que depressa se reduzem a cinzas na fogueira devastadora.

No Oriente, o Japão debate-se com a China.

As democracias apetrecham rapidamente os arsenais com que a\-ma\-nhã imporão de-novo ao mundo o seu poderio. Mais 2 anos e elas presidirão, insuflando-\-lhes o senso da paz, ao destino dos povos.

É uma boa oportunidade de Hitler chamar a si a Áustria, enquanto aquele
poderio adverso não é atingido. Em 1938 anexa os sudetes. O tratado de
Munich, agora assinado pelas gran\-des potências de um e outro lado, vai opôr-se de òra-avante a que novos esbu\-lhos se produzam sôbre os paises mais fracos. Mas o tratado de Munich é um papel como os outros, e atraz dos sudetes
vai toda a Tcheco há pouco mais de um mês. Pouco antes tinha sido anexada
Memel.

Hitler necessita expandir-se, e ameaça todos os obstáculos que se oponham à sua necessidade de espaço vital, enquanto as democracias se agacham
timidamente. Já não restam dúvidas que a Alemanha pretende ser senhora da
Europa, como em 1914 sonhava ser seu imperador Gui\-lherme II.

Poucos dias antes da partida para a Madeira, já em plena semana santa,
a inquieta\-ção da Europa Central dá motivo a que os diários noticiosos encham mais uma vez as suas colunas de letras gordas. É desta vez a Albânia
que pede socôrro, ameaçada de invasão pela Itália. O terror da rapinagem
próxima foi de curta dura\-ção. Os pródromos foram sumários. E, numa noite
a Albânia tornou-se quasi imprevistamente uma colónia italiana, não sem ter
procurado defender com denodo e até ao último reduto a sua independên\-cia.

A atmosfera interna\-cio\-nal conserva-se sobrecarregadíssima. A Alemanha
e a Itália são o fulcro do desassossêgo mundial, com a sua megalomania invasora, ameaçando anexar tudo à insaciabilidade das suas exi\-gên\-cias territoriais, semeando o pânico, precipitando as invasões em vertigem louca. Que
revindica\-ções viriam mais? As cobiçadas colónias portuguesas, há muito condenadas à parti\-lha entre os nossos amigos inglêses e alemãis? O próprio
Portugal, chave do Atlântico e do Mediterrâneo? Ah, não! Nêste, mais que a
amizade, havia os interesses da Inglaterra a manter íntegra a sua independên\-cia, e só êsse facto nos dava a certeza de ainda pisarmos solo nosso ao
abordarmos a costa...

Guerra! --  imploram uns, como único remédio salvador com que se extermine
aquele foco de desordem. Paz! -- reclamam outros, aqueles que tremem ao ver
a chacina das lan\-ças e obuzes, a destrui\-ção duma civiliza\-ção.

As democracias ameaçam sair do marasmo para pôr côbro à voracidade invasora, mas ainda se não sentem suficientemente fortes. Só em 1940. A Fran\-ça
e a Inglaterra estão bem certas da fidelidade do seu recente tratado de
alian\-ça, e contam com o auxílio da Rússia. Os Estados Unidos da América não
enfileirarão a seu lado? É uma pregunta que todos fazem.

Com a termina\-ção da guerra de Espanha, cabe agora a vez da Itália revindicar compensa\-ções para o auxilio que trouxe à vitória de Franco. Voltase para Gibraltar, a porta do Mediterâneo, de que pretende as\-se\-nho\-re\-ar-se
para me\-lhor dominar êste mar. As esquadras ingleza e italiana começam de se
exibir. Portugal está ameaçado de servir agora de teatro de guerra.

Eis a enervante expectativa com que as últimas notícias recebidas na
Madeira eram disputadas com sofreguidão pelos continentais, como agora, no
alto mar, o são as notícias do rádio. Portugal, teatro de guerra! Aquele torrão em que nascemos, que vimos progredir, condenado a ser a\-ma\-nhã um montão
de ruínas! Cidades arrasadas, campos destruídos, mãis enlutadas pelos fi\-lhos
mortos!

As últimas notícias de bordo dão-nos agora conta da assinatura da
alian\-ça entre a Inglaterra e a Polónia, destinada especialmente a evitar
que a cidade livre de Dantzig, já ameaçada, caia sob o domínio alemão. Roosevelt, do lado de lá do Atlântico, chama á razão os revindicadores insaciáveis
mostrando-\-lhes a responsabilidade duma guerra, e apela para o
seu bom-senso no sentido de se evitar a efusão de sangue.

E a incerteza do dia de a\-ma\-nhã cai não só sôbre todos os pequenos
povos, com o pesadelo tremendo da absor\-ção, como sôbre toda a Europa, perante
a iminência duma conflagra\-ção mundial...

O domingo amanheceu com um céu límpido. Não fazia mar nenhum. Dir-se-ia que o barco estava parado num quieto lago. E assim se conservou o
dia inteiro, permitindo todos os folguedos a bordo se bem que indivíduos
houvesse que não apareceram, como aconteceu com Princesa. O almôço, embora
pouca gente comparecesse na mesa, como de costume, e o chá, decorreram por
isso com normalidade.

O jantar estava anunciado como sendo de gala, oferecido pelo comandante do vapor aos excursionistas, com o pedido especial da assistência de
todos aqueles que o pudessem fazer. Pela primeira vez a sala de jantar se
encheu. Á última hora foi dispensado o traje de cerimónia, como havia acontecido em todos os jantares de bordo.

Eugé\-nio fôra convidado pelo Comissário para se sentar na sua mesa.

Uma interessante surpreza estava reservada aos convivas nesta refei\-ção. Levantados os pratos de sôpa, apagam-se as luzes dum sector da sala. Fusão de fios ou qualquer outro acidente, conjecturavam uns e outros. Em breve se apagam as restantes, e os convivas mergu\-lham nas trevas. Após uns
momentos de geral inquieta\-ção, uma formatura de criados entra na sala rasgando a escuridão com pequenas luzes. Cada qual com uma travessa de lagosta, com os o\-lhos substituídos por minúsculas lâmpadas electricas e encimadas por uma torre cuja cúpula era uma taça igualmente iluminada.

O efeito foi deveras surpreendente. Reacendeu-se a ilumina\-ção e serviu-se a lagosta no meio de quente ova\-ção à oficialidade e a toda a tripula\-ção sem esque\-cer o cozinheiro.

O serviço de jantar foi primoroso. O Comandante, e com êle toda a oficialidade, que em toda a viagem não puderam ter mais requintes de gentileza para os excursionistas, quizeram fechar com chave de ouro a fidalguia
do trato com que sempre se houveram para com êles. Até o baile desta noite, o último a bordo, promovido pelos oficiais, teve especial anima\-ção.

Amanheceu o dia de 2ª feira com um horizonte límpido como os anteriores, e o mar calmo. Acrescia agora a saudade dos momentos vividos, no semblante de todos. Aproximava-se cada vez mais a terra. 100 mi\-lhas, 90, 80...
A chegada a Lisboa está marcada para as 15 horas, e já os binóculos acusam
o perfil indeciso do Cabo Espichel ao longe. O barco continua na sua rota,
com uma velocidade de 12 mi\-lhas á hora. Almoça-se apressadamente para gozar o panorama da entrada em Lisboa. A costa começa a desenhar-se cada
vez mais distintamente, e junto dela vêem-se já bastantes barcos na sua
faina. As gaivotas anunciam agora, com o seu aparecimento, a curta distância da terra.

O cruzeiro vai terminar. Como no cinema, passa adiante dos o\-lhos curiosos o grandioso trecho panorâmico da entrada da barra. Cascais, Estoril,
Parede, Carcavelos, as torres de S. Julião e Bugio, a Serra de Sintra ao longe, Paço de Arcos, Belém, os Jerónimos, o casario de Lisboa, e eis o desembarque em Santos sob um calor abafado de trovão.

Separam-se os excursionistas com as fa\-mí\-lias que aguardavam a sua
chegada. Uns ainda confraternizaram á noite com jantar e baile de despedida no Hotel Aviz. Outros seguiram para as suas residên\-cias.

Eugé\-nio tomou o primeiro comboio para o Porto, onde chegou cêrca da
meia noite para retomar a sua habitual pacatez após oito dias de activa
e inolvidável recrea\-ção.

\begin{center}\bf\large Princesa\end{center}

Passaram-se muitos dias sôbre o regresso da Madeira. Eugé\-nio sacudira
já da sua mente a imagem dos seus antigos e persistentes sonhos.

 A aproxima\-ção de Princesa acabara de desfazer as últimas ilusões a seu respeito.
 Já não era aquele carácter ideal, de forma\-ção moral cheia de beleza, de
conduta impecável.

O desdém começa a invadir Eugé\-nio. Efectivamente Princesa descera do
alto pedestal de mu\-lher superior á banalidade da mu\-lher comum. Até fisicamente, Princesa parecia já outra aos o\-lhos de Eugé\-nio. Eram agora os defeitos
procurados, físicos ou caracterológicos, que o impressionavam, o\-lhando-os á luz
da razão consciente, depois de afastada a nebulosa da paixão..

Já não via nela os o\-lhos encantadores, a beleza do seu rosto oval de tez
branco -- pálida, a ex\-pres\-são mística dum semblante magestoso, e outras quejandas
propriedades físicas, paralelamente a uma educa\-ção superior a coroar um temperamento indefectível; perpassavam agora pela mente de Eugé\-nio somente as
qualidades pejorativas que lhe notara a bordo -- o amor-próprio exagerado com
que a ouvira contrariar os seus Pais, numa voz débil de falsete denotando uns
pulmões pouco resistentes a condizer com a magreza de fei\-ções que de há tempos se lhe vem notando.

Acrescia a incongruência de Princesa naquela noite fatídica, que se mantinha inexplicável sob todos os aspectos. Que razões a levaram a proceder assim,
se Eugé\-nio nunca teve a mínima incorrec\-ção para com a sua pessoa?
Quer êle constituisse no seu pensamento, o antigo apaixonado, quer a convidasse
a dan\-çar como qualquer indiferente, Eugé\-nio nunca lhe havia dado motivo para
receber de Princesa a mínima desconsidera\-ção. De-resto, mais do que um legítimo
direito, o convite impunha-se como um dever de polidez de quem havia já dan\-çado com todas as raparigas excepto consigo.

Por outro lado, como qualquer indiferente, Eugé\-nio podia bem orgu\-lhar-se
de não ter qualquer acto na sua conduta que o envergonhasse; e sob êsse aspecto ela não poderia ter dúvida sem aceder ao seu convite, como acedera ao de
todos os demais, sem excluir de pessoas cuja reputa\-ção havia sido por si própria posta outrora em dúvida.

Não podiam todos êstes factos deixar de influenciar profundamente o espírito de Eugé\-nio, que viveu semanas de atroz revolta interior tanto mais penosa quanto ela era recalcada. Com efeito, exceptuando a evasiva de Princesa
em dan\-çar consigo, que Eugé\-nio declinou a Marília, nenhuma outra exterioriza\-ção sua deixou perceber a quem quer que fôsse o que se lhe ia no íntimo.

Por outro lado, da bisbi\-lhotice feminina que tanto falara nas vésperas
da excursão, nada chegava agora aos ouvidos de Eugé\-nio em redor de quem se formava como que um extranho vácuo. Sem dúvida que a curiosidade daquelas mesmas que carrearam abundantes notícias à fa\-mí\-lia Frazão, antes da viagem,
não se daria por satisfeita sem conhecer pormenores do passado durante a excursão, e haviam de comentar atravez da lente amplifi\-cativa e maliciosamente
deturpadora os factos que provavelmente Princesa narrou.

Semanas passadas ainda nada tinha, porém, chegado aos ouvidos de Eugé\-nio.
Nem nas visitas frequentes que fazia a casa de Marília nada transparecia, e o
facto era tanto mais de estranhar quanto é certo que esta prometera espontâneamente indagar de Princesa, após o regresso da Madeira, e como coisa sua, im\-pres\-sões sôbre Eugé\-nio.

Que misterioso silêncio, pois, aquele, feito á volta das im\-pres\-sões de
Princesa, e da sua repercussão na sociedade feminina! E as semanas rolavam,
e as visitas sucediam-se a casa de Marília sempre na expectativa de que esta,
ou a sua irmã, tomassem a iniciativa de cortar tal silêncio -- iniciativa de
que teimava em se coibir Eugé\-nio.

Mas que poderia interessar agora o que se passava ou dizia? Se a figura
de Princesa se varrera do espírito de Eugé\-nio, como explicar a curiosidade
que era de esperar de Eugé\-nio, partindo da hipótese que esque\-cera definitivamente Princesa.

Mas estaria de facto absolutamente livre o espírito de Eugé\-nio? Não
é impunemente que uma simpatia se radica e se avoluma no decurso de tantos
anos.

Eugé\-nio assistia na verdade a uma duplica\-ção do seu sêr. Por um lado,
a realidade dos factos observados directamente que diminuiam aos seus o\-lhos
a figura outrora angélica de Princesa; por outro lado, a revivescência duma
simpatia tão entranhada no seu espírito. E esta revivescência do passado
suplantava intimamente a realidade actual. O consciente pretendia ver a realidade do presente, sim, mas o inconsciente esmagava-o com o seu poder mais
elevado.

As noites sucediam-se até, mais do que nunca, em sonhos múltiplos e variados em que Princesa era sistematicamente a protagonista, e  êste facto não
podia deixar de ter abundante actua\-ção durante o dia no espírito de Eugé\-nio.
Com efeito, ao vê-la de novo passar na rua, já a sua figura, de belo rosto
oval e o\-lhos magnéticos, o fazia outra vez vibrar da emo\-ção dos apaixonados
e esque\-cer qualquer agravo recebido.

E no entanto tornava-se necessário esquecê-la -- bem o sabia nas horas
de razão.

No seu espírito travava-se intensa luta. Momentos de razão consciente,
horas em que o inconsciente operava com intenso labor... E não havia já
quem lhe ouvisse uma palavra do que se lhe ia na alma, tão recalcados agora
no íntimo se conservavam os seus pensamentos sem os confiar a pessoa alguma.
Isolava-se a ler e escrever, e êsse isolamento era o único consôlo dum espírito consumido pelas intempéries da adversidade.

Via em redor de si rapazes sem as suas faculdades, sem qualidades absolutamente nenhumas,  serem felizes, amando e serem amados, e só êle não encontrava um oásis no deserto da sua vida afectiva, êle que bem sabia ter faculdades superiores a muitos outros.

Por quê a hostilidade sistemática de Princesa? Na Madeira, em plena excursão, nos passeios em conjunto, ela acercava-se frequèntes vezes do grupo
de Eugé\-nio com familariedade natural, sem qualquer espécie de constrangimento, o que parecia denotar uma atitude muito diferente para com êste. Mas
aquela fatídica noite do Casino...

Como sempre, incompreensível!

Eugé\-nio assistia positivamente a uma duplica\-ção ciclotímica da sua
personalidade. Era umas vezes a razão consciente a afastá-lo das circumvizinhan\-ças de Princesa, procurando esquecê-la na convic\-ção de que nem os afectos dela eram concordantes com os seus, nem tam pouco êle os podia manter
porque ela já não constituia o seu ideal de beleza moral e física. Ressoavam-lhe então aos ouvidos as críticas que escutou múltiplas vezes sôbre os habitos de gran\-deza e de mundanismo que de há tempos levava, e recordava as
desconsidera\-ções dela recebidas.

Outras vezes imperava o inconsciente apoiado na invetera\-ção do hábito,
e Eugé\-nio via-a mais bela fisicamente, e desmentia energicamente as críticas
ouvidas a seu respeito, teimando em só dar crédito ao que via com os seus
o\-lhos, e o que via nela era bem pelo contrário a modéstia na apresenta\-ção
sem qualquer aparato, e a vida caseira e recatada de verdadeira dona de casa
com que se distinguia ao seu espírito. Ao mesmo tempo imaginava justifi\-ca\-ções para qualquer acto de menos cortezia dimanado da parte dela. E os seus
o\-lhares representavam-se na mente de Eugé\-nio como luz vinda duma estrela
de primeira gran\-deza, que o fulminava a ponto de o fazer desviar os seus
o\-lhos tímidos. Chegava a mal dizer o temperamento emotivo que o levava a esconder-se quan\-do a via, quan\-do é certo que tantas ocasiões tivera em se acercar do grupo dela e por essa forma insinuar-se no seu espírito.

Sucediam-se assim as fases antagónicas de actividade e depressão no
seu estado de humor em rela\-ção á sua paixão, fases opostas em si, mas tendo
um traço de união a cingí-las, com o único traço a ligar as duas personalidades de Eugé\-nio: o sentimento de respeito, íntegro, permanente, que nunca lhe
permitiu qualquer acto ou palavra em desabôno da pessoa amada que, no fundo,
era sempre Princesa.

Entretanto pela cidade era voz corrente que Eugé\-nio estava noivo.
O boato corria veloz, mencionando-se o nome de Princesa, como sendo
a noiva, entre os que a conheciam, ou a fa\-mí\-lia Frazão, ou simplesmente o local da resistência entre aqueles que só pela morada a podiam identifi\-car. Ou
então, falando dela, atribuiam-\-lhe um noivo médico, vagamente, ou adiantavam
o nome de Eugé\-nio.

Por uma das poucas tardes calmosas dêste verão, Eugé\-nio encontrou-se
com um professor de letras das suas rela\-ções. Ainda bem poucos dias antes
haviam cavaqueado ambos, sem que nada fizesse prever o rumo da conversa de
hoje.

-- Oh doutor: Por quem é desculpe-me o desabafo de noutrodia, começou
o professor ao mesmo tempo que estendia a mão a Eugé\-nio. Fui inconveniente
e, sobretudo, importuno, ao fazer lhe a apologia do celibato, supondo-o comungar nas minhas ideias. Quem diria então que estava a falar com um noivo, que
aliás teve a gentileza de não querer opôr-me objec\-ções? Fui incorrecto.
Mas creia que de nada sabia, e agora que mo comunicaram aqui me tem a pedir\-lhe mil desculpas.

O ilustre letrado defendia de-facto o celibato com o mesmo entusiasmo
com que o egoísta defende avaramente na posse indivisa de todos os bens
materiais que possa alcan\-çar, justifi\-cando-se com a época de vida dificultosa
que não permite a distrac\-ção duma pequena parcela de sacrifício em troca
dum afecto, a que é impermeável, e com o incómodo de ter de aturar uma prole
impertinente. Em abono da sua tese desfiara argumentos ponderosos, certamente, sob o aspecto material, mas não merecendo uma palavra de discussão da parte de Eugé\-nio, para quem os liames do casamento teem qualquer coisa de mais
sublime que o pedagogo nunca compreendera, como nunca sentira quanto de edifi\-cante e construtivo tem o amor.

Que seria do homem, se a par das suas fadigas com que luta nas
angústias dêste escarcéu que é a vida, de incertezas e ilusões e espinhos, não tivesse o sublime prazer de se deliciar com os perfumados
eflúvios, com o êxtase de chocar os seus o\-lhos nêsse arrebol de belezas que em si encerra a rosa humana?

Se nos intervalos do seu monótono e afadigoso traba\-lho cotidiano
não tivesse em quem contemplar a graciosidade e multiformidade de
belezas, que só na jovem mu\-lher encontra, como só dela pode receber o
confortável amparo e carinho na senilidade?

Talqualmente a flôr nos fere a retina com as suas belas côres e os
seus aromas, em que a natureza se envaidece de mostrar retratada a
sua transcendên\-cia, a mu\-lher, no estado de perfei\-ção que a  ca\-ra\-cte\-ri\-za,
com a delicadeza da sua complei\-ção, nos arrebata estendendo-nos a sua
rede sedutora por cujas ma\-lhas se entrevê um idílio de quimeras, se
planeiam mil castelos de areia... Porque, como a flôr que reune à beleza a altivez do seu porte e a salubridade da atmosfera que a envolve, assim a mu\-lher à beleza alia a fecundidade do seu cora\-ção que nos
prodigaliza uma multidão infinita de carinhos com que nos consola
nos momentos de tristeza e nos fortifi\-ca nos transes de fraqueza
por que passamos.

O cora\-ção do homem só pela san\-ção dos mais puros laços de amôr no sagrado matrimónio se
completa. Sem dúvida os dois sexos não se
podem isolar porque formam fundamentalmente os alicerces em que se
baseia a conserva\-ção da espécie e a riqueza das Na\-ções, e o bem estar
da Humanidade.

Nada há que mais  sensibilize o espírito que ver encarnada a
igualdade, a harmonia dos gé\-nios, a homogeneidade do pensamento, em um
casal de que rescendente a felicidade, baseado nos laços do amôr, única base sôbre que se pode implantar solidamente o futuro dos nubentes, e
cada vez tendendo mais a serem estreitados pelo desabrochar das novas
vergônteas dêles resultantes, comêço de uma nova gera\-ção.

É que o amôr, êsse grandioso laço como que sôbrenatural, é o mais
forte sentimento por que à mãi doem os vagidos do recém-nascido das
suas entranhas, que com ternura oscula em seus braços! É que o amôr
é o mais inquebrantável elo que indelèvelmente vinca a união de dois
cora\-ções amantes que se fascinaram, criando-\-lhes um mundo de venturas!
É que o amôr, além disso, é êsse manto infinito que protege e ampara
uma humanidade inteira, fraternizando-a em uma doce e pura harmonia!...

Aquele homem não compreende, porém o amor no seu signifi\-cado transcendente, mas só o amor a que se podia chamar mercenário se êste adjectivo
inutilizasse de por si a ideia que encerra tão bela ex\-pres\-são. Aquele cora\-ção empedernido nunca experimentou os afectos duma fa\-mí\-lia constituída.
Vegetava no concubinato, e todo o seu enlêvo ia para os canários que possuia
em gran\-de número. De-resto viajava e descreveu como viu vários países estranjeiros por onde passou.

Era um revoltado. Os homens cultos não teem a protec\-ção do Estado que
deviam ter, não realçam na sociedade como era mister, e a ideia de se ver nivelado com outras camadas sociais, analfabetas ou quasi, obcecava-o ao ponto
de defender uma cultocracia organizada que se impuzesse nos destinos da
na\-ção. Assim pensava o pedagogo, egoistamente, como o obreiro manual podia
defender a manucracia, esquecendo que é do concurso comum, e por igual, que
resulta o progresso da sociedade. E esquecendo que a hierarquia das
profissões se diluiu já como a das
castas, e o orgu\-lho dos intelectuais,
seme\-lhante ao dos nobres, não tem
mais razão de existir, devendo bastar-\-lhes
a vantagem que aqueles ainda
possuem de verem me\-lhor remunerado
o seu traba\-lho e realizado em condi\-ções mais confortáveis.

Os argumentos com que viera em apoio do celibato não assentavam em
bases sérias para que houvessem merecido sequer a aten\-ção de Eugé\-nio, por
ventura solicitada para outros assuntos.

Eugé\-nio negou, como infundado, o boato que erroneamente se havia propalado a seu respeito, embora não convencesse o letrado.

-- Que é da boa praxe esconderem-se os preparativos do casamento, replicou, e você, meu caro Dr.\ Eugé\-nio, não foge á regra com a sua negativa. Mas
o facto é já conhecido, e estou bem informado até da pessoa de quem se trata,
por sinal senhora muito prendada e merecedora da felicidade que você lhe
proporcionará. Quere saber mais? Habita uma casa quasi ao fundo da Avenida dos Combatentes, que faz esquina para a rua por onde passa o electrico.
Só não me recordo se me disseram do lado direito ou esquerdo. E não me recordo de quem mo disse por muito que quisesse satisfazer a sua curiosidade.

Eugé\-nio ouviu estupefacto esta narrativa. Donde parte o boato? Habitando uma zona da cidade muito distante da sua, onde podia êle ter co\-lhido
tal informa\-ção?

O interlocutor não se lembrava, e Eugé\-nio muito menos o podia imaginar.

Decorria esta conversa três mêses após o regresso da excursão à Madeira, e outras se haviam de suceder no mesmo gôsto, a breve trecho.

Ainda não haveriam passado três semanas sôbre êste encontro quan\-do, numa noite, Eugé\-nio se avistou em casa de Marília com um sujeito que êle via
pela primeira vez. Também era a primeira vez que ali se encontrava o
Sr.\ Gentil, que tal era o seu nome, homem de meia idade e estatura mediana,
com uns o\-lhos vivos sobressaindo da sua cara arredondada, e uma pera bem cuidada a esconder-\-lhe o mento ; as unhas envernizadas condiziam com o esmero
da indumentária.

Não demoraram as apresenta\-ções. Falou-se sôbre variados assuntos, e de pessoas diversas. O Sr.\ Gentil tornava-se na verdade insinuante pela boa
disposi\-ção exteriorizada em bom e ininterrupto cavaco, em que era exímio, particularmente especializado no género da chalaça, e pelas maneiras polidas;
mostrava além disso ser muito relacionado na cidade, como homem de negócios
que era.

-- Há dias, disse êle ainda não havia passado uma hora de con\-ví\-vio, estava eu á porta do Grande Hotel do Porto conversando com o Sr.\ Frazão e o
Dr.\ Segismundo. A conversa derivou em determinada altura para a pregunta
que fiz de quan\-do se casava uma das fi\-lhas dêste que consta estar noiva. E
virando-se para Eugé\-nio: sabe a resposta do Dr.? Que ainda se não pensa
nisso. Que a fi\-lha mais ve\-lha do Frazão, essa sim, é que está noiva dum médico.

Eugé\-nio ficou perplexo num misto de estranheza e embaraço ao interrogar-se a si próprio: Como é que um indivíduo que se lhe apresenta pela primeira vez e que nunca vira, mostrava, com conhecimento de causa, saber da sua
simpatia? Como é que o Dr.\ Segismundo, amigo inseparável do Sr.\ Frazão, como
inseparáveis são as fi\-lhas de um e do outro, se faz eco de uma atoarda que não pode
desconhecer como menos verdadeira?

Marília e a irmã apressaram-se, mal o Sr.\ Gentil virou costas, a afirmar
a Eugé\-nio que não pensasse ter partido delas, porventura, a indiscre\-ção
que habilitara o novo visitante a abordar tal assunto. Sim, porque era natural supô-lo perante tão inesperada revela\-ção, acrescentaram em uníssono.

Eugé\-nio suposera-o, de-facto, á custa de não atinar com outra explicação. E de novo ficou sem saber a origem e o objectivo de tal boato.

No entanto as conversas dêste género não se lhe tornavam impertinentes,
nem Eugé\-nio fazia por dar essa impressão. Não o aborreciam. O seu sabor
era-\-lhe sempre grato. E como que apostadas em se precipitarem das mais desconexas origens, elas sucediam-se agora frequentemente a dar mais vulto ao
boato propalado por todos os cantos da cidade.

Ao declinar dum dia nublado de Agosto, no regresso da visita a um doente, Eugé\-nio encontrou-se na Constitui\-ção com um colega e antigo condiscípulo,
com quem aliás nunca mantivera outras rela\-ções além das de boa camaradagem.

-- Estive a falar muito de si há dias, começou o colega. Não imagina a
que respeito? Assuntos delicados que não vale a pena trazer a lume.

Eugé\-nio não podia imaginar, efectivamente, dado o minguado con\-ví\-vio com
o interlocutor. Insistiu que dissesse, ao ver o embaraço do colega para iniciar.

-- Já sei que está em breve para casar. Ainda bem que é com pessoa que,
se bem que a não conheça, sei possuir os me\-lhores predicados. Por isso o felicito.

E preparava-se para se despedir, mas Eugé\-nio é que já não deixou o colega sem que êste o informasse donde proviera a infundamentada informa\-ção,
o que só conseguiu após gran\-de insistência.

-- A pessoa que mo disse não a conhece você, nem ela a você. É uma senhora Cunha, de meia idade, residindo lá para baixo, longe daqui. Eu conto como
foi: Ao ser-\-lhe apresentado há coisa de um mês, logo indagou há quanto tempo
terminei a formatura e, satisfeita a sua curiosidade, preguntou-me se conhecia um médico que deveria ser do meu tempo, chamado Eugé\-nio. Conheço-o muito bem e até fomos condiscípulos, respondi. Exprimiu-me satisfa\-ção em poder
por meu intermédio saber informes sôbre o seu carácter, porquanto tal médico
é o noivo duma sua amiga, que ela muito preza, e a quem deseja imensas felicidades, alegando-me temer que o tal Dr.\ Eugé\-nio, que ela muito bem conhece
de nome, fôsse pessoa indigna da sua amiga querida. E agora desculpe-me repeti-lo diante de si aquilo que entendi di\-\mbox{zer-lhe}, que foi sómente o que %JNO
sinto: Falei das suas qualidades, da sua vida recatada, disse-lhe enfim
quanto estima me merece a sua pessoa desde os bancos da Faculdade, em que
nos conhecemos, até hoje. E ela assegurou-me estimar tanto os informes ouvidos a seu respeito, como estima a fi\-lha do Sr.\ Frazão.

E assim ficou Eugé\-nio mais uma vez ciente de quanto está espa\-lhada a
infundada notícia. Mais uma vez ficou sem atinar a origem e o propósito
daquilo que corria a seu respeito e de Princesa. A origem não podia deixar
de estar na fa\-mí\-lia Frazão. Mas o objectivo é que se conserva deveras enigmático, como enigmática foi sempre a conduta de Princesa para consigo.

\begin{center}\bf\large Um verão confuso\end{center}

O verão decorre êste ano nebuloso e frio, como nebulosa é a tétrica
expectativa duma situa\-ção interna\-cio\-nal indesejável e frio o ânimo de todos aqueles que
albergam em si um pouco de sensibilidade. Tempos de guerra em que a atmosfera se mantém pesada a irritar os nervos, proémio da borrasca que ameaça
continuamente descarregar sôbre a Europa inteira.

Agora está na berlinda a aspira\-ção da Alemanha em anexar Dantzig, a
cidade livre que lhe daria preciosa saída para o Báltico. A Polónia, reforçada com a alian\-ça ingleza, não cede os seus direitos sôbre aquele pôrto
de mar. Mas a Alemanha exige-o.

A Inglaterra e a Fran\-ça unidas, procuram a alian\-ça da Rússia, cujas
negocia\-ções já se arrastam há mêses em estudo, sem resultado positivo por
ora. Na esperan\-ça certa de a conseguirem, começam aquelas potências a bradar com firmeza a sua disposi\-ção de socorrerem a Polónia.

Em meados de Agosto surge a situa\-ção mais imprevista que considerar
se pode. De-repente, e sem que nada o desse a perceber, em plena fase das
negocia\-ções com a Inglaterra e a Fran\-ça, a Rússia assina uma alian\-ça com a
Alemanha perante a estupefac\-ção de todo mundo.

A Alemanha, ôntem inimiga figadal da sua nova aliada, rasga uma vez mais
o tratado anti-Komintern em que se mancomunava com outras potências no sentido de combater o comunismo; a mesma Alemanha que dizia o pior que se
pode dizer da Rússia, aprègoando uma ideologia política incompatível com a
sua, e que ainda há pouco combateu nos campos de Espanha contra tropas daquela na\-ção; a Rússia, amiga da Fran\-ça e da Inglaterra, e a elas ligada por
afinidades políticas, rompe com as negocia\-ções entaboladas. Alemanha e Rússia aparecem finalmente, dum instante para o outro, inesperadamente, unidas.

A política é assim. O interêsse sobreleva à ideologia. Quem não tem
visto partidos ôntem irreconciliáveis, mancomunarem-se a\-ma\-nhã entre si para
aniquilar um terceiro, primitivamente aliado, desde que os interesses duma
elei\-ção, ou outros, o imponham? Com os indivíduos sucede a mesmíssima coisa.
Entretanto os facciosos não o vêem, não o sentem, com a razão obscurecida
pela obcessão, não descortinam que com o seu sectarismo são levados para
onde me\-lhor apraz  aos chefes que os dominam como a carneiros abúlicos, e manejam como a autómatos, para satisfa\-ção exclusiva de vaidades ou interesses
pessoais...

A coerência de ideias implica rectidão de carácter, e o carácter está
em crise há muito.

De-resto as correntes de opiniões, crenças ou sectarismos de qualquer
natureza, são fi\-lhas do nosso comodismo mental. Pensar livremente, ter personalidade, acarreta dispêndio de raciocínio ou de responsabilidade, e é apanágio dum escol superior. Perfi\-lhar ideias é mais cómodo que criá-las.
A massa popular é devedora aos seus chefes de
pensarem por si, de lhe pouparem raciocínio e responsabilidade. Por isso
não é de mais que lhes pague com a satisfa\-ção das suas vaidades ou interesses pessoais. Só o comodismo explica e aceita o sectarismo. Comodismo ou
deficiência mental.

Hitler ainda ôntem, em holocausto às ideias anti-comunistas, sacrificou
mi\-lhares de vidas. Hoje não tem escrúpulo em se apoiar no comunismo, espesinhando ideias para defender interesses. Os seus apaniguados aceitam sem
discussão a nova directriz e, para justifi\-carem a sua atitude, já encontram
muitos pontos de contacto entre as duas doutrinas, ôntem irreconciliáveis...

A-quan\-do-da invasão etíope, em nome duma civiliza\-ção cristã, os italianos cometeram a maior casta de barbaridades contra uma na\-ção indefeza e
pacífi\-ca, embora cheia de heroicidade. Por último recorreram mesmo aos gases
asfixiantes, sem o que não teriam atingido o seu fim. Pois o catolicismo
apoiou incoerentemente a bárbara invasão.

Na hora presente, as ideologias políticas da Europa chocam com as crenças religiosas, em flagrante confusão de ideais. Acrescentemos-\-lhes os tratados de alian\-ça, conhecidos e secretos, existentes, e a confusão é ainda maior.
Adicionemos-\-lhe por último aqueles que, sem opinião, crença ou ideal que não
seja o interêsse, esperam a última hora para verem para onde pende o fiel da
balan\-ça antes de se pronunciarem.

Nêste momento o Japão protesta acerbamente contra a inesperada alian\-ça
germano-soviética, contrária na essência ao pacto anti-Komintern que aquela
na\-ção assinou igualmente.

A Espanha parece seguir no encalço do Japão, bem vincada tem na sua
mente a carnificina e a destrui\-ção que assolou recentemente o país.

A Itália, na\-ção tradicionalmente católica, manter-se-á na esteira da
Alemanha, profundamente ateia, e hoje ligada ao seu maior inimigo comum de
ôntem?

A Grécia, União Sul-Africana, a Austrália, prometem o seu auxílio ao
bloco franco-inglez, assim como a Hungria, amiga da Polónia.

A Eslováquia, embora ameaçada pela Alemanha, mas inimiga da Hungria,
enfileirará ao lado daquela.

Entretanto as tropas alemãs concentram-se junto da fronteira polaca,
prestes a romper fogo. A Polónia aguarda com calma e firmeza os acontecimentos. A guerra europeia considera-se inevitável. A ansiedade de todo o
mundo é manifesta.

Roosevelt dirige novo apêlo em favor da paz aos contendores. O Papa
secunda-o. Pelo menos trazem um compasso de espera que permite aliviar
um pouco a tensão que a todos esmaga.

Hitler mede forças; não a responsabilidade da luta sangrenta, mas as
possibilidades de vitória. A Rússia que ameaçava fazer aumentar o poderio em favor do bloco adversário, é bem agora uma fôrça a seu lado. Depende nês\-te momento da atitude da Itália o desiquilíbrio das forças dos
dois blocos contrários, afora os imprevistos que sempre aparecem até antes
do final da contenda... A Áustria, a Tcheco, a Albânia, procurarão de-certo
recobrar a sua independên\-cia ao ouvirem soar os primeiros  clarins de guerra
e tornar-se-ão outras tantas causas de fraqueza do eixo Berlim-Roma.

Estamos na última semana de Agosto.

\centerline{\rule[-1em]{0pt}{2.5em}. . . . . . . . . . . .}

Estava destinado o ano de 1939 para ser manchada uma vez mais a his\-tó\-ria da Europa com o sangue da guerra. No 1º de Setembro a Alemanha de Hitler
invade de súbito a Polónia.

Eis o rasti\-lho acêso.

\fotoC

Os exércitos franco-británicos, fieis à alian\-ça polaca, declaram guerra
á Alemanha, invadindo-a, não sem chamarem á razão Hitler e o intimarem a reco\-lher as suas tropas.

A Polónia fez-\-lhe frente, não obstante a despropor\-ção das forças. A Rússia,
duas semanas após a entrada dos alemãis, invadiu por sua vez a parte que consigo confinava. A Polónia inscreveu uma nova página de glória na sua his\-tó\-ria com a bravura com que defendeu até ao último reduto a independên\-cia
contra os dois invasores. Mais uma na\-ção desaparecida sob a voracidade insaciável das gran\-des potências.

As tropas francesas e britânicas, comandadas as de terra por Gamelin,
e as do mar e ar por
(...) \footnote{Espaço deixado em branco.},
continuam na sua ac\-ção. As primeiras estão
desde início instaladas em território alemão, e forçam Sarrebruck a renderse, enquanto a esquadra marítima dá caça á praga dos submarinos inimigos,
e a aérea auxilia uma e outra e sobrevoa Berlim.

Mas Hitler pede a paz. Declara não querer guerra com o Ocidente nada
mais pretende que a conquista da Polónia, termo último das suas revindica\-ções, a não ser as colónias que foram suas. O Ocidente porém não lhe dá a
paz sem a liberta\-ção da Polónia e das na\-ções anteriormente invadidas; e
mais: não trata com o govêrno nazista, cujos antecedentes desmerecem há
muito da confian\-ça de todo mundo, mas somente com um govêrno que me\-lhores
garantias represente.

A Rússia agora começa a revindicar mais larga saída para o mar Báltico.
A Finlândia, a Estónia, a Letónia, a Lituânia, amedrontam-se da invasão soviética. Mas a Alemanha também pretende comparticipa\-ção. Começam aqui as dissidên\-cias. O pacto germano-soviético não obriga a Rússia a auxiliar militarmente a Alemanha, e esta, sem o almejado auxílio daquela na\-ção, e economicamente bloqueada, volve-se para o Ocidente, sozinha, saindo pela primeira vez
da defeza para arremeter, nervosamente, na ofensiva, num estertor de morte
próxima. A Turquia alia-se á Fran\-ça e á Inglaterra. A Itália ainda se não
manifestou.

Estamos em meados dum Outubro frio, em que á calamidade da guerra se
junta a da fome, após um verão nada criador que tornou escassas as co\-lheitas.
É certo que tudo isso, como se pouco fôra, não modificou ainda o aspecto normal do mundanismo, que continua nas suas reuniões dan\-çantes, placidamente
a\-lheio ao que se passa. No estrangeiro, as máscaras anti-gás com espe\-lho e
aprestos para as pinturas do frívolo sexo, estão na moda, para exibir em substitui\-ção das carteiras de mão. Em Portugal só a ganância de extorquir o público consumidor, aproveitando a oportunidade duma situa\-ção anómala, nos lembra a realidade triste da calamidade que ameaça alastrar cada vez mais.

Que surprezas nos trará, como finalidade, a actual guerra? Ainda perdura no espírito de todos a radical influência da guerra de 1914-18, quer nas
ideias políticas, quer na economia colectiva e privada. As necessidades do
traba\-lhador manual aumentam, e nivelam-se com as do intelectual ou do burocrata. Na sociedade já quasi se distinguem somente aqueles que produzem,
manual ou intelectualmente, daqueles que nada produzem. Só o traba\-lho dignifi\-ca, e não a diferença de origem ou de reservas amontoadas. Por outro lado
a docilidade de servidores modificou-se imenso, como que adquirissem a consciência do valor com que contribuem para a sociedade. Ainda outro resultado
da guerra foi o desenvolvimento prodigioso da me\-câ\-ni\-ca, belo em princípio,
mas funesto pelas consequências desastrosas que trouxe dispensando uma gran\-de parte das necessidades sem operariado manual. Resumindo: demissão de traba\-lhadores atirados para a miséria; aumento de necessidades, agravando a miséria daqueles que não puderam acompanhar as novas condi\-ções de vida profundamente modifi\-cadas, de entre as quais avulta a ostenta\-ção e o comodismo.

A guerra de 1914 trouxe mais comodidades á vida, mas que se pagam caro, a sujei\-ção á modéstia desapareceu, as necessidades aumentaram correlativamente,
mas diminuiu o traba\-lho que a máquina substituiu, e daí um gran\-de desequilíbrio económico geral. Se a vida hoje é mais fácil, a miséria é maior. A efervescência da humanidade inquieta os governos que não conseguem pão para
a turba. Só governos fortes se aguentam.
Carlyle, citado num livro francês, de
Boigey, mede a felicidade do homem
por meio de uma frac\-ção, cujo numerador exprime as nossas possibilidades
e o denominador os nossos desejos ou
ambi\-ções. Ora, sob o aspecto material,
há dois processos de me\-lhorar a felicidade: 1º -- aumentando o numerador,
isto é, as receitas: 2º -- reduzindo o
denominador, isto é, as despesas. Este
último processo é sem dúvida o mais
seguro, mas geralmente procura-se realizar o primeiro muito incerto e inefi\-caz. A guerra de 1914 teve finalidade precisamente inversa, como a de
diminuir o numerador e aumentar simultaneamente o denominador da nossa
vida. Eis a causa do grave desiquilíbrio.

Por outro lado, as máquinas produzindo muito mais, com menos dispêndio,
enrique\-ceram uma minoria. A ostenta\-ção atinge aqui o má\-ximo. O mundo torna-se pequeno para dar largas ao esbanjamento da sua ociosidade. Ao mesmo
tempo, o modesto traba\-lhador de outrora tem uma no\-ção diferente do direito
de viver, que não possuia. A miséria de que vê rodeados os seus fi\-lhos, sem
o sustento necessário, após um traba\-lho árduo, contrasta com o supérfluo do
magnate. Este nega-\-lhe o preciso, para des\-per\-di\-çar na luxúria. O traba\-lhador
é pela fôrça das circunstâncias, agora, já despido da antiga docilidade, um
revoltado.

Á mesma revolta conduzem muitos outros. É o abastado capitalista que
aferro\-lha improdutivamente o dinheiro. É o gran\-de industrial que regateia
ao pessoal o suficiente para viver e o despede na ve\-lhice sem reforma, enquanto esbanja na orgia ou nos festins
aquilo que tanto suor custou aos seus operários. São os gran\-des proprietários que gozam na cidade aquilo que os outros moirejaram nos campos, sem encargos de administra\-ção do que muitas vezes nem sequer nunca conheceram. São
as mu\-lheres que dão largas a uma vida ociosa, na exibi\-ção de indumentárias
caricatas de luxúria e impudor, realçadas por joias extravagantes. São os
cirurgiões que se negam a salvar a vida dum pobre por êste não poder satisfazer a elevada quantia exigida. São os gastrónomos que se refastelam a
digerir iguarias caras, servidas por criados fardados, entre paredes cobertas
de pratas e iluminadas por candelabros vistosos, á vista de um mendigo que
passa a cambalear de fome ou negando a um faminto os sobejos, que prefere
deitar fora. São todos aqueles, em suma, capitalistas, industriais, comerciantes, agricultores,
que abusam do capital realizado com o auxílio, de outrem a quem negam o salário indispensável. Direitos de gozar o supérfluo, para uns; dever de morrer na miséria para os outros...

Eis os gran\-des inimigos da sociedade, êstes miseráveis que da explora\-ção vivem, que à custa da explora\-ção ostentam uma vida faustosa, e que não
teem repugnância em ver os seus assalariados morrerem de fome, á míngua de
pão para si e para os fi\-lhos, enquanto êles gastam nos vícios somas avultadas de dinheiro! Eis os verdadeiros responsáveis pela revolta do povo tra\-lhador e faminto que cresce como cogumelos na podridão.
Repudiam o sistema comunista, mas são eles os maiores culpados das ideias comunistas que precipitam. Não são comunistas, pois, mas são
comunizantes êstes tiranos do capital.

Conta-se que numa das maiores fá\-bri\-cas do Norte uma comissão de operários se dirigiu um dia ao dono a pedir-\-lhe um aumento de salário.

-- Não podemos fazer face á carestia da vida, explicaram. Temos mu\-lher
e fi\-lhos, que passam fome não obstante o nosso traba\-lho intensivo.

-- Impossível, respondeu-\-lhes o industrial. As minhas despezas não o
consentem.

Alguém do lado atreveu-se a recordar-\-lhe que no avultado rendimento
da sua avantajada fortuna nem sequer se notaria um pequeno acréscimo do
salário.

-- Enganam-se, objectou o magnate. Só um fi\-lho que tenho em Paris me
absorve a totalidade dos rendimentos.

-- E V.\ Exª acha justo, retorquiu um dos da comissão, que para um seu fi\-lho gastar desalmadamente dinheiro, em bacanais, se passe fome nas casas dos
humildes que com V.\ Exª colaboram?...

São pois todos êstes comunizantes que, com os seus desvarios, desorganizam e dissolvem a sociedade, abalando o que esta tem de mais belo -- a harmonia, conduzindo ao nervosismo e á efervescência constante os indivíduos.

Mas os desvarios são tantos sem quantidade, e tão variados em qualidade,
que emprêsa difícil seria enumerá-los. Ainda há bem pouco tempo correu célere pela cidade o seguinte episódio, passado num salão de chá da baixa,
onde, nas tardes de certos dias da semana tidos como da moda, as mesas costumam
estar replectas. Êstes dias depressa se revelam ao transeunte curioso pelo
estacionamento de gran\-de fileira de automóveis que então ali se juntam, uns
conhecidos como pertencentes a fa\-mí\-lias falidas material ou moralmente, outros a rapazes tendo por única profissão a conquista duma noiva cujo dote
\-lhe assegure um futuro de calma ociosidade, outros a indivíduos insolventes;
uns a divorciados por culpa da mu\-lher que o mundanismo to\-lheu, outros a divorciados por culpa do homem que não nasceu para constituir lar. Dentro
prepondera o elemento feminino representado pela mais requintada sociedade
elegante que empolga habitualmente com a sua presença as reuniões chiques,
as sessões de moda dos teatros, os casinos, fazendo bri\-lhar joias caras compradas a crédito, ou pagas com desfalques, e para quem a vida se limita ao
gôzo enquanto o lar se dissolve; senhoras casadas ou raparigas solteiras,
tomando o seu chá ou queimando cigarros, de pernas terçadas, a apreciar em
quem passa na rua, ou a criticarem o caso anunciado daquele titular arruinado que vai casar-se com uma herdeira rica mas boçal. Não é somente o sangue
azul que ali fervi\-lha, aquele sangue que outrora tinha o exclusivo do ócio,
mas também o sangue amarelo do oiro, e um e outro não são mais que a degenera\-ção do sangue verme\-lho-rutilante dum escol de traba\-lho afirmado em acrescentar novas páginas á his\-tó\-ria ou por qualquer modo em tornar activas as
energias latentes da na\-ção.

Mas vamos ao episódio. Numa tarde em que o salão estava á cunha, um borborinho chamou a aten\-ção de toda a assistência para a mesa onde duas senhoras gesticulavam raivosamente, ameaçando-se mutuamente; da ameaça passam
rapidamente a vias de facto, engalfinham-se e a mesa rola pelo chão já transformado em estendal de bolos e fragmentos de chávenas e esti\-lhaços de cristal. Acodem criados a reco\-lher a mesa mutilada e limpar o pavimento, e interveem circunstantes a apaziguar os ânimos, enquanto naquele selecto ambiente
de perfumes e cores garridas, incidentalmente confuso, se cruzam com os gritos nervosos as censuras contra o escândalo e as interroga\-ções sôbre o motivo de tal desaforo. Eram para mais parentes muito próximas as senhoras desavindas, e ambas casadas. O mistério desvendou-se, porém, depressa, ao espa\-lhar-se que a razão da contenda foi a passagem dum oficial do exército, também casado mas ainda credor dos ilícitos afectos de ambas as rivais...

A guerra de 1914 acarretando outras exi\-gên\-cias de vida fez com que as
profissões liberais se pejassem em detrimento das manuais. Os campos fi\-caram
sem braços, e as cidades congestionaram-se na espectativa duma vida mais
fácil, com mais comodidades e prazeres e, presumivelmente, com me\-lhores proventos. As vagas do Estado, no comércio, ou nos escritórios, começaram a disputar-se por indivíduos com habilita\-ções superiores ás exigidas, tal a luta pela
vida que se verifi\-ca perante a plétora que cada um vai encontrar na profissão a que se destinava. Aparecem novas profissões, de mal limitado âmbito,
cujos interêsses se vão chocar com os das já existentes: a dos enfermeiros,
por exemplo, vegeta á sombra do comodismo dos médicos e dizima-\-lhes a actividade, invadindo a sua esfera o que, com a plétora, contribue muito, para fazer
cair os profissionais da medicina num proletariado próximo que já
os ameaça.

Outra profissão, fácil, deriva do aumento do automobilismo: onde quer
que se aglomerem carros, logo alguns moços correm a revindicar o direito de
guardar... objectos que aliás não teem guarda, como mero expediente para
fazer jus a uma espórtula. Em certas ruas, por exemplo, onde nos dias de jôgo
da bola se aglomeram centenas de carros, é fácil observar como os moços correm em azáfama constante a oferecer os seus serviços aos que chegam: depois,
ao verem que não veem mais, desaparecem, para só voltarem aos seus postos
quan\-do o jogo está a terminar, com o fim de cobrarem a gorgeta por uma fictícia guarda. Entretanto os garotos mexem impunemente nos carros, certos de não serem
incomodados...

Junto dos teatros e cinematógrafos os contractadores açambarcam os
bi\-lhetes, por que pedem mais caro, sem terem prestado qualquer serviço ao
público que não poupa passo algum para os adquirir, como aliás aconteceria
se êsses homens os fôssem oferecer pelas ruas mais afastadas.

Eis aqui exemplos de profissões que se não justifi\-cam, enquanto que os
campos lutam com a falta de braços, e que revertem em puro parasitismo. Originaram-se nas novas necessidades que a guerra de 1914 trouxe á vida.

Essas necessidades levaram ainda a criar processos ilícitos de ampliar
os proventos: as fraudes alimentícias tomaram vulto de flagêlo social. Estas
são as primeiras responsáveis pelo acréscimo de doenças verifi\-cadas na
humanidade, ao lado dos gases tóxicos, espa\-lhados na atmosfera
a-quan\-do-da conflagra\-ção europeia.

Enfim, a guerra de 1914 modificou radicalmente a economia dos povos e
a saúde do povo. Que nos trará a actual? É a interroga\-ção que todos fazem,
e para cuja resposta não será necessário ser demasiado pessimista augurando-\-lhe calamitosas consequências.

Para já, nota-se a confusão das ideias políticas, apanágio dos se\-ctá\-rios. Ainda há pouco tempo se podiam reduzir a duas as correntes ideológicas  por\-tu\-guêsas: Uma conservadora, outra democrática, ou, mais modernamente,
a dos na\-cio\-nalistas e a dos oposicionistas. Na presente conjuntura, a barreira entre conservadores e liberais diluiu-se. Há germanófilos e germanófobos.
A Alemanha, dizem uns, necessita de ocupar o merecido lugar na Europa como
na\-ção que é dum povo de admiráveis qualidades de actividade. Para outros
impõe-se o aniquilamento daquele foco de efervescência europeia. Os primeiros consideram arbitrária a forma como o tratado de Versa\-lhes desmantelou
o povo vencido em 1918, enquanto que os segundos afirmam que o desmantelamento ficou aquém do necessário para a manuten\-ção duma paz duradoura.
Ideias germanófilas e germanófobas não se sobrepõem, porém, ás ideologias anteriormente existentes. Se conservadores e democráticos se conluiaram para, na sua maioria, apoiarem  as democracias francêsa e britânica, opondo-se á megalomania germânica, outros há que, do lado conservador, manteem-se em atitude hostil para com as democracias ás quais preferem o comunismo verme\-lho, ou, do lado democrático, evoluiram para ideias mais avan\-çadas que outra coisa não são que comunistas. Eis aqui a confusão: conservadores, na sua maioria católicos, feitos simpatizantes da Alemanha, que é ateia e aliada dos comunistas; democráticos infieis á sua doutrina, apologistas dum sistema totalitário, como o é o alemão, e mancomunados com uma ditadura como o é a russa. Outros ainda desejariam o impossível: apoiar a Alemanha e não a Rússia, e ainda outros apoiar a Rússia e não a Alemanha. Finalmente há quem preconize a modifi\-ca\-ção radical dos dois sistemas antagonistas que se debatem: em lugar das democracias procurarem aniquilar o germano-sovietismo, deveriam coligar-se com a Alemanha para formarem barreira comum contra o perigo comunista.

 As ideias, muitas vezes incoerentes, explicam-se. Domina no espírito de uns a solidariedade religiosa noutros a política, noutros a raça, noutros ainda o interêsse particular. Na  actual conjuntura prevalece, porém, o sentimentalismo que condena as invasões ou as admite, a dividir em dois gran\-des sectores a opinião pública,  independentemente de ideologias que se confundiram, porque, tomadas em bloco, são fundamentalmente insustentáveis e só discutíveis parceladamente.

Eis o que para já nos trouxe a actual guerra, além das muitas vidas ceifadas e dos mi\-lhares de contos que se desperdiçam diariamente, afundando navios ou queimando torpedos, e que me\-lhor destino tinham na redu\-ção da miséria dos necessitados.

\centerline{\rule[-1em]{0pt}{2.5em}. . . . . . . . . . .  .}

Assim se passou êste verão nervosamente. A catástrofe iminente, e mais do que isso, a inconstância do tempo, frio e chuvoso, não permitiu gran\-des exibi\-ções mundanas nesta época.
As praias e termas, ressentiram-se. A Foz esteve desanimada como nunca.

Miramar foi a praia da moda êste ano, com a inaugura\-ção do Parque da
Gândara que lhe insuflou desusada actividade vital. Ali, os arraiais minhotos, e chás dan\-çantes aos domingos, descongestionavam a sociedade elegante
do Porto.

A fa\-mí\-lia Frazão, açodada pelos cuidados que a irmã de Princesa começou
a inspirar com a sua febrícula renitente, teve de procurar es\-tân\-cia, a conse\-lho médico, onde pudesse beneficiar simultaneamente dos ares do mar e dos
pinheirais. A ridente praia de Miramar, com os seus palacetes ajardinados e
as suas avenidas muito limpas, foi o retiro esco\-lhido. O sossêgo ali é absoluto, só interrompido pelos silvos dos comboios, e os ares são magníficos. A
proximidade do Porto é outro factor a juntar àqueles.

Eugé\-nio teve ensejo de por ali ir nalguns domingos. Só uma das vezes
viu Princesa. Tê-la-ia visto de-certo todas as outras vezes se entrasse no Parque.

Mas este, tendo para Eugé\-nio acentuado sabor a mundanismo,
e\-no\-ja\-va-o, e nunca lá entrou. E em Setembro retirou-se para a sua pacata
aldeia a saborear as delícias do campo, onde não há mundanismo mas a simplicidade e a pureza. Antes, porém, de sair para a aldeia, um cuidado teve: o de
se despedir da sociedade de que participava o sr. Frazão, a trôco de qualquer pretexto; efectivamente, se o estatuto de tal colectividade era belo,
a sua prática limitava-se a dois números primordiais -- a gastronomia e o
exibicionismo. Nada interessava, pois, a sociedade a Eugé\-nio, com a agravante
de se tornar dispendiosa e incompatível com os seus hábitos de economia.


\centerline{\rule[-1em]{0pt}{2.5em}. . . . . . . .}


O casamento do Zèquinha, realizado durante à ausência de Eugé\-nio, era
ainda, quan\-do êste regressou ao Porto no princípio de Outubro, o facto da
ocasião.

Foi o Toninho quem se apressou a referi-lo a Eugé\-nio. Um dia os noivos
passam, a pé, a caminho do registo civil. Alguns metros atraz caminhavam as
mãis de um e outro. Dois dias depois efectuou-se o acto religioso na capela
da casa da noiva, jantaram com número limitado de convidados, e os noivos
foram até ao Bussaco. Mais dois dias passados já cá estavam.

E virando-se para Eugé\-nio: -- Já viu casamento mais pelintra?

Eugé\-nio afirmou que sim, que já vira casamentos, embora poucos, com tanta
soma de bom-senso. Que o que acabara de ouvir narrar não é mais que a atitude daqueles que vêem no matrimónio um acto íntimo e não a exposi\-ção espectaculosa dos noivos, desfilando em longo cortejo de automóveis perante a
curiosidade do povoleu e porventura sujeitos ás aprecia\-ções, por
vezes, maliciosas, vindas da massa anónima dos espectadores que abrem fileiras á saída da igreja. E os múltiplos convidados que vão fazer? Exteriorizar, em brindes, a hipocrisia dos votos que fazem pela felicidade dos noivos?
Mas êste desejo só é bem sentido pelas fa\-mí\-lias dos nubentes, e a elas se
deve limitar a gala. Grande comezaina e muitos foguetes, para quê? Me\-lhor
comemora\-ção da cerimónia seria servir uma refei\-ção aos pobres. E rematou
alegando que, se alguma coisa tiver concorrido para a simplicidade do casamento do Zequinha, com as conversas que com ele teve nesse sentido, por muito feliz se dá
pois que é assim que compreende toda e qualquer cerimónia íntima.

O Toninho foi levado a lastimar-se por se basear demasiado nos costumes, sem nunca se ter ainda lembrado de pensar que na realidade é assim mesmo. Mas um médico... objectou êle. Um médico tem responsabilidades sociais.

-- Tem sim, como toda a pessoa, professe ela a advocacia, o ensino ou as
artes. Como todas as pessoas, o médico tem a responsabilidade do juízo na
conduta, e o juízo conduz à modéstia e á sobriedade, como únicos princípios
duma vida sã, e á personalidade que lhe traça a rota das ideias e hábitos
sem a peia dos princípios e costumes que foram ou são apanágio de mentalidades antiquadas.

A-propósito: Quando se casa você?

A inconstância do Toninho não lhe permitia persistir no mesmo em\-prê\-go,
como êle denominava os namoriscos, mais de um mês. Por vezes, ao fim de uma
semana de conquista, ou até menos, ele estava desejoso de arranjar pretexto
para abandonar o em\-prê\-go. A sua perícia consistia na conquista, e uma vez
o acto consumado, desaparecia rapidamente o interêsse.

Com a transmontana parecia que se daria a primeira excep\-ção, mas tal
não aconteceu. Após uns idílios há um ano, pela mesma ocasião em que Eugé\-nio a conheceu, o
entusiasmo de Toninho desvaneceu-se como uma espírala de fumo. Pouco tempo
depois era êle próprio que falava da sua antiga apaixonada com desdém: Pinta-se, monta a cavalo, guia Vau-8 e joga hockey em patins. Tal qual o Sr.
Dr.\ a caracterizou, acrescentava êle.

Em Milita também o Toninho nunca mais pensou. De-resto os predicados
que a exornavam há um ano para traz modifi\-caram-se muito ao dobrar os 19
anos. O carmim dos lábios já não era o seu, como seu não era o carmim das
faces. Os pretendentes a falar com ela formavam legião, e aguardavam a sua
vez que se não faria demorar. Havia um só requisito a exigir aos candidatos para que vissem deferida a sua pretensão: rechonchuda coloca\-ção que
garantisse ao candidato vida desafogada. Se tivesse meios de fortuna, me\-lhor
ainda.

-- Que quere? dizia a mãi de Milita á vizinha do lado. Para manter a
minha fi\-lha no estadão em que a criei não serve qualquer rapaz. Com ordenado abaixo de um conto, escusa quem quer de aparecer.

Os pretendentes sucediam-se, de facto, fascinados por aquele palmo de
cara, para serem rapidamente despedidos por não atingirem o mí\-ni\-mo de rendimento exigido que suportasse aquele estadão, ou para se despedirem eles ao
verem o desfalque de juízo produzido naquela gente.

\begin{center}\bf\large Um baile de Carnaval\end{center}

A guerra continua, devastadora e tétrica, como maldi\-ção caída sôbre
a Europa. Recai mais a aten\-ção do povo sôbre o Norte, onde há 2 mêses a
Finlândia, invadida pela Rússia, luta pela sua independên\-cia.

Foi no princípio de Dezembro que o colosso soviético atravessou
as fronteiras da sua vizinha, justifi\-cando-se perante o mundo com acusa\-ções de maus tratos dela recebidos. A lamúria era plagiada da Alemanha
no recente ataque á Polónia. O que é certo é que desde há 2 mêses, na mais
desigual das lutas -- 3 mi\-lhões para 140 -- a Finlândia tem mostrado uma
heroicidade e valor militar que suplantam todas as conjecturas optimistas que sôbre tal se aventassem. Em 15 dias calculavam os russos englobar nos seus domínios a desventurada Finlândia; há 2 mêses que a enorme
superioridade numérica dos invasores é subjugada pela superioridade intelectual dos invadidos.

E o mundo inteiro palpita de emo\-ção e exulta de contentamento nos
transes gloriosos por que a Finlândia, crèdora da simpatia geral, resiste aos ímpetos do adversário espumando rancor nas suas arremetidas. Ainda mais que na conquista da Polónia, a bravura do pequeno povo finlandez
faz vibrar em uníssono a alma de todo universo cada vez mais sequioso
de justiça contra a barbárie imperialista.

A Fran\-ça e a Inglaterra continuam a sua luta contra a Alemanha, que
antes pretendem fazer render pelo cêrco económico que pela dizima\-ção de
vidas. Por isso a guerra no ocidente é mais calma e não chama tanto as
aten\-ções.

O inverno corre, entretanto, excepcionalmente frio. O Janeiro é assinalado por nevões que cobrem toda a Europa e até o norte portuguez, com
a sua alvura incomparável; os lagos finlandezes gelam na sua totalidade
e até o mar Báltico numa distância de 60 Km da costa. Dir-se-ia que a
natureza acode ao pequeno povo, dificultando a marcha dos rapinantes invasores, e mostrando a todo mundo a necessidade de paz com o manto alvo
que a todos os europeus cobriu.

Improvizam-se bailes e récitas de caridade a favor das vítimas finlandezas. Uns e outras parece que pela primeira vez dão lucro para minorar a miséria a que se destinam. A sociedade elegante, ao encontrar
mais um pretexto para dar largas á sua ociosidade, torna-se desta vez
útil. A iniciativa resulta feliz, embora a finalidade real seja discutível
como o meio empregado. De resto, o gôzo mundano decorre em todas as suas
modalidades, plácido e sereno, como se nada houvesse a denegrir a humanidade. A quadra de Carnaval, porém, só de nome ainda existe, mas mais por
imposi\-ção da autoridade e da invernia de domingo gordo e terça-feira de
entrudo que pela sensibilidade embotada dos homens; só de portas a dentro
se permitiram folias.

Tinha Eugé\-nio por esta ocasião o tempo todo tomado no exercício dos
seus deveres profissionais, e as noites que êle já normalmente aproveitava
para o merecido descanso eram desta vez ainda mais esperadas para recuperar as energias gastas durante o dia. Mas, contrafeito ou não, todas as
desculpas foram indeferidas para passar a noite do sábado gordo em casa
da fa\-mí\-lia de Marília, onde se realizava uma reunião dan\-çante, e teve de
assistir. Ali, entre numerosas fa\-mí\-lias amigas da casa, estava todo presente o grupo de Princesa. Era Laurinda e a irmã, suas primas, e as duas
irmãs do colega de Eugé\-nio que tão bem informado se mostrara a seu respeito há anos. Só faltavam as outras duas irmãs que Eugé\-nio conhecera
nesta mesma casa, vestidas de preto, há um ano, e cuja falta justifi\-caram
por doença. Ofélia e Marília ali estavam, e eis as pessoas a que se limitava o grupo, outrora numeroso, de Princesa, por quanto as restantes as havia afastado a constitui\-ção do lar.

Eugé\-nio passou toda a noite agradavelmente, dan\-çando sempre, e as dêste grupo fôram as mais preferidas por êle. Nos intervalos, passavam na
 tela da discussão os vestidos fantasiados das raparigas, os galanteios
observados pelos o\-lhos curiosos ou importunos, a compostura dos rapazes,
a anima\-ção de todos.

Duas raparigas sentadas, conversavam. Uma, a mais nova, de 17 anos, virada para a sua amiga, de 25, ensinava a esta a forma de captivar
a simpatia dos rapazes.

-- Tú não os o\-lhas de frente, quan\-do êles falam, explicava a mais nova. Assim não arranjas namôro. Eu o\-lho-os bem, para os conquistar.

Noutro sector da  sala, uma rapariga pedia conse\-lho à companheira.

-- Estou indecisa, dizia. Tenho dois pretendentes: um, muito rico, mas sem
modo de vida; o outro é pobre, mas possui uma licenciatura. Por qual
optar?

De resto apreciava-se a elegância dos pares dan\-çantes, a pintura
das raparigas, o serviço das comezainas e a simpatia maior ou menor dos
componentes do sexo contrário que se acotovelavam na sala, á mistura com
o jôgo da metra\-lha carnavalesca.

Laurinda derivava todas as conversas para o cepticismo com que via
todos os homens, que acoimava de incoerentes. Só quem soubesse a vida
íntima desta rapariga, compreendia a razão da sua má-vontade para com o sexo forte que
\-lhe proporcionara a maior das desilusões. Namorava ela com um rapaz, de
quem gostava muito, quan\-do êste se casou quasi repentinamente com outra
rapariga. A causa da troca foi o interêsse material, dizem as amigas;
Laurinda é pobre, embora dotada de excelentes qualidades, e o rapaz preferiu uma rapariga rica que lhe permitisse a posse do automóvel que hoje
tem. Mas Eugé\-nio é que não podia deixar passar sem protesto a generalizaza\-ção dum facto desa\-gra\-dá\-vel, que êle era o primeiro a lastimar, mas não
achava razoável pagar o justo pelo pecador. E nin\-guém tinha mais autoridade que Eugé\-nio para proclamar a coerência dos homens... Por isso
interpelou-a nêsse sentido:

-- Então para si todos os homens são incoerentes? Há-os sim, mas
também os há bem coerentes. E sublinhou com a tonalidade da voz esta
afirma\-ção. Assim, como de resto, há mu\-lheres incoerentes, e talvez em maior
número que entre os homens.

-- Bem sei que há excep\-ções, respondeu Laurinda que bem percebera a
alusão de Eugé\-nio feita a Princesa. Mas são excep\-ções para mais confirmarem a regra.

-- Então, ata\-lhou Eugé\-nio, permita-me que só como excep\-ção tome igualmente a coerência no seu sexo.

Na rua, quan\-do dentro a algazarra serenava um pouco, ouviam-se cair
bátegas de água que formavam enxurrada.

E ao nascer o dia de domingo, e só então, toda aquela gente abandonou a casa do Largo da Lapa que tão boa guarida lhe proporcionou durante uma noite inteira.

Na noite seguinte, por via certamente da actividade do inconsciente, Eugé\-nio sonhou estar em casa de Princesa. Esta requisitou os seus
serviços profissionais para um parente distante, á porta de cuja casa
o acompanhou. Chegados ali, Princesa disse que fôsse Eugé\-nio visitar o
doente, que ela esperaria cá fora. Eugé\-nio em vão tentou dissuadi-la
de fi\-car cá fora sòzinha e, perante a renitência de Princesa, correu a
casa dêle a buscar o seu automóvel onde então mais comodamente ela podia
aguardar o regresso da visita.

E acordou.

\begin{center}\bf\large As Baronezas\footnote{A lápis: (10-3-40).}\end{center}

Chegara o mês de Março. O ceu agora desanuviado contrasta com o dos
intermináveis mêses de inverno que o antecederam, de copiosa chuva, como
a atmosfera agora tépida contrasta com o frio que abundou.

O ar dos campos e as flores que os exornam são chamarizes que de
òra avante descongestionam as cidades aos domingos. Com efeito, todo o
indivíduo sente a imperiosa necessidade de espairecer após uma semana
de traba\-lho, e só quem não pode abandonar a cidade nos dias de descanso
é que o não faz, porque só longe é que o repouso é completo, desembaraçado dos ruídos, do bulício e da pasmaceira das ruas ao domingo. Largam
por essas estradas fora aqueles que possuem algo de seu nos arredores,
que visitar; largam aqueles que nada teem, a passar o dia por algum
sítio mais pitoresco, com as suas merendas; largam outros sem destino, de
automóvel, a galgar em algumas dezenas de léguas por essas terriolas fora,
para comerem longe, em casa de pasto ou dos farnéis que levam.

Esgotam-se então as especialidades que em arte de pastelaria são
conhecidas por êsse Norte além -- os bolos de amor de Casais Novos, os jesuítas de Santo Tirso, as lérias e os foguetes do Reis, de Amarante, as
clarinhas de Fão, e outras doçarias mais. Em algumas destas es\-tân\-cias é mesmo de
bom tom tomar chá aos domingos, e com isso enriquece o pasteleiro que,
sôbre levar caro para manter a hierarquia conquistada, não tem mãos a
medir no seu rechonchudo negócio onde só entra quem pode pagar.

Um lugar há, porém, que se torna obrigatório em tais dias para toda
a pessoa com pretensões a chique, ou porque tal se convencionasse entre
a sociedade elegante, ou porque êsse lugar constitui já de facto uma
praia que usufrui foros de aristocrática. É Vila do Conde. A sua
casa de chá é, sem favor, esmeradamente servida, e a nova instala\-ção é a prova
evidente que a casa deu ao dono bom rendimento, com o bom doce fabricado.
Disputam-se ali as mesas nas tardes dos domingos de sol, não tanto pelo
esmero do serviço, mas sobretudo porque abundam os pretendentes a
chiques.

Foi num dêstes domingos, em vésperas do solstício da Primavera, que
a casa do "bom doce" de Vila do Conde estava replecta por volta das 5
da tarde. Só duas outras fa\-mí\-lias presentes eram da localidade; todas
as mais eram do Pôrto. Os automóveis chegavam constantemente a despejar
novos freguezes, que tinham de aguardar vez para se sentarem.

No centro da sala, haviam-se juntado 2 mesas para abancarem duas
fa\-mí\-lias da mais elegante estirpe tripeira: os barões do Corvo com
seus dois rebentos, do sexo feminino, e os Ortigões. Formavam os papás
barões com os papás Ortigões sec\-ção à parte de grave cavaqueira em que
nos não vamos intrometer, mas cujo assunto deveria á-certa andar em redor da situa\-ção crítica que no momento passa a Finlândia, possivelmente
em vésperas de capitula\-ção por causa da tão comentada atitude da Suécia
e da Noruega, que não só a não auxiliam como não deixam passar os aliados em seu socôrro. Formavam grupo autónomo as fi\-lhas de um e outro lado, as baronezas do Carvão, como no Pôrto eram conhecidas (talvez por
analogia de côr com a ave que ostentavam no seu brazão), e as amigas
Ortigões em número de tres, e umas e outras bem se podiam incluir numa escala de idades
entre os dezoito e os vinte e cinco anos. Fisicamente nada devia, nenhuma delas, à boniteza, mas as baronesas sempre suplantavam as Ortigões em fealdade, com os seus dentes superiores projectados para fora dos lábios
por debaixo duns narizes aduncos de papagaio.

A conversa\-ção de tão elegantes raparigas realçava, pela sua anima\-ção, sôbre a dos papás. Talvez que pelo hábito do corte,  adstrito ao sexo feminino, em que a tesoura constitui o instrumento tão precioso como imprescindível, o belo-sexo tem sempre que dizer porque, á míngua de outro
assunto, sobeja sempre o que sôbre as amigas ausentes há que criticar.
Entremos um pouco na indiscre\-ção, e ouçamos o grupo.

-- Não acham que fizemos bem desculpando-nos de não podermos ir
ao chá dos anos de Julieta? preguntou Ambrósia, a mais ve\-lha das baronezas do Carvão, embora a mais baixa e espevitada das irmãs.

-- Muito bem! apoiaram todas.

-- Vocês ainda as lá vão aturar ao jantar, interrompeu uma Ortigão,
enquanto que nós descartámo-nos delas, e só lá vai uma de nós porque
parecia mal não aparecer nin\-guém.

-- Que se passará a estas horas em casa das Frazões? preguntou a
Henriqueta, a mais nova das baronezas, como quem descarrega a consciência
dizendo qualquer coisa.

-- Ora o que se há -de passar? respondeu a Ambrósia. Aquilo que se
verifi\-ca sempre nas reuniões daquela casa. Tudo gente ``pires".

-- Uf! interrompe uma das Ortigões, virando a cara. Que roda ali se
junta! Que insipidez de amigas teem aquelas raparigas!

-- Que mistura! ata\-lha Henriqueta. Como é que a gente se há-de sentir ali bem?! Aquelas primas são tão antipáticas, credo!

-- E aquelas duas do Pinheiro Manso, e as de Barros Lima, e as da Lapa,
de quem nem sei o nome nem nunca procurei sabê- lo? acrescentou uma das
Ortigões. Oxalá que tudo aquilo debande antes do jantar.

-- Oxalá! repetiram as baronezas em uníssono.

-- E fôssemos nós aturar aquelas sensaboronas! continuou outra das
Ortigões. Que tarde de tão mau passadio nos estava reservada!

-- Isto aqui entre nós, que nin\-guém nos ouve, recomeçou Ambrósia em
voz segredada para as outras, nós precisamos de presumir um pouco mais
na convivência. Mesmo as Frazões não sei por que artes se atrelaram á
nossa roda... Sim, porque a mim e á mana devem elas entrada na sociedade. Desde há poucos anos a esta parte é que aquelas raparigas aprenderam a viver a vida, e a nós o devem. Até então as Frazões eram para
ali como as outras que a esta hora estão em sua casa. Só resta agora
desembaraçá-las daqueles empeci\-lhos que ainda as visitam. Felizmente
já se casaram muitas das do grupo, e as outras morreram, segundo elas contam.
Isso é que devia ser uma sociedade completa, quan\-do se juntavam todas!

-- O que é pena, interveiu a baroneza mais nova com ares de contristada, é que a Julieta não pudesse ir aos bailes do Carnaval êste ano.
Aquela doença dela... Quan\-do a vida lhe começava a sorrir é que adoece. É
azar! Com o andamento que leva, nem para as próximas festas do Centenário ela está em condi\-ções de as gozar. Eu não sei que doença é aquela,
mas não é coisa boa; quan\-do a febre é teimosa e os médicos começam a
falar em sais de ouro, como eu já ouvi, é fraco sinal. E é tão boa, coitada,
tão ingénua! A Princesa, essa, já se tem tornado um pouco petulante; toda
a sombra de homem que vê lhe parece a do doutor a persegui-la, e as conversas dela derivam sempre para as pretensões dele: êle foi à Madeira
por causa dela, êle só acompanhava na i\-lha com pessoas amigas dela com
o fito de procurar ser-\-lhe apresentado, êle foi para a sociedade do pai
para se insinuar no espírito da fi\-lha, e até, para cúmulo, comprou um carro
para lhe agradar... Aquilo nela já é gabarolice demasiada. É verdade:
já lhes constou que os dois estão noivos? É voz corrente...

-- Já, já, responderam ao mesmo tempo muitas vozes.

-- Se ca\-lhar êle nem para ela o\-lha, obtemperou com uma garga\-lhada
uma das Ortigões que até aí se conservara calada, o\-lhando a palradora que
a antecedeu, como quem aguarda confirma\-ção à sua dúvida.

-- Isso não sei. Só sei que nunca o vi senão na excursão á Madeira,
e nunca nos nossos bailes. No verão passado, todos os domingos em que a
sociedade elegante do Pôrto se reunia na Gândara, êle tinha todo ensejo
de lá aparecer. Pois nunca o enxerguei. Apenas uma vez, ao passar por
nós um carro na estrada de Miramar, ouvi dizer a Princesa: "Ele ali vai";
êsse "êle" não podia ser outro senão de quem falava amiùdadas vezes. Mas
o doutor lá seguiu em direc\-ção á Granja, como quem vai de passeio, e ninguem mais o viu.

E após uma pausa prosseguiu:

-- Eu sou amiga dela, mas isso não me impede de dizer aquilo que entendo ser a verdade. Pois o\-lha se aquele rapaz não tem muito que o pretenda... E isto cá entre nós: Que atractivos tem Princesa para que um
rapaz ande babado por ela?.. Nem sequer sabe esconder com um pouco de
rouge a palidez daquelas faces, que é o me\-lhor chamariz para os rapazes.
Até nisso é insípida, credo!... Já vês quanto um doutor há-de andar
babado por aquilo que cheira a bota de elástico... Porque, sabes, todo
o rapaz aprecia o senso estético da mu\-lher na arte com que se pinta, com
que cuida das sobrance\-lhas, das pestanas e das unhas...

-- Oh se aprecia... ata\-lharam as Ortigões. São exi\-gên\-cias a que nos
não podemos furtar, que a sociedade nos impõe.

-- Infelizmente, interveio Henriqueta, são exi\-gên\-cias que ultrapassam
já os limites da burguesia e desta forma não nos distinguem  muitas vezes das outras de meia tigela.

-- É como acontece no campo, confirmou ingenuamente a Henriqueta, que
por vezes tinha uns assomos de literatura campestre. Sob as mesmas côres
variegadas e lindas adejam no espaço borboletas benéfi\-cas ao lado das
ruins. Notem como a do útil bicho da seda é até a mais modesta na sua
alvura, enquanto que a do pavão da noite, insecto altamente nocivo, é a
mais garrida.

-- Eu detesto o campo, disse Ambrósia para derivar a conversa cujo
rumo não lhe agradava agora. E detesto-o pela miséria que transparece
de tudo aquilo que rescende a nitreira. Até as mãos daquela
gente que no-las estende em demasiada sem-cerimónia...

-- Os dias custam tanto a passar, confirmou a Henriqueta. Sempre a
mesma coisa todos os anos, as mesmas árvores, a mesma terra, a mesma gente,
as mesmas lamúrias dos caseiros a queixarem-se de não terem pão!...

-- Que horror! ata\-lhou a Ambrósia virando-se para as Ortigões. Que
nôjo. E vivem bem aqueles pobres diabos, sabes? Não teem pretensões, e o
mundo para êles resume-se ao palmo de terra que cultivam. As exi\-gên\-cias
dêles são só o pão de cada dia.

-- Só são exigentes, emendou a Henriqueta, em pedirem continuamente
obras. As conversas dêles só se referem a um socalco caído nos últimos
temporais, ou ao mi\-lho estragado por falta de cómodos... É só no que
sabem falar. Ora tu bem vês que as terras não dão para se despender um
centavo com elas, e mesmo que dessem, nós temos muito mais em que gastar
o dinheiro. Pois não é verdade?

-- Bem faz o meu Pai, rematou Ambrósia: ás cartas recebidas dos feitores corre-\-lhes a vista, e, ou falam do envio de dinheiros das rendas
ou vão logo para o cesto dos papéis. Pouco se importa êle com lamúrias.
Ainda há dias recebeu a notícia de ter acabado de derruir um muro que
de há anos vinha desmoronando, no principal campo duma fazenda, com a ameaça de o caseiro abandonar a terra porque, diz a carta, destruído aquele
campo não poderá pagar a renda... Muito nos rimos da desconchavada
ameaça. Que o arrange êle, se quiser, á custa do que nos tem roubado...
Senão vejam o cordão de ouro que a fi\-lha dêsse caseiro levou no casamento...

A conversa\-ção prometia eternizar-se entre tão ilustre sociedade.
Já o sol baixava sôbre o mar, donde vinha agora um pouco de frescura quasi desa\-gra\-dá\-vel.

-- Agasa\-lhem-se minhas fi\-lhas, interrompeu a mamã baroneza, já receosa de que a coriza viesse empecer as preciosas vergônteas que ao mundo
dera. E voltando-se para Ambrósia: -- Já estás encatarroada, minha fi\-lha;
a\-ma\-nhã fi\-cas na cama, antes que surja daí alguma pneumonia.

-- Eu posso lá estar na cama! resmungou a Ambrósia. E virando-se para
a mãi: Esqueces, porventura, os compromissos que temos para a\-ma\-nhã? A
recep\-ção em casa das Camanhos e a ceia americana nas Godinhos?

A energia da resposta, acompanhada dum jôgo fisionómico em que
os o\-lhos cintilantes tomaram parte importante, fez reconsiderar a mamã
que na verdade os deveres mundanos eram incompatíveis com qualquer possibilidade
de pneumonia:

-- É verdade, fi\-lha, não me recordava.

-- Se eu tivesse o vagar que teem os camponezes lá da aldeia...
disse entre um enco\-lher de ombros a Ambrósia, como revivescência da conversa que a mamã viera interromper. Santa paz a daquela gente, sem preocupa\-ções como as nossas!...

-- Vamos embora, remata o barão. São horas de irmos até ao Frazão.
E os dois automóveis rolaram pela estrada fora, cobrindo em meia
hora a distância ao Pôrto.

Era quasi noite quan\-do as duas fa\-mí\-lias deram ingresso na casa da
festejada. Da fa\-mí\-lia Ortigão, porém, só os papás e a fi\-lha mais nova apareceram, porque á fi\-lha mais ve\-lha sobreviera um ligeiro incómodo de saúde e a outra irmã fi\-cara a fazer-\-lhe companhia.

Trocaram-se então beijos entre as recém-chegadas e a Julieta, de
mistura com abraços de que a custo esta se desvenci\-lhava, ao mesmo tempo
que se formulavam desculpas por não poderem ter passado a tarde com ela.

-- Que tarde tão bem passada, se tivessemos podido comparecer!
rosnava lamuriosamente a Ambrósia.

-- Nem imaginas o nosso pesar de não podermos comungar contigo a
alegria de tarde inteira, mas acredita que nos foi inteiramente impossível, juntou a Henriqueta.

-- Minhas irmãs estão retidas em casa, terminou a Ortigão, uma por
doença e a outra por lhe fi\-car a fazer companhia. E eu sou a  portadora
duma imensidade de beijos da sua parte, que tenho de acrescentar aos meus.

A imensidade dos beijos era na verdade infindável como que a devo\-rarem-se mutuamente as faces já todas lambujadas.

-- Passastes bem a tarde, não é verdade, rematou Julieta?

-- Mas falamos sempre de ti e, a-par da tristeza de estarmos longe,
contentava-nos a quasi-certeza de estares bem acompanhada, respondeu a
Ambrósia piscando o ô\-lho para a Ortigão.

-- Estive, sim. Ainda há momentos saíram as últimas amigas, que não
quiseram fi\-car para o jantar. Só restam cá as minhas primas e as Garcias,
e agora vocês.

-- Que pena, ata\-lhou maliciosamente Henriqueta, não nos encontrarmos
todas á noite. São tão simpáticas...

Decorreu o jantar com aquela magnitude de casa avantajada em têres
que alberga sob os seus tectos várias fa\-mí\-lias convidadas, embora de tão
heterogéneo sangue. Sobejam as luzes dos candelabros a reflectirem-se
nas paredes cobertas de pratas; erguem-se floreiras de cristal pejadas
de vistosos verdes e multi-coloridas flores; amontoam-se os pratos sôbre
a mesa congestionada de iguarias caras.

Julieta ocupara o lugar de honra ao centro da mesa. As baronezas
tinham-\-lhe segredado que desejavam fi\-car juntas das Ortigões e de Princesa, e o grupo formado por todas estas ocupava todo lado direito da
festejada. Á esquerda, viam-se as Garcias e as primas Coe\-lhos; as cabeceiras e o lado oposto eram ocupados pelos papás e convidados.

Sucederam-se os brindes na me\-lhor intimidade de todos os lados, e a
vozearia entrecruzava-se muitas vezes na maior das alegrias.

As Garcias e as Coe\-lhos conversavam também animadamente entre si.
Preguntavam-se umas ás outras em voz cochichada se já haviam reparado
bem na pobreza do estendal de prendas oferecidas á festejada Julieta,
exposto sôbre a cama do seu quarto de dormir. A nenhuma tinha passado
despercebido o contraste entre as oferendas de outrora com as de hoje.
Que de há poucos anos para trás a cama cobria-se de múltiplas e valiosas
dádivas com que nós outras, as antigas amigas -- dizia uma das Coe\-lhos --
contribuíamos para a sua festa natalicia, manifestando-\-lhe a nossa afei\-ção; hoje é aquilo que vêem.

-- Amor com amor se paga, ata\-lhou logo outra das Garcias. Desde que
as Corvos passaram a fazer parte da sua roda, estas Frazões esque\-ceram-se
das suas antigas amigas.

-- Não me esquece a partida que Princesa fez á Marília, acrescentou
outra das Garcias. Aquela desculpa de não poder visitá-la no seu aniversário por ter de ir para a praia... e ainda para mais dentro de um sobrescrito comercial!

-- Estas minhas primas, de facto, continuou uma das Coe\-lhos, modifi\-caram-se tanto de há 4 anos para cá... O\-lhem das amigas antigas quantas
cá apareceram esta tarde! É que quanto nos eram simpáticas noutros tempos como hoje se nos estão a tornar antipáticas.

-- Princesa tem-se feito tão altiva... disse outra das Garcias. E
viran\-do-se para as Coe\-lhos ali rememorou os tempos da mocidade do vasto
grupo, em que as Frazões recebiam da mãi dinheiro para comprarem um camarote no São João, nas noites de Carnaval, e sujeitavam-se a cadeiras na
plateia para com a diferença de preço poderem beneficiar outras amigas
sem recursos. Que belos tempos êsses, terminou com um suspiro. Eram-nos
tão simpáticas, então, estas raparigas, e agora parece que já nos vêem com
sobranceria.

-- Nem admira, objectou ironicamente uma das Coe\-lhos. Nós não temos
títulos nobiliárquicos nem dinheiro para os comprar, para que a nossa
companhia se imponha como a das ilustres Corvos... Bem vês que as minhas primas agora subiram na escala social, e já se arrogam o direito
de me pôrem na rua á meia-noite, como o fizeram há tempos quan\-do eu, por
ordem da sua mãi, cortava a liga\-ção telefónica com aquele lambisgóia por
quem Princesa se babava toda.

-- A propósito, interrompeu uma Garcia: Eugé\-nio continua a fazer
a côrte a Princesa?

E como as outras enco\-lhessem os ombros em sinal de ignorância, a
Garcia continuou: Pois não era feio par. Que quere ela? A êle conheci-o eu naquele assalto na noite de sábado gordo em casa de Ofélia e
Marília, e por sinal que o não achei nada antipático como ela o espa\-lhava
no nosso grupo. E por falar em assaltos: Até nêsse ponto Princesa se
mostrou quem é; o\-lhem como ela trocou a nossa companhia nessa noite de
reinadio, entre fa\-mí\-lia, entre as suas antigas amigas, para ir ao baile
do Grande Hotel do Pôrto com a outra sociedade das baronezas e parceiras
seme\-lhantes. Não repararam na fotografia do Primeiro de Janeiro com que
vestido ela lá apareceu? Como ela está mudada!..

-- É a escola das baronezas, replicou outra Garcia.

-- Sabem que Eugé\-nio anda a fazer um diário da sua vida, onde põe
em destaque os seus amores com Princesa? preguntou a Coe\-lho mais ve\-lha.
É o que consta, afirmou.

E perante a curiosidade do grupo, relatou que até já o preguntara
uma vez a Ofélia, de cuja fa\-mí\-lia Eugé\-nio é médico e onde vai muitas
vezes, entrando ali com certa familariedade. Ofélia, porém, não se abriu,
sempre foi dizendo que já viu duas fo\-lhas avulsas dum romance que
Eugé\-nio andava a fazer, que romance devia ser porque aquelas fo\-lhas que
Eugé\-nio lhe mostrou por se referirem á casa da Lapa deviam constituir
o episódio dum livro. Quisera saber mais coisas a êste respeito, mas
Ofélia alegou ignorar, embora seja pouco crível que o não saiba.

Depois falou de Eugé\-nio. É muito correcto. Querem ver uma pequena
faceta do seu carácter que denota a muita considera\-ção que lhe merecem
os mais humildes? Assisti eu, por acaso, á cena. Ao entrar num carro
eléctrico, mostrou o bi\-lhete ao condutor sem que êste lho pedisse,
para lhe poupar a desloca\-ção. Esta faceta define o Eugé\-nio, principalmente se pusermos a sua conduta em contraste com a de toda a gente que
entra e vai sentar-se comodamente, obrigando o condutor a um contínuo
vai-vem, quan\-do ainda por cima o não recebe insolentemente. Pois Eugé\-nio
procede em todos os casos com idêntico civismo, hoje tão raro!

-- Nem vejo -- prosseguiu a Coe\-lho -- razão para que Princesa o ridicularizasse por aquela forma na Foz, lembram-se? Bem sei que isso está
no seu modo de ser. O primeiro pretendente que teve, o primo, era o "barão
do semicúpio" por ter uma casa de artigos sanitários; mas êle soube virar-\-lhe as costas a tempo. O segundo pretendente, o irmão das Ortigões,
foi cognominado "O Pinchinhas" pela forma como dan\-çava; e o mais curioso foi a Sr.ª D.\ Maria barafustar quan\-do êle apareceu casado com uma
loira toda pintada, porque afinal, dizia ela, ”o Pinchinhas não dava aprêço
nenhum ao natural da minha Princesa, nem ao seu tipo moreno, por quanto o
trocou por muito diversa mu\-lher..."

Por seu turno, as baronezas conversavam animadamente. Tinham levado
de seguida todas as fitas que se exibem nos cinemas do Pôrto, depois que
acabaram os bailes de Carnaval. Desde então já es\-tre\-aram 2 vestidos de
noite, cada uma das manas. Já não usavam o rouge Pompeia, mas viraram-se
agora para o Tangée que é me\-lhor e mais caro e está agora na moda. E o
baton e o pó de arroz também não o querem doutra marca. Na semana passada fôram a Lisboa mas sem demora; o papá ainda precisava de fi\-car lá
uns dias por motivo de negócios, mas elas é que não podiam interromper
as massagens encetadas no Instituto de Beleza; a viagem foi, como de costume de automóvel, pois há 15 anos que não utilizam outro meio de transporte.

-- E o teu apaixonado? preguntou Julieta a Henriqueta.

-- É verdade, respondeu: agora me lembro que deixei em casa uma carta
dêle recebida de manhã que ainda não abri. E por causa disso estou em
o mandar embora. Porque é uma massada, de manhã cedo vir a creada com a
correspondên\-cia enquanto estou reco\-lhida. Ainda não era bem meio-dia, e
já me batia á porta aquela parva: "menina Henriqueta, tem aqui uma carta".
Confesso-te que me apeteceu lan\-çar pela janela fora a salva de prata que
me estendia, com o conteúdo. Descompu-la pela milésima vez por me vir incomodar, e peguei na carta que atirei para cima da toilette. E nunca mais
me lembrei nem da carta... nem do remetente.

E mudando de tom: Coitado, êle é massador mas tem bons predicados.
Dan\-ça muito bem, veste a primor, é elegante e frequenta a me\-lhor sociedade.
Em todas as sessões da moda êle lá está, quer se trate de cinemas, ou de
bailes. Porque nem doutra forma eu o tolerava. É rialmente simpático, e
tem dinheiro. Há porém um defeito nele que não me agrada: é a sua teimosia em usar água da Colónia de Flores del Campo, de que eu não gosto;
já lhe tenho dito que prefiro que use a Houbigant. Mas, enfim, remedeia
até encontrar outro me\-lhor...

Na roda das mamãs discutia-se o modêlo dos chapeus comprados na véspera pelas fi\-lhas dos Senhores do Corvo.

-- A verdade é esta, dizia a senhora baroneza: no nosso tempo corriam-nos se nos apresentassemos com seme\-lhantes modelos. Mas agora estão na moda e eu não posso privar as minhas fi\-lhas de se exibirem consoante os últimos figurinos de Paris. Lá de bonito nada teem aqueles chapeus que
que mais parecem de pa\-lhaços, pequenos e afunilados, que nem o cocuruto
da cabeça escondem, e ainda para mais abertos em cima!

-- La noblesse oblige, corroborou a mamã Ortigão a quem a alusão á
gran\-de capital veio recordar o universal lugar comum. As minhas 3 fi\-lhas
quiseram chapeus todos diferentes, que tanto são precisamente os que estão em voga. O da Francelina é uma torsade que em nada difere -- Deus me
perdoe -- das rodi\-lhas que as camponezas põem na cabeça para carregarem
com cestos, mas vá lá uma pessoa dizer-\-lho, que toda ela se irrita; é moda e pronto! A Anacleta quis um dos tais de funil arrombado. E a Natércia esco\-lheu aquele modêlo de banda, direito para cima, que deixa a cabeça toda a amostra, como de resto acontece com os outros.

Na cozinha o numeroso pessoal criticava a seu modo o dia. Pu\-\mbox{nha-se}
em contraste a despesa feita com o chá e o jantar dos anos de Julieta
com as necessidades de tanta gente.

-- É que, isto aqui entre nós, dizia uma servente, só com os créscimos
do chá e do jantar de hoje eu fazia um banquete que chegava para empanzinar os pobres da minha freguezia.

Depois falava-se da assistência ao jantar:

-- Aquelas meninas baronezas são tão feias, senhor me perdoe, objectava a cozinheira, e aquelas Ortigões andam tão pintadas que até parecem
mascaradas, santo Deus. Que a culpa é dos pais que não teem um chicote
com que as obriguem a andar mais decentes.

-- E aqueles aneis, todos iguais, dos fidalgos do Corvo, que até parece que trazem um carvão negro agarrado, exclamou um criado!

-- A criada de sala das meninas Frazões explicou então perante aquela
assembleia que as meninas já uma vez lhe haviam posto em pratos limpos
a signifi\-ca\-ção daqueles aneis. Os brasões são solares, porque o sangue
daquela gente é genuíno; não são como aqueles que se vêem amiúde divididos
em dois ou mais compartimentos a representar misturas de outras tantas
fa\-mí\-lias. As armas são constituídas por um corvo, que deu o nome aos srs.
barões, que é aquilo que se vê muito preto por que é feito com uma pedra
preciosa que se chama azeviche. O resto do brazão, que envolve as armas,
chamam-\-lhe o paquife que é uma ornamenta\-ção sem valor algum.

-- Bravo! apoiaram as circunstantes, estupefactas com tão sábia li\-ção. Bem se vê que priva com as amigas das baronezas.

-- Amigas de agora, ata\-lhou a criada, a quem o rubor tinha invadido as
faces. Amigas de agora, repetiu. Porque cá as meninas dantes não as podiam ver, e troçavam-nas muito quan\-do as baronezas cá vinham. E foi até
duma dessas vezes que elas me contaram isso, acrescentando que o brazão
era de pureza solar, sem miscelânea de sangues de outras fa\-mí\-lias, porque
o título havia sido comprado há pouco tempo. Até lhes chamavam as baronezas do carvão, concluiu rindo-se.

-- Não que, isto entre nós, apoiou a cozinheira: As meninas em outro
tempo tinham uma convivência mais sã que esta de agora. Ele aparece aí
cada uma, t'arrenego!... Até aquela pa\-lhaça que aqui vem dar massagens á
menina Julieta, uma mu\-lher casada tôda mascarada que até parece outra coisa.

-- Pchiu, ouviu-se dum lado. Não digam mal dessa criatura, que o nosso
patrão se o sabe põe-nos imediatamente na rua...

-- Hein? preguntaram algumas criadas com a curiosidade aguçada por
tal revela\-ção.

-- São cá coisas, que se não podem dizer, respondeu o que havia falado.
É o que se diz lá por fora. Eu como nunca vi, nada posso acrescentar.

-- Má língua, bradou a cozinheira.

-- É má língua, é..., repontou o mesmo, com ares de quem está no segredo dos deuses. Ora diga-me cá: como é que a massagista pode ter automovel, e o marido outro? Logo 2 automóveis numa casa, e a girarem todo o
dia, cada qual para seu lado...

-- Lá que o patrão não se ensaia muito para isso, é verdade, disse a
cozinheira já vencida. Ai que as más línguas falam tanto do Sr.\ Frazão...

Eis as conversas com que todos os sectores da festiva casa preenchiam o tempo, auxiliando com a boa disposi\-ção as difíceis digestões de
tão empanzinados convivas.

\begin{center}\bf\large A derrota da Fran\-ça\end{center}

O estado de guerra continuava, por ora, sob a forma de convulsões
isoladas a sacudir aqui e além a Europa.

A Finlândia fôra ingloriamente vencida, na mais desigual das lutas,
pela Rússia, e obrigada a aceitar um tratado de paz em 12 de Março sob
pesadíssimas condi\-ções, como a cedên\-cia de territórios no istmo da Carélia
e nas margens do Ládoga, e ainda o estabelecimento de bases militares na
península de Hangoe, a entrada do Golfo da Finlândia, etc.

Em 9 de Abril a Alemanha invade a Noruega, cujas águas estão cercadas
pelas minas inglesas, de que se torna necessário desembaraçar a
todo o custo, em virtude dêste país ser a única via para o transporte do
ferro sueco desde que o golfo de Bótnia se encontra inavegável pelos gelos. A Noruega resistiu e, com o pronto auxílio dos aliados,
fez-\-lhe frente. Entre o Mar do Norte e o Báltico, desde Skagerrak até
Narvik, foram formidáveis e encarniçadas as bata\-lhas aero-navais que então se desenrolaram até Maio.

Simultaneamente á da Noruega verificou-se a invasão da Dinamarca.
Esta, porém, entregou-se sem a mínima resistência.

Os contendores batiam-se, pois, a distância dos seus paizes, como que
a experimentarem a fôrça do adversário, mutuamente, antes de terem de se
haver com êle directamente. De resto, aguardava-se dum momento para o outro o início da verdadeira guerra, da guerra ainda não começada, do encontro frente a frente dos colossos em litígio.

As linhas de defesa, de Maginot e de Siegfried, consideram-se intransponíveis, e por isso pela fronteira franco-alemã nada há a temer.

Já se haviam feito tentativas de invasão germânica através da Holanda e Bélgica. Era ali, de facto, o ponto vulnerável, pronto a ser atravessado dum momento para o outro. A Alemanha vendo cada vez mais apertado
o cêrco económico contra si -- e a escassez do ferro era de primordial importância vital na guerra -- estrebuchava furiosamente. Já muitos a consideravam no estertor da morte.

Eis, porém, que repentinamente, na madrugada de 10 de Maio, a Alemanha,
em arremetida violenta, invade a Holanda e Bélgica que  mal tempo tiveram para pôrem em ac\-ção as suas defesas.

Em Inglaterra, o govêrno de Chamberlain é imediatamente substituído
pelo de Churchill, de carácter mais bélico.

Foi na realidade o comêço da nova gran\-de guerra. Houve gazetas que
começaram a numerar desde então os dias da guerra, nos seus cabeça\-lhos
espaventosos.

Graças a um novo exército, de paraquedistas, já experimentado na invasão russa á Finlândia, mas aperfeiçoado pelos alemãis, os are\-ódro\-mos
dos paizes invadidos foram rapidamente subornados, e até os diques holandeses, no momento em que iam ser postos a funcionar contra o invasor, fôram tomados de assalto.

A invasão alemã tornou-se assim fulminante, e em 5 dias capitulava
a Holanda. Ficando então o território belga, já replecto de tropas anglofrancesas, a servir de teatro de guerra contra o inimigo comum, começou
imediatamente a travar-se a gran\-de bata\-lha do Mosa, no dia 14, entre Liège
e Sédan. O 16 de Maio era tristemente ca\-ra\-cte\-ri\-zado pela transferência
do govêrno belga para Ostende, em virtude de estar iminente a rendi\-ção
de Bruxelas, verifi\-cada no dia seguinte.

A invasão-relâmpago, como lhe chamaram, desnorteava os aliados. As
rédeas do govêrno francês passam em 18 para as mãos de Reynaud, o qual
dois dias depois substitui Gamelin por Weigand no comando superior das
tropas aliadas. Está nesta altura a bata\-lha do Mosa no seu auge, tendo os
alemãis conseguido abrir uma enorme brecha até Amiens e Arras, em direc\-ção á lancha, surpreendendo os aliados com o esplendor do seu exército
motorizado posto em actividade.

A guerra de trincheiras fôra totalmente
 destronada, e substituida pela guerra de movimento, extremamente rápida,
 para a qual os aliados não
estavam preparados. Cada dia traz avanços aos alemãis que vão até 50Km.
ou mais, de profundidade, e em 22 de Maio a brecha atingia Abbeville na
foz do Soma, já a poucos quilómetros do mar da Mancha.

O desapontamento era geral, a par da desorienta\-ção perante tão inesperado como inédito avanço. O parlamento inglês confere ao govêr\-no autoriza\-ção para socializar pessoas e bens de todos os cidadãos in\-glêses,
atenta a gravidade do momento.

Entretanto a situa\-ção é cada vez mais delicada. As colunas motorizadas de Abbeville progridem, e no dia 23 travam-se combates nas proximidades de Bolonha e Calais. As tropas anglo-franco-belgas que operam
na Bélgica e noroeste da Fran\-ça estão agora completamente envolvidas
por terra.

A rendi\-ção do rei dos belgas, em 28, vem agora tornar mais aguda a
crise em que se encontram os aliados. O momento é solemne. A tragédia
afigura-se inevitável; mas a derrota da Fran\-ça, apesar de tudo, é hipótese
inconsistente, muito embora os rumores propalados sôbre a desmoraliza\-ção
dos seus exércitos.

A evacua\-ção das tropas envolvidas, num total de 300 e tal mil homens,
conseguiu fazer-se através de Dunkerque para o território ingles, sob a
chuva incessante da metra\-lha alemã; a heróica resistência desta cidade
termina após o reembarque dos últimos soldados, em 4 de Junho, que abandonaram todo material de guerra ao inimigo.

Na véspera havia a avia\-ção alemã bombardeado Paris. Para esta capital começam agora a virar-se as tropas alemãs, sucedendo-se a mais dura
bata\-lha que a his\-tó\-ria regista, pois contam-se por meio mi\-lhão os invasores e por 4 mil os carros de assalto blindados que atacam macissamente,
numa extensa frente que vai desde o mar até Montmedy, procurando atravessar o rio Aisne.

Por outro lado, desde há certo tempo que as duas partes adversárias
vinham fazendo pressão sôbre a Itália no sentido de conseguirem a sua
adesão, mas ultimamente os aliados já se contentavam com a neutralidade
daquele paiz cuja simpatia se verifi\-cava ser cada vez mais flagrantemente favorável á Alemanha.

O dia 10 de Junho era assinalado pela declara\-ção de guerra da Itália
á Fran\-ça e á Inglaterra. Era a última machadada que havia de precipitar
a tragédia prestes a desenrolar-se. Redobram de intensidade os pedidos
de armamento á America do Norte para a Fran\-ça agonizante. A guerra, agora no Mediterrânio, ameaça alastrar a mais paizes. A armada e a avia\-ção
britânicas começam a ter duros encontros com os italianos, quer nas costas da metrópole, quer nas suas colónias africanas.

Entretanto os alemãis, já conquistada tôda a costa norte da Fran\-ça,
avan\-çam sôbre Paris com tanta mais facilidade e rapidez quanto maior é
a desmoraliza\-ção das tropas, cuja resistência é cada vez menor. No dia
13 é mesmo abandonada a gran\-de capital ao inimigo, antes que fique destruída.

A 2ª gran\-de bata\-lha, a da Fran\-ça, a mais gigantesca e encarniçada
bata\-lha de que há memória, iniciada em 5 de Junho, estava praticamente
ganha pela Alemanha que, com rapidez vertiginosa, infligira a maior derrota que o adversário conta na sua his\-tó\-ria. Crê-se no epílogo do gran\-de
drama em que se vão jogar os destinos da raça latina.

Em 22 era finalmente assinado o armistício entre a Fran\-ça e a Alemanha, para cessarem as hostilidades somente 6 horas após a assinatura
da Itália o que só no dia 24 se verificou. E no dia 25, pelas 1 1/2 horas da madrugada, cessa a guerra com a Fran\-ça vencida, que hoje decreta
luto na\-cio\-nal no seu território.

Estava terminado o gran\-de insucesso. Começa agora a discussão das
causas que o tornaram possível com surpreza de todos aqueles que julgavam invencível a Fran\-ça. O alastramento das ideias comunistas, que teriam levado os governos a menosprezar armamentos, tornados escassos e,
sobretudo, antiquados perante a evolu\-ção da técnica militar; o decréscimo evidente da natalidade francesa, desfalcando gravemente a demografia,
como processo de maltusianismo comunizante e dissolvente sobretudo das
classes mais favorecidas; a dissolu\-ção, enfim, da fa\-mí\-lia, na onda de prazer
e egoísmo materialista que impera na sociedade, ávida só de regalias, queixando-se da sobrecarga de deveres, e tendo como consequência imediata a
falta de disciplina que tanto se fez sentir ainda na frente das bata\-lhas
onde os soldados franceses se recusaram a cumprir as ordens superiores
de avanço, depondo as armas ao som do hino interna\-cio\-nal.

Contrastando com a da na\-ção vencida, a disciplina alemã é um do\-gma:
o alemão é um autómato sem personalidade, sem direitos, e só impera nele
o dever de se bater pela sua pátria e pelo sonho de dominar o mundo.

Entretanto a Rússia, sempre calada até aí, invade a 28 de Junho a
província romena da Bessarábia e o norte da Bucovina, ricos celeiros
de trigo, o primeiro dos quais lhe havia pertencido até 1918. Precedeu
a invasão um ultimato a que a Roménia não teve outro remédio se não ceder.

Entre as condi\-ções do armistício germano-francês conseguiram nobremente os plenipotenciários aliados que a sua esquadra não pudesse ser
utilizada pelos alemãis contra a Inglaterra no decurso da luta que agora
prosseguia contra esta. Os ingleses, porém, não confiados no compromisso
alemão, resolveram dar caça á esquadra francesa, antes que ela fôsse porventura aproveitada contra si pela Alemanha. A bata\-lha de Oran, no Mediterrânio, em 3 de Ju\-lho, foi a primeira das várias que se haviam de travar
no sentido de dominar a esquadra francesa, e motivou o rompimento de rela\-ções entre a Fran\-ça e a Inglaterra verifi\-cado no dia 5.

Na noite de 4 para 5 era Gibraltar por sua vez objecto de ataque
levado a efeito por aviões de na\-cio\-nalidade desconhecida, ataque esse que
de futuro se iria repetir amiúde.

As bata\-lhas sucedem-se agora irregularmente, entre as avia\-ções e as
esquadras inglêsa e alemã ou italiana, sôbre a Mancha, como nas gran\-des
tentativas de invasão germânica de 10 de Ju\-lho e 9 e 12 de Agosto, ou
nos diversos mares que banham domínios ingleses ou italianos ou sôbre
êsses mesmos territórios e ainda sôbre a Alemanha.

Por outro lado, a Fran\-ça arrazada modifi\-cava a sua fei\-ção política.
O chefe do govêrno, Pétain, assumiu ditatorialmente as fun\-ções de chefe
de estado em 12 de Ju\-lho, depois de convidar Lebrun a abdicar.

Assim decorreu a primeira parte dêste verão, triste como o anterior,
prenhe de surprezas que o noticiário telegráfico ou radiofundido nos
trazia constantemente, abatendo quasi sempre os ânimos ainda que fortes
perante a série ininterrupta de desastrosos revezes sofridos pelos aliados, raras vezes os levantando um pouco em face de qualquer pequeno êxito
obtido.

Pode afirmar-se que a ávida sofreguidão com que eram esperadas as
notícias hora a hora, e as emo\-ções sofridas, traziam esgotados os nervos
de tôda a gente a quem não faltasse de todo a sensibilidade.

O que será o dia de a\-ma\-nhã? -- Era a pregunta que todos se faziam,
ao ver baquear a civiliza\-ção latina e perante o jugo teutónico que ameaçava tudo absorver para em tôda a parte implantar as suas ideias político-económicas.

\fotoD

Portugal era dos rarís\-si\-mos paizes da Europa onde o sossêgo era
relativamente de apreciar. Todavia o ambiente de terror em que todos
mergu\-lhavam constituia campo magnífico para a germina\-ção do boato. Era
então a exi\-gên\-cia dos dois adversários ao mesmo tempo, junto do govêrno
por\-tu\-guês, para na nossa costa estabelecerem bases navais. Era a Espanha,
esfomeada e intranquila desde a sua guerra civil, a exigir a anexa\-ção
do nosso solo, tendo chegado ali a publicar-se que o 1940 teria de ser
a desforra do 1640. Era mais lido aquilo que, no mesmo género, pudesse
acudir á imagina\-ção dos profissionais, posto a correr de bôca em bôca
por entre um público deprimido ante tanta surpreza interna\-cio\-nal.

Chegara o mês de Ju\-lho, frio, a pronunciar um verão nada criador como
já o fôra o do ano passado, e tristonho como o ambiente interna\-cio\-nal. Continuava Portugal a constituir uma das raras na\-ções onde ainda havia bem-estar material. A fome e o luto que assolavam a maior parte da Europa
não se tinham feito sentir aqui, assim como o permanente desassossêgo dos
ataques aéreos.

Se não fôra a boataria derrotista referente ao paiz, ou a solidária
vibra\-ção dos nacionais com os estranjeiros que sofrem os horrores da
guerra, dir-se-ia que a vida em Portugal era a mais edénica e despreocupada vida que era dado viver-se á superfície da terra. Assim
a consideravam os numerosos refugiados das na\-ções invadidas aco\-lhidos no
nosso paiz, e que comunicavam a cidades e praias, onde se fixaram, acentuado sabor cosmopolita por vezes.

Para mais, festejavam-se nês\-te recanto ocidental dois centenários
-- o oitavo da funda\-ção da na\-cio\-nalidade, e o terceiro da restaura\-ção da
independên\-cia -- e as pompas prolongavam-se desde Junho a Dezembro com
a participa\-ção sucessiva de todas as localidades ligadas á his\-tó\-ria portuguesa. A cruz azul sôbre fundo branco, que constituira a bandeira
ostentada há oito séculos por Afonso Henriques, flutuava altaneira
 nos mastros ou nas janelas de todos os prédios oficiais ou particulares,
 e até na frente dos automóveis circulantes na via pública, ao lado das bandeiras
na\-cio\-nal e da restaura\-ção. As gazetas enchiam páginas com gran\-des fotografias dos cortejos organizados pelo paiz fora, e as boas-vindas ás
embaixadas especiais das na\-ções que em gran\-de número se haviam dignado
fazer-se representar. A Exposi\-ção do Mundo Por\-tu\-guês, em Lisboa, alardava-se como pedindo meças ás gran\-des exposi\-ções internacionais realizadas lá fora.

{\small
\newpage\rule{0pt}{5em}\begin{center}\sc\normalsize Memória biográfi\-ca\end{center}

%%Memória biográfi\-ca

Ludgero Lopes Parreira nasceu a 22 de Junho de 1900 em Penha-Longa, Marco de Canaveses, fi\-lho do professor Henrique Fernandes Lopes Parreira (1843-1906) e de Emília Augusta de São José (1861-1900). Casou-se a 23 de Março de 1945 com Ângela Martins Albuquerque Oliveira Parreira, na Igreja de S. José, em Lisboa. O seu único fi\-lho, Dr.\ José Henrique Oliveira Parreira, nasceu a 19 de Março de 1946 na freguesia de Paranhos, Porto.

%%1972.12.17: Casamento de fi\-lho Dr José Henrique Parreira com Dr.\ Maria José Alves Moreira Parreira, em Penafiel
%%Falecido: 1 de set de 1974 (com a idade de 74) no Porto-Bonfim-Rua do Lima, 52
%%Funeral: Marrancos, Vila Verde.
%%
%%Profissão: Médico
%%Liceu Rodrigues de Freitas – Porto e B.C.G.-- Porto
%%Também foi Médico Escolar das Escolas do Porto (na época, existiam dois Médicos escolares para todas as Escolas do Porto)

Teve como irmãos o inspector escolar Acácio Fernandes Lopes Parreira (1885-1955),
%%casou com Prof. Maria Rocha da Concei\-ção em 31 Janeiro 1909, em São Cristóvão de Nogueira, Cinfães, Viseu, Portugal. Eles eram os pais de quatro fi\-lhos e duas fi\-lhas.. Morreu em 24 Outubro de 1955, em Marco de Canaveses, Porto, Portugal, com a idade de 69 anos
Alcino Lopes Parreira (1886-1969),
%% Alcino Fernandes nasceu em 23 Novembro de 1886, in Penha Longa, Marco de Canaveses, Porto, Portugal.Casou com Germana Ferreira Dias Torres em 23 Maio 1909, no Bonfim, Porto, Portugal. Viveu no Porto, Portugal em 1909. Morreu em 24 Junho 1969, em São João do Souto, Braga, Portugal, com a idade de 82 anos
Demétrio Lopes Parreira (1894-1960) e José Lopes Parreira (1891-?).

Segue-se um curriculum vitae seu, que compilou em 1971.

\vskip 1em\begin{center}\sc\normalsize Curriculum Vitae de Ludgero Lopes Parreira
\end{center}

\noindent
-- Médico efectivo dos Serviços Médico-Sociais desde há 23 anos, na cidade do Porto, a
quem foi oferecido há já 9 anos um lugar de Inspector -- Médico nesses serviços, que então não lhe convinha.
\\
 -- Curso de Medicina Sanitária com 17 valores.
\\
 -- Aprova\-ção em dois concursos por provas públicas, para médicos escolares em 1934 e para assistentes de vacina\-ção do B.C.G.\ em 1961, cujos lugares desempenhou até 1970.
\\
 -- Louvor publicado no Diário do Governo, 2ª série, nº 97, de 27-4-34.
\\
 -- Oferta duma bolsa de estudo no estrangeiro pela Funda\-ção Gulbenkian, em 1956, por intermédio do então Governador Civil Dr.\ Domingos Braga da Cruz e do Diretor do Instituto Superior de Higiene do Porto Dr.\ Gonçalves Ferreira, que não pôde ser aproveitada.
\\
 -- Inquéritos alimentares na Régua, em 1954, de colabora\-ção com o Dr.\ Rocha Faria, e no Porto em 1969-70.
\\
 -- Publica\-ção de mais de 3 dezenas de traba\-lhos respeitantes a vários sectores de actividade médica e não médica, dentre os quais um, não publicado senão parcialmente, versou o tema “O Seguro na Doença e os Serviços Médico-Sociais”. Eis a lista desses traba\-lhos:

\begin{itemize}
\item[]
1 -- A Higiene na Tuberculose (Conferência realizada pela Semana da Tuberculose em 4-6-31, no Liceu de Portalegre).
\item[]
2 -- Legisla\-ção Médica -- Porto, 1932 (esg.).
\item[]
3 -- Cadastro Sanitário do Liceu Mouzinho da Silveira (Tese apresentada ao concurso para médicos escolares).
\item[]
4 -- Un Caso Clínico de Adenites Bacilares -- Barcelona, 1933, Separata de La Clínica nº 4.
\item[]
5 -- 1º Suplemento (relativos ao ano de 1933) da Legisla\-ção Médica -- Porto, 1934.
\item[]
6 -- A Higiene no combate da Tuberculose (Conf.ª realizada pela Semana da Tuberculose, em 5-5-34, na Escola Industrial de Portalegre.
\item[]
7 -- Preceitos de Higiene Física, Higiene Moral, Higiene Intelectual -- Porto, 1935.
\item[]
8 -- Higiene Moral – A Forma\-ção do Carácter -- Porto, 1936.
\item[]
9 -- A Medicina Escolar (Palestra realizada no Clube dos Rotários do Porto, em 17-11-38).
\item[]
10 -- Os Desportos na Forma\-ção Moral (em A Saúde Escolar, nº 18, de Dezembro de 1937).
\item[]
11 -- Cruzeiro à Madeira (Palestra feita no Clube dos Rotários do Porto, em 27-7-39, sobre a viagem realizada em Abril deste ano.
\item[]
12 -- Vizinhan\-ças incómodas, insalubres ou deterioradas dos estabelecimentos de ensino (em A Saúde Escolar, nº 26, de Abril de 1939, e noutras revistas).
\item[]
13 -- A Lei de 1917 e a Saúde Escolar (em A Saúde Escolar, nº 27, de Junho de 1939).
\item[]
12 (sic) -- Vizinhan\-ças incômodas, insalubres ou deletérias dos estabelecimentos de ensino (em A Saúde Escolar, nº 26, de Abril de 1939, e noutras revistas).
\item[]
14 -- Bases para um Estatuto do Exercício de Medicina e Artes Afins (na Gazeta Médica, nº 2 e 3 -- Lisboa, 1940).
\item[]
15 -- Higiene Elementar (Geral e Escolar) -- Porto, 1941.
\item[]
16 -- Anomalias Vertebrais (dois casos de espondiloesquise) -- Sep.\ do Arquivo de Anatomia e Antropologia, vol.\ 22 -- Lisboa, 1942.
\item[]
17 -- A Medicina Escolar -- Palestra radiofónica emitida pela Ideal Rádio, na noite de 25-3-943, a convite da delega\-ção no Porto da Cruz Verme\-lha.
\item[]
18 -- Antropometria Escolar -- Sep.\ do Jornal do Médico nºs 381 e 412 -- Porto, 1950.
\item[]
19 -- O Imposto Profissional --  Separatas do Jornal do Médico nº 180 e nº 371 -- Porto, 1946 e 1950.
\item[]
20 -- A Saúde Escolar a Instru\-ção Primária do Porto (conferência). Sep.\ do Jornal do Médico nº 40 3 -- Porto, 1950.
\item[]
21 -- Alimenta\-ção nas Cantinas da M.P.\ -- Lisboa 1951.
\item[]
23 (sic) -- A Profilaxia Estomatológica na Saúde Escolar (conferên\-cia). Separata da Odontoestomatologia Portuguesa, nº 9 -- Lisboa, 1952.
\item[]
24 -- A Saúde Escolar no ensino Primário e Infantil (Tese apresentada no I Congresso de Protec\-ção à Infância) -- Lisboa, 1952. Sep.\ do Jornal do Médico nº 528, de 7-3-953.
\item[]
25 -- Vacina\-ção pelo B.C.G.\ nas escolas Primárias (Conferência). Sep.\ do Jornal Médico nº 557 -- Porto, 1953.
\item[]
26 -- Normas Científi\-cas da Alimenta\-ção (Conf.\ na delega\-ção do Porto da Sociedade Portuguesa de Ciências Veterinárias, em 30- 10-954.
\item[]
27 -- No\-ções Indispensáveis sobre a Vacina\-ção Antituberculosa (Li\-ção ao Curso de Aperfeiçoamento do Professorado Primário do Porto, em 29 e 30-6-956).
\item[]
28 -- A Refei\-ção dos Escolares -- uma experiência com proteínas animais. Sep.\ do Jornal do Médico nº 727 -- Porto 1956.
\item[]
29 -- A Assistência Alimentar através das cantinas escolares -- Sep.\ do Jornal Médico nº 383 -- Porto, 1959. % 1l entar porquet.69 das Cantinas Escolares la
\item[]
30 -- A Reforma da Medicina Escolar -- Sep.\ de O Médico nº 402 -- Porto, 1959.
\item[]
31 -- A Refei\-ção nas Cantinas Escolares do Porto -- Sep.\ do Jornal do Medico nº 880 -- Porto, 1959.
\item[]
32 -- As Condi\-ções Higiênicas dos Edifícios Escolares do Porto -- Sep.\ do Boletim Cultural da Cãmara Municipal do Porto, vol.\ XXVIII, Fasc. 1-2 e 3-4. Porto, 1965.
\item[]
33 -- A Soja na Nutri\-ção -- Sep.\ do Jornal do Médico nº 922, de 24 de Setembro de 1960.
\item[]
34 -- Aplica\-ção simultânea das vacinas do B.C.G.\ e tríplice (de colabora\-ção com o Dr.\ J.M.\ Rodrigues Pereira) -- sep.\ do Jornal do Médico nº 1173, de Ju\-lho de 1965.
\item[]
35 -- Assistência alimentar aos Escolares na Preven\-ção dos Estados Subnutricionais (Conferência proferida em 22-7-1969, na Figueira da Foz, ao 1º Curso de aperfeiçoamento para Médicos Escolares em Educa\-ção Físico-Desportiva-Saúde-Escolar, nº 20, de Outubro de 1969).
\item[]
36 -- O Seguro na Doença e os Serviços Médico -- Sociais (não publicado).
\end{itemize}
-- Colabora\-ção agrícola, e outra, dispersa por vários semanários noti\-ciosos do Marco, Caldas das Taipas e Guimarães, em alguns números da Gazeta das Aldeias (Porto), na revista de Medicina Veterinária nº 220, de Junho de 1920 (Lisboa), e ainda na Mensagem dos Campos e em O Pelicano (Porto). E de medicina escolar, em O Diário de lisboa de 28-8-34, e em alguns números de A Saúde Escolar (Lisboa) e do Boletim do Liceu Rodrigues de Freitas (Porto).

\vskip 1em\noindent
Novembro de 1971.
%%__________________________________________________________
%%
%%
%%Notas: A viagem à Madeira é em Abril de !939
%%           A data que se encontra no término do romance é Agosto de 1939 (deve o tio Parreira ter acrescentando notas após à data referida dado que fala da guerra.
} % small

\newpage

~\vfill \sf\footnotesize\centering
Impressão:\\Gráfica da Graciosa
\\Ponte de Lima
\\Dezembro de 2025
\thispagestyle{empty}

\end{document}

